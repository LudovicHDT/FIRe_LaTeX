%% Code source du rapport de 2e année
%% Copyright (C) 2016  Ludovic Hondet

%% This program is free software: you can redistribute it and/or modify
%% it under the terms of the GNU General Public License as published by
%% the Free Software Foundation, either version 3 of the License, or
%% (at your option) any later version.

%% This program is distributed in the hope that it will be useful,
%% but WITHOUT ANY WARRANTY; without even the implied warranty of
%% MERCHANTABILITY or FITNESS FOR A PARTICULAR PURPOSE.  See the
%% GNU General Public License for more details.
				  
%% You should have received a copy of the GNU General Public License
%% along with this program.  If not, see <http://www.gnu.org/licenses/>.

\documentclass[12pt]{report}
\usepackage{fontspec}
\usepackage{polyglossia}
\usepackage[top=2.5cm, bottom=2.5cm, left=3.5cm, right=2cm]{geometry}
\usepackage{array,tabulary,graphicx}
\usepackage{tikz}
\usepackage{minted}
\usepackage{amsmath}
\usepackage{wrapfig}
\usepackage[nottoc]{tocbibind}
\usepackage{frbib}
\usepackage{natbib}



\setdefaultlanguage{french}
\setotherlanguage{english}
\graphicspath{{images/}}
\setmainfont[Ligatures=TeX]{Latin Modern Roman}
\bibliographystyle{abbrvnat-fr}



\title{Influence des houppiers selon leur feuillage sur la lumière qui les traverse\\
\normalsize{Centre INRA Poitou-Charentes}}
\author{Ludovic Hondet}
\date{\today}

\begin{document}



\begin{titlepage}    
    \centering
    {\LARGE Ministère de l'Agriculture, de l'Alimentation,
    de la Pêche, de la Ruralité et de l'Aménagement du territoire \par}
    \vspace{6mm}
    {\huge \textbf{École Nationale Supérieure des Sciences Agronomiques de Bordeaux
    Aquitaine} \par}
    \vspace{3mm}
    {\normalsize 1 cours du Général de Gaulle - CS 40201 - 33175 Gradignan CEDEX \par}
    \vspace{15mm}
    {\LARGE Mémoire de stage de S8 dominante Foresterie de Bordeaux Sciences Agro \par}
    \vspace{15mm}
    {\LARGE \textbf{Influence des houppiers selon leur feuillage sur la lumière qui les traverse} \par}
    
    \vfill
    
    {\Large{Hondet Ludovic} \hfill \Large{- 2016 -} \hfill \includegraphics[width=30mm]{logo_BxScAgro.jpg}}
\end{titlepage}

\newpage
\thispagestyle{empty}
~
\newpage



\maketitle

\newpage
\thispagestyle{empty}
\setcounter{page}{0}
\noindent Le fichier source de ce document est disponibles sur GitHub à l'adresse:\\
\verb?https://github.com/LudovicHDT/FIRe_LaTeX?\\
\noindent Les codes sources des scripts écrits pour ce rapport sont disponible à l'adresse:\\
\verb?https://github.com/LudovicHDT/A7R?\\
Si vous désirez des documents ou figures dont il est fait mention dans le rapport mais qui
ne sont pas en annexes, vous pouvez vous adresser à l'auteur:\\
\verb?ludohon@gmail.com?

{\footnotesize
\begin{verbatim}

Copyright (C) 2016  Ludovic Hondet

This document is free document: you can redistribute it and/or modify
it under the terms of the GNU General Public License as published by
the Free Software Foundation, either version 3 of the License, or
(at your option) any later version.

This document is distributed in the hope that it will be useful,
but WITHOUT ANY WARRANTY; without even the implied warranty of
MERCHANTABILITY or FITNESS FOR A PARTICULAR PURPOSE.  See the
GNU General Public License for more details.

You should have received a copy of the GNU General Public License
along with this program.  If not, see <http://www.gnu.org/licenses/>.
\end{verbatim}
}
\newpage

\tableofcontents


\chapter*{Préface - remerciements}
\addcontentsline{toc}{chapter}{\protect\numberline{}Préface - remerciements}

Le centre INRA Poitou-Charentes fut officiellement créé en 1985, des équipes
de l'INRA étaient déjà présentes en 1960 dans l'ancienne région. Le centre est répartis sur plusieurs communes:
Lusignan, Rouillé, Le Magneraud et Saint-Laurent-de-la-Prée. Aujourd'hui il
existe deux sites avec plusieurs unités sur la commune de Lusignan. J'ai
travaillé sur les deux sites de Lusignan que sont le site du Chêne et le site
des Verrines mais mes missions se sont déroulées au sein de l'équipe
d'écophysiologie de l'URP3F du site du Chêne. Plusieurs unités peuvent coexister
sur un même site, chacune de ces unités peut être constituée de plusieurs
équipes. Le site du Chêne est constitué d'une unité de recherche
pluridisciplinaire prairies et plantes fourragères et d'une unité d'appui.
L'unité de recherche est divisée en 2 équipes: une équipe d'écophysiologie et
une équipe de génétique.\\

J'ai effectué un stage instructif à l'INRA Poitou-Charentes de Lusignan sur un
sujet complexe mais intéressant. Celui-ci répondait parfaitement à mes attentes,
c'est-à-dire travailler sur un projet pluridisciplinaire avant de commencer ma
dernière année en spécialisation foresterie. Je remercie Didier Combes, mon
maître de stage, pour avoir adapté le sujet qu'il me proposait de manière à me
permettre de travailler sur un sujet de recherche pluridisciplinaire. Je
remercie Ela Frak qui a répondu à mes nombreuses questions tout au long du stage.
Je remercie les techniciens pour leur disponibilité et leur aide précieuse sans
lesquelles aucun projet de recherche ne pourrait aboutir. Je remercie tous les chercheurs pour
les échanges constructifs que j'ai pu avoir avec eux. Je remercie tous mes
collègues stagiaires, thésards et employés avec lesquelles j'ai passé du bon
temps. Enfin je remercie toute l'équipe pour son sens de l'humour.

\chapter{Introduction}

L'agroforesterie est une alternative aux monocultures massivement développées
après-guerre. Celles-ci ont été développées dans l'optique de produire à
haut rendement, leur développement a nécessité la mise en place d'une
industrie chimique pour produire des intrants et des produits phytosanitaires.
Cependant, il est vite apparu que ces cultures à forts intrants posaient des
problèmes de durabilité et environnementaux: pollution des cours d'eau par
lixiviation, tassement des sols, sensibilité aux maladies et aux
ravageurs du fait de l'absence d'espèces protectrices, etc. À partir des années
1970, la mise en place d'une durabilité, ou soutenabilité, des systèmes est devenue
nécessaire pour assurer la pérennité de l'humanité, c'est dans ce contexte que des
centres de recherche ont commencé à étudier l'agroforesterie.
L'agroforesterie est l'association sur une même parcelle d'arbres et de cultures,
elle n'entraîne pas une diminution du rendement, ni par concurrence entre espèces,
ni par une contrainte de mécanisation~\citep{AF_ref50}. Bien au contraire, elle
augmente le rendement en biomasse lorsque l'on associe des espèces
complémentaires~\citep{AF_ref51}.

Ces associations par complémentarités s'inspirent des cultures traditionnelles
et des mélangent présents dans la nature dont nous avons des connaissances
empiriques, la recherche scientifique en agroforesterie se développe afin de
transformer ces connaissances en connaissances scientifiques. À ce jour, nous
savons qu'il existe de fortes interactions entre les arbres et les cultures dans
les systèmes agroforestiers~\citep{AF_ref48,AF_ref49,AF_ref47},
des interactions racinaires et des interactions pour la lumière. Dans un système
agroforestier établi, c'est-à-dire avec des arbres de plus 5 ans et une strate
herbacée, il n'y a pas de compétition pour les éléments minéraux du sol puisque
les racines des arbres plongent bien plus profondément que celles des espèces
herbacées, en revanche il existe bel et bien des interactions avec la lumière.
Une couverture arbustive induit des modifications quantitatives et qualitatives
de la lumière~\citep{MAR_ref36} et plus spécifiquement du PAR
(Photosynthetically Active Radiation), ces modifications sont quantitatives à
cause de la variation de l'irradiance et qualitatives à cause des variations du
spectre lumineux. Cette modification de lumière qui atteint le sous-bois suppose
qu'il y a une variation continue de la lumière tout au long de sa traversée des
houppiers, autrement dit qu'il existe un gradient de lumière le long du houppier
\citep{MAR_ref31,MAR_ref35}. Ainsi les arbres cherchent à optimiser ce
gradient et cette captation de la lumière pour leur propre croissance face à
leurs voisins~\citep{MAR_ref52} en faisant varier les caractères morphologiques de leur
houppier (hauteur, largeur, etc.). Cette optimisation de la captation de la lumière a été mise
en évidence sur des pêchers en leur imposant des tailles, houppiers en
parallélogrammes, pyramides, etc.~\citep{MAR_ref26}. En plus de ces recherches
sur le PAR, il a été montré que les caractères morphologiques des houppiers, et
l'architecture des plantes chlorophylliennes en général, sont influencés par des
variations du MAR (Morphogenetically Active Radiation)~\citep{MAR_ref35}, les réponses
morphogénétiques des plantes sont induites par des photorécepteurs qui capte le flux
photonique UVA-bleu entre 350 et 500nm et par des photorécepteurs très particuliers: les
phytochromes. Les phytochromes existent sous deux formes inconvertibles, une
forme Pr qui présente un pic d'absorption à 660nm et une forme Pfr qui présente
un pic à 730nm, la fréquence de réversibilité de ces 2 formes dépend directement
des rapports de ses longueurs d'onde que l'on appelle $\zeta$.

En parallèle, on a découvert que les herbacées sont directement influencés, en
terme de morphogenèse, par la lumière reçue~\citep{MAR_ref16}. En effet, $\zeta$,
le rapport rouge sur infrarouge (R:FR) a une grande influence sur la vitesse de
croissance du fétuque (\textit{Festuca arundinacea}) alors que la lumière bleue
à une influence sur la direction de croissance de la plante~\citep{BioVeg_ref43}.\\

L'effet global d'un houppier sur la lumière et les mécanismes d'absorption de
celle-ci par les plantes sont connus, à l'inverse, dans le cas de
l'agroforesterie, il semble que l'interaction entre les arbre et les herbes ont
été peu étudiés, plus précisément, peu d'études mettent en évidence qu'une
différence des caractères morphologiques des houppiers peut avoir une influence sur
la transmission de la lumière. Il est possible de formuler une
problématique: Existe-t-il une différence de transmission de la lumière par les
houppiers, compte tenu des différences morphologiques qui peuvent exister?\\

Pour répondre à cette question, les dimensions des caractères morphologiques des
houppiers ont été étudiées ainsi que les lumières qui frappent et traversent les
houppiers, autrement dit les
lumières incidentes et transmises. Plusieurs méthodes ont été mises au
point pour déterminer ces caractères. Une outil employé par des laboratoires de
l'INRA est le digitaliseur magnétique, l'outil se compose d'une sphère
intégrante, d'un boîtier d'acquisition, d'un ordinateur et d'un pointeur. La
sphère produit un champ magnétique autour de l'objet à virtualiser, l'opérateur
va pointer des points remarquables (extrémité d'une branche, emplacement d'une
fourche, extrémités du houppier, etc.) pour les intégrer virtuellement dans un
modèle sur l'ordinateur. Par ce procédé, l'opérateur obtient un modèle
virtualisé de son arbre et peut ainsi en extraire les caractères morphologiques.
Une autre méthode a été mise au point pour déterminer l'interception de la lumière par
le houppier. Cette méthode ne détermine pas les caractères
morphologiques mais elle permet de mettre en relation la lumière interceptée avec
le PAR et le MAR transmis. Elle utilise des photos de l'ombre projetée de la canopée au
sol \citep{MAR_ref32}. Pour cela, les opérateurs prennent des photos de l'ombre
projetée au sol le long des rangées d'arbre. La dernière méthodes permet
d'obtenir les caractères de la morphologie des houppiers et de connaître, par
exemple, la surface foliaire théorique disponible pour intercepter la lumière.
Elle permet de déterminer la hauteur de l'arbre, la hauteur du houppier, la
largeur du houppier selon les axes Est-Ouest et Nord-Sud, le volume apparent du
houppier, la surface foliaire totale, la densité de la surface foliaire (LAD)
moyenne. Le principal outil de cette méthode est le logiciel Tree Analyzer
\citep{MAR_ref24,MAR_ref25}. Il détermine les caractères morphologiques à
partir de photographies des arbres.

Les plantes exploitent le PAR pour faire la photosynthèse, le MAR a une forte
influence sur la morphogenèse des plantes. C'est donc ces 2 rayonnements qui
ont été étudié pour répondre à la problématique. Le matériel qui a été utilisé,
MARscope, a été développé par le laboratoire INRA Poitou-Charentes. MARscope
capte l'irradiance, il calcul les flux photoniques
du PAR (400 à 700nm), de l'UVA-bleu entre 350 et 500nm, du bleu élargi entre,
du bleu strict entre, du vert entre, du rouge entre, du proche infrarouge entre,
du rouge strict entre et de l'infrarouge strict entre. Ces flux photoniques
lui permettent de calculer des valeurs d'équilibres chimiques spécifiques aux réactions
de la photosynthèse, $\zeta$ et le rapport rouge sur
infrarouge proche. Parmi toutes ces données mises à disposition, le flux
photoniques global (PAR), le flux photonique UVA-bleu et $\zeta$ ont été
exploités pour répondre à notre problématique.

Nous avons répondu à notre problématique en deux étapes. Nous avons commencé par
établir le protocole d'acquisition des images pour Tree Analyzer. Le protocole
présenté dans l'article \citet{MAR_ref25} a été testé puis à été amélioré pour
raccourcir le temps nécessaire à l'acquisition des images. Ensuite nous avons
établi le protocole de mesure pour l'acquisition du PAR et du MAR des lumière
incidentes et transmises. Le même ordre a été suivi pour présenter et analyser
les résultats. Nous avons en premier lieu analysé les résultats des relations
allométriques et de Tree Analyzer, puis nous avons présenté les analyses issues
des données de MARscope. Nous avons analysé les résultats des deux expériences
en vis à vis pour pouvoir établir des graphiques qui nous ont permis de répondre
à notre problématique.


\chapter{Matériel et méthodes}

\section{Parcelle et matériel biologique}


L'étude a été conduite sur une parcelle agroforestière entre le 6 juillet et le
29 août 2016. Elle est constituée de chênes verts (\textit{Quercus ilex}), de frênes communs
(\textit{Fraxinus excelsior}), de mûriers blancs (\textit{Morus alba}),
d'ormes Lutèce (\textit{Ulmus minor}) et d'aulnes corses
(\textit{Alnus cordata}). Elle est destinée au pâturage par des vaches. Seules 3
espèces ont été étudiées parmi les cinq
présentes, le \textit{F. excelsior} pour ses feuilles composées,
l'\textit{A. cordata} pour ses feuilles simples et le \textit{M. alba} pour ses
feuilles simples mais pourtant si particulières géométriquement. Ces 3 espèces
et l'\textit{U. minor} seront conduites en têtard afin d'être disponibles en
tant que pâturage pour le bétail. Le \textit{Q. ilex} sera également conduit
en têtard mais à 3m, au lieu de 2m pour les autres arbres, cet arbre sempervirent
fournira un abri naturel au bétail été comme hiver. La futur conduite en têtard
n'aura aucun effet sur les résultats de l'expérience qui a été menée.

\begin{figure}
  \centering
  \def\svgwidth{\linewidth}
  \input{plan_M2.pdf_tex}
  \caption{Schéma de la parcelle agroforestière étudiée sur laquelle sont
  menées les expérimentations\label{fig:parcelle}}
\end{figure}

Sur la parcelle (voir figure~\ref{fig:parcelle}) l'inter-rang est de 20m et chaque arbre est
espacé de 3m au sein de chaque rang.



\section{Estimation des caractères morphologiques}


\subsection{Estimation des caractères morphologiques d'un pied de luzerne en tant qu'expérience témoin}

Nous nous sommes inspirés du protocole des prises de vues présenté par les
concepteurs de Tree Analyzer \citep{MAR_ref24, MAR_ref25} pour tester notre
propre protocole. Celui-ci a été mené sur un pied de luzerne. Pour 
obtenir les caractères morphologiques des houppiers qui ont été utiles,
c'est-à-dire hauteur, diamètre Est-Ouest et Nord-Sud, volume, surface foliaire
totale, densité de surface foliaire (LAD) et l'indice foliaire (LAI), le
logiciel, Tree Analyzer, a besoin de 8 prises de vue. Ces clichés sont
pris tout autour du sujet par tranche de 45°. Dans le
cas de la luzerne (figure~\ref{fig:luzerne}) j'ai fait tourner le pot au
lieu de déplacer le montage. Il a fallu relever pour chaque photographie la
hauteur,l'angle de l'appareil photo et la distance de l'appareil au sujet. Des
paramètres supplémentaires, tels que l'inclinaison des feuilles et la surface
foliaire moyenne, directement mesurés sur le plan de luzerne, ont été
nécessaires en entrées pour calculer les caractéristiques morphologiques (voir appendice~\ref{App:appxA}). 
Un traitement d'image a été nécessaire pour 
utiliser les photographies dans le logiciel, celui-ci ne savait gérer que des
images en noir et blanc au format .bmp. Pour traiter les images j'ai 
utilisé ImageJ\cite{ImageJ_ref39}, c'est pour cela qu'un drap rouge (figure~\ref{fig:luzerne})
est étendu en guise de fond: pour faciliter le traitement d'image par une
différenciation du sujet avec le fond.

\begin{figure}[h]
  \centering
  \includegraphics[width=140mm]{bozo.jpg}
  \caption{Photographie du montage des prises de vues de la luzerne\label{fig:luzerne}}
\end{figure}

L'autre intérêt du pied de luzerne était qu'il a été facile, de part sa taille, de confronter deux
méthodes de mesure pour déterminer la surface foliaire totale. La deuxième
méthode se base sur un comptage et une mesure des surfaces des feuilles à l'aide
de scans. Le programme Mesure de surface de feuilles v2.py\citep{Python_ref40}
a permis de mesurer la surface foliaire au pixel près des scans.


\subsection{Établissement des relations allométriques}

Des relations allométriques ont été créés pour mesurer la surface foliaire par
une autre méthode que par Tree Analyzer. Ce logiciel est un outil prototypique,
sa fiabilité n'est donc pas avérée et il a donc fallu s'assurer que ses
résultats soient suffisamment exactes pour être analysés. J'ai procéder à
un échantillonnage de feuilles et de branches qui a permis d'établir des relations
allométriques: relation entre surface d'une feuille et longueur de la nervure
principale, relation entre longueur d'un axe (branche) et nombre de feuilles sur
cet axe. Ces relations avaient plusieurs fonctions: elles
ont permis une mesure de la surface foliaire de chaque arbre plus rapidement que
si on mesurait chacune des feuilles et elles nous ont permis de valider des
hypothèses sur la répartition des feuilles le long des branches.

Le premier type de relations allométriques a permis de mettre en relation la surface
des feuilles avec leur longueur (ou du pétiole qui porte les
folioles chez \textit{F. excelsior}). Pour cela un échantillon de 9 feuilles a
été prélevé pour \textit{F. excelsior}, 35 pour \textit{M. alba} et 21 pour \textit{A. cordata}.
Toutes ces feuilles ont été scannées puis analysées à l'aide d'un script développé
pour l'expérience pour déterminer la surface de chacune. Dans un deuxième temps
la longueur du limbe de chaque feuille a été mesurée. Après avoir enregistrées
toutes les valeurs j'ai tracé la première relation allométrique.

La seconde relation allométrique a été construite de la même manière. Celle-ci relie
la longueur d'un axe avec le nombre de feuilles sur cet axe. Des échantillons d'une
vingtaine de branches ont été prélevés pour chacune des 3 espèces. Grâce aux points
que nous avons rentré nous j'ai tracé la seconde relation allométrique.

\subsection{Détermination des caractères morphologiques}

Après avoir établi les relations allométriques qui ont permis de vérifier la
fiabilité des résultats de Tree Analyzer sur le pied de luzerne, nous avons adapté le protocole
d'acquisition à des arbres. Ce protocole était le même que pour le pied de luzerne à la différence que le ciel
contrastait suffisamment avec les houppiers des arbres lorsque je seuillais, il
n'était donc pas nécessaire de couvrir tout le houppier mais uniquement la partie
basse pour qu'elle se détache du fond vert et de la prairie en dessous.
De plus d'autres scripts ont été utilisés pour automatiser une partie
du traitement. L'utilisation
d'un drap rouge pour la moitié basse du houppier, au lieu d'étendre le drap sur
toute la hauteur de la plante (voir figure~\ref{fig:luzerne}), a permis d'alléger
la structure qui supporte le drap rouge, donc de gagner
du temps lors des prises de vues. Par cette méthode, le post-traitement était un
peu plus long mais on en gagnait avec un script d'automatisation, globalement, la
combinaison des méthodes a raccourci le temps nécessaire à l'acquisition et au
traitement des images.

\begin{figure}
  \centering
  \def\svgwidth{\linewidth}
  \input{montage.pdf_tex}
  \caption{Photographie du montage de prises de vues pour un \textit{F. excelsior}\label{fig:montage_finale}}
\end{figure}

\section{Étude des lumières incidentes et transmises}

\subsection{Choix du matériel et contraintes liées}


La machine utilisée pour l'analyse des lumières incidentes et transmises, MARscope,
est un prototype. Elle permettait, grâce à une interface et son logiciel
de traitement, de quantifier et de qualifier la lumière qui frappait les capteurs
à partir de l'irradiance reçue. Grâce à cette mesure, la machine était capable de
calculer les flux photoniques qui ont permis de répondre à notre problématique:
$\zeta$ (calculé à partir de 2 flux photoniques), le flux photonique
global et le flux photonique UVA-bleu. Le matériel informatique était dans un caisson sur roues
(voir figure~\ref{fig:MARscope}).

\begin{figure}[h]
  \centering
  \includegraphics[width=140mm]{MARscope.JPG}
  \caption{Photographie du MARscope lors de l'étalonnage\label{fig:MARscope}}
\end{figure}

7 capteurs cosinus, 6 pour la lumière transmise et un pour la lumière incidente,
étaient reliés au MARscope via des fibres optiques. Ils permettaient de capter la
lumière puis la pondéraient par l'angle d'incidence
des rayons. La valeur d'irradiance restait inchangée
lorsque le rayon frappait orthogonalement, à l'inverse, la valeur pondérée du rayon
diminuait au fur et à mesure que l'angle par rapport à la verticale augmentait. La
fonction qui liait la pondération et l'angle d'incidence était une fonction
cosinus. Cette intégration du rayonnement
impliquait que les capteurs soient tous d'aplomb sinon chacun d'eux
n'intégreraient pas le même signal et les mesures seraient faussées.


\subsection{Placement des capteurs et traitement de l'information}


L'objectif principal de l'expérimentation était de quantifier et de qualifier les
modifications que la lumière subissait lorsqu'elle traverse un houppier, cela
nécessitait donc de mettre les capteurs à l'ombre. Par définition les capteurs cosinus
intègrent la valeur réelle des rayons orthogonaux à leur surface et pondèrent
les rayons rasants, donc les capteurs ont dû être placés de
telle sorte qu'ils soient à l'ombre lorsque le soleil était au plus haut dans le
ciel, ceci nous a amené à placer les capteurs à l'ombre des arbres à 12h UTC,
autrement dit à 14h heure locale. Comme les rangées étaient orientées Est-Ouest,
il en a découlé que les capteurs devaient être placés au Nord de chaque arbre.


\begin{wrapfigure}{l}{0.5\textwidth}
  \vspace{-20pt}
  \begin{center}
    \includegraphics[width=0.45\textwidth]{capteur_Morus.png}
  \end{center}
  \caption{Photographie d'un capteur sous un \textit{M. alba}\label{fig:capteur}}
\end{wrapfigure}


De plus, comme c'était l'ombre qui allait nous permettre de répondre à
notre problématique, les capteurs ont
été placés de telle sorte que l'ombre du milieu du houppier frappe directement
le capteur. Pour calculer leur placement nous avons utilisé le théorème de
Pythagore:
\[
    a = \frac{o}{\tan(\alpha)}
\]
Connaissant l'angle d'élévation du soleil à 14h ($\alpha$) et la hauteur du
point le plus dense du houppier (\textit{o}), il a été possible d'en déduire à quelle
distance (\textit{a}) il a fallu mettre chacun des capteurs (voir figure~\ref{fig:capteur}).

L'analyse des données a commencé immédiatement après avoir démarré
l'acquisition des lumières incidentes et résultantes.
Des scripts R~\citep{R_ref38} (voir appendice~\ref{App:appxB}) ont été écrits pour
analyser les données utiles parmi la masse mis à disposition par le
MARscope, 3 flux photoniques ont été particulièrement intéressants: $\zeta$, le flux
photonique global et le flux photonique UVA-bleu. $\zeta$, la variable qui
permet aux plantes de déterminer si elles sont à l'ombre, est le rapport de
la lumière rouge sur l'infrarouge proche (R:FR), elle
induit une accélération de la croissance en hauteur des plantes, autrement dit
une élongation de l'entre-n\oe{}ud. Ce rapport était émis par le rayonnement
solaire mais aussi par réflexion de la lumière sur les plantes environnantes, il
indiquait donc à la plante si elle était en compétition. La deuxième variable qui
qui a une influence sur la morphogenèse, est le flux photonique UVA-bleu,
compris entre 350 et 500nm. La lumière de cette longueur d'onde a 2 effets sur
la plante, elle est absorbé par la chlorophylle dans le processus de la
photosynthèse et indique dans quelle
direction doit pousser la plante pour se rapprocher de la lumière. Ces 2
variables sont donc des indicateurs environnementaux qui induisent la
morphologie de la plante, donc ils font partie du MAR (Morphogenetically Active
Radiation). La dernière variable intéressante pour notre étude était le flux
photonique global, compris entre 400 et 700nm, elle représente l'ensemble du rayonnement
utilisable par la plante pour la photosynthèse, elle correspond donc au PAR
(Photosynthetically Active Radiation).



\chapter{Résultats et analyse}


\section{Relations allométriques}

\subsection{Relations allométriques entre surface et longueur des feuilles}

Les résultats des relations allométriques ont pu être interprétés avant
le traitement complet des données de MARscope. La figure~\ref{fig:feuilles}
représente les relations allométriques entre la surface et la longueur des
feuilles. La régression qui a été appliquée était une
fonction de puissance (définition d'une relation d'allométrie selon Julian
Huxley, 1936). Chaque espèce a sa propre relation allométrique avec des
coefficients de détermination ne descendant pas en dessous de 0.93, \textit{
F. excelsior} étant l'espèce qui a le plus petit coefficient. Tout d'abord, il
est important de remarquer que des folioles de \textit{F. excelsior} ont été
échantillonnés et non des feuilles, nous avons fait le même échantillonnage pour
les feuilles mais leur courbe de régression écrasait les autres courbes.

Visuellement, lorsqu'une courbe est au dessus d'une autre, cela signifie que la
surface de la feuille sera plus élevé pour une même longueur de limbe. Dans le
cas des feuilles de frêne, qui n'ont pas été représentées, cela signifie
qu'elles sont capables d'optimiser leur surface pour une même longueur de
pétiole.

\subsection{Relations allométriques entre longueur des axes et nombre de
feuilles par axe}

La figure~\ref{fig:branches} représente les relations allométriques 
entre la longueur des axes et le nombre de feuilles par axe, j'ai appliqué la
définition de relation allométrique pour établir les
équations. De manière générale les coefficients de détermination sont un peu
moins bon puisqu'on est entre 0.93 pour \textit{F. excelsior} et 0.72 pour
\textit{A. cordata}, ces plus faibles valeurs s'expliquent par notre méthode
d'échantillonnage. En effet, nous avons relevé la présence réelle de feuilles et
non la présence théorique, des feuilles ont pu tomber car elles ont été abîmées
par exemple.

De manière générale, lorsqu'une courbe est au dessus d'une autre, cela signifie
qu'il y aura plus de feuilles sur l'axe pour une
espèce comparé aux courbes des autres espèces qui sont en dessous. Ces
différences du nombre de feuilles peuvent s'expliquer par la longueur de l'entre
n\oe{}ud ou l'arrangement des feuilles sur l'axe.

\begin{figure}
    \centering
    \includegraphics[width=170mm]{rel_feuilles.png}
    \caption{Relations allométriques longueur/surface de feuilles\label{fig:feuilles}}
\end{figure}
\begin{figure}
    \centering
    \includegraphics[width=170mm]{branches.png}
    \caption{Relations allométriques longueur des axes / nombre de feuilles par axe\label{fig:branches}}
\end{figure}

\section{Influence de la morphologie du houppier sur la transmission de la lumière}

\subsection{Des ombrages spécifiques à chaque espèce}

Les scripts qui ont été écrits ont permis de tracer 2 types de graphiques: des
nuages de points et des boîtes à moustaches. Le premier groupe de graphiques
(voir figure~\ref{fig:flux}) représente la variation de nos 3 flux, le flux
photonique global, le flux d'UVA-bleu et $\zeta$ au cours du temps pour les capteurs
1, 3 et 4, un frêne, un aulne et un mûrier. Nous avons utilisé $\zeta$ pour
déterminer à quel moment les capteurs sont à l'ombre, il a été décidé de prendre
0.85 comme seuil d'ombrage. Les boîtes à moustaches ont été obtenues en ne
conservant que les valeurs inférieurs à 0.85.

Sur les 3 graphiques de la figure~\ref{fig:flux}, on voit que les capteurs sont
passés à l'ombre vers 11h30. À l'inverse, l'heure de sortie de la zone d'ombre
semble dépendre des arbres, le frêne se caractérisait par un houppier plus
étroit, un volume foliaire plus faible et un LAD plus faible que l'aulne et le
mûrier, il était donc logique que le capteur 1 soit moins à l'ombre que les
capteurs 3 et 4. On peut remarquer sur nos 3 graphiques que les valeurs des 3
flux sont plus faibles avant le passage de l'ombre, cela s'explique par des
passages nuageux. L'idéal aurait été d'avoir un ciel bleu du matin au soir mais
malheureusement le cas ne c'est jamais présenté pendant toute notre campagne de
mesures.

Les boîtes à moustaches (voir figure~\ref{fig:boxplots}) ont pour représentation de la
moyenne des cercles rouges, elles ont été calculées avec les points qui sont
sous le seuil de 0.85, les extrémités représentent les points extrêmes. $\zeta$
minimum représente l'intensité d'ombrage maximal qu'un arbre peut induire, alors
que la moyenne est un indicateur de la proportion de trouées dans le feuillage,
autrement dit c'est un indicateur non quantifié de la densité du feuillage.

\begin{figure}[h]
  \centering
  \begin{tiny}
    % Created by tikzDevice version 0.10.1 on 2016-11-13 11:36:45
% !TEX encoding = UTF-8 Unicode
\begin{tikzpicture}[x=1pt,y=1pt, scale=0.70]
\definecolor{fillColor}{RGB}{255,255,255}
\path[use as bounding box,fill=fillColor,fill opacity=0.00] (0,0) rectangle (722.70,289.08);
\begin{scope}
\path[clip] ( 47.52, 47.52) rectangle (238.26,241.56);
\definecolor{drawColor}{RGB}{255,0,0}
\definecolor{fillColor}{RGB}{255,0,0}

\path[draw=drawColor,line width= 0.4pt,line join=round,line cap=round,fill=fillColor] ( 54.58, 97.11) circle (  1.49);
\definecolor{drawColor}{RGB}{0,0,0}
\definecolor{fillColor}{RGB}{0,0,0}

\path[draw=drawColor,line width= 0.4pt,line join=round,line cap=round,fill=fillColor] ( 54.60,111.68) circle (  1.49);
\definecolor{drawColor}{RGB}{255,0,0}
\definecolor{fillColor}{RGB}{255,0,0}

\path[draw=drawColor,line width= 0.4pt,line join=round,line cap=round,fill=fillColor] ( 55.04, 99.87) circle (  1.49);
\definecolor{drawColor}{RGB}{0,0,0}
\definecolor{fillColor}{RGB}{0,0,0}

\path[draw=drawColor,line width= 0.4pt,line join=round,line cap=round,fill=fillColor] ( 55.06,116.96) circle (  1.49);
\definecolor{drawColor}{RGB}{255,0,0}
\definecolor{fillColor}{RGB}{255,0,0}

\path[draw=drawColor,line width= 0.4pt,line join=round,line cap=round,fill=fillColor] ( 55.48,101.14) circle (  1.49);
\definecolor{drawColor}{RGB}{0,0,0}
\definecolor{fillColor}{RGB}{0,0,0}

\path[draw=drawColor,line width= 0.4pt,line join=round,line cap=round,fill=fillColor] ( 55.50,130.55) circle (  1.49);
\definecolor{drawColor}{RGB}{255,0,0}
\definecolor{fillColor}{RGB}{255,0,0}

\path[draw=drawColor,line width= 0.4pt,line join=round,line cap=round,fill=fillColor] ( 55.93,102.52) circle (  1.49);
\definecolor{drawColor}{RGB}{0,0,0}
\definecolor{fillColor}{RGB}{0,0,0}

\path[draw=drawColor,line width= 0.4pt,line join=round,line cap=round,fill=fillColor] ( 55.94,121.43) circle (  1.49);
\definecolor{drawColor}{RGB}{255,0,0}
\definecolor{fillColor}{RGB}{255,0,0}

\path[draw=drawColor,line width= 0.4pt,line join=round,line cap=round,fill=fillColor] ( 56.39,102.85) circle (  1.49);
\definecolor{drawColor}{RGB}{0,0,0}
\definecolor{fillColor}{RGB}{0,0,0}

\path[draw=drawColor,line width= 0.4pt,line join=round,line cap=round,fill=fillColor] ( 56.40,116.15) circle (  1.49);
\definecolor{drawColor}{RGB}{255,0,0}
\definecolor{fillColor}{RGB}{255,0,0}

\path[draw=drawColor,line width= 0.4pt,line join=round,line cap=round,fill=fillColor] ( 56.84,101.13) circle (  1.49);
\definecolor{drawColor}{RGB}{0,0,0}
\definecolor{fillColor}{RGB}{0,0,0}

\path[draw=drawColor,line width= 0.4pt,line join=round,line cap=round,fill=fillColor] ( 56.86,110.70) circle (  1.49);
\definecolor{drawColor}{RGB}{255,0,0}
\definecolor{fillColor}{RGB}{255,0,0}

\path[draw=drawColor,line width= 0.4pt,line join=round,line cap=round,fill=fillColor] ( 57.29, 98.80) circle (  1.49);
\definecolor{drawColor}{RGB}{0,0,0}
\definecolor{fillColor}{RGB}{0,0,0}

\path[draw=drawColor,line width= 0.4pt,line join=round,line cap=round,fill=fillColor] ( 57.30,107.60) circle (  1.49);
\definecolor{drawColor}{RGB}{255,0,0}
\definecolor{fillColor}{RGB}{255,0,0}

\path[draw=drawColor,line width= 0.4pt,line join=round,line cap=round,fill=fillColor] ( 57.76, 97.48) circle (  1.49);
\definecolor{drawColor}{RGB}{0,0,0}
\definecolor{fillColor}{RGB}{0,0,0}

\path[draw=drawColor,line width= 0.4pt,line join=round,line cap=round,fill=fillColor] ( 57.78,106.36) circle (  1.49);
\definecolor{drawColor}{RGB}{255,0,0}
\definecolor{fillColor}{RGB}{255,0,0}

\path[draw=drawColor,line width= 0.4pt,line join=round,line cap=round,fill=fillColor] ( 58.22, 96.93) circle (  1.49);
\definecolor{drawColor}{RGB}{0,0,0}
\definecolor{fillColor}{RGB}{0,0,0}

\path[draw=drawColor,line width= 0.4pt,line join=round,line cap=round,fill=fillColor] ( 58.23,106.91) circle (  1.49);
\definecolor{drawColor}{RGB}{255,0,0}
\definecolor{fillColor}{RGB}{255,0,0}

\path[draw=drawColor,line width= 0.4pt,line join=round,line cap=round,fill=fillColor] ( 58.66, 97.21) circle (  1.49);
\definecolor{drawColor}{RGB}{0,0,0}
\definecolor{fillColor}{RGB}{0,0,0}

\path[draw=drawColor,line width= 0.4pt,line join=round,line cap=round,fill=fillColor] ( 58.69,108.43) circle (  1.49);
\definecolor{drawColor}{RGB}{255,0,0}
\definecolor{fillColor}{RGB}{255,0,0}

\path[draw=drawColor,line width= 0.4pt,line join=round,line cap=round,fill=fillColor] ( 59.12, 98.11) circle (  1.49);
\definecolor{drawColor}{RGB}{0,0,0}
\definecolor{fillColor}{RGB}{0,0,0}

\path[draw=drawColor,line width= 0.4pt,line join=round,line cap=round,fill=fillColor] ( 59.14,110.49) circle (  1.49);
\definecolor{drawColor}{RGB}{255,0,0}
\definecolor{fillColor}{RGB}{255,0,0}

\path[draw=drawColor,line width= 0.4pt,line join=round,line cap=round,fill=fillColor] ( 59.56, 98.98) circle (  1.49);
\definecolor{drawColor}{RGB}{0,0,0}
\definecolor{fillColor}{RGB}{0,0,0}

\path[draw=drawColor,line width= 0.4pt,line join=round,line cap=round,fill=fillColor] ( 59.58,112.44) circle (  1.49);
\definecolor{drawColor}{RGB}{255,0,0}
\definecolor{fillColor}{RGB}{255,0,0}

\path[draw=drawColor,line width= 0.4pt,line join=round,line cap=round,fill=fillColor] ( 60.04,100.25) circle (  1.49);
\definecolor{drawColor}{RGB}{0,0,0}
\definecolor{fillColor}{RGB}{0,0,0}

\path[draw=drawColor,line width= 0.4pt,line join=round,line cap=round,fill=fillColor] ( 60.05,112.47) circle (  1.49);
\definecolor{drawColor}{RGB}{255,0,0}
\definecolor{fillColor}{RGB}{255,0,0}

\path[draw=drawColor,line width= 0.4pt,line join=round,line cap=round,fill=fillColor] ( 60.49,100.90) circle (  1.49);
\definecolor{drawColor}{RGB}{0,0,0}
\definecolor{fillColor}{RGB}{0,0,0}

\path[draw=drawColor,line width= 0.4pt,line join=round,line cap=round,fill=fillColor] ( 60.51,112.52) circle (  1.49);
\definecolor{drawColor}{RGB}{255,0,0}
\definecolor{fillColor}{RGB}{255,0,0}

\path[draw=drawColor,line width= 0.4pt,line join=round,line cap=round,fill=fillColor] ( 60.95,100.98) circle (  1.49);
\definecolor{drawColor}{RGB}{0,0,0}
\definecolor{fillColor}{RGB}{0,0,0}

\path[draw=drawColor,line width= 0.4pt,line join=round,line cap=round,fill=fillColor] ( 60.97,114.07) circle (  1.49);
\definecolor{drawColor}{RGB}{255,0,0}
\definecolor{fillColor}{RGB}{255,0,0}

\path[draw=drawColor,line width= 0.4pt,line join=round,line cap=round,fill=fillColor] ( 61.41,101.02) circle (  1.49);
\definecolor{drawColor}{RGB}{0,0,0}
\definecolor{fillColor}{RGB}{0,0,0}

\path[draw=drawColor,line width= 0.4pt,line join=round,line cap=round,fill=fillColor] ( 61.43,109.23) circle (  1.49);
\definecolor{drawColor}{RGB}{255,0,0}
\definecolor{fillColor}{RGB}{255,0,0}

\path[draw=drawColor,line width= 0.4pt,line join=round,line cap=round,fill=fillColor] ( 61.92,101.39) circle (  1.49);
\definecolor{drawColor}{RGB}{0,0,0}
\definecolor{fillColor}{RGB}{0,0,0}

\path[draw=drawColor,line width= 0.4pt,line join=round,line cap=round,fill=fillColor] ( 61.93,117.78) circle (  1.49);
\definecolor{drawColor}{RGB}{255,0,0}
\definecolor{fillColor}{RGB}{255,0,0}

\path[draw=drawColor,line width= 0.4pt,line join=round,line cap=round,fill=fillColor] ( 62.38,102.13) circle (  1.49);
\definecolor{drawColor}{RGB}{0,0,0}
\definecolor{fillColor}{RGB}{0,0,0}

\path[draw=drawColor,line width= 0.4pt,line join=round,line cap=round,fill=fillColor] ( 62.39,120.10) circle (  1.49);
\definecolor{drawColor}{RGB}{255,0,0}
\definecolor{fillColor}{RGB}{255,0,0}

\path[draw=drawColor,line width= 0.4pt,line join=round,line cap=round,fill=fillColor] ( 62.83,102.84) circle (  1.49);
\definecolor{drawColor}{RGB}{0,0,0}
\definecolor{fillColor}{RGB}{0,0,0}

\path[draw=drawColor,line width= 0.4pt,line join=round,line cap=round,fill=fillColor] ( 62.85,123.96) circle (  1.49);
\definecolor{drawColor}{RGB}{255,0,0}
\definecolor{fillColor}{RGB}{255,0,0}

\path[draw=drawColor,line width= 0.4pt,line join=round,line cap=round,fill=fillColor] ( 63.28,103.34) circle (  1.49);
\definecolor{drawColor}{RGB}{0,0,0}
\definecolor{fillColor}{RGB}{0,0,0}

\path[draw=drawColor,line width= 0.4pt,line join=round,line cap=round,fill=fillColor] ( 63.29,120.85) circle (  1.49);
\definecolor{drawColor}{RGB}{255,0,0}
\definecolor{fillColor}{RGB}{255,0,0}

\path[draw=drawColor,line width= 0.4pt,line join=round,line cap=round,fill=fillColor] ( 63.74,103.77) circle (  1.49);
\definecolor{drawColor}{RGB}{0,0,0}
\definecolor{fillColor}{RGB}{0,0,0}

\path[draw=drawColor,line width= 0.4pt,line join=round,line cap=round,fill=fillColor] ( 63.75,116.37) circle (  1.49);
\definecolor{drawColor}{RGB}{255,0,0}
\definecolor{fillColor}{RGB}{255,0,0}

\path[draw=drawColor,line width= 0.4pt,line join=round,line cap=round,fill=fillColor] ( 64.19,102.42) circle (  1.49);
\definecolor{drawColor}{RGB}{0,0,0}
\definecolor{fillColor}{RGB}{0,0,0}

\path[draw=drawColor,line width= 0.4pt,line join=round,line cap=round,fill=fillColor] ( 64.21,115.40) circle (  1.49);
\definecolor{drawColor}{RGB}{255,0,0}
\definecolor{fillColor}{RGB}{255,0,0}

\path[draw=drawColor,line width= 0.4pt,line join=round,line cap=round,fill=fillColor] ( 64.65,101.73) circle (  1.49);
\definecolor{drawColor}{RGB}{0,0,0}
\definecolor{fillColor}{RGB}{0,0,0}

\path[draw=drawColor,line width= 0.4pt,line join=round,line cap=round,fill=fillColor] ( 64.67,116.95) circle (  1.49);
\definecolor{drawColor}{RGB}{255,0,0}
\definecolor{fillColor}{RGB}{255,0,0}

\path[draw=drawColor,line width= 0.4pt,line join=round,line cap=round,fill=fillColor] ( 65.13,101.79) circle (  1.49);
\definecolor{drawColor}{RGB}{0,0,0}
\definecolor{fillColor}{RGB}{0,0,0}

\path[draw=drawColor,line width= 0.4pt,line join=round,line cap=round,fill=fillColor] ( 65.14,119.74) circle (  1.49);
\definecolor{drawColor}{RGB}{255,0,0}
\definecolor{fillColor}{RGB}{255,0,0}

\path[draw=drawColor,line width= 0.4pt,line join=round,line cap=round,fill=fillColor] ( 65.58,102.26) circle (  1.49);
\definecolor{drawColor}{RGB}{0,0,0}
\definecolor{fillColor}{RGB}{0,0,0}

\path[draw=drawColor,line width= 0.4pt,line join=round,line cap=round,fill=fillColor] ( 65.60,119.53) circle (  1.49);
\definecolor{drawColor}{RGB}{255,0,0}
\definecolor{fillColor}{RGB}{255,0,0}

\path[draw=drawColor,line width= 0.4pt,line join=round,line cap=round,fill=fillColor] ( 66.03,102.68) circle (  1.49);
\definecolor{drawColor}{RGB}{0,0,0}
\definecolor{fillColor}{RGB}{0,0,0}

\path[draw=drawColor,line width= 0.4pt,line join=round,line cap=round,fill=fillColor] ( 66.04,119.09) circle (  1.49);
\definecolor{drawColor}{RGB}{255,0,0}
\definecolor{fillColor}{RGB}{255,0,0}

\path[draw=drawColor,line width= 0.4pt,line join=round,line cap=round,fill=fillColor] ( 66.49,101.57) circle (  1.49);
\definecolor{drawColor}{RGB}{0,0,0}
\definecolor{fillColor}{RGB}{0,0,0}

\path[draw=drawColor,line width= 0.4pt,line join=round,line cap=round,fill=fillColor] ( 66.52,117.95) circle (  1.49);
\definecolor{drawColor}{RGB}{255,0,0}
\definecolor{fillColor}{RGB}{255,0,0}

\path[draw=drawColor,line width= 0.4pt,line join=round,line cap=round,fill=fillColor] ( 66.94,101.33) circle (  1.49);
\definecolor{drawColor}{RGB}{0,0,0}
\definecolor{fillColor}{RGB}{0,0,0}

\path[draw=drawColor,line width= 0.4pt,line join=round,line cap=round,fill=fillColor] ( 66.96,114.55) circle (  1.49);
\definecolor{drawColor}{RGB}{255,0,0}
\definecolor{fillColor}{RGB}{255,0,0}

\path[draw=drawColor,line width= 0.4pt,line join=round,line cap=round,fill=fillColor] ( 67.40,101.86) circle (  1.49);
\definecolor{drawColor}{RGB}{0,0,0}
\definecolor{fillColor}{RGB}{0,0,0}

\path[draw=drawColor,line width= 0.4pt,line join=round,line cap=round,fill=fillColor] ( 67.42,112.82) circle (  1.49);
\definecolor{drawColor}{RGB}{255,0,0}
\definecolor{fillColor}{RGB}{255,0,0}

\path[draw=drawColor,line width= 0.4pt,line join=round,line cap=round,fill=fillColor] ( 67.89,101.09) circle (  1.49);
\definecolor{drawColor}{RGB}{0,0,0}
\definecolor{fillColor}{RGB}{0,0,0}

\path[draw=drawColor,line width= 0.4pt,line join=round,line cap=round,fill=fillColor] ( 67.91,111.65) circle (  1.49);
\definecolor{drawColor}{RGB}{255,0,0}
\definecolor{fillColor}{RGB}{255,0,0}

\path[draw=drawColor,line width= 0.4pt,line join=round,line cap=round,fill=fillColor] ( 68.35,100.98) circle (  1.49);
\definecolor{drawColor}{RGB}{0,0,0}
\definecolor{fillColor}{RGB}{0,0,0}

\path[draw=drawColor,line width= 0.4pt,line join=round,line cap=round,fill=fillColor] ( 68.37,111.08) circle (  1.49);
\definecolor{drawColor}{RGB}{255,0,0}
\definecolor{fillColor}{RGB}{255,0,0}

\path[draw=drawColor,line width= 0.4pt,line join=round,line cap=round,fill=fillColor] ( 68.81,100.62) circle (  1.49);
\definecolor{drawColor}{RGB}{0,0,0}
\definecolor{fillColor}{RGB}{0,0,0}

\path[draw=drawColor,line width= 0.4pt,line join=round,line cap=round,fill=fillColor] ( 68.83,111.18) circle (  1.49);
\definecolor{drawColor}{RGB}{255,0,0}
\definecolor{fillColor}{RGB}{255,0,0}

\path[draw=drawColor,line width= 0.4pt,line join=round,line cap=round,fill=fillColor] ( 69.27,100.57) circle (  1.49);
\definecolor{drawColor}{RGB}{0,0,0}
\definecolor{fillColor}{RGB}{0,0,0}

\path[draw=drawColor,line width= 0.4pt,line join=round,line cap=round,fill=fillColor] ( 69.28,107.41) circle (  1.49);
\definecolor{drawColor}{RGB}{255,0,0}
\definecolor{fillColor}{RGB}{255,0,0}

\path[draw=drawColor,line width= 0.4pt,line join=round,line cap=round,fill=fillColor] ( 69.73,101.17) circle (  1.49);
\definecolor{drawColor}{RGB}{0,0,0}
\definecolor{fillColor}{RGB}{0,0,0}

\path[draw=drawColor,line width= 0.4pt,line join=round,line cap=round,fill=fillColor] ( 69.74,113.50) circle (  1.49);
\definecolor{drawColor}{RGB}{255,0,0}
\definecolor{fillColor}{RGB}{255,0,0}

\path[draw=drawColor,line width= 0.4pt,line join=round,line cap=round,fill=fillColor] ( 70.18,101.67) circle (  1.49);
\definecolor{drawColor}{RGB}{0,0,0}
\definecolor{fillColor}{RGB}{0,0,0}

\path[draw=drawColor,line width= 0.4pt,line join=round,line cap=round,fill=fillColor] ( 70.20,117.26) circle (  1.49);
\definecolor{drawColor}{RGB}{255,0,0}
\definecolor{fillColor}{RGB}{255,0,0}

\path[draw=drawColor,line width= 0.4pt,line join=round,line cap=round,fill=fillColor] ( 70.64,102.48) circle (  1.49);
\definecolor{drawColor}{RGB}{0,0,0}
\definecolor{fillColor}{RGB}{0,0,0}

\path[draw=drawColor,line width= 0.4pt,line join=round,line cap=round,fill=fillColor] ( 70.66,120.86) circle (  1.49);
\definecolor{drawColor}{RGB}{255,0,0}
\definecolor{fillColor}{RGB}{255,0,0}

\path[draw=drawColor,line width= 0.4pt,line join=round,line cap=round,fill=fillColor] ( 71.10,104.49) circle (  1.49);
\definecolor{drawColor}{RGB}{0,0,0}
\definecolor{fillColor}{RGB}{0,0,0}

\path[draw=drawColor,line width= 0.4pt,line join=round,line cap=round,fill=fillColor] ( 71.12,126.64) circle (  1.49);
\definecolor{drawColor}{RGB}{255,0,0}
\definecolor{fillColor}{RGB}{255,0,0}

\path[draw=drawColor,line width= 0.4pt,line join=round,line cap=round,fill=fillColor] ( 71.56,104.41) circle (  1.49);
\definecolor{drawColor}{RGB}{0,0,0}
\definecolor{fillColor}{RGB}{0,0,0}

\path[draw=drawColor,line width= 0.4pt,line join=round,line cap=round,fill=fillColor] ( 71.58,126.58) circle (  1.49);
\definecolor{drawColor}{RGB}{255,0,0}
\definecolor{fillColor}{RGB}{255,0,0}

\path[draw=drawColor,line width= 0.4pt,line join=round,line cap=round,fill=fillColor] ( 72.02,106.48) circle (  1.49);
\definecolor{drawColor}{RGB}{0,0,0}
\definecolor{fillColor}{RGB}{0,0,0}

\path[draw=drawColor,line width= 0.4pt,line join=round,line cap=round,fill=fillColor] ( 72.03,129.46) circle (  1.49);
\definecolor{drawColor}{RGB}{255,0,0}
\definecolor{fillColor}{RGB}{255,0,0}

\path[draw=drawColor,line width= 0.4pt,line join=round,line cap=round,fill=fillColor] ( 72.46,107.15) circle (  1.49);
\definecolor{drawColor}{RGB}{0,0,0}
\definecolor{fillColor}{RGB}{0,0,0}

\path[draw=drawColor,line width= 0.4pt,line join=round,line cap=round,fill=fillColor] ( 72.49,129.61) circle (  1.49);
\definecolor{drawColor}{RGB}{255,0,0}
\definecolor{fillColor}{RGB}{255,0,0}

\path[draw=drawColor,line width= 0.4pt,line join=round,line cap=round,fill=fillColor] ( 72.92,108.08) circle (  1.49);
\definecolor{drawColor}{RGB}{0,0,0}
\definecolor{fillColor}{RGB}{0,0,0}

\path[draw=drawColor,line width= 0.4pt,line join=round,line cap=round,fill=fillColor] ( 72.93,131.97) circle (  1.49);
\definecolor{drawColor}{RGB}{255,0,0}
\definecolor{fillColor}{RGB}{255,0,0}

\path[draw=drawColor,line width= 0.4pt,line join=round,line cap=round,fill=fillColor] ( 73.36,108.90) circle (  1.49);
\definecolor{drawColor}{RGB}{0,0,0}
\definecolor{fillColor}{RGB}{0,0,0}

\path[draw=drawColor,line width= 0.4pt,line join=round,line cap=round,fill=fillColor] ( 73.38,145.05) circle (  1.49);
\definecolor{drawColor}{RGB}{255,0,0}
\definecolor{fillColor}{RGB}{255,0,0}

\path[draw=drawColor,line width= 0.4pt,line join=round,line cap=round,fill=fillColor] ( 73.82,109.23) circle (  1.49);
\definecolor{drawColor}{RGB}{0,0,0}
\definecolor{fillColor}{RGB}{0,0,0}

\path[draw=drawColor,line width= 0.4pt,line join=round,line cap=round,fill=fillColor] ( 73.84,154.41) circle (  1.49);
\definecolor{drawColor}{RGB}{255,0,0}
\definecolor{fillColor}{RGB}{255,0,0}

\path[draw=drawColor,line width= 0.4pt,line join=round,line cap=round,fill=fillColor] ( 74.26,109.97) circle (  1.49);
\definecolor{drawColor}{RGB}{0,0,0}
\definecolor{fillColor}{RGB}{0,0,0}

\path[draw=drawColor,line width= 0.4pt,line join=round,line cap=round,fill=fillColor] ( 74.28,173.36) circle (  1.49);
\definecolor{drawColor}{RGB}{255,0,0}
\definecolor{fillColor}{RGB}{255,0,0}

\path[draw=drawColor,line width= 0.4pt,line join=round,line cap=round,fill=fillColor] ( 74.70,108.86) circle (  1.49);
\definecolor{drawColor}{RGB}{0,0,0}
\definecolor{fillColor}{RGB}{0,0,0}

\path[draw=drawColor,line width= 0.4pt,line join=round,line cap=round,fill=fillColor] ( 74.72,142.41) circle (  1.49);
\definecolor{drawColor}{RGB}{255,0,0}
\definecolor{fillColor}{RGB}{255,0,0}

\path[draw=drawColor,line width= 0.4pt,line join=round,line cap=round,fill=fillColor] ( 75.16,106.93) circle (  1.49);
\definecolor{drawColor}{RGB}{0,0,0}
\definecolor{fillColor}{RGB}{0,0,0}

\path[draw=drawColor,line width= 0.4pt,line join=round,line cap=round,fill=fillColor] ( 75.18,136.57) circle (  1.49);
\definecolor{drawColor}{RGB}{255,0,0}
\definecolor{fillColor}{RGB}{255,0,0}

\path[draw=drawColor,line width= 0.4pt,line join=round,line cap=round,fill=fillColor] ( 75.60,105.37) circle (  1.49);
\definecolor{drawColor}{RGB}{0,0,0}
\definecolor{fillColor}{RGB}{0,0,0}

\path[draw=drawColor,line width= 0.4pt,line join=round,line cap=round,fill=fillColor] ( 75.62,132.39) circle (  1.49);
\definecolor{drawColor}{RGB}{255,0,0}
\definecolor{fillColor}{RGB}{255,0,0}

\path[draw=drawColor,line width= 0.4pt,line join=round,line cap=round,fill=fillColor] ( 76.06,103.24) circle (  1.49);
\definecolor{drawColor}{RGB}{0,0,0}
\definecolor{fillColor}{RGB}{0,0,0}

\path[draw=drawColor,line width= 0.4pt,line join=round,line cap=round,fill=fillColor] ( 76.08,153.68) circle (  1.49);
\definecolor{drawColor}{RGB}{255,0,0}
\definecolor{fillColor}{RGB}{255,0,0}

\path[draw=drawColor,line width= 0.4pt,line join=round,line cap=round,fill=fillColor] ( 76.52,102.17) circle (  1.49);
\definecolor{drawColor}{RGB}{0,0,0}
\definecolor{fillColor}{RGB}{0,0,0}

\path[draw=drawColor,line width= 0.4pt,line join=round,line cap=round,fill=fillColor] ( 76.54,232.85) circle (  1.49);
\definecolor{drawColor}{RGB}{255,0,0}
\definecolor{fillColor}{RGB}{255,0,0}

\path[draw=drawColor,line width= 0.4pt,line join=round,line cap=round,fill=fillColor] ( 76.96,100.19) circle (  1.49);
\definecolor{drawColor}{RGB}{0,0,0}
\definecolor{fillColor}{RGB}{0,0,0}

\path[draw=drawColor,line width= 0.4pt,line join=round,line cap=round,fill=fillColor] ( 76.98,156.14) circle (  1.49);
\definecolor{drawColor}{RGB}{255,0,0}
\definecolor{fillColor}{RGB}{255,0,0}

\path[draw=drawColor,line width= 0.4pt,line join=round,line cap=round,fill=fillColor] ( 77.44, 98.57) circle (  1.49);
\definecolor{drawColor}{RGB}{0,0,0}
\definecolor{fillColor}{RGB}{0,0,0}

\path[draw=drawColor,line width= 0.4pt,line join=round,line cap=round,fill=fillColor] ( 77.45,234.37) circle (  1.49);
\definecolor{drawColor}{RGB}{255,0,0}
\definecolor{fillColor}{RGB}{255,0,0}

\path[draw=drawColor,line width= 0.4pt,line join=round,line cap=round,fill=fillColor] ( 77.94, 96.92) circle (  1.49);
\definecolor{drawColor}{RGB}{0,0,0}
\definecolor{fillColor}{RGB}{0,0,0}

\path[draw=drawColor,line width= 0.4pt,line join=round,line cap=round,fill=fillColor] ( 77.98,231.32) circle (  1.49);
\definecolor{drawColor}{RGB}{255,0,0}
\definecolor{fillColor}{RGB}{255,0,0}

\path[draw=drawColor,line width= 0.4pt,line join=round,line cap=round,fill=fillColor] ( 78.40, 95.07) circle (  1.49);
\definecolor{drawColor}{RGB}{0,0,0}
\definecolor{fillColor}{RGB}{0,0,0}

\path[draw=drawColor,line width= 0.4pt,line join=round,line cap=round,fill=fillColor] ( 78.42,210.63) circle (  1.49);
\definecolor{drawColor}{RGB}{255,0,0}
\definecolor{fillColor}{RGB}{255,0,0}

\path[draw=drawColor,line width= 0.4pt,line join=round,line cap=round,fill=fillColor] ( 78.84, 93.86) circle (  1.49);
\definecolor{drawColor}{RGB}{0,0,0}
\definecolor{fillColor}{RGB}{0,0,0}

\path[draw=drawColor,line width= 0.4pt,line join=round,line cap=round,fill=fillColor] ( 78.86,165.96) circle (  1.49);
\definecolor{drawColor}{RGB}{255,0,0}
\definecolor{fillColor}{RGB}{255,0,0}

\path[draw=drawColor,line width= 0.4pt,line join=round,line cap=round,fill=fillColor] ( 79.30, 89.92) circle (  1.49);
\definecolor{drawColor}{RGB}{0,0,0}
\definecolor{fillColor}{RGB}{0,0,0}

\path[draw=drawColor,line width= 0.4pt,line join=round,line cap=round,fill=fillColor] ( 79.32,138.75) circle (  1.49);
\definecolor{drawColor}{RGB}{255,0,0}
\definecolor{fillColor}{RGB}{255,0,0}

\path[draw=drawColor,line width= 0.4pt,line join=round,line cap=round,fill=fillColor] ( 79.78, 88.97) circle (  1.49);
\definecolor{drawColor}{RGB}{0,0,0}
\definecolor{fillColor}{RGB}{0,0,0}

\path[draw=drawColor,line width= 0.4pt,line join=round,line cap=round,fill=fillColor] ( 79.79,108.88) circle (  1.49);
\definecolor{drawColor}{RGB}{255,0,0}
\definecolor{fillColor}{RGB}{255,0,0}

\path[draw=drawColor,line width= 0.4pt,line join=round,line cap=round,fill=fillColor] ( 80.25, 89.00) circle (  1.49);
\definecolor{drawColor}{RGB}{0,0,0}
\definecolor{fillColor}{RGB}{0,0,0}

\path[draw=drawColor,line width= 0.4pt,line join=round,line cap=round,fill=fillColor] ( 80.27,107.56) circle (  1.49);
\definecolor{drawColor}{RGB}{255,0,0}
\definecolor{fillColor}{RGB}{255,0,0}

\path[draw=drawColor,line width= 0.4pt,line join=round,line cap=round,fill=fillColor] ( 80.76, 88.32) circle (  1.49);
\definecolor{drawColor}{RGB}{0,0,0}
\definecolor{fillColor}{RGB}{0,0,0}

\path[draw=drawColor,line width= 0.4pt,line join=round,line cap=round,fill=fillColor] ( 80.78,109.16) circle (  1.49);
\definecolor{drawColor}{RGB}{255,0,0}
\definecolor{fillColor}{RGB}{255,0,0}

\path[draw=drawColor,line width= 0.4pt,line join=round,line cap=round,fill=fillColor] ( 81.25, 86.79) circle (  1.49);
\definecolor{drawColor}{RGB}{0,0,0}
\definecolor{fillColor}{RGB}{0,0,0}

\path[draw=drawColor,line width= 0.4pt,line join=round,line cap=round,fill=fillColor] ( 81.27,116.75) circle (  1.49);
\definecolor{drawColor}{RGB}{255,0,0}
\definecolor{fillColor}{RGB}{255,0,0}

\path[draw=drawColor,line width= 0.4pt,line join=round,line cap=round,fill=fillColor] ( 81.73, 84.13) circle (  1.49);
\definecolor{drawColor}{RGB}{0,0,0}
\definecolor{fillColor}{RGB}{0,0,0}

\path[draw=drawColor,line width= 0.4pt,line join=round,line cap=round,fill=fillColor] ( 81.74,115.55) circle (  1.49);
\definecolor{drawColor}{RGB}{255,0,0}
\definecolor{fillColor}{RGB}{255,0,0}

\path[draw=drawColor,line width= 0.4pt,line join=round,line cap=round,fill=fillColor] ( 82.20, 81.48) circle (  1.49);
\definecolor{drawColor}{RGB}{0,0,0}
\definecolor{fillColor}{RGB}{0,0,0}

\path[draw=drawColor,line width= 0.4pt,line join=round,line cap=round,fill=fillColor] ( 82.22,139.64) circle (  1.49);
\definecolor{drawColor}{RGB}{255,0,0}
\definecolor{fillColor}{RGB}{255,0,0}

\path[draw=drawColor,line width= 0.4pt,line join=round,line cap=round,fill=fillColor] ( 82.69, 79.59) circle (  1.49);
\definecolor{drawColor}{RGB}{0,0,0}
\definecolor{fillColor}{RGB}{0,0,0}

\path[draw=drawColor,line width= 0.4pt,line join=round,line cap=round,fill=fillColor] ( 82.71,205.93) circle (  1.49);
\definecolor{drawColor}{RGB}{255,0,0}
\definecolor{fillColor}{RGB}{255,0,0}

\path[draw=drawColor,line width= 0.4pt,line join=round,line cap=round,fill=fillColor] ( 83.17, 79.98) circle (  1.49);
\definecolor{drawColor}{RGB}{0,0,0}
\definecolor{fillColor}{RGB}{0,0,0}

\path[draw=drawColor,line width= 0.4pt,line join=round,line cap=round,fill=fillColor] ( 83.18,216.27) circle (  1.49);
\definecolor{drawColor}{RGB}{255,0,0}
\definecolor{fillColor}{RGB}{255,0,0}

\path[draw=drawColor,line width= 0.4pt,line join=round,line cap=round,fill=fillColor] ( 83.62, 78.93) circle (  1.49);
\definecolor{drawColor}{RGB}{0,0,0}
\definecolor{fillColor}{RGB}{0,0,0}

\path[draw=drawColor,line width= 0.4pt,line join=round,line cap=round,fill=fillColor] ( 83.64,206.36) circle (  1.49);
\definecolor{drawColor}{RGB}{255,0,0}
\definecolor{fillColor}{RGB}{255,0,0}

\path[draw=drawColor,line width= 0.4pt,line join=round,line cap=round,fill=fillColor] ( 84.10, 76.64) circle (  1.49);
\definecolor{drawColor}{RGB}{0,0,0}
\definecolor{fillColor}{RGB}{0,0,0}

\path[draw=drawColor,line width= 0.4pt,line join=round,line cap=round,fill=fillColor] ( 84.12,198.50) circle (  1.49);
\definecolor{drawColor}{RGB}{255,0,0}
\definecolor{fillColor}{RGB}{255,0,0}

\path[draw=drawColor,line width= 0.4pt,line join=round,line cap=round,fill=fillColor] ( 84.57, 76.43) circle (  1.49);
\definecolor{drawColor}{RGB}{0,0,0}
\definecolor{fillColor}{RGB}{0,0,0}

\path[draw=drawColor,line width= 0.4pt,line join=round,line cap=round,fill=fillColor] ( 84.59,195.36) circle (  1.49);
\definecolor{drawColor}{RGB}{255,0,0}
\definecolor{fillColor}{RGB}{255,0,0}

\path[draw=drawColor,line width= 0.4pt,line join=round,line cap=round,fill=fillColor] ( 85.05, 73.19) circle (  1.49);
\definecolor{drawColor}{RGB}{0,0,0}
\definecolor{fillColor}{RGB}{0,0,0}

\path[draw=drawColor,line width= 0.4pt,line join=round,line cap=round,fill=fillColor] ( 85.06,195.59) circle (  1.49);
\definecolor{drawColor}{RGB}{255,0,0}
\definecolor{fillColor}{RGB}{255,0,0}

\path[draw=drawColor,line width= 0.4pt,line join=round,line cap=round,fill=fillColor] ( 85.57, 71.06) circle (  1.49);
\definecolor{drawColor}{RGB}{0,0,0}
\definecolor{fillColor}{RGB}{0,0,0}

\path[draw=drawColor,line width= 0.4pt,line join=round,line cap=round,fill=fillColor] ( 85.59,188.52) circle (  1.49);
\definecolor{drawColor}{RGB}{255,0,0}
\definecolor{fillColor}{RGB}{255,0,0}

\path[draw=drawColor,line width= 0.4pt,line join=round,line cap=round,fill=fillColor] ( 86.06, 69.35) circle (  1.49);
\definecolor{drawColor}{RGB}{0,0,0}
\definecolor{fillColor}{RGB}{0,0,0}

\path[draw=drawColor,line width= 0.4pt,line join=round,line cap=round,fill=fillColor] ( 86.08,195.18) circle (  1.49);
\definecolor{drawColor}{RGB}{255,0,0}
\definecolor{fillColor}{RGB}{255,0,0}

\path[draw=drawColor,line width= 0.4pt,line join=round,line cap=round,fill=fillColor] ( 86.54, 68.66) circle (  1.49);
\definecolor{drawColor}{RGB}{0,0,0}
\definecolor{fillColor}{RGB}{0,0,0}

\path[draw=drawColor,line width= 0.4pt,line join=round,line cap=round,fill=fillColor] ( 86.55,180.32) circle (  1.49);
\definecolor{drawColor}{RGB}{255,0,0}
\definecolor{fillColor}{RGB}{255,0,0}

\path[draw=drawColor,line width= 0.4pt,line join=round,line cap=round,fill=fillColor] ( 87.09, 68.94) circle (  1.49);
\definecolor{drawColor}{RGB}{0,0,0}
\definecolor{fillColor}{RGB}{0,0,0}

\path[draw=drawColor,line width= 0.4pt,line join=round,line cap=round,fill=fillColor] ( 87.11,198.97) circle (  1.49);
\definecolor{drawColor}{RGB}{255,0,0}
\definecolor{fillColor}{RGB}{255,0,0}

\path[draw=drawColor,line width= 0.4pt,line join=round,line cap=round,fill=fillColor] ( 87.59, 69.68) circle (  1.49);
\definecolor{drawColor}{RGB}{0,0,0}
\definecolor{fillColor}{RGB}{0,0,0}

\path[draw=drawColor,line width= 0.4pt,line join=round,line cap=round,fill=fillColor] ( 87.60,203.43) circle (  1.49);
\definecolor{drawColor}{RGB}{255,0,0}
\definecolor{fillColor}{RGB}{255,0,0}

\path[draw=drawColor,line width= 0.4pt,line join=round,line cap=round,fill=fillColor] ( 88.08, 70.59) circle (  1.49);
\definecolor{drawColor}{RGB}{0,0,0}
\definecolor{fillColor}{RGB}{0,0,0}

\path[draw=drawColor,line width= 0.4pt,line join=round,line cap=round,fill=fillColor] ( 88.09,202.39) circle (  1.49);
\definecolor{drawColor}{RGB}{255,0,0}
\definecolor{fillColor}{RGB}{255,0,0}

\path[draw=drawColor,line width= 0.4pt,line join=round,line cap=round,fill=fillColor] ( 88.60, 70.50) circle (  1.49);
\definecolor{drawColor}{RGB}{0,0,0}
\definecolor{fillColor}{RGB}{0,0,0}

\path[draw=drawColor,line width= 0.4pt,line join=round,line cap=round,fill=fillColor] ( 88.62,202.28) circle (  1.49);
\definecolor{drawColor}{RGB}{255,0,0}
\definecolor{fillColor}{RGB}{255,0,0}

\path[draw=drawColor,line width= 0.4pt,line join=round,line cap=round,fill=fillColor] ( 89.08, 70.92) circle (  1.49);
\definecolor{drawColor}{RGB}{0,0,0}
\definecolor{fillColor}{RGB}{0,0,0}

\path[draw=drawColor,line width= 0.4pt,line join=round,line cap=round,fill=fillColor] ( 89.09,127.90) circle (  1.49);
\definecolor{drawColor}{RGB}{255,0,0}
\definecolor{fillColor}{RGB}{255,0,0}

\path[draw=drawColor,line width= 0.4pt,line join=round,line cap=round,fill=fillColor] ( 89.55, 71.45) circle (  1.49);
\definecolor{drawColor}{RGB}{0,0,0}
\definecolor{fillColor}{RGB}{0,0,0}

\path[draw=drawColor,line width= 0.4pt,line join=round,line cap=round,fill=fillColor] ( 89.58,175.10) circle (  1.49);
\definecolor{drawColor}{RGB}{255,0,0}
\definecolor{fillColor}{RGB}{255,0,0}

\path[draw=drawColor,line width= 0.4pt,line join=round,line cap=round,fill=fillColor] ( 90.04, 71.77) circle (  1.49);
\definecolor{drawColor}{RGB}{0,0,0}
\definecolor{fillColor}{RGB}{0,0,0}

\path[draw=drawColor,line width= 0.4pt,line join=round,line cap=round,fill=fillColor] ( 90.07,184.33) circle (  1.49);
\definecolor{drawColor}{RGB}{255,0,0}
\definecolor{fillColor}{RGB}{255,0,0}

\path[draw=drawColor,line width= 0.4pt,line join=round,line cap=round,fill=fillColor] ( 90.53, 72.44) circle (  1.49);
\definecolor{drawColor}{RGB}{0,0,0}
\definecolor{fillColor}{RGB}{0,0,0}

\path[draw=drawColor,line width= 0.4pt,line join=round,line cap=round,fill=fillColor] ( 90.55,183.95) circle (  1.49);
\definecolor{drawColor}{RGB}{255,0,0}
\definecolor{fillColor}{RGB}{255,0,0}

\path[draw=drawColor,line width= 0.4pt,line join=round,line cap=round,fill=fillColor] ( 91.01, 73.26) circle (  1.49);
\definecolor{drawColor}{RGB}{0,0,0}
\definecolor{fillColor}{RGB}{0,0,0}

\path[draw=drawColor,line width= 0.4pt,line join=round,line cap=round,fill=fillColor] ( 91.02,206.71) circle (  1.49);
\definecolor{drawColor}{RGB}{255,0,0}
\definecolor{fillColor}{RGB}{255,0,0}

\path[draw=drawColor,line width= 0.4pt,line join=round,line cap=round,fill=fillColor] ( 91.48, 75.93) circle (  1.49);
\definecolor{drawColor}{RGB}{0,0,0}
\definecolor{fillColor}{RGB}{0,0,0}

\path[draw=drawColor,line width= 0.4pt,line join=round,line cap=round,fill=fillColor] ( 91.50,208.79) circle (  1.49);
\definecolor{drawColor}{RGB}{255,0,0}
\definecolor{fillColor}{RGB}{255,0,0}

\path[draw=drawColor,line width= 0.4pt,line join=round,line cap=round,fill=fillColor] ( 91.94, 77.91) circle (  1.49);
\definecolor{drawColor}{RGB}{0,0,0}
\definecolor{fillColor}{RGB}{0,0,0}

\path[draw=drawColor,line width= 0.4pt,line join=round,line cap=round,fill=fillColor] ( 91.96,196.80) circle (  1.49);
\definecolor{drawColor}{RGB}{255,0,0}
\definecolor{fillColor}{RGB}{255,0,0}

\path[draw=drawColor,line width= 0.4pt,line join=round,line cap=round,fill=fillColor] ( 92.41, 78.03) circle (  1.49);
\definecolor{drawColor}{RGB}{0,0,0}
\definecolor{fillColor}{RGB}{0,0,0}

\path[draw=drawColor,line width= 0.4pt,line join=round,line cap=round,fill=fillColor] ( 92.43,212.47) circle (  1.49);
\definecolor{drawColor}{RGB}{255,0,0}
\definecolor{fillColor}{RGB}{255,0,0}

\path[draw=drawColor,line width= 0.4pt,line join=round,line cap=round,fill=fillColor] ( 92.91, 77.04) circle (  1.49);
\definecolor{drawColor}{RGB}{0,0,0}
\definecolor{fillColor}{RGB}{0,0,0}

\path[draw=drawColor,line width= 0.4pt,line join=round,line cap=round,fill=fillColor] ( 92.92,123.31) circle (  1.49);
\definecolor{drawColor}{RGB}{255,0,0}
\definecolor{fillColor}{RGB}{255,0,0}

\path[draw=drawColor,line width= 0.4pt,line join=round,line cap=round,fill=fillColor] ( 93.38, 78.34) circle (  1.49);
\definecolor{drawColor}{RGB}{0,0,0}
\definecolor{fillColor}{RGB}{0,0,0}

\path[draw=drawColor,line width= 0.4pt,line join=round,line cap=round,fill=fillColor] ( 93.40,195.83) circle (  1.49);
\definecolor{drawColor}{RGB}{255,0,0}
\definecolor{fillColor}{RGB}{255,0,0}

\path[draw=drawColor,line width= 0.4pt,line join=round,line cap=round,fill=fillColor] ( 93.86, 80.06) circle (  1.49);
\definecolor{drawColor}{RGB}{0,0,0}
\definecolor{fillColor}{RGB}{0,0,0}

\path[draw=drawColor,line width= 0.4pt,line join=round,line cap=round,fill=fillColor] ( 93.87,208.88) circle (  1.49);
\definecolor{drawColor}{RGB}{255,0,0}
\definecolor{fillColor}{RGB}{255,0,0}

\path[draw=drawColor,line width= 0.4pt,line join=round,line cap=round,fill=fillColor] ( 94.36, 80.62) circle (  1.49);
\definecolor{drawColor}{RGB}{0,0,0}
\definecolor{fillColor}{RGB}{0,0,0}

\path[draw=drawColor,line width= 0.4pt,line join=round,line cap=round,fill=fillColor] ( 94.38,208.50) circle (  1.49);
\definecolor{drawColor}{RGB}{255,0,0}
\definecolor{fillColor}{RGB}{255,0,0}

\path[draw=drawColor,line width= 0.4pt,line join=round,line cap=round,fill=fillColor] ( 94.84, 77.23) circle (  1.49);
\definecolor{drawColor}{RGB}{0,0,0}
\definecolor{fillColor}{RGB}{0,0,0}

\path[draw=drawColor,line width= 0.4pt,line join=round,line cap=round,fill=fillColor] ( 94.85,208.05) circle (  1.49);
\definecolor{drawColor}{RGB}{255,0,0}
\definecolor{fillColor}{RGB}{255,0,0}

\path[draw=drawColor,line width= 0.4pt,line join=round,line cap=round,fill=fillColor] ( 95.33, 76.13) circle (  1.49);
\definecolor{drawColor}{RGB}{0,0,0}
\definecolor{fillColor}{RGB}{0,0,0}

\path[draw=drawColor,line width= 0.4pt,line join=round,line cap=round,fill=fillColor] ( 95.36,204.05) circle (  1.49);
\definecolor{drawColor}{RGB}{255,0,0}
\definecolor{fillColor}{RGB}{255,0,0}

\path[draw=drawColor,line width= 0.4pt,line join=round,line cap=round,fill=fillColor] ( 95.80, 75.09) circle (  1.49);
\definecolor{drawColor}{RGB}{0,0,0}
\definecolor{fillColor}{RGB}{0,0,0}

\path[draw=drawColor,line width= 0.4pt,line join=round,line cap=round,fill=fillColor] ( 95.82,198.32) circle (  1.49);
\definecolor{drawColor}{RGB}{255,0,0}
\definecolor{fillColor}{RGB}{255,0,0}

\path[draw=drawColor,line width= 0.4pt,line join=round,line cap=round,fill=fillColor] ( 96.29, 73.89) circle (  1.49);
\definecolor{drawColor}{RGB}{0,0,0}
\definecolor{fillColor}{RGB}{0,0,0}

\path[draw=drawColor,line width= 0.4pt,line join=round,line cap=round,fill=fillColor] ( 96.31,200.69) circle (  1.49);
\definecolor{drawColor}{RGB}{255,0,0}
\definecolor{fillColor}{RGB}{255,0,0}

\path[draw=drawColor,line width= 0.4pt,line join=round,line cap=round,fill=fillColor] ( 96.77, 73.35) circle (  1.49);
\definecolor{drawColor}{RGB}{0,0,0}
\definecolor{fillColor}{RGB}{0,0,0}

\path[draw=drawColor,line width= 0.4pt,line join=round,line cap=round,fill=fillColor] ( 96.79,201.00) circle (  1.49);
\definecolor{drawColor}{RGB}{255,0,0}
\definecolor{fillColor}{RGB}{255,0,0}

\path[draw=drawColor,line width= 0.4pt,line join=round,line cap=round,fill=fillColor] ( 97.26, 73.05) circle (  1.49);
\definecolor{drawColor}{RGB}{0,0,0}
\definecolor{fillColor}{RGB}{0,0,0}

\path[draw=drawColor,line width= 0.4pt,line join=round,line cap=round,fill=fillColor] ( 97.28,191.85) circle (  1.49);
\definecolor{drawColor}{RGB}{255,0,0}
\definecolor{fillColor}{RGB}{255,0,0}

\path[draw=drawColor,line width= 0.4pt,line join=round,line cap=round,fill=fillColor] ( 97.72, 71.08) circle (  1.49);
\definecolor{drawColor}{RGB}{0,0,0}
\definecolor{fillColor}{RGB}{0,0,0}

\path[draw=drawColor,line width= 0.4pt,line join=round,line cap=round,fill=fillColor] ( 97.73,196.95) circle (  1.49);
\definecolor{drawColor}{RGB}{255,0,0}
\definecolor{fillColor}{RGB}{255,0,0}

\path[draw=drawColor,line width= 0.4pt,line join=round,line cap=round,fill=fillColor] ( 98.21, 70.19) circle (  1.49);
\definecolor{drawColor}{RGB}{0,0,0}
\definecolor{fillColor}{RGB}{0,0,0}

\path[draw=drawColor,line width= 0.4pt,line join=round,line cap=round,fill=fillColor] ( 98.23,168.46) circle (  1.49);
\definecolor{drawColor}{RGB}{255,0,0}
\definecolor{fillColor}{RGB}{255,0,0}

\path[draw=drawColor,line width= 0.4pt,line join=round,line cap=round,fill=fillColor] ( 98.68, 68.79) circle (  1.49);
\definecolor{drawColor}{RGB}{0,0,0}
\definecolor{fillColor}{RGB}{0,0,0}

\path[draw=drawColor,line width= 0.4pt,line join=round,line cap=round,fill=fillColor] ( 98.70,175.83) circle (  1.49);
\definecolor{drawColor}{RGB}{255,0,0}
\definecolor{fillColor}{RGB}{255,0,0}

\path[draw=drawColor,line width= 0.4pt,line join=round,line cap=round,fill=fillColor] ( 99.16, 68.72) circle (  1.49);
\definecolor{drawColor}{RGB}{0,0,0}
\definecolor{fillColor}{RGB}{0,0,0}

\path[draw=drawColor,line width= 0.4pt,line join=round,line cap=round,fill=fillColor] ( 99.18,146.63) circle (  1.49);
\definecolor{drawColor}{RGB}{255,0,0}
\definecolor{fillColor}{RGB}{255,0,0}

\path[draw=drawColor,line width= 0.4pt,line join=round,line cap=round,fill=fillColor] ( 99.65, 68.32) circle (  1.49);
\definecolor{drawColor}{RGB}{0,0,0}
\definecolor{fillColor}{RGB}{0,0,0}

\path[draw=drawColor,line width= 0.4pt,line join=round,line cap=round,fill=fillColor] ( 99.67,164.01) circle (  1.49);
\definecolor{drawColor}{RGB}{255,0,0}
\definecolor{fillColor}{RGB}{255,0,0}

\path[draw=drawColor,line width= 0.4pt,line join=round,line cap=round,fill=fillColor] (100.12, 69.62) circle (  1.49);
\definecolor{drawColor}{RGB}{0,0,0}
\definecolor{fillColor}{RGB}{0,0,0}

\path[draw=drawColor,line width= 0.4pt,line join=round,line cap=round,fill=fillColor] (100.14,187.23) circle (  1.49);
\definecolor{drawColor}{RGB}{255,0,0}
\definecolor{fillColor}{RGB}{255,0,0}

\path[draw=drawColor,line width= 0.4pt,line join=round,line cap=round,fill=fillColor] (100.60, 68.70) circle (  1.49);
\definecolor{drawColor}{RGB}{0,0,0}
\definecolor{fillColor}{RGB}{0,0,0}

\path[draw=drawColor,line width= 0.4pt,line join=round,line cap=round,fill=fillColor] (100.62,177.26) circle (  1.49);
\definecolor{drawColor}{RGB}{255,0,0}
\definecolor{fillColor}{RGB}{255,0,0}

\path[draw=drawColor,line width= 0.4pt,line join=round,line cap=round,fill=fillColor] (101.16, 65.55) circle (  1.49);
\definecolor{drawColor}{RGB}{0,0,0}
\definecolor{fillColor}{RGB}{0,0,0}

\path[draw=drawColor,line width= 0.4pt,line join=round,line cap=round,fill=fillColor] (101.17, 92.84) circle (  1.49);
\definecolor{drawColor}{RGB}{255,0,0}
\definecolor{fillColor}{RGB}{255,0,0}

\path[draw=drawColor,line width= 0.4pt,line join=round,line cap=round,fill=fillColor] (101.70, 63.66) circle (  1.49);
\definecolor{drawColor}{RGB}{0,0,0}
\definecolor{fillColor}{RGB}{0,0,0}

\path[draw=drawColor,line width= 0.4pt,line join=round,line cap=round,fill=fillColor] (101.71,117.34) circle (  1.49);
\definecolor{drawColor}{RGB}{255,0,0}
\definecolor{fillColor}{RGB}{255,0,0}

\path[draw=drawColor,line width= 0.4pt,line join=round,line cap=round,fill=fillColor] (102.19, 62.96) circle (  1.49);
\definecolor{drawColor}{RGB}{0,0,0}
\definecolor{fillColor}{RGB}{0,0,0}

\path[draw=drawColor,line width= 0.4pt,line join=round,line cap=round,fill=fillColor] (102.20,199.14) circle (  1.49);
\definecolor{drawColor}{RGB}{255,0,0}
\definecolor{fillColor}{RGB}{255,0,0}

\path[draw=drawColor,line width= 0.4pt,line join=round,line cap=round,fill=fillColor] (102.68, 69.90) circle (  1.49);
\definecolor{drawColor}{RGB}{0,0,0}
\definecolor{fillColor}{RGB}{0,0,0}

\path[draw=drawColor,line width= 0.4pt,line join=round,line cap=round,fill=fillColor] (102.69,181.56) circle (  1.49);
\definecolor{drawColor}{RGB}{255,0,0}
\definecolor{fillColor}{RGB}{255,0,0}

\path[draw=drawColor,line width= 0.4pt,line join=round,line cap=round,fill=fillColor] (103.15, 63.29) circle (  1.49);
\definecolor{drawColor}{RGB}{0,0,0}
\definecolor{fillColor}{RGB}{0,0,0}

\path[draw=drawColor,line width= 0.4pt,line join=round,line cap=round,fill=fillColor] (103.17,191.06) circle (  1.49);
\definecolor{drawColor}{RGB}{255,0,0}
\definecolor{fillColor}{RGB}{255,0,0}

\path[draw=drawColor,line width= 0.4pt,line join=round,line cap=round,fill=fillColor] (103.63, 62.43) circle (  1.49);
\definecolor{drawColor}{RGB}{0,0,0}
\definecolor{fillColor}{RGB}{0,0,0}

\path[draw=drawColor,line width= 0.4pt,line join=round,line cap=round,fill=fillColor] (103.64,193.04) circle (  1.49);
\definecolor{drawColor}{RGB}{255,0,0}
\definecolor{fillColor}{RGB}{255,0,0}

\path[draw=drawColor,line width= 0.4pt,line join=round,line cap=round,fill=fillColor] (104.12, 62.51) circle (  1.49);
\definecolor{drawColor}{RGB}{0,0,0}
\definecolor{fillColor}{RGB}{0,0,0}

\path[draw=drawColor,line width= 0.4pt,line join=round,line cap=round,fill=fillColor] (104.14,191.77) circle (  1.49);
\definecolor{drawColor}{RGB}{255,0,0}
\definecolor{fillColor}{RGB}{255,0,0}

\path[draw=drawColor,line width= 0.4pt,line join=round,line cap=round,fill=fillColor] (104.61, 62.85) circle (  1.49);
\definecolor{drawColor}{RGB}{0,0,0}
\definecolor{fillColor}{RGB}{0,0,0}

\path[draw=drawColor,line width= 0.4pt,line join=round,line cap=round,fill=fillColor] (104.63,192.01) circle (  1.49);
\definecolor{drawColor}{RGB}{255,0,0}
\definecolor{fillColor}{RGB}{255,0,0}

\path[draw=drawColor,line width= 0.4pt,line join=round,line cap=round,fill=fillColor] (105.08, 65.25) circle (  1.49);
\definecolor{drawColor}{RGB}{0,0,0}
\definecolor{fillColor}{RGB}{0,0,0}

\path[draw=drawColor,line width= 0.4pt,line join=round,line cap=round,fill=fillColor] (105.10,188.92) circle (  1.49);
\definecolor{drawColor}{RGB}{255,0,0}
\definecolor{fillColor}{RGB}{255,0,0}

\path[draw=drawColor,line width= 0.4pt,line join=round,line cap=round,fill=fillColor] (105.56, 63.73) circle (  1.49);
\definecolor{drawColor}{RGB}{0,0,0}
\definecolor{fillColor}{RGB}{0,0,0}

\path[draw=drawColor,line width= 0.4pt,line join=round,line cap=round,fill=fillColor] (105.58,184.25) circle (  1.49);
\definecolor{drawColor}{RGB}{255,0,0}
\definecolor{fillColor}{RGB}{255,0,0}

\path[draw=drawColor,line width= 0.4pt,line join=round,line cap=round,fill=fillColor] (106.15, 97.30) circle (  1.49);
\definecolor{drawColor}{RGB}{0,0,0}
\definecolor{fillColor}{RGB}{0,0,0}

\path[draw=drawColor,line width= 0.4pt,line join=round,line cap=round,fill=fillColor] (106.16,140.71) circle (  1.49);
\definecolor{drawColor}{RGB}{255,0,0}
\definecolor{fillColor}{RGB}{255,0,0}

\path[draw=drawColor,line width= 0.4pt,line join=round,line cap=round,fill=fillColor] (106.66, 57.88) circle (  1.49);
\definecolor{drawColor}{RGB}{0,0,0}
\definecolor{fillColor}{RGB}{0,0,0}

\path[draw=drawColor,line width= 0.4pt,line join=round,line cap=round,fill=fillColor] (106.67,191.54) circle (  1.49);
\definecolor{drawColor}{RGB}{255,0,0}
\definecolor{fillColor}{RGB}{255,0,0}

\path[draw=drawColor,line width= 0.4pt,line join=round,line cap=round,fill=fillColor] (107.16, 63.00) circle (  1.49);
\definecolor{drawColor}{RGB}{0,0,0}
\definecolor{fillColor}{RGB}{0,0,0}

\path[draw=drawColor,line width= 0.4pt,line join=round,line cap=round,fill=fillColor] (107.18,179.89) circle (  1.49);
\definecolor{drawColor}{RGB}{255,0,0}
\definecolor{fillColor}{RGB}{255,0,0}

\path[draw=drawColor,line width= 0.4pt,line join=round,line cap=round,fill=fillColor] (107.70, 74.41) circle (  1.49);
\definecolor{drawColor}{RGB}{0,0,0}
\definecolor{fillColor}{RGB}{0,0,0}

\path[draw=drawColor,line width= 0.4pt,line join=round,line cap=round,fill=fillColor] (107.72,178.89) circle (  1.49);
\definecolor{drawColor}{RGB}{255,0,0}
\definecolor{fillColor}{RGB}{255,0,0}

\path[draw=drawColor,line width= 0.4pt,line join=round,line cap=round,fill=fillColor] (108.39, 58.22) circle (  1.49);
\definecolor{drawColor}{RGB}{0,0,0}
\definecolor{fillColor}{RGB}{0,0,0}

\path[draw=drawColor,line width= 0.4pt,line join=round,line cap=round,fill=fillColor] (108.41,112.77) circle (  1.49);
\definecolor{drawColor}{RGB}{255,0,0}
\definecolor{fillColor}{RGB}{255,0,0}

\path[draw=drawColor,line width= 0.4pt,line join=round,line cap=round,fill=fillColor] (108.93, 58.18) circle (  1.49);
\definecolor{drawColor}{RGB}{0,0,0}
\definecolor{fillColor}{RGB}{0,0,0}

\path[draw=drawColor,line width= 0.4pt,line join=round,line cap=round,fill=fillColor] (108.95,188.61) circle (  1.49);
\definecolor{drawColor}{RGB}{255,0,0}
\definecolor{fillColor}{RGB}{255,0,0}

\path[draw=drawColor,line width= 0.4pt,line join=round,line cap=round,fill=fillColor] (109.42, 57.63) circle (  1.49);
\definecolor{drawColor}{RGB}{0,0,0}
\definecolor{fillColor}{RGB}{0,0,0}

\path[draw=drawColor,line width= 0.4pt,line join=round,line cap=round,fill=fillColor] (109.44,185.85) circle (  1.49);
\definecolor{drawColor}{RGB}{255,0,0}
\definecolor{fillColor}{RGB}{255,0,0}

\path[draw=drawColor,line width= 0.4pt,line join=round,line cap=round,fill=fillColor] (109.93, 57.17) circle (  1.49);
\definecolor{drawColor}{RGB}{0,0,0}
\definecolor{fillColor}{RGB}{0,0,0}

\path[draw=drawColor,line width= 0.4pt,line join=round,line cap=round,fill=fillColor] (109.95,186.57) circle (  1.49);
\definecolor{drawColor}{RGB}{255,0,0}
\definecolor{fillColor}{RGB}{255,0,0}

\path[draw=drawColor,line width= 0.4pt,line join=round,line cap=round,fill=fillColor] (110.45, 56.69) circle (  1.49);
\definecolor{drawColor}{RGB}{0,0,0}
\definecolor{fillColor}{RGB}{0,0,0}

\path[draw=drawColor,line width= 0.4pt,line join=round,line cap=round,fill=fillColor] (110.47,186.19) circle (  1.49);
\definecolor{drawColor}{RGB}{255,0,0}
\definecolor{fillColor}{RGB}{255,0,0}

\path[draw=drawColor,line width= 0.4pt,line join=round,line cap=round,fill=fillColor] (110.94, 55.76) circle (  1.49);
\definecolor{drawColor}{RGB}{0,0,0}
\definecolor{fillColor}{RGB}{0,0,0}

\path[draw=drawColor,line width= 0.4pt,line join=round,line cap=round,fill=fillColor] (110.96,183.37) circle (  1.49);
\definecolor{drawColor}{RGB}{255,0,0}
\definecolor{fillColor}{RGB}{255,0,0}

\path[draw=drawColor,line width= 0.4pt,line join=round,line cap=round,fill=fillColor] (111.45, 54.89) circle (  1.49);
\definecolor{drawColor}{RGB}{0,0,0}
\definecolor{fillColor}{RGB}{0,0,0}

\path[draw=drawColor,line width= 0.4pt,line join=round,line cap=round,fill=fillColor] (111.47,184.72) circle (  1.49);
\definecolor{drawColor}{RGB}{255,0,0}
\definecolor{fillColor}{RGB}{255,0,0}

\path[draw=drawColor,line width= 0.4pt,line join=round,line cap=round,fill=fillColor] (111.99, 57.91) circle (  1.49);
\definecolor{drawColor}{RGB}{0,0,0}
\definecolor{fillColor}{RGB}{0,0,0}

\path[draw=drawColor,line width= 0.4pt,line join=round,line cap=round,fill=fillColor] (112.01,184.72) circle (  1.49);
\definecolor{drawColor}{RGB}{255,0,0}
\definecolor{fillColor}{RGB}{255,0,0}

\path[draw=drawColor,line width= 0.4pt,line join=round,line cap=round,fill=fillColor] (112.48, 59.32) circle (  1.49);
\definecolor{drawColor}{RGB}{0,0,0}
\definecolor{fillColor}{RGB}{0,0,0}

\path[draw=drawColor,line width= 0.4pt,line join=round,line cap=round,fill=fillColor] (112.52,160.14) circle (  1.49);
\definecolor{drawColor}{RGB}{255,0,0}
\definecolor{fillColor}{RGB}{255,0,0}

\path[draw=drawColor,line width= 0.4pt,line join=round,line cap=round,fill=fillColor] (112.99, 62.39) circle (  1.49);
\definecolor{drawColor}{RGB}{0,0,0}
\definecolor{fillColor}{RGB}{0,0,0}

\path[draw=drawColor,line width= 0.4pt,line join=round,line cap=round,fill=fillColor] (113.01,184.37) circle (  1.49);
\definecolor{drawColor}{RGB}{255,0,0}
\definecolor{fillColor}{RGB}{255,0,0}

\path[draw=drawColor,line width= 0.4pt,line join=round,line cap=round,fill=fillColor] (113.50, 85.77) circle (  1.49);
\definecolor{drawColor}{RGB}{0,0,0}
\definecolor{fillColor}{RGB}{0,0,0}

\path[draw=drawColor,line width= 0.4pt,line join=round,line cap=round,fill=fillColor] (113.51,184.13) circle (  1.49);
\definecolor{drawColor}{RGB}{255,0,0}
\definecolor{fillColor}{RGB}{255,0,0}

\path[draw=drawColor,line width= 0.4pt,line join=round,line cap=round,fill=fillColor] (114.01, 55.32) circle (  1.49);
\definecolor{drawColor}{RGB}{0,0,0}
\definecolor{fillColor}{RGB}{0,0,0}

\path[draw=drawColor,line width= 0.4pt,line join=round,line cap=round,fill=fillColor] (114.02,183.25) circle (  1.49);
\definecolor{drawColor}{RGB}{255,0,0}
\definecolor{fillColor}{RGB}{255,0,0}

\path[draw=drawColor,line width= 0.4pt,line join=round,line cap=round,fill=fillColor] (114.51, 57.99) circle (  1.49);
\definecolor{drawColor}{RGB}{0,0,0}
\definecolor{fillColor}{RGB}{0,0,0}

\path[draw=drawColor,line width= 0.4pt,line join=round,line cap=round,fill=fillColor] (114.53,184.74) circle (  1.49);
\definecolor{drawColor}{RGB}{255,0,0}
\definecolor{fillColor}{RGB}{255,0,0}

\path[draw=drawColor,line width= 0.4pt,line join=round,line cap=round,fill=fillColor] (115.02, 58.49) circle (  1.49);
\definecolor{drawColor}{RGB}{0,0,0}
\definecolor{fillColor}{RGB}{0,0,0}

\path[draw=drawColor,line width= 0.4pt,line join=round,line cap=round,fill=fillColor] (115.04,184.22) circle (  1.49);
\definecolor{drawColor}{RGB}{255,0,0}
\definecolor{fillColor}{RGB}{255,0,0}

\path[draw=drawColor,line width= 0.4pt,line join=round,line cap=round,fill=fillColor] (115.53, 68.23) circle (  1.49);
\definecolor{drawColor}{RGB}{0,0,0}
\definecolor{fillColor}{RGB}{0,0,0}

\path[draw=drawColor,line width= 0.4pt,line join=round,line cap=round,fill=fillColor] (115.54,183.68) circle (  1.49);
\definecolor{drawColor}{RGB}{255,0,0}
\definecolor{fillColor}{RGB}{255,0,0}

\path[draw=drawColor,line width= 0.4pt,line join=round,line cap=round,fill=fillColor] (116.02, 66.02) circle (  1.49);
\definecolor{drawColor}{RGB}{0,0,0}
\definecolor{fillColor}{RGB}{0,0,0}

\path[draw=drawColor,line width= 0.4pt,line join=round,line cap=round,fill=fillColor] (116.04,183.65) circle (  1.49);
\definecolor{drawColor}{RGB}{255,0,0}
\definecolor{fillColor}{RGB}{255,0,0}

\path[draw=drawColor,line width= 0.4pt,line join=round,line cap=round,fill=fillColor] (116.51, 57.71) circle (  1.49);
\definecolor{drawColor}{RGB}{0,0,0}
\definecolor{fillColor}{RGB}{0,0,0}

\path[draw=drawColor,line width= 0.4pt,line join=round,line cap=round,fill=fillColor] (116.53,161.33) circle (  1.49);
\definecolor{drawColor}{RGB}{255,0,0}
\definecolor{fillColor}{RGB}{255,0,0}

\path[draw=drawColor,line width= 0.4pt,line join=round,line cap=round,fill=fillColor] (117.02, 90.46) circle (  1.49);
\definecolor{drawColor}{RGB}{0,0,0}
\definecolor{fillColor}{RGB}{0,0,0}

\path[draw=drawColor,line width= 0.4pt,line join=round,line cap=round,fill=fillColor] (117.03,184.63) circle (  1.49);
\definecolor{drawColor}{RGB}{255,0,0}
\definecolor{fillColor}{RGB}{255,0,0}

\path[draw=drawColor,line width= 0.4pt,line join=round,line cap=round,fill=fillColor] (117.56, 56.32) circle (  1.49);
\definecolor{drawColor}{RGB}{0,0,0}
\definecolor{fillColor}{RGB}{0,0,0}

\path[draw=drawColor,line width= 0.4pt,line join=round,line cap=round,fill=fillColor] (117.59,183.67) circle (  1.49);
\definecolor{drawColor}{RGB}{255,0,0}
\definecolor{fillColor}{RGB}{255,0,0}

\path[draw=drawColor,line width= 0.4pt,line join=round,line cap=round,fill=fillColor] (118.07, 55.74) circle (  1.49);
\definecolor{drawColor}{RGB}{0,0,0}
\definecolor{fillColor}{RGB}{0,0,0}

\path[draw=drawColor,line width= 0.4pt,line join=round,line cap=round,fill=fillColor] (118.08,182.47) circle (  1.49);
\definecolor{drawColor}{RGB}{255,0,0}
\definecolor{fillColor}{RGB}{255,0,0}

\path[draw=drawColor,line width= 0.4pt,line join=round,line cap=round,fill=fillColor] (118.59,102.34) circle (  1.49);
\definecolor{drawColor}{RGB}{0,0,0}
\definecolor{fillColor}{RGB}{0,0,0}

\path[draw=drawColor,line width= 0.4pt,line join=round,line cap=round,fill=fillColor] (118.61,182.73) circle (  1.49);
\definecolor{drawColor}{RGB}{255,0,0}
\definecolor{fillColor}{RGB}{255,0,0}

\path[draw=drawColor,line width= 0.4pt,line join=round,line cap=round,fill=fillColor] (119.20, 88.26) circle (  1.49);
\definecolor{drawColor}{RGB}{0,0,0}
\definecolor{fillColor}{RGB}{0,0,0}

\path[draw=drawColor,line width= 0.4pt,line join=round,line cap=round,fill=fillColor] (119.21,183.21) circle (  1.49);
\definecolor{drawColor}{RGB}{255,0,0}
\definecolor{fillColor}{RGB}{255,0,0}

\path[draw=drawColor,line width= 0.4pt,line join=round,line cap=round,fill=fillColor] (119.70,139.41) circle (  1.49);
\definecolor{drawColor}{RGB}{0,0,0}
\definecolor{fillColor}{RGB}{0,0,0}

\path[draw=drawColor,line width= 0.4pt,line join=round,line cap=round,fill=fillColor] (119.72,184.59) circle (  1.49);
\definecolor{drawColor}{RGB}{255,0,0}
\definecolor{fillColor}{RGB}{255,0,0}

\path[draw=drawColor,line width= 0.4pt,line join=round,line cap=round,fill=fillColor] (120.19, 68.90) circle (  1.49);
\definecolor{drawColor}{RGB}{0,0,0}
\definecolor{fillColor}{RGB}{0,0,0}

\path[draw=drawColor,line width= 0.4pt,line join=round,line cap=round,fill=fillColor] (120.23, 91.23) circle (  1.49);
\definecolor{drawColor}{RGB}{255,0,0}
\definecolor{fillColor}{RGB}{255,0,0}

\path[draw=drawColor,line width= 0.4pt,line join=round,line cap=round,fill=fillColor] (120.85, 54.71) circle (  1.49);
\definecolor{drawColor}{RGB}{0,0,0}
\definecolor{fillColor}{RGB}{0,0,0}

\path[draw=drawColor,line width= 0.4pt,line join=round,line cap=round,fill=fillColor] (120.86,153.44) circle (  1.49);
\definecolor{drawColor}{RGB}{255,0,0}
\definecolor{fillColor}{RGB}{255,0,0}

\path[draw=drawColor,line width= 0.4pt,line join=round,line cap=round,fill=fillColor] (121.57, 83.70) circle (  1.49);
\definecolor{drawColor}{RGB}{0,0,0}
\definecolor{fillColor}{RGB}{0,0,0}

\path[draw=drawColor,line width= 0.4pt,line join=round,line cap=round,fill=fillColor] (121.59,152.10) circle (  1.49);
\definecolor{drawColor}{RGB}{255,0,0}
\definecolor{fillColor}{RGB}{255,0,0}

\path[draw=drawColor,line width= 0.4pt,line join=round,line cap=round,fill=fillColor] (122.09, 68.64) circle (  1.49);
\definecolor{drawColor}{RGB}{0,0,0}
\definecolor{fillColor}{RGB}{0,0,0}

\path[draw=drawColor,line width= 0.4pt,line join=round,line cap=round,fill=fillColor] (122.11,154.37) circle (  1.49);
\definecolor{drawColor}{RGB}{255,0,0}
\definecolor{fillColor}{RGB}{255,0,0}

\path[draw=drawColor,line width= 0.4pt,line join=round,line cap=round,fill=fillColor] (122.60,100.44) circle (  1.49);
\definecolor{drawColor}{RGB}{0,0,0}
\definecolor{fillColor}{RGB}{0,0,0}

\path[draw=drawColor,line width= 0.4pt,line join=round,line cap=round,fill=fillColor] (122.62,153.21) circle (  1.49);
\definecolor{drawColor}{RGB}{255,0,0}
\definecolor{fillColor}{RGB}{255,0,0}

\path[draw=drawColor,line width= 0.4pt,line join=round,line cap=round,fill=fillColor] (123.25,112.16) circle (  1.49);
\definecolor{drawColor}{RGB}{0,0,0}
\definecolor{fillColor}{RGB}{0,0,0}

\path[draw=drawColor,line width= 0.4pt,line join=round,line cap=round,fill=fillColor] (123.27,153.52) circle (  1.49);
\definecolor{drawColor}{RGB}{255,0,0}
\definecolor{fillColor}{RGB}{255,0,0}

\path[draw=drawColor,line width= 0.4pt,line join=round,line cap=round,fill=fillColor] (123.84,119.55) circle (  1.49);
\definecolor{drawColor}{RGB}{0,0,0}
\definecolor{fillColor}{RGB}{0,0,0}

\path[draw=drawColor,line width= 0.4pt,line join=round,line cap=round,fill=fillColor] (123.86,183.10) circle (  1.49);
\definecolor{drawColor}{RGB}{255,0,0}
\definecolor{fillColor}{RGB}{255,0,0}

\path[draw=drawColor,line width= 0.4pt,line join=round,line cap=round,fill=fillColor] (124.34, 57.12) circle (  1.49);
\definecolor{drawColor}{RGB}{0,0,0}
\definecolor{fillColor}{RGB}{0,0,0}

\path[draw=drawColor,line width= 0.4pt,line join=round,line cap=round,fill=fillColor] (124.35,182.12) circle (  1.49);
\definecolor{drawColor}{RGB}{255,0,0}
\definecolor{fillColor}{RGB}{255,0,0}

\path[draw=drawColor,line width= 0.4pt,line join=round,line cap=round,fill=fillColor] (124.88,177.61) circle (  1.49);
\definecolor{drawColor}{RGB}{0,0,0}
\definecolor{fillColor}{RGB}{0,0,0}

\path[draw=drawColor,line width= 0.4pt,line join=round,line cap=round,fill=fillColor] (124.89,181.36) circle (  1.49);
\definecolor{drawColor}{RGB}{255,0,0}
\definecolor{fillColor}{RGB}{255,0,0}

\path[draw=drawColor,line width= 0.4pt,line join=round,line cap=round,fill=fillColor] (125.40,186.87) circle (  1.49);
\definecolor{drawColor}{RGB}{0,0,0}
\definecolor{fillColor}{RGB}{0,0,0}

\path[draw=drawColor,line width= 0.4pt,line join=round,line cap=round,fill=fillColor] (125.42,179.39) circle (  1.49);
\definecolor{drawColor}{RGB}{255,0,0}
\definecolor{fillColor}{RGB}{255,0,0}

\path[draw=drawColor,line width= 0.4pt,line join=round,line cap=round,fill=fillColor] (125.99,189.77) circle (  1.49);
\definecolor{drawColor}{RGB}{0,0,0}
\definecolor{fillColor}{RGB}{0,0,0}

\path[draw=drawColor,line width= 0.4pt,line join=round,line cap=round,fill=fillColor] (126.00,181.82) circle (  1.49);
\definecolor{drawColor}{RGB}{255,0,0}
\definecolor{fillColor}{RGB}{255,0,0}

\path[draw=drawColor,line width= 0.4pt,line join=round,line cap=round,fill=fillColor] (126.56,191.65) circle (  1.49);
\definecolor{drawColor}{RGB}{0,0,0}
\definecolor{fillColor}{RGB}{0,0,0}

\path[draw=drawColor,line width= 0.4pt,line join=round,line cap=round,fill=fillColor] (126.58,183.22) circle (  1.49);
\definecolor{drawColor}{RGB}{255,0,0}
\definecolor{fillColor}{RGB}{255,0,0}

\path[draw=drawColor,line width= 0.4pt,line join=round,line cap=round,fill=fillColor] (127.05,190.23) circle (  1.49);
\definecolor{drawColor}{RGB}{0,0,0}
\definecolor{fillColor}{RGB}{0,0,0}

\path[draw=drawColor,line width= 0.4pt,line join=round,line cap=round,fill=fillColor] (127.07,181.89) circle (  1.49);
\definecolor{drawColor}{RGB}{255,0,0}
\definecolor{fillColor}{RGB}{255,0,0}

\path[draw=drawColor,line width= 0.4pt,line join=round,line cap=round,fill=fillColor] (127.56,192.06) circle (  1.49);
\definecolor{drawColor}{RGB}{0,0,0}
\definecolor{fillColor}{RGB}{0,0,0}

\path[draw=drawColor,line width= 0.4pt,line join=round,line cap=round,fill=fillColor] (127.58,159.21) circle (  1.49);
\definecolor{drawColor}{RGB}{255,0,0}
\definecolor{fillColor}{RGB}{255,0,0}

\path[draw=drawColor,line width= 0.4pt,line join=round,line cap=round,fill=fillColor] (128.07,191.97) circle (  1.49);
\definecolor{drawColor}{RGB}{0,0,0}
\definecolor{fillColor}{RGB}{0,0,0}

\path[draw=drawColor,line width= 0.4pt,line join=round,line cap=round,fill=fillColor] (128.08,183.10) circle (  1.49);
\definecolor{drawColor}{RGB}{255,0,0}
\definecolor{fillColor}{RGB}{255,0,0}

\path[draw=drawColor,line width= 0.4pt,line join=round,line cap=round,fill=fillColor] (128.56,192.33) circle (  1.49);
\definecolor{drawColor}{RGB}{0,0,0}
\definecolor{fillColor}{RGB}{0,0,0}

\path[draw=drawColor,line width= 0.4pt,line join=round,line cap=round,fill=fillColor] (128.57,183.13) circle (  1.49);
\definecolor{drawColor}{RGB}{255,0,0}
\definecolor{fillColor}{RGB}{255,0,0}

\path[draw=drawColor,line width= 0.4pt,line join=round,line cap=round,fill=fillColor] (129.12,192.91) circle (  1.49);
\definecolor{drawColor}{RGB}{0,0,0}
\definecolor{fillColor}{RGB}{0,0,0}

\path[draw=drawColor,line width= 0.4pt,line join=round,line cap=round,fill=fillColor] (129.13, 72.00) circle (  1.49);
\definecolor{drawColor}{RGB}{255,0,0}
\definecolor{fillColor}{RGB}{255,0,0}

\path[draw=drawColor,line width= 0.4pt,line join=round,line cap=round,fill=fillColor] (129.67,192.92) circle (  1.49);
\definecolor{drawColor}{RGB}{0,0,0}
\definecolor{fillColor}{RGB}{0,0,0}

\path[draw=drawColor,line width= 0.4pt,line join=round,line cap=round,fill=fillColor] (129.69,183.44) circle (  1.49);
\definecolor{drawColor}{RGB}{255,0,0}
\definecolor{fillColor}{RGB}{255,0,0}

\path[draw=drawColor,line width= 0.4pt,line join=round,line cap=round,fill=fillColor] (130.18,193.87) circle (  1.49);
\definecolor{drawColor}{RGB}{0,0,0}
\definecolor{fillColor}{RGB}{0,0,0}

\path[draw=drawColor,line width= 0.4pt,line join=round,line cap=round,fill=fillColor] (130.20,183.76) circle (  1.49);
\definecolor{drawColor}{RGB}{255,0,0}
\definecolor{fillColor}{RGB}{255,0,0}

\path[draw=drawColor,line width= 0.4pt,line join=round,line cap=round,fill=fillColor] (130.65,194.26) circle (  1.49);
\definecolor{drawColor}{RGB}{0,0,0}
\definecolor{fillColor}{RGB}{0,0,0}

\path[draw=drawColor,line width= 0.4pt,line join=round,line cap=round,fill=fillColor] (130.67,154.22) circle (  1.49);
\definecolor{drawColor}{RGB}{255,0,0}
\definecolor{fillColor}{RGB}{255,0,0}

\path[draw=drawColor,line width= 0.4pt,line join=round,line cap=round,fill=fillColor] (131.14,192.09) circle (  1.49);
\definecolor{drawColor}{RGB}{0,0,0}
\definecolor{fillColor}{RGB}{0,0,0}

\path[draw=drawColor,line width= 0.4pt,line join=round,line cap=round,fill=fillColor] (131.16,182.95) circle (  1.49);
\definecolor{drawColor}{RGB}{255,0,0}
\definecolor{fillColor}{RGB}{255,0,0}

\path[draw=drawColor,line width= 0.4pt,line join=round,line cap=round,fill=fillColor] (131.62,192.01) circle (  1.49);
\definecolor{drawColor}{RGB}{0,0,0}
\definecolor{fillColor}{RGB}{0,0,0}

\path[draw=drawColor,line width= 0.4pt,line join=round,line cap=round,fill=fillColor] (131.64,182.90) circle (  1.49);
\definecolor{drawColor}{RGB}{255,0,0}
\definecolor{fillColor}{RGB}{255,0,0}

\path[draw=drawColor,line width= 0.4pt,line join=round,line cap=round,fill=fillColor] (132.19,117.94) circle (  1.49);
\definecolor{drawColor}{RGB}{0,0,0}
\definecolor{fillColor}{RGB}{0,0,0}

\path[draw=drawColor,line width= 0.4pt,line join=round,line cap=round,fill=fillColor] (132.21,174.32) circle (  1.49);
\definecolor{drawColor}{RGB}{255,0,0}
\definecolor{fillColor}{RGB}{255,0,0}

\path[draw=drawColor,line width= 0.4pt,line join=round,line cap=round,fill=fillColor] (132.75,148.12) circle (  1.49);
\definecolor{drawColor}{RGB}{0,0,0}
\definecolor{fillColor}{RGB}{0,0,0}

\path[draw=drawColor,line width= 0.4pt,line join=round,line cap=round,fill=fillColor] (132.77,183.48) circle (  1.49);
\definecolor{drawColor}{RGB}{255,0,0}
\definecolor{fillColor}{RGB}{255,0,0}

\path[draw=drawColor,line width= 0.4pt,line join=round,line cap=round,fill=fillColor] (133.27,158.61) circle (  1.49);
\definecolor{drawColor}{RGB}{0,0,0}
\definecolor{fillColor}{RGB}{0,0,0}

\path[draw=drawColor,line width= 0.4pt,line join=round,line cap=round,fill=fillColor] (133.29,183.07) circle (  1.49);
\definecolor{drawColor}{RGB}{255,0,0}
\definecolor{fillColor}{RGB}{255,0,0}

\path[draw=drawColor,line width= 0.4pt,line join=round,line cap=round,fill=fillColor] (133.80,182.47) circle (  1.49);
\definecolor{drawColor}{RGB}{0,0,0}
\definecolor{fillColor}{RGB}{0,0,0}

\path[draw=drawColor,line width= 0.4pt,line join=round,line cap=round,fill=fillColor] (133.81,182.78) circle (  1.49);
\definecolor{drawColor}{RGB}{255,0,0}
\definecolor{fillColor}{RGB}{255,0,0}

\path[draw=drawColor,line width= 0.4pt,line join=round,line cap=round,fill=fillColor] (134.34,190.27) circle (  1.49);
\definecolor{drawColor}{RGB}{0,0,0}
\definecolor{fillColor}{RGB}{0,0,0}

\path[draw=drawColor,line width= 0.4pt,line join=round,line cap=round,fill=fillColor] (134.35,182.89) circle (  1.49);
\definecolor{drawColor}{RGB}{255,0,0}
\definecolor{fillColor}{RGB}{255,0,0}

\path[draw=drawColor,line width= 0.4pt,line join=round,line cap=round,fill=fillColor] (134.84,173.65) circle (  1.49);
\definecolor{drawColor}{RGB}{0,0,0}
\definecolor{fillColor}{RGB}{0,0,0}

\path[draw=drawColor,line width= 0.4pt,line join=round,line cap=round,fill=fillColor] (134.86,182.79) circle (  1.49);
\definecolor{drawColor}{RGB}{255,0,0}
\definecolor{fillColor}{RGB}{255,0,0}

\path[draw=drawColor,line width= 0.4pt,line join=round,line cap=round,fill=fillColor] (135.35,193.90) circle (  1.49);
\definecolor{drawColor}{RGB}{0,0,0}
\definecolor{fillColor}{RGB}{0,0,0}

\path[draw=drawColor,line width= 0.4pt,line join=round,line cap=round,fill=fillColor] (135.37,183.54) circle (  1.49);
\definecolor{drawColor}{RGB}{255,0,0}
\definecolor{fillColor}{RGB}{255,0,0}

\path[draw=drawColor,line width= 0.4pt,line join=round,line cap=round,fill=fillColor] (135.91,137.71) circle (  1.49);
\definecolor{drawColor}{RGB}{0,0,0}
\definecolor{fillColor}{RGB}{0,0,0}

\path[draw=drawColor,line width= 0.4pt,line join=round,line cap=round,fill=fillColor] (135.92,179.83) circle (  1.49);
\definecolor{drawColor}{RGB}{255,0,0}
\definecolor{fillColor}{RGB}{255,0,0}

\path[draw=drawColor,line width= 0.4pt,line join=round,line cap=round,fill=fillColor] (136.43,124.60) circle (  1.49);
\definecolor{drawColor}{RGB}{0,0,0}
\definecolor{fillColor}{RGB}{0,0,0}

\path[draw=drawColor,line width= 0.4pt,line join=round,line cap=round,fill=fillColor] (136.45,182.53) circle (  1.49);
\definecolor{drawColor}{RGB}{255,0,0}
\definecolor{fillColor}{RGB}{255,0,0}

\path[draw=drawColor,line width= 0.4pt,line join=round,line cap=round,fill=fillColor] (136.92,176.55) circle (  1.49);
\definecolor{drawColor}{RGB}{0,0,0}
\definecolor{fillColor}{RGB}{0,0,0}

\path[draw=drawColor,line width= 0.4pt,line join=round,line cap=round,fill=fillColor] (136.94,173.53) circle (  1.49);
\definecolor{drawColor}{RGB}{255,0,0}
\definecolor{fillColor}{RGB}{255,0,0}

\path[draw=drawColor,line width= 0.4pt,line join=round,line cap=round,fill=fillColor] (137.48,180.81) circle (  1.49);
\definecolor{drawColor}{RGB}{0,0,0}
\definecolor{fillColor}{RGB}{0,0,0}

\path[draw=drawColor,line width= 0.4pt,line join=round,line cap=round,fill=fillColor] (137.51, 68.92) circle (  1.49);
\definecolor{drawColor}{RGB}{255,0,0}
\definecolor{fillColor}{RGB}{255,0,0}

\path[draw=drawColor,line width= 0.4pt,line join=round,line cap=round,fill=fillColor] (138.02,191.44) circle (  1.49);
\definecolor{drawColor}{RGB}{0,0,0}
\definecolor{fillColor}{RGB}{0,0,0}

\path[draw=drawColor,line width= 0.4pt,line join=round,line cap=round,fill=fillColor] (138.04,183.28) circle (  1.49);
\definecolor{drawColor}{RGB}{255,0,0}
\definecolor{fillColor}{RGB}{255,0,0}

\path[draw=drawColor,line width= 0.4pt,line join=round,line cap=round,fill=fillColor] (138.49,187.32) circle (  1.49);
\definecolor{drawColor}{RGB}{0,0,0}
\definecolor{fillColor}{RGB}{0,0,0}

\path[draw=drawColor,line width= 0.4pt,line join=round,line cap=round,fill=fillColor] (138.51,183.31) circle (  1.49);
\definecolor{drawColor}{RGB}{255,0,0}
\definecolor{fillColor}{RGB}{255,0,0}

\path[draw=drawColor,line width= 0.4pt,line join=round,line cap=round,fill=fillColor] (138.99, 67.63) circle (  1.49);
\definecolor{drawColor}{RGB}{0,0,0}
\definecolor{fillColor}{RGB}{0,0,0}

\path[draw=drawColor,line width= 0.4pt,line join=round,line cap=round,fill=fillColor] (139.00,182.37) circle (  1.49);
\definecolor{drawColor}{RGB}{255,0,0}
\definecolor{fillColor}{RGB}{255,0,0}

\path[draw=drawColor,line width= 0.4pt,line join=round,line cap=round,fill=fillColor] (139.49,193.35) circle (  1.49);
\definecolor{drawColor}{RGB}{0,0,0}
\definecolor{fillColor}{RGB}{0,0,0}

\path[draw=drawColor,line width= 0.4pt,line join=round,line cap=round,fill=fillColor] (139.51,182.80) circle (  1.49);
\definecolor{drawColor}{RGB}{255,0,0}
\definecolor{fillColor}{RGB}{255,0,0}

\path[draw=drawColor,line width= 0.4pt,line join=round,line cap=round,fill=fillColor] (139.97,193.64) circle (  1.49);
\definecolor{drawColor}{RGB}{0,0,0}
\definecolor{fillColor}{RGB}{0,0,0}

\path[draw=drawColor,line width= 0.4pt,line join=round,line cap=round,fill=fillColor] (139.98,133.90) circle (  1.49);
\definecolor{drawColor}{RGB}{255,0,0}
\definecolor{fillColor}{RGB}{255,0,0}

\path[draw=drawColor,line width= 0.4pt,line join=round,line cap=round,fill=fillColor] (140.49,193.63) circle (  1.49);
\definecolor{drawColor}{RGB}{0,0,0}
\definecolor{fillColor}{RGB}{0,0,0}

\path[draw=drawColor,line width= 0.4pt,line join=round,line cap=round,fill=fillColor] (140.51,182.52) circle (  1.49);
\definecolor{drawColor}{RGB}{255,0,0}
\definecolor{fillColor}{RGB}{255,0,0}

\path[draw=drawColor,line width= 0.4pt,line join=round,line cap=round,fill=fillColor] (140.98,193.46) circle (  1.49);
\definecolor{drawColor}{RGB}{0,0,0}
\definecolor{fillColor}{RGB}{0,0,0}

\path[draw=drawColor,line width= 0.4pt,line join=round,line cap=round,fill=fillColor] (141.00,182.73) circle (  1.49);
\definecolor{drawColor}{RGB}{255,0,0}
\definecolor{fillColor}{RGB}{255,0,0}

\path[draw=drawColor,line width= 0.4pt,line join=round,line cap=round,fill=fillColor] (141.51,192.86) circle (  1.49);
\definecolor{drawColor}{RGB}{0,0,0}
\definecolor{fillColor}{RGB}{0,0,0}

\path[draw=drawColor,line width= 0.4pt,line join=round,line cap=round,fill=fillColor] (141.52,182.63) circle (  1.49);
\definecolor{drawColor}{RGB}{255,0,0}
\definecolor{fillColor}{RGB}{255,0,0}

\path[draw=drawColor,line width= 0.4pt,line join=round,line cap=round,fill=fillColor] (142.11,192.47) circle (  1.49);
\definecolor{drawColor}{RGB}{0,0,0}
\definecolor{fillColor}{RGB}{0,0,0}

\path[draw=drawColor,line width= 0.4pt,line join=round,line cap=round,fill=fillColor] (142.13,182.16) circle (  1.49);
\definecolor{drawColor}{RGB}{255,0,0}
\definecolor{fillColor}{RGB}{255,0,0}

\path[draw=drawColor,line width= 0.4pt,line join=round,line cap=round,fill=fillColor] (142.60,192.59) circle (  1.49);
\definecolor{drawColor}{RGB}{0,0,0}
\definecolor{fillColor}{RGB}{0,0,0}

\path[draw=drawColor,line width= 0.4pt,line join=round,line cap=round,fill=fillColor] (142.62,181.99) circle (  1.49);
\definecolor{drawColor}{RGB}{255,0,0}
\definecolor{fillColor}{RGB}{255,0,0}

\path[draw=drawColor,line width= 0.4pt,line join=round,line cap=round,fill=fillColor] (143.09,192.22) circle (  1.49);
\definecolor{drawColor}{RGB}{0,0,0}
\definecolor{fillColor}{RGB}{0,0,0}

\path[draw=drawColor,line width= 0.4pt,line join=round,line cap=round,fill=fillColor] (143.11,182.01) circle (  1.49);
\definecolor{drawColor}{RGB}{255,0,0}
\definecolor{fillColor}{RGB}{255,0,0}

\path[draw=drawColor,line width= 0.4pt,line join=round,line cap=round,fill=fillColor] (143.57,192.19) circle (  1.49);
\definecolor{drawColor}{RGB}{0,0,0}
\definecolor{fillColor}{RGB}{0,0,0}

\path[draw=drawColor,line width= 0.4pt,line join=round,line cap=round,fill=fillColor] (143.59,182.04) circle (  1.49);
\definecolor{drawColor}{RGB}{255,0,0}
\definecolor{fillColor}{RGB}{255,0,0}

\path[draw=drawColor,line width= 0.4pt,line join=round,line cap=round,fill=fillColor] (144.09,104.72) circle (  1.49);
\definecolor{drawColor}{RGB}{0,0,0}
\definecolor{fillColor}{RGB}{0,0,0}

\path[draw=drawColor,line width= 0.4pt,line join=round,line cap=round,fill=fillColor] (144.11,181.88) circle (  1.49);
\definecolor{drawColor}{RGB}{255,0,0}
\definecolor{fillColor}{RGB}{255,0,0}

\path[draw=drawColor,line width= 0.4pt,line join=round,line cap=round,fill=fillColor] (144.60,192.25) circle (  1.49);
\definecolor{drawColor}{RGB}{0,0,0}
\definecolor{fillColor}{RGB}{0,0,0}

\path[draw=drawColor,line width= 0.4pt,line join=round,line cap=round,fill=fillColor] (144.62,182.24) circle (  1.49);
\definecolor{drawColor}{RGB}{255,0,0}
\definecolor{fillColor}{RGB}{255,0,0}

\path[draw=drawColor,line width= 0.4pt,line join=round,line cap=round,fill=fillColor] (145.11,192.39) circle (  1.49);
\definecolor{drawColor}{RGB}{0,0,0}
\definecolor{fillColor}{RGB}{0,0,0}

\path[draw=drawColor,line width= 0.4pt,line join=round,line cap=round,fill=fillColor] (145.12,182.05) circle (  1.49);
\definecolor{drawColor}{RGB}{255,0,0}
\definecolor{fillColor}{RGB}{255,0,0}

\path[draw=drawColor,line width= 0.4pt,line join=round,line cap=round,fill=fillColor] (145.60,192.90) circle (  1.49);
\definecolor{drawColor}{RGB}{0,0,0}
\definecolor{fillColor}{RGB}{0,0,0}

\path[draw=drawColor,line width= 0.4pt,line join=round,line cap=round,fill=fillColor] (145.62,176.40) circle (  1.49);
\definecolor{drawColor}{RGB}{255,0,0}
\definecolor{fillColor}{RGB}{255,0,0}

\path[draw=drawColor,line width= 0.4pt,line join=round,line cap=round,fill=fillColor] (146.11,170.19) circle (  1.49);
\definecolor{drawColor}{RGB}{0,0,0}
\definecolor{fillColor}{RGB}{0,0,0}

\path[draw=drawColor,line width= 0.4pt,line join=round,line cap=round,fill=fillColor] (146.12,113.30) circle (  1.49);
\definecolor{drawColor}{RGB}{255,0,0}
\definecolor{fillColor}{RGB}{255,0,0}

\path[draw=drawColor,line width= 0.4pt,line join=round,line cap=round,fill=fillColor] (146.60,192.73) circle (  1.49);
\definecolor{drawColor}{RGB}{0,0,0}
\definecolor{fillColor}{RGB}{0,0,0}

\path[draw=drawColor,line width= 0.4pt,line join=round,line cap=round,fill=fillColor] (146.61,182.22) circle (  1.49);
\definecolor{drawColor}{RGB}{255,0,0}
\definecolor{fillColor}{RGB}{255,0,0}

\path[draw=drawColor,line width= 0.4pt,line join=round,line cap=round,fill=fillColor] (147.11,191.95) circle (  1.49);
\definecolor{drawColor}{RGB}{0,0,0}
\definecolor{fillColor}{RGB}{0,0,0}

\path[draw=drawColor,line width= 0.4pt,line join=round,line cap=round,fill=fillColor] (147.12,165.93) circle (  1.49);
\definecolor{drawColor}{RGB}{255,0,0}
\definecolor{fillColor}{RGB}{255,0,0}

\path[draw=drawColor,line width= 0.4pt,line join=round,line cap=round,fill=fillColor] (147.61,192.11) circle (  1.49);
\definecolor{drawColor}{RGB}{0,0,0}
\definecolor{fillColor}{RGB}{0,0,0}

\path[draw=drawColor,line width= 0.4pt,line join=round,line cap=round,fill=fillColor] (147.63,182.13) circle (  1.49);
\definecolor{drawColor}{RGB}{255,0,0}
\definecolor{fillColor}{RGB}{255,0,0}

\path[draw=drawColor,line width= 0.4pt,line join=round,line cap=round,fill=fillColor] (148.09,191.48) circle (  1.49);
\definecolor{drawColor}{RGB}{0,0,0}
\definecolor{fillColor}{RGB}{0,0,0}

\path[draw=drawColor,line width= 0.4pt,line join=round,line cap=round,fill=fillColor] (148.10,180.16) circle (  1.49);
\definecolor{drawColor}{RGB}{255,0,0}
\definecolor{fillColor}{RGB}{255,0,0}

\path[draw=drawColor,line width= 0.4pt,line join=round,line cap=round,fill=fillColor] (148.61,175.52) circle (  1.49);
\definecolor{drawColor}{RGB}{0,0,0}
\definecolor{fillColor}{RGB}{0,0,0}

\path[draw=drawColor,line width= 0.4pt,line join=round,line cap=round,fill=fillColor] (148.63,180.77) circle (  1.49);
\definecolor{drawColor}{RGB}{255,0,0}
\definecolor{fillColor}{RGB}{255,0,0}

\path[draw=drawColor,line width= 0.4pt,line join=round,line cap=round,fill=fillColor] (149.10,180.93) circle (  1.49);
\definecolor{drawColor}{RGB}{0,0,0}
\definecolor{fillColor}{RGB}{0,0,0}

\path[draw=drawColor,line width= 0.4pt,line join=round,line cap=round,fill=fillColor] (149.12,181.00) circle (  1.49);
\definecolor{drawColor}{RGB}{255,0,0}
\definecolor{fillColor}{RGB}{255,0,0}

\path[draw=drawColor,line width= 0.4pt,line join=round,line cap=round,fill=fillColor] (149.69,193.18) circle (  1.49);
\definecolor{drawColor}{RGB}{0,0,0}
\definecolor{fillColor}{RGB}{0,0,0}

\path[draw=drawColor,line width= 0.4pt,line join=round,line cap=round,fill=fillColor] (149.71,183.10) circle (  1.49);
\definecolor{drawColor}{RGB}{255,0,0}
\definecolor{fillColor}{RGB}{255,0,0}

\path[draw=drawColor,line width= 0.4pt,line join=round,line cap=round,fill=fillColor] (150.22,193.94) circle (  1.49);
\definecolor{drawColor}{RGB}{0,0,0}
\definecolor{fillColor}{RGB}{0,0,0}

\path[draw=drawColor,line width= 0.4pt,line join=round,line cap=round,fill=fillColor] (150.23,183.61) circle (  1.49);
\definecolor{drawColor}{RGB}{255,0,0}
\definecolor{fillColor}{RGB}{255,0,0}

\path[draw=drawColor,line width= 0.4pt,line join=round,line cap=round,fill=fillColor] (150.67,193.63) circle (  1.49);
\definecolor{drawColor}{RGB}{0,0,0}
\definecolor{fillColor}{RGB}{0,0,0}

\path[draw=drawColor,line width= 0.4pt,line join=round,line cap=round,fill=fillColor] (150.69,182.86) circle (  1.49);
\definecolor{drawColor}{RGB}{255,0,0}
\definecolor{fillColor}{RGB}{255,0,0}

\path[draw=drawColor,line width= 0.4pt,line join=round,line cap=round,fill=fillColor] (151.15,192.72) circle (  1.49);
\definecolor{drawColor}{RGB}{0,0,0}
\definecolor{fillColor}{RGB}{0,0,0}

\path[draw=drawColor,line width= 0.4pt,line join=round,line cap=round,fill=fillColor] (151.16,183.03) circle (  1.49);
\definecolor{drawColor}{RGB}{255,0,0}
\definecolor{fillColor}{RGB}{255,0,0}

\path[draw=drawColor,line width= 0.4pt,line join=round,line cap=round,fill=fillColor] (151.69,191.93) circle (  1.49);
\definecolor{drawColor}{RGB}{0,0,0}
\definecolor{fillColor}{RGB}{0,0,0}

\path[draw=drawColor,line width= 0.4pt,line join=round,line cap=round,fill=fillColor] (151.71,182.17) circle (  1.49);
\definecolor{drawColor}{RGB}{255,0,0}
\definecolor{fillColor}{RGB}{255,0,0}

\path[draw=drawColor,line width= 0.4pt,line join=round,line cap=round,fill=fillColor] (152.21,191.98) circle (  1.49);
\definecolor{drawColor}{RGB}{0,0,0}
\definecolor{fillColor}{RGB}{0,0,0}

\path[draw=drawColor,line width= 0.4pt,line join=round,line cap=round,fill=fillColor] (152.23,180.71) circle (  1.49);
\definecolor{drawColor}{RGB}{255,0,0}
\definecolor{fillColor}{RGB}{255,0,0}

\path[draw=drawColor,line width= 0.4pt,line join=round,line cap=round,fill=fillColor] (152.70,192.45) circle (  1.49);
\definecolor{drawColor}{RGB}{0,0,0}
\definecolor{fillColor}{RGB}{0,0,0}

\path[draw=drawColor,line width= 0.4pt,line join=round,line cap=round,fill=fillColor] (152.74,182.40) circle (  1.49);
\definecolor{drawColor}{RGB}{255,0,0}
\definecolor{fillColor}{RGB}{255,0,0}

\path[draw=drawColor,line width= 0.4pt,line join=round,line cap=round,fill=fillColor] (153.19,190.57) circle (  1.49);
\definecolor{drawColor}{RGB}{0,0,0}
\definecolor{fillColor}{RGB}{0,0,0}

\path[draw=drawColor,line width= 0.4pt,line join=round,line cap=round,fill=fillColor] (153.21,178.90) circle (  1.49);
\definecolor{drawColor}{RGB}{255,0,0}
\definecolor{fillColor}{RGB}{255,0,0}

\path[draw=drawColor,line width= 0.4pt,line join=round,line cap=round,fill=fillColor] (153.70,190.97) circle (  1.49);
\definecolor{drawColor}{RGB}{0,0,0}
\definecolor{fillColor}{RGB}{0,0,0}

\path[draw=drawColor,line width= 0.4pt,line join=round,line cap=round,fill=fillColor] (153.72,181.38) circle (  1.49);
\definecolor{drawColor}{RGB}{255,0,0}
\definecolor{fillColor}{RGB}{255,0,0}

\path[draw=drawColor,line width= 0.4pt,line join=round,line cap=round,fill=fillColor] (154.21,191.26) circle (  1.49);
\definecolor{drawColor}{RGB}{0,0,0}
\definecolor{fillColor}{RGB}{0,0,0}

\path[draw=drawColor,line width= 0.4pt,line join=round,line cap=round,fill=fillColor] (154.23,181.37) circle (  1.49);
\definecolor{drawColor}{RGB}{255,0,0}
\definecolor{fillColor}{RGB}{255,0,0}

\path[draw=drawColor,line width= 0.4pt,line join=round,line cap=round,fill=fillColor] (154.73,191.01) circle (  1.49);
\definecolor{drawColor}{RGB}{0,0,0}
\definecolor{fillColor}{RGB}{0,0,0}

\path[draw=drawColor,line width= 0.4pt,line join=round,line cap=round,fill=fillColor] (154.75,181.40) circle (  1.49);
\definecolor{drawColor}{RGB}{255,0,0}
\definecolor{fillColor}{RGB}{255,0,0}

\path[draw=drawColor,line width= 0.4pt,line join=round,line cap=round,fill=fillColor] (155.24,191.03) circle (  1.49);
\definecolor{drawColor}{RGB}{0,0,0}
\definecolor{fillColor}{RGB}{0,0,0}

\path[draw=drawColor,line width= 0.4pt,line join=round,line cap=round,fill=fillColor] (155.26,181.55) circle (  1.49);
\definecolor{drawColor}{RGB}{255,0,0}
\definecolor{fillColor}{RGB}{255,0,0}

\path[draw=drawColor,line width= 0.4pt,line join=round,line cap=round,fill=fillColor] (155.75,190.26) circle (  1.49);
\definecolor{drawColor}{RGB}{0,0,0}
\definecolor{fillColor}{RGB}{0,0,0}

\path[draw=drawColor,line width= 0.4pt,line join=round,line cap=round,fill=fillColor] (155.76,180.76) circle (  1.49);
\definecolor{drawColor}{RGB}{255,0,0}
\definecolor{fillColor}{RGB}{255,0,0}

\path[draw=drawColor,line width= 0.4pt,line join=round,line cap=round,fill=fillColor] (156.24,190.62) circle (  1.49);
\definecolor{drawColor}{RGB}{0,0,0}
\definecolor{fillColor}{RGB}{0,0,0}

\path[draw=drawColor,line width= 0.4pt,line join=round,line cap=round,fill=fillColor] (156.26,180.37) circle (  1.49);
\definecolor{drawColor}{RGB}{255,0,0}
\definecolor{fillColor}{RGB}{255,0,0}

\path[draw=drawColor,line width= 0.4pt,line join=round,line cap=round,fill=fillColor] (156.75,190.62) circle (  1.49);
\definecolor{drawColor}{RGB}{0,0,0}
\definecolor{fillColor}{RGB}{0,0,0}

\path[draw=drawColor,line width= 0.4pt,line join=round,line cap=round,fill=fillColor] (156.76,179.33) circle (  1.49);
\definecolor{drawColor}{RGB}{255,0,0}
\definecolor{fillColor}{RGB}{255,0,0}

\path[draw=drawColor,line width= 0.4pt,line join=round,line cap=round,fill=fillColor] (157.25,190.89) circle (  1.49);
\definecolor{drawColor}{RGB}{0,0,0}
\definecolor{fillColor}{RGB}{0,0,0}

\path[draw=drawColor,line width= 0.4pt,line join=round,line cap=round,fill=fillColor] (157.27,180.91) circle (  1.49);
\definecolor{drawColor}{RGB}{255,0,0}
\definecolor{fillColor}{RGB}{255,0,0}

\path[draw=drawColor,line width= 0.4pt,line join=round,line cap=round,fill=fillColor] (157.76,191.34) circle (  1.49);
\definecolor{drawColor}{RGB}{0,0,0}
\definecolor{fillColor}{RGB}{0,0,0}

\path[draw=drawColor,line width= 0.4pt,line join=round,line cap=round,fill=fillColor] (157.78,128.35) circle (  1.49);
\definecolor{drawColor}{RGB}{255,0,0}
\definecolor{fillColor}{RGB}{255,0,0}

\path[draw=drawColor,line width= 0.4pt,line join=round,line cap=round,fill=fillColor] (158.25,191.68) circle (  1.49);
\definecolor{drawColor}{RGB}{0,0,0}
\definecolor{fillColor}{RGB}{0,0,0}

\path[draw=drawColor,line width= 0.4pt,line join=round,line cap=round,fill=fillColor] (158.29,181.27) circle (  1.49);
\definecolor{drawColor}{RGB}{255,0,0}
\definecolor{fillColor}{RGB}{255,0,0}

\path[draw=drawColor,line width= 0.4pt,line join=round,line cap=round,fill=fillColor] (158.78,191.56) circle (  1.49);
\definecolor{drawColor}{RGB}{0,0,0}
\definecolor{fillColor}{RGB}{0,0,0}

\path[draw=drawColor,line width= 0.4pt,line join=round,line cap=round,fill=fillColor] (158.81, 79.99) circle (  1.49);
\definecolor{drawColor}{RGB}{255,0,0}
\definecolor{fillColor}{RGB}{255,0,0}

\path[draw=drawColor,line width= 0.4pt,line join=round,line cap=round,fill=fillColor] (159.33,191.24) circle (  1.49);
\definecolor{drawColor}{RGB}{0,0,0}
\definecolor{fillColor}{RGB}{0,0,0}

\path[draw=drawColor,line width= 0.4pt,line join=round,line cap=round,fill=fillColor] (159.35,181.06) circle (  1.49);
\definecolor{drawColor}{RGB}{255,0,0}
\definecolor{fillColor}{RGB}{255,0,0}

\path[draw=drawColor,line width= 0.4pt,line join=round,line cap=round,fill=fillColor] (159.86,191.17) circle (  1.49);
\definecolor{drawColor}{RGB}{0,0,0}
\definecolor{fillColor}{RGB}{0,0,0}

\path[draw=drawColor,line width= 0.4pt,line join=round,line cap=round,fill=fillColor] (159.87,181.14) circle (  1.49);
\definecolor{drawColor}{RGB}{255,0,0}
\definecolor{fillColor}{RGB}{255,0,0}

\path[draw=drawColor,line width= 0.4pt,line join=round,line cap=round,fill=fillColor] (160.38,191.90) circle (  1.49);
\definecolor{drawColor}{RGB}{0,0,0}
\definecolor{fillColor}{RGB}{0,0,0}

\path[draw=drawColor,line width= 0.4pt,line join=round,line cap=round,fill=fillColor] (160.40,181.35) circle (  1.49);
\definecolor{drawColor}{RGB}{255,0,0}
\definecolor{fillColor}{RGB}{255,0,0}

\path[draw=drawColor,line width= 0.4pt,line join=round,line cap=round,fill=fillColor] (160.86,191.78) circle (  1.49);
\definecolor{drawColor}{RGB}{0,0,0}
\definecolor{fillColor}{RGB}{0,0,0}

\path[draw=drawColor,line width= 0.4pt,line join=round,line cap=round,fill=fillColor] (160.87,181.03) circle (  1.49);
\definecolor{drawColor}{RGB}{255,0,0}
\definecolor{fillColor}{RGB}{255,0,0}

\path[draw=drawColor,line width= 0.4pt,line join=round,line cap=round,fill=fillColor] (161.40,191.44) circle (  1.49);
\definecolor{drawColor}{RGB}{0,0,0}
\definecolor{fillColor}{RGB}{0,0,0}

\path[draw=drawColor,line width= 0.4pt,line join=round,line cap=round,fill=fillColor] (161.41,181.10) circle (  1.49);
\definecolor{drawColor}{RGB}{255,0,0}
\definecolor{fillColor}{RGB}{255,0,0}

\path[draw=drawColor,line width= 0.4pt,line join=round,line cap=round,fill=fillColor] (161.89,191.33) circle (  1.49);
\definecolor{drawColor}{RGB}{0,0,0}
\definecolor{fillColor}{RGB}{0,0,0}

\path[draw=drawColor,line width= 0.4pt,line join=round,line cap=round,fill=fillColor] (161.90,180.95) circle (  1.49);
\definecolor{drawColor}{RGB}{255,0,0}
\definecolor{fillColor}{RGB}{255,0,0}

\path[draw=drawColor,line width= 0.4pt,line join=round,line cap=round,fill=fillColor] (162.41,191.77) circle (  1.49);
\definecolor{drawColor}{RGB}{0,0,0}
\definecolor{fillColor}{RGB}{0,0,0}

\path[draw=drawColor,line width= 0.4pt,line join=round,line cap=round,fill=fillColor] (162.44,181.01) circle (  1.49);
\definecolor{drawColor}{RGB}{255,0,0}
\definecolor{fillColor}{RGB}{255,0,0}

\path[draw=drawColor,line width= 0.4pt,line join=round,line cap=round,fill=fillColor] (163.11,191.28) circle (  1.49);
\definecolor{drawColor}{RGB}{0,0,0}
\definecolor{fillColor}{RGB}{0,0,0}

\path[draw=drawColor,line width= 0.4pt,line join=round,line cap=round,fill=fillColor] (163.13,177.07) circle (  1.49);
\definecolor{drawColor}{RGB}{255,0,0}
\definecolor{fillColor}{RGB}{255,0,0}

\path[draw=drawColor,line width= 0.4pt,line join=round,line cap=round,fill=fillColor] (163.59,190.76) circle (  1.49);
\definecolor{drawColor}{RGB}{0,0,0}
\definecolor{fillColor}{RGB}{0,0,0}

\path[draw=drawColor,line width= 0.4pt,line join=round,line cap=round,fill=fillColor] (163.61,180.29) circle (  1.49);
\definecolor{drawColor}{RGB}{255,0,0}
\definecolor{fillColor}{RGB}{255,0,0}

\path[draw=drawColor,line width= 0.4pt,line join=round,line cap=round,fill=fillColor] (164.10,190.08) circle (  1.49);
\definecolor{drawColor}{RGB}{0,0,0}
\definecolor{fillColor}{RGB}{0,0,0}

\path[draw=drawColor,line width= 0.4pt,line join=round,line cap=round,fill=fillColor] (164.11,179.58) circle (  1.49);
\definecolor{drawColor}{RGB}{255,0,0}
\definecolor{fillColor}{RGB}{255,0,0}

\path[draw=drawColor,line width= 0.4pt,line join=round,line cap=round,fill=fillColor] (164.57,189.55) circle (  1.49);
\definecolor{drawColor}{RGB}{0,0,0}
\definecolor{fillColor}{RGB}{0,0,0}

\path[draw=drawColor,line width= 0.4pt,line join=round,line cap=round,fill=fillColor] (164.59,179.38) circle (  1.49);
\definecolor{drawColor}{RGB}{255,0,0}
\definecolor{fillColor}{RGB}{255,0,0}

\path[draw=drawColor,line width= 0.4pt,line join=round,line cap=round,fill=fillColor] (165.10,189.74) circle (  1.49);
\definecolor{drawColor}{RGB}{0,0,0}
\definecolor{fillColor}{RGB}{0,0,0}

\path[draw=drawColor,line width= 0.4pt,line join=round,line cap=round,fill=fillColor] (165.11,177.05) circle (  1.49);
\definecolor{drawColor}{RGB}{255,0,0}
\definecolor{fillColor}{RGB}{255,0,0}

\path[draw=drawColor,line width= 0.4pt,line join=round,line cap=round,fill=fillColor] (165.59,190.50) circle (  1.49);
\definecolor{drawColor}{RGB}{0,0,0}
\definecolor{fillColor}{RGB}{0,0,0}

\path[draw=drawColor,line width= 0.4pt,line join=round,line cap=round,fill=fillColor] (165.60,179.52) circle (  1.49);
\definecolor{drawColor}{RGB}{255,0,0}
\definecolor{fillColor}{RGB}{255,0,0}

\path[draw=drawColor,line width= 0.4pt,line join=round,line cap=round,fill=fillColor] (166.14,190.13) circle (  1.49);
\definecolor{drawColor}{RGB}{0,0,0}
\definecolor{fillColor}{RGB}{0,0,0}

\path[draw=drawColor,line width= 0.4pt,line join=round,line cap=round,fill=fillColor] (166.16,180.23) circle (  1.49);
\definecolor{drawColor}{RGB}{255,0,0}
\definecolor{fillColor}{RGB}{255,0,0}

\path[draw=drawColor,line width= 0.4pt,line join=round,line cap=round,fill=fillColor] (166.65,190.30) circle (  1.49);
\definecolor{drawColor}{RGB}{0,0,0}
\definecolor{fillColor}{RGB}{0,0,0}

\path[draw=drawColor,line width= 0.4pt,line join=round,line cap=round,fill=fillColor] (166.67,180.37) circle (  1.49);
\definecolor{drawColor}{RGB}{255,0,0}
\definecolor{fillColor}{RGB}{255,0,0}

\path[draw=drawColor,line width= 0.4pt,line join=round,line cap=round,fill=fillColor] (167.16,190.11) circle (  1.49);
\definecolor{drawColor}{RGB}{0,0,0}
\definecolor{fillColor}{RGB}{0,0,0}

\path[draw=drawColor,line width= 0.4pt,line join=round,line cap=round,fill=fillColor] (167.17,179.85) circle (  1.49);
\definecolor{drawColor}{RGB}{255,0,0}
\definecolor{fillColor}{RGB}{255,0,0}

\path[draw=drawColor,line width= 0.4pt,line join=round,line cap=round,fill=fillColor] (167.67,190.14) circle (  1.49);
\definecolor{drawColor}{RGB}{0,0,0}
\definecolor{fillColor}{RGB}{0,0,0}

\path[draw=drawColor,line width= 0.4pt,line join=round,line cap=round,fill=fillColor] (167.68,179.98) circle (  1.49);
\definecolor{drawColor}{RGB}{255,0,0}
\definecolor{fillColor}{RGB}{255,0,0}

\path[draw=drawColor,line width= 0.4pt,line join=round,line cap=round,fill=fillColor] (168.16,190.30) circle (  1.49);
\definecolor{drawColor}{RGB}{0,0,0}
\definecolor{fillColor}{RGB}{0,0,0}

\path[draw=drawColor,line width= 0.4pt,line join=round,line cap=round,fill=fillColor] (168.17,179.63) circle (  1.49);
\definecolor{drawColor}{RGB}{255,0,0}
\definecolor{fillColor}{RGB}{255,0,0}

\path[draw=drawColor,line width= 0.4pt,line join=round,line cap=round,fill=fillColor] (168.81,190.26) circle (  1.49);
\definecolor{drawColor}{RGB}{0,0,0}
\definecolor{fillColor}{RGB}{0,0,0}

\path[draw=drawColor,line width= 0.4pt,line join=round,line cap=round,fill=fillColor] (168.83,176.49) circle (  1.49);
\definecolor{drawColor}{RGB}{255,0,0}
\definecolor{fillColor}{RGB}{255,0,0}

\path[draw=drawColor,line width= 0.4pt,line join=round,line cap=round,fill=fillColor] (169.30,190.59) circle (  1.49);
\definecolor{drawColor}{RGB}{0,0,0}
\definecolor{fillColor}{RGB}{0,0,0}

\path[draw=drawColor,line width= 0.4pt,line join=round,line cap=round,fill=fillColor] (169.32,179.65) circle (  1.49);
\definecolor{drawColor}{RGB}{255,0,0}
\definecolor{fillColor}{RGB}{255,0,0}

\path[draw=drawColor,line width= 0.4pt,line join=round,line cap=round,fill=fillColor] (169.79,190.26) circle (  1.49);
\definecolor{drawColor}{RGB}{0,0,0}
\definecolor{fillColor}{RGB}{0,0,0}

\path[draw=drawColor,line width= 0.4pt,line join=round,line cap=round,fill=fillColor] (169.81,179.65) circle (  1.49);
\definecolor{drawColor}{RGB}{255,0,0}
\definecolor{fillColor}{RGB}{255,0,0}

\path[draw=drawColor,line width= 0.4pt,line join=round,line cap=round,fill=fillColor] (170.35,190.09) circle (  1.49);
\definecolor{drawColor}{RGB}{0,0,0}
\definecolor{fillColor}{RGB}{0,0,0}

\path[draw=drawColor,line width= 0.4pt,line join=round,line cap=round,fill=fillColor] (170.37,179.82) circle (  1.49);
\definecolor{drawColor}{RGB}{255,0,0}
\definecolor{fillColor}{RGB}{255,0,0}

\path[draw=drawColor,line width= 0.4pt,line join=round,line cap=round,fill=fillColor] (170.86,190.22) circle (  1.49);
\definecolor{drawColor}{RGB}{0,0,0}
\definecolor{fillColor}{RGB}{0,0,0}

\path[draw=drawColor,line width= 0.4pt,line join=round,line cap=round,fill=fillColor] (170.87,179.90) circle (  1.49);
\definecolor{drawColor}{RGB}{255,0,0}
\definecolor{fillColor}{RGB}{255,0,0}

\path[draw=drawColor,line width= 0.4pt,line join=round,line cap=round,fill=fillColor] (171.40,190.69) circle (  1.49);
\definecolor{drawColor}{RGB}{0,0,0}
\definecolor{fillColor}{RGB}{0,0,0}

\path[draw=drawColor,line width= 0.4pt,line join=round,line cap=round,fill=fillColor] (171.41,180.00) circle (  1.49);
\definecolor{drawColor}{RGB}{255,0,0}
\definecolor{fillColor}{RGB}{255,0,0}

\path[draw=drawColor,line width= 0.4pt,line join=round,line cap=round,fill=fillColor] (171.89,190.47) circle (  1.49);
\definecolor{drawColor}{RGB}{0,0,0}
\definecolor{fillColor}{RGB}{0,0,0}

\path[draw=drawColor,line width= 0.4pt,line join=round,line cap=round,fill=fillColor] (171.91,179.62) circle (  1.49);
\definecolor{drawColor}{RGB}{255,0,0}
\definecolor{fillColor}{RGB}{255,0,0}

\path[draw=drawColor,line width= 0.4pt,line join=round,line cap=round,fill=fillColor] (172.36,189.96) circle (  1.49);
\definecolor{drawColor}{RGB}{0,0,0}
\definecolor{fillColor}{RGB}{0,0,0}

\path[draw=drawColor,line width= 0.4pt,line join=round,line cap=round,fill=fillColor] (172.38,179.29) circle (  1.49);
\definecolor{drawColor}{RGB}{255,0,0}
\definecolor{fillColor}{RGB}{255,0,0}

\path[draw=drawColor,line width= 0.4pt,line join=round,line cap=round,fill=fillColor] (172.89,189.51) circle (  1.49);
\definecolor{drawColor}{RGB}{0,0,0}
\definecolor{fillColor}{RGB}{0,0,0}

\path[draw=drawColor,line width= 0.4pt,line join=round,line cap=round,fill=fillColor] (172.90,179.01) circle (  1.49);
\definecolor{drawColor}{RGB}{255,0,0}
\definecolor{fillColor}{RGB}{255,0,0}

\path[draw=drawColor,line width= 0.4pt,line join=round,line cap=round,fill=fillColor] (173.39,189.37) circle (  1.49);
\definecolor{drawColor}{RGB}{0,0,0}
\definecolor{fillColor}{RGB}{0,0,0}

\path[draw=drawColor,line width= 0.4pt,line join=round,line cap=round,fill=fillColor] (173.41,178.83) circle (  1.49);
\definecolor{drawColor}{RGB}{255,0,0}
\definecolor{fillColor}{RGB}{255,0,0}

\path[draw=drawColor,line width= 0.4pt,line join=round,line cap=round,fill=fillColor] (173.97,189.63) circle (  1.49);
\definecolor{drawColor}{RGB}{0,0,0}
\definecolor{fillColor}{RGB}{0,0,0}

\path[draw=drawColor,line width= 0.4pt,line join=round,line cap=round,fill=fillColor] (173.98,178.88) circle (  1.49);
\definecolor{drawColor}{RGB}{255,0,0}
\definecolor{fillColor}{RGB}{255,0,0}

\path[draw=drawColor,line width= 0.4pt,line join=round,line cap=round,fill=fillColor] (174.56,189.06) circle (  1.49);
\definecolor{drawColor}{RGB}{0,0,0}
\definecolor{fillColor}{RGB}{0,0,0}

\path[draw=drawColor,line width= 0.4pt,line join=round,line cap=round,fill=fillColor] (174.57,178.69) circle (  1.49);
\definecolor{drawColor}{RGB}{255,0,0}
\definecolor{fillColor}{RGB}{255,0,0}

\path[draw=drawColor,line width= 0.4pt,line join=round,line cap=round,fill=fillColor] (175.06,188.74) circle (  1.49);
\definecolor{drawColor}{RGB}{0,0,0}
\definecolor{fillColor}{RGB}{0,0,0}

\path[draw=drawColor,line width= 0.4pt,line join=round,line cap=round,fill=fillColor] (175.08,178.15) circle (  1.49);
\definecolor{drawColor}{RGB}{255,0,0}
\definecolor{fillColor}{RGB}{255,0,0}

\path[draw=drawColor,line width= 0.4pt,line join=round,line cap=round,fill=fillColor] (175.54,188.56) circle (  1.49);
\definecolor{drawColor}{RGB}{0,0,0}
\definecolor{fillColor}{RGB}{0,0,0}

\path[draw=drawColor,line width= 0.4pt,line join=round,line cap=round,fill=fillColor] (175.56,178.19) circle (  1.49);
\definecolor{drawColor}{RGB}{255,0,0}
\definecolor{fillColor}{RGB}{255,0,0}

\path[draw=drawColor,line width= 0.4pt,line join=round,line cap=round,fill=fillColor] (176.10,188.78) circle (  1.49);
\definecolor{drawColor}{RGB}{0,0,0}
\definecolor{fillColor}{RGB}{0,0,0}

\path[draw=drawColor,line width= 0.4pt,line join=round,line cap=round,fill=fillColor] (176.11,178.45) circle (  1.49);
\definecolor{drawColor}{RGB}{255,0,0}
\definecolor{fillColor}{RGB}{255,0,0}

\path[draw=drawColor,line width= 0.4pt,line join=round,line cap=round,fill=fillColor] (176.60,188.73) circle (  1.49);
\definecolor{drawColor}{RGB}{0,0,0}
\definecolor{fillColor}{RGB}{0,0,0}

\path[draw=drawColor,line width= 0.4pt,line join=round,line cap=round,fill=fillColor] (176.62,178.48) circle (  1.49);
\definecolor{drawColor}{RGB}{255,0,0}
\definecolor{fillColor}{RGB}{255,0,0}

\path[draw=drawColor,line width= 0.4pt,line join=round,line cap=round,fill=fillColor] (177.08,188.77) circle (  1.49);
\definecolor{drawColor}{RGB}{0,0,0}
\definecolor{fillColor}{RGB}{0,0,0}

\path[draw=drawColor,line width= 0.4pt,line join=round,line cap=round,fill=fillColor] (177.09,178.55) circle (  1.49);
\definecolor{drawColor}{RGB}{255,0,0}
\definecolor{fillColor}{RGB}{255,0,0}

\path[draw=drawColor,line width= 0.4pt,line join=round,line cap=round,fill=fillColor] (177.55,189.02) circle (  1.49);
\definecolor{drawColor}{RGB}{0,0,0}
\definecolor{fillColor}{RGB}{0,0,0}

\path[draw=drawColor,line width= 0.4pt,line join=round,line cap=round,fill=fillColor] (177.57,178.68) circle (  1.49);
\definecolor{drawColor}{RGB}{255,0,0}
\definecolor{fillColor}{RGB}{255,0,0}

\path[draw=drawColor,line width= 0.4pt,line join=round,line cap=round,fill=fillColor] (178.03,188.81) circle (  1.49);
\definecolor{drawColor}{RGB}{0,0,0}
\definecolor{fillColor}{RGB}{0,0,0}

\path[draw=drawColor,line width= 0.4pt,line join=round,line cap=round,fill=fillColor] (178.04,178.86) circle (  1.49);
\definecolor{drawColor}{RGB}{255,0,0}
\definecolor{fillColor}{RGB}{255,0,0}

\path[draw=drawColor,line width= 0.4pt,line join=round,line cap=round,fill=fillColor] (178.50,188.89) circle (  1.49);
\definecolor{drawColor}{RGB}{0,0,0}
\definecolor{fillColor}{RGB}{0,0,0}

\path[draw=drawColor,line width= 0.4pt,line join=round,line cap=round,fill=fillColor] (178.52,178.84) circle (  1.49);
\definecolor{drawColor}{RGB}{255,0,0}
\definecolor{fillColor}{RGB}{255,0,0}

\path[draw=drawColor,line width= 0.4pt,line join=round,line cap=round,fill=fillColor] (179.03,188.83) circle (  1.49);
\definecolor{drawColor}{RGB}{0,0,0}
\definecolor{fillColor}{RGB}{0,0,0}

\path[draw=drawColor,line width= 0.4pt,line join=round,line cap=round,fill=fillColor] (179.04,171.70) circle (  1.49);
\definecolor{drawColor}{RGB}{255,0,0}
\definecolor{fillColor}{RGB}{255,0,0}

\path[draw=drawColor,line width= 0.4pt,line join=round,line cap=round,fill=fillColor] (179.50,188.28) circle (  1.49);
\definecolor{drawColor}{RGB}{0,0,0}
\definecolor{fillColor}{RGB}{0,0,0}

\path[draw=drawColor,line width= 0.4pt,line join=round,line cap=round,fill=fillColor] (179.52,177.35) circle (  1.49);
\definecolor{drawColor}{RGB}{255,0,0}
\definecolor{fillColor}{RGB}{255,0,0}

\path[draw=drawColor,line width= 0.4pt,line join=round,line cap=round,fill=fillColor] (180.12,188.17) circle (  1.49);
\definecolor{drawColor}{RGB}{0,0,0}
\definecolor{fillColor}{RGB}{0,0,0}

\path[draw=drawColor,line width= 0.4pt,line join=round,line cap=round,fill=fillColor] (180.14,168.43) circle (  1.49);
\definecolor{drawColor}{RGB}{255,0,0}
\definecolor{fillColor}{RGB}{255,0,0}

\path[draw=drawColor,line width= 0.4pt,line join=round,line cap=round,fill=fillColor] (180.70,187.71) circle (  1.49);
\definecolor{drawColor}{RGB}{0,0,0}
\definecolor{fillColor}{RGB}{0,0,0}

\path[draw=drawColor,line width= 0.4pt,line join=round,line cap=round,fill=fillColor] (180.71,177.55) circle (  1.49);
\definecolor{drawColor}{RGB}{255,0,0}
\definecolor{fillColor}{RGB}{255,0,0}

\path[draw=drawColor,line width= 0.4pt,line join=round,line cap=round,fill=fillColor] (181.25,187.10) circle (  1.49);
\definecolor{drawColor}{RGB}{0,0,0}
\definecolor{fillColor}{RGB}{0,0,0}

\path[draw=drawColor,line width= 0.4pt,line join=round,line cap=round,fill=fillColor] (181.27,177.14) circle (  1.49);
\definecolor{drawColor}{RGB}{255,0,0}
\definecolor{fillColor}{RGB}{255,0,0}

\path[draw=drawColor,line width= 0.4pt,line join=round,line cap=round,fill=fillColor] (181.71,187.50) circle (  1.49);
\definecolor{drawColor}{RGB}{0,0,0}
\definecolor{fillColor}{RGB}{0,0,0}

\path[draw=drawColor,line width= 0.4pt,line join=round,line cap=round,fill=fillColor] (181.73,177.19) circle (  1.49);
\definecolor{drawColor}{RGB}{255,0,0}
\definecolor{fillColor}{RGB}{255,0,0}

\path[draw=drawColor,line width= 0.4pt,line join=round,line cap=round,fill=fillColor] (182.17,187.95) circle (  1.49);
\definecolor{drawColor}{RGB}{0,0,0}
\definecolor{fillColor}{RGB}{0,0,0}

\path[draw=drawColor,line width= 0.4pt,line join=round,line cap=round,fill=fillColor] (182.19,177.71) circle (  1.49);
\definecolor{drawColor}{RGB}{255,0,0}
\definecolor{fillColor}{RGB}{255,0,0}

\path[draw=drawColor,line width= 0.4pt,line join=round,line cap=round,fill=fillColor] (182.63,188.07) circle (  1.49);
\definecolor{drawColor}{RGB}{0,0,0}
\definecolor{fillColor}{RGB}{0,0,0}

\path[draw=drawColor,line width= 0.4pt,line join=round,line cap=round,fill=fillColor] (182.64,177.85) circle (  1.49);
\definecolor{drawColor}{RGB}{255,0,0}
\definecolor{fillColor}{RGB}{255,0,0}

\path[draw=drawColor,line width= 0.4pt,line join=round,line cap=round,fill=fillColor] (183.10,187.74) circle (  1.49);
\definecolor{drawColor}{RGB}{0,0,0}
\definecolor{fillColor}{RGB}{0,0,0}

\path[draw=drawColor,line width= 0.4pt,line join=round,line cap=round,fill=fillColor] (183.12,177.59) circle (  1.49);
\definecolor{drawColor}{RGB}{255,0,0}
\definecolor{fillColor}{RGB}{255,0,0}

\path[draw=drawColor,line width= 0.4pt,line join=round,line cap=round,fill=fillColor] (183.56,187.84) circle (  1.49);
\definecolor{drawColor}{RGB}{0,0,0}
\definecolor{fillColor}{RGB}{0,0,0}

\path[draw=drawColor,line width= 0.4pt,line join=round,line cap=round,fill=fillColor] (183.58,177.45) circle (  1.49);
\definecolor{drawColor}{RGB}{255,0,0}
\definecolor{fillColor}{RGB}{255,0,0}

\path[draw=drawColor,line width= 0.4pt,line join=round,line cap=round,fill=fillColor] (184.03,187.47) circle (  1.49);
\definecolor{drawColor}{RGB}{0,0,0}
\definecolor{fillColor}{RGB}{0,0,0}

\path[draw=drawColor,line width= 0.4pt,line join=round,line cap=round,fill=fillColor] (184.05,177.42) circle (  1.49);
\definecolor{drawColor}{RGB}{255,0,0}
\definecolor{fillColor}{RGB}{255,0,0}

\path[draw=drawColor,line width= 0.4pt,line join=round,line cap=round,fill=fillColor] (184.64,187.54) circle (  1.49);
\definecolor{drawColor}{RGB}{0,0,0}
\definecolor{fillColor}{RGB}{0,0,0}

\path[draw=drawColor,line width= 0.4pt,line join=round,line cap=round,fill=fillColor] (184.66,177.48) circle (  1.49);
\definecolor{drawColor}{RGB}{255,0,0}
\definecolor{fillColor}{RGB}{255,0,0}

\path[draw=drawColor,line width= 0.4pt,line join=round,line cap=round,fill=fillColor] (185.16,187.62) circle (  1.49);
\definecolor{drawColor}{RGB}{0,0,0}
\definecolor{fillColor}{RGB}{0,0,0}

\path[draw=drawColor,line width= 0.4pt,line join=round,line cap=round,fill=fillColor] (185.18,177.27) circle (  1.49);
\definecolor{drawColor}{RGB}{255,0,0}
\definecolor{fillColor}{RGB}{255,0,0}

\path[draw=drawColor,line width= 0.4pt,line join=round,line cap=round,fill=fillColor] (185.72,187.36) circle (  1.49);
\definecolor{drawColor}{RGB}{0,0,0}
\definecolor{fillColor}{RGB}{0,0,0}

\path[draw=drawColor,line width= 0.4pt,line join=round,line cap=round,fill=fillColor] (185.74,177.42) circle (  1.49);
\definecolor{drawColor}{RGB}{255,0,0}
\definecolor{fillColor}{RGB}{255,0,0}

\path[draw=drawColor,line width= 0.4pt,line join=round,line cap=round,fill=fillColor] (186.18,187.86) circle (  1.49);
\definecolor{drawColor}{RGB}{0,0,0}
\definecolor{fillColor}{RGB}{0,0,0}

\path[draw=drawColor,line width= 0.4pt,line join=round,line cap=round,fill=fillColor] (186.21,177.42) circle (  1.49);
\definecolor{drawColor}{RGB}{255,0,0}
\definecolor{fillColor}{RGB}{255,0,0}

\path[draw=drawColor,line width= 0.4pt,line join=round,line cap=round,fill=fillColor] (186.62,187.30) circle (  1.49);
\definecolor{drawColor}{RGB}{0,0,0}
\definecolor{fillColor}{RGB}{0,0,0}

\path[draw=drawColor,line width= 0.4pt,line join=round,line cap=round,fill=fillColor] (186.65,177.28) circle (  1.49);
\definecolor{drawColor}{RGB}{255,0,0}
\definecolor{fillColor}{RGB}{255,0,0}

\path[draw=drawColor,line width= 0.4pt,line join=round,line cap=round,fill=fillColor] (187.11,187.08) circle (  1.49);
\definecolor{drawColor}{RGB}{0,0,0}
\definecolor{fillColor}{RGB}{0,0,0}

\path[draw=drawColor,line width= 0.4pt,line join=round,line cap=round,fill=fillColor] (187.13,176.50) circle (  1.49);
\definecolor{drawColor}{RGB}{255,0,0}
\definecolor{fillColor}{RGB}{255,0,0}

\path[draw=drawColor,line width= 0.4pt,line join=round,line cap=round,fill=fillColor] (187.67,186.55) circle (  1.49);
\definecolor{drawColor}{RGB}{0,0,0}
\definecolor{fillColor}{RGB}{0,0,0}

\path[draw=drawColor,line width= 0.4pt,line join=round,line cap=round,fill=fillColor] (187.69,176.53) circle (  1.49);
\definecolor{drawColor}{RGB}{255,0,0}
\definecolor{fillColor}{RGB}{255,0,0}

\path[draw=drawColor,line width= 0.4pt,line join=round,line cap=round,fill=fillColor] (188.16,186.65) circle (  1.49);
\definecolor{drawColor}{RGB}{0,0,0}
\definecolor{fillColor}{RGB}{0,0,0}

\path[draw=drawColor,line width= 0.4pt,line join=round,line cap=round,fill=fillColor] (188.18,174.09) circle (  1.49);
\definecolor{drawColor}{RGB}{255,0,0}
\definecolor{fillColor}{RGB}{255,0,0}

\path[draw=drawColor,line width= 0.4pt,line join=round,line cap=round,fill=fillColor] (188.63,186.64) circle (  1.49);
\definecolor{drawColor}{RGB}{0,0,0}
\definecolor{fillColor}{RGB}{0,0,0}

\path[draw=drawColor,line width= 0.4pt,line join=round,line cap=round,fill=fillColor] (188.65,126.17) circle (  1.49);
\definecolor{drawColor}{RGB}{255,0,0}
\definecolor{fillColor}{RGB}{255,0,0}

\path[draw=drawColor,line width= 0.4pt,line join=round,line cap=round,fill=fillColor] (189.11,186.66) circle (  1.49);
\definecolor{drawColor}{RGB}{0,0,0}
\definecolor{fillColor}{RGB}{0,0,0}

\path[draw=drawColor,line width= 0.4pt,line join=round,line cap=round,fill=fillColor] (189.13,176.58) circle (  1.49);
\definecolor{drawColor}{RGB}{255,0,0}
\definecolor{fillColor}{RGB}{255,0,0}

\path[draw=drawColor,line width= 0.4pt,line join=round,line cap=round,fill=fillColor] (189.57,186.27) circle (  1.49);
\definecolor{drawColor}{RGB}{0,0,0}
\definecolor{fillColor}{RGB}{0,0,0}

\path[draw=drawColor,line width= 0.4pt,line join=round,line cap=round,fill=fillColor] (189.60,176.64) circle (  1.49);
\definecolor{drawColor}{RGB}{255,0,0}
\definecolor{fillColor}{RGB}{255,0,0}

\path[draw=drawColor,line width= 0.4pt,line join=round,line cap=round,fill=fillColor] (190.04,186.06) circle (  1.49);
\definecolor{drawColor}{RGB}{0,0,0}
\definecolor{fillColor}{RGB}{0,0,0}

\path[draw=drawColor,line width= 0.4pt,line join=round,line cap=round,fill=fillColor] (190.06,174.15) circle (  1.49);
\definecolor{drawColor}{RGB}{255,0,0}
\definecolor{fillColor}{RGB}{255,0,0}

\path[draw=drawColor,line width= 0.4pt,line join=round,line cap=round,fill=fillColor] (190.52,185.90) circle (  1.49);
\definecolor{drawColor}{RGB}{0,0,0}
\definecolor{fillColor}{RGB}{0,0,0}

\path[draw=drawColor,line width= 0.4pt,line join=round,line cap=round,fill=fillColor] (190.53,175.88) circle (  1.49);
\definecolor{drawColor}{RGB}{255,0,0}
\definecolor{fillColor}{RGB}{255,0,0}

\path[draw=drawColor,line width= 0.4pt,line join=round,line cap=round,fill=fillColor] (190.99,186.09) circle (  1.49);
\definecolor{drawColor}{RGB}{0,0,0}
\definecolor{fillColor}{RGB}{0,0,0}

\path[draw=drawColor,line width= 0.4pt,line join=round,line cap=round,fill=fillColor] (191.01,169.15) circle (  1.49);
\definecolor{drawColor}{RGB}{255,0,0}
\definecolor{fillColor}{RGB}{255,0,0}

\path[draw=drawColor,line width= 0.4pt,line join=round,line cap=round,fill=fillColor] (191.48,186.04) circle (  1.49);
\definecolor{drawColor}{RGB}{0,0,0}
\definecolor{fillColor}{RGB}{0,0,0}

\path[draw=drawColor,line width= 0.4pt,line join=round,line cap=round,fill=fillColor] (191.50,161.00) circle (  1.49);
\definecolor{drawColor}{RGB}{255,0,0}
\definecolor{fillColor}{RGB}{255,0,0}

\path[draw=drawColor,line width= 0.4pt,line join=round,line cap=round,fill=fillColor] (191.94,185.85) circle (  1.49);
\definecolor{drawColor}{RGB}{0,0,0}
\definecolor{fillColor}{RGB}{0,0,0}

\path[draw=drawColor,line width= 0.4pt,line join=round,line cap=round,fill=fillColor] (191.97,173.04) circle (  1.49);
\definecolor{drawColor}{RGB}{255,0,0}
\definecolor{fillColor}{RGB}{255,0,0}

\path[draw=drawColor,line width= 0.4pt,line join=round,line cap=round,fill=fillColor] (192.42,185.59) circle (  1.49);
\definecolor{drawColor}{RGB}{0,0,0}
\definecolor{fillColor}{RGB}{0,0,0}

\path[draw=drawColor,line width= 0.4pt,line join=round,line cap=round,fill=fillColor] (192.43,175.55) circle (  1.49);
\definecolor{drawColor}{RGB}{255,0,0}
\definecolor{fillColor}{RGB}{255,0,0}

\path[draw=drawColor,line width= 0.4pt,line join=round,line cap=round,fill=fillColor] (192.94,185.91) circle (  1.49);
\definecolor{drawColor}{RGB}{0,0,0}
\definecolor{fillColor}{RGB}{0,0,0}

\path[draw=drawColor,line width= 0.4pt,line join=round,line cap=round,fill=fillColor] (192.96,169.20) circle (  1.49);
\definecolor{drawColor}{RGB}{255,0,0}
\definecolor{fillColor}{RGB}{255,0,0}

\path[draw=drawColor,line width= 0.4pt,line join=round,line cap=round,fill=fillColor] (193.45,185.60) circle (  1.49);
\definecolor{drawColor}{RGB}{0,0,0}
\definecolor{fillColor}{RGB}{0,0,0}

\path[draw=drawColor,line width= 0.4pt,line join=round,line cap=round,fill=fillColor] (193.46,173.33) circle (  1.49);
\definecolor{drawColor}{RGB}{255,0,0}
\definecolor{fillColor}{RGB}{255,0,0}

\path[draw=drawColor,line width= 0.4pt,line join=round,line cap=round,fill=fillColor] (193.92,184.65) circle (  1.49);
\definecolor{drawColor}{RGB}{0,0,0}
\definecolor{fillColor}{RGB}{0,0,0}

\path[draw=drawColor,line width= 0.4pt,line join=round,line cap=round,fill=fillColor] (193.94,175.17) circle (  1.49);
\definecolor{drawColor}{RGB}{255,0,0}
\definecolor{fillColor}{RGB}{255,0,0}

\path[draw=drawColor,line width= 0.4pt,line join=round,line cap=round,fill=fillColor] (194.41,184.64) circle (  1.49);
\definecolor{drawColor}{RGB}{0,0,0}
\definecolor{fillColor}{RGB}{0,0,0}

\path[draw=drawColor,line width= 0.4pt,line join=round,line cap=round,fill=fillColor] (194.43,174.77) circle (  1.49);
\definecolor{drawColor}{RGB}{255,0,0}
\definecolor{fillColor}{RGB}{255,0,0}

\path[draw=drawColor,line width= 0.4pt,line join=round,line cap=round,fill=fillColor] (194.89,184.20) circle (  1.49);
\definecolor{drawColor}{RGB}{0,0,0}
\definecolor{fillColor}{RGB}{0,0,0}

\path[draw=drawColor,line width= 0.4pt,line join=round,line cap=round,fill=fillColor] (194.90,174.09) circle (  1.49);
\definecolor{drawColor}{RGB}{255,0,0}
\definecolor{fillColor}{RGB}{255,0,0}

\path[draw=drawColor,line width= 0.4pt,line join=round,line cap=round,fill=fillColor] (195.36,184.01) circle (  1.49);
\definecolor{drawColor}{RGB}{0,0,0}
\definecolor{fillColor}{RGB}{0,0,0}

\path[draw=drawColor,line width= 0.4pt,line join=round,line cap=round,fill=fillColor] (195.38,174.36) circle (  1.49);
\definecolor{drawColor}{RGB}{255,0,0}
\definecolor{fillColor}{RGB}{255,0,0}

\path[draw=drawColor,line width= 0.4pt,line join=round,line cap=round,fill=fillColor] (195.82,184.11) circle (  1.49);
\definecolor{drawColor}{RGB}{0,0,0}
\definecolor{fillColor}{RGB}{0,0,0}

\path[draw=drawColor,line width= 0.4pt,line join=round,line cap=round,fill=fillColor] (195.84,174.26) circle (  1.49);
\definecolor{drawColor}{RGB}{255,0,0}
\definecolor{fillColor}{RGB}{255,0,0}

\path[draw=drawColor,line width= 0.4pt,line join=round,line cap=round,fill=fillColor] (196.26,184.16) circle (  1.49);
\definecolor{drawColor}{RGB}{0,0,0}
\definecolor{fillColor}{RGB}{0,0,0}

\path[draw=drawColor,line width= 0.4pt,line join=round,line cap=round,fill=fillColor] (196.28,174.33) circle (  1.49);
\definecolor{drawColor}{RGB}{255,0,0}
\definecolor{fillColor}{RGB}{255,0,0}

\path[draw=drawColor,line width= 0.4pt,line join=round,line cap=round,fill=fillColor] (196.75,183.80) circle (  1.49);
\definecolor{drawColor}{RGB}{0,0,0}
\definecolor{fillColor}{RGB}{0,0,0}

\path[draw=drawColor,line width= 0.4pt,line join=round,line cap=round,fill=fillColor] (196.77,174.44) circle (  1.49);
\definecolor{drawColor}{RGB}{255,0,0}
\definecolor{fillColor}{RGB}{255,0,0}

\path[draw=drawColor,line width= 0.4pt,line join=round,line cap=round,fill=fillColor] (197.21,183.44) circle (  1.49);
\definecolor{drawColor}{RGB}{0,0,0}
\definecolor{fillColor}{RGB}{0,0,0}

\path[draw=drawColor,line width= 0.4pt,line join=round,line cap=round,fill=fillColor] (197.23,172.76) circle (  1.49);
\definecolor{drawColor}{RGB}{255,0,0}
\definecolor{fillColor}{RGB}{255,0,0}

\path[draw=drawColor,line width= 0.4pt,line join=round,line cap=round,fill=fillColor] (197.67,183.19) circle (  1.49);
\definecolor{drawColor}{RGB}{0,0,0}
\definecolor{fillColor}{RGB}{0,0,0}

\path[draw=drawColor,line width= 0.4pt,line join=round,line cap=round,fill=fillColor] (197.69,173.47) circle (  1.49);
\definecolor{drawColor}{RGB}{255,0,0}
\definecolor{fillColor}{RGB}{255,0,0}

\path[draw=drawColor,line width= 0.4pt,line join=round,line cap=round,fill=fillColor] (198.13,183.36) circle (  1.49);
\definecolor{drawColor}{RGB}{0,0,0}
\definecolor{fillColor}{RGB}{0,0,0}

\path[draw=drawColor,line width= 0.4pt,line join=round,line cap=round,fill=fillColor] (198.15,173.47) circle (  1.49);
\definecolor{drawColor}{RGB}{255,0,0}
\definecolor{fillColor}{RGB}{255,0,0}

\path[draw=drawColor,line width= 0.4pt,line join=round,line cap=round,fill=fillColor] (198.59,183.29) circle (  1.49);
\definecolor{drawColor}{RGB}{0,0,0}
\definecolor{fillColor}{RGB}{0,0,0}

\path[draw=drawColor,line width= 0.4pt,line join=round,line cap=round,fill=fillColor] (198.60,173.81) circle (  1.49);
\definecolor{drawColor}{RGB}{255,0,0}
\definecolor{fillColor}{RGB}{255,0,0}

\path[draw=drawColor,line width= 0.4pt,line join=round,line cap=round,fill=fillColor] (199.06,183.34) circle (  1.49);
\definecolor{drawColor}{RGB}{0,0,0}
\definecolor{fillColor}{RGB}{0,0,0}

\path[draw=drawColor,line width= 0.4pt,line join=round,line cap=round,fill=fillColor] (199.08,173.21) circle (  1.49);
\definecolor{drawColor}{RGB}{255,0,0}
\definecolor{fillColor}{RGB}{255,0,0}

\path[draw=drawColor,line width= 0.4pt,line join=round,line cap=round,fill=fillColor] (199.54,183.46) circle (  1.49);
\definecolor{drawColor}{RGB}{0,0,0}
\definecolor{fillColor}{RGB}{0,0,0}

\path[draw=drawColor,line width= 0.4pt,line join=round,line cap=round,fill=fillColor] (199.55,173.77) circle (  1.49);
\definecolor{drawColor}{RGB}{255,0,0}
\definecolor{fillColor}{RGB}{255,0,0}

\path[draw=drawColor,line width= 0.4pt,line join=round,line cap=round,fill=fillColor] (200.01,183.29) circle (  1.49);
\definecolor{drawColor}{RGB}{0,0,0}
\definecolor{fillColor}{RGB}{0,0,0}

\path[draw=drawColor,line width= 0.4pt,line join=round,line cap=round,fill=fillColor] (200.03,174.02) circle (  1.49);
\definecolor{drawColor}{RGB}{255,0,0}
\definecolor{fillColor}{RGB}{255,0,0}

\path[draw=drawColor,line width= 0.4pt,line join=round,line cap=round,fill=fillColor] (200.47,183.72) circle (  1.49);
\definecolor{drawColor}{RGB}{0,0,0}
\definecolor{fillColor}{RGB}{0,0,0}

\path[draw=drawColor,line width= 0.4pt,line join=round,line cap=round,fill=fillColor] (200.49,173.95) circle (  1.49);
\definecolor{drawColor}{RGB}{255,0,0}
\definecolor{fillColor}{RGB}{255,0,0}

\path[draw=drawColor,line width= 0.4pt,line join=round,line cap=round,fill=fillColor] (200.94,183.77) circle (  1.49);
\definecolor{drawColor}{RGB}{0,0,0}
\definecolor{fillColor}{RGB}{0,0,0}

\path[draw=drawColor,line width= 0.4pt,line join=round,line cap=round,fill=fillColor] (200.96,174.08) circle (  1.49);
\definecolor{drawColor}{RGB}{255,0,0}
\definecolor{fillColor}{RGB}{255,0,0}

\path[draw=drawColor,line width= 0.4pt,line join=round,line cap=round,fill=fillColor] (201.40,183.78) circle (  1.49);
\definecolor{drawColor}{RGB}{0,0,0}
\definecolor{fillColor}{RGB}{0,0,0}

\path[draw=drawColor,line width= 0.4pt,line join=round,line cap=round,fill=fillColor] (201.42,173.86) circle (  1.49);
\definecolor{drawColor}{RGB}{255,0,0}
\definecolor{fillColor}{RGB}{255,0,0}

\path[draw=drawColor,line width= 0.4pt,line join=round,line cap=round,fill=fillColor] (201.84,183.38) circle (  1.49);
\definecolor{drawColor}{RGB}{0,0,0}
\definecolor{fillColor}{RGB}{0,0,0}

\path[draw=drawColor,line width= 0.4pt,line join=round,line cap=round,fill=fillColor] (201.86,168.53) circle (  1.49);
\definecolor{drawColor}{RGB}{255,0,0}
\definecolor{fillColor}{RGB}{255,0,0}

\path[draw=drawColor,line width= 0.4pt,line join=round,line cap=round,fill=fillColor] (202.32,183.19) circle (  1.49);
\definecolor{drawColor}{RGB}{0,0,0}
\definecolor{fillColor}{RGB}{0,0,0}

\path[draw=drawColor,line width= 0.4pt,line join=round,line cap=round,fill=fillColor] (202.34,173.25) circle (  1.49);
\definecolor{drawColor}{RGB}{255,0,0}
\definecolor{fillColor}{RGB}{255,0,0}

\path[draw=drawColor,line width= 0.4pt,line join=round,line cap=round,fill=fillColor] (202.79,182.83) circle (  1.49);
\definecolor{drawColor}{RGB}{0,0,0}
\definecolor{fillColor}{RGB}{0,0,0}

\path[draw=drawColor,line width= 0.4pt,line join=round,line cap=round,fill=fillColor] (202.81,173.26) circle (  1.49);
\definecolor{drawColor}{RGB}{255,0,0}
\definecolor{fillColor}{RGB}{255,0,0}

\path[draw=drawColor,line width= 0.4pt,line join=round,line cap=round,fill=fillColor] (203.27,182.85) circle (  1.49);
\definecolor{drawColor}{RGB}{0,0,0}
\definecolor{fillColor}{RGB}{0,0,0}

\path[draw=drawColor,line width= 0.4pt,line join=round,line cap=round,fill=fillColor] (203.29,172.68) circle (  1.49);
\definecolor{drawColor}{RGB}{255,0,0}
\definecolor{fillColor}{RGB}{255,0,0}

\path[draw=drawColor,line width= 0.4pt,line join=round,line cap=round,fill=fillColor] (203.81,182.38) circle (  1.49);
\definecolor{drawColor}{RGB}{0,0,0}
\definecolor{fillColor}{RGB}{0,0,0}

\path[draw=drawColor,line width= 0.4pt,line join=round,line cap=round,fill=fillColor] (203.83,172.58) circle (  1.49);
\definecolor{drawColor}{RGB}{255,0,0}
\definecolor{fillColor}{RGB}{255,0,0}

\path[draw=drawColor,line width= 0.4pt,line join=round,line cap=round,fill=fillColor] (204.30,182.11) circle (  1.49);
\definecolor{drawColor}{RGB}{0,0,0}
\definecolor{fillColor}{RGB}{0,0,0}

\path[draw=drawColor,line width= 0.4pt,line join=round,line cap=round,fill=fillColor] (204.32,172.53) circle (  1.49);
\definecolor{drawColor}{RGB}{255,0,0}
\definecolor{fillColor}{RGB}{255,0,0}

\path[draw=drawColor,line width= 0.4pt,line join=round,line cap=round,fill=fillColor] (204.76,182.39) circle (  1.49);
\definecolor{drawColor}{RGB}{0,0,0}
\definecolor{fillColor}{RGB}{0,0,0}

\path[draw=drawColor,line width= 0.4pt,line join=round,line cap=round,fill=fillColor] (204.78,171.99) circle (  1.49);
\definecolor{drawColor}{RGB}{255,0,0}
\definecolor{fillColor}{RGB}{255,0,0}

\path[draw=drawColor,line width= 0.4pt,line join=round,line cap=round,fill=fillColor] (205.18,182.14) circle (  1.49);
\definecolor{drawColor}{RGB}{0,0,0}
\definecolor{fillColor}{RGB}{0,0,0}

\path[draw=drawColor,line width= 0.4pt,line join=round,line cap=round,fill=fillColor] (205.20,171.77) circle (  1.49);
\definecolor{drawColor}{RGB}{255,0,0}
\definecolor{fillColor}{RGB}{255,0,0}

\path[draw=drawColor,line width= 0.4pt,line join=round,line cap=round,fill=fillColor] (205.72,182.05) circle (  1.49);
\definecolor{drawColor}{RGB}{0,0,0}
\definecolor{fillColor}{RGB}{0,0,0}

\path[draw=drawColor,line width= 0.4pt,line join=round,line cap=round,fill=fillColor] (205.76,133.23) circle (  1.49);
\definecolor{drawColor}{RGB}{255,0,0}
\definecolor{fillColor}{RGB}{255,0,0}

\path[draw=drawColor,line width= 0.4pt,line join=round,line cap=round,fill=fillColor] (206.23,182.26) circle (  1.49);
\definecolor{drawColor}{RGB}{0,0,0}
\definecolor{fillColor}{RGB}{0,0,0}

\path[draw=drawColor,line width= 0.4pt,line join=round,line cap=round,fill=fillColor] (206.25,172.57) circle (  1.49);
\definecolor{drawColor}{RGB}{255,0,0}
\definecolor{fillColor}{RGB}{255,0,0}

\path[draw=drawColor,line width= 0.4pt,line join=round,line cap=round,fill=fillColor] (206.72,182.33) circle (  1.49);
\definecolor{drawColor}{RGB}{0,0,0}
\definecolor{fillColor}{RGB}{0,0,0}

\path[draw=drawColor,line width= 0.4pt,line join=round,line cap=round,fill=fillColor] (206.74,149.35) circle (  1.49);
\definecolor{drawColor}{RGB}{255,0,0}
\definecolor{fillColor}{RGB}{255,0,0}

\path[draw=drawColor,line width= 0.4pt,line join=round,line cap=round,fill=fillColor] (207.21,181.84) circle (  1.49);
\definecolor{drawColor}{RGB}{0,0,0}
\definecolor{fillColor}{RGB}{0,0,0}

\path[draw=drawColor,line width= 0.4pt,line join=round,line cap=round,fill=fillColor] (207.23,170.18) circle (  1.49);
\definecolor{drawColor}{RGB}{255,0,0}
\definecolor{fillColor}{RGB}{255,0,0}

\path[draw=drawColor,line width= 0.4pt,line join=round,line cap=round,fill=fillColor] (207.66,181.46) circle (  1.49);
\definecolor{drawColor}{RGB}{0,0,0}
\definecolor{fillColor}{RGB}{0,0,0}

\path[draw=drawColor,line width= 0.4pt,line join=round,line cap=round,fill=fillColor] (207.67,171.92) circle (  1.49);
\definecolor{drawColor}{RGB}{255,0,0}
\definecolor{fillColor}{RGB}{255,0,0}

\path[draw=drawColor,line width= 0.4pt,line join=round,line cap=round,fill=fillColor] (208.15,181.54) circle (  1.49);
\definecolor{drawColor}{RGB}{0,0,0}
\definecolor{fillColor}{RGB}{0,0,0}

\path[draw=drawColor,line width= 0.4pt,line join=round,line cap=round,fill=fillColor] (208.20, 57.03) circle (  1.49);
\definecolor{drawColor}{RGB}{255,0,0}
\definecolor{fillColor}{RGB}{255,0,0}

\path[draw=drawColor,line width= 0.4pt,line join=round,line cap=round,fill=fillColor] (208.65,181.22) circle (  1.49);
\definecolor{drawColor}{RGB}{0,0,0}
\definecolor{fillColor}{RGB}{0,0,0}

\path[draw=drawColor,line width= 0.4pt,line join=round,line cap=round,fill=fillColor] (208.67,145.67) circle (  1.49);
\definecolor{drawColor}{RGB}{255,0,0}
\definecolor{fillColor}{RGB}{255,0,0}

\path[draw=drawColor,line width= 0.4pt,line join=round,line cap=round,fill=fillColor] (209.10,180.82) circle (  1.49);
\definecolor{drawColor}{RGB}{0,0,0}
\definecolor{fillColor}{RGB}{0,0,0}

\path[draw=drawColor,line width= 0.4pt,line join=round,line cap=round,fill=fillColor] (209.11,171.58) circle (  1.49);
\definecolor{drawColor}{RGB}{255,0,0}
\definecolor{fillColor}{RGB}{255,0,0}

\path[draw=drawColor,line width= 0.4pt,line join=round,line cap=round,fill=fillColor] (209.59,181.32) circle (  1.49);
\definecolor{drawColor}{RGB}{0,0,0}
\definecolor{fillColor}{RGB}{0,0,0}

\path[draw=drawColor,line width= 0.4pt,line join=round,line cap=round,fill=fillColor] (209.60,171.47) circle (  1.49);
\definecolor{drawColor}{RGB}{255,0,0}
\definecolor{fillColor}{RGB}{255,0,0}

\path[draw=drawColor,line width= 0.4pt,line join=round,line cap=round,fill=fillColor] (210.05,180.88) circle (  1.49);
\definecolor{drawColor}{RGB}{0,0,0}
\definecolor{fillColor}{RGB}{0,0,0}

\path[draw=drawColor,line width= 0.4pt,line join=round,line cap=round,fill=fillColor] (210.06,171.19) circle (  1.49);
\definecolor{drawColor}{RGB}{255,0,0}
\definecolor{fillColor}{RGB}{255,0,0}

\path[draw=drawColor,line width= 0.4pt,line join=round,line cap=round,fill=fillColor] (210.54,180.60) circle (  1.49);
\definecolor{drawColor}{RGB}{0,0,0}
\definecolor{fillColor}{RGB}{0,0,0}

\path[draw=drawColor,line width= 0.4pt,line join=round,line cap=round,fill=fillColor] (210.55,171.03) circle (  1.49);
\definecolor{drawColor}{RGB}{255,0,0}
\definecolor{fillColor}{RGB}{255,0,0}

\path[draw=drawColor,line width= 0.4pt,line join=round,line cap=round,fill=fillColor] (211.06,180.70) circle (  1.49);
\definecolor{drawColor}{RGB}{0,0,0}
\definecolor{fillColor}{RGB}{0,0,0}

\path[draw=drawColor,line width= 0.4pt,line join=round,line cap=round,fill=fillColor] (211.08,170.98) circle (  1.49);
\definecolor{drawColor}{RGB}{255,0,0}
\definecolor{fillColor}{RGB}{255,0,0}

\path[draw=drawColor,line width= 0.4pt,line join=round,line cap=round,fill=fillColor] (211.54,180.78) circle (  1.49);
\definecolor{drawColor}{RGB}{0,0,0}
\definecolor{fillColor}{RGB}{0,0,0}

\path[draw=drawColor,line width= 0.4pt,line join=round,line cap=round,fill=fillColor] (211.55,170.36) circle (  1.49);
\definecolor{drawColor}{RGB}{255,0,0}
\definecolor{fillColor}{RGB}{255,0,0}

\path[draw=drawColor,line width= 0.4pt,line join=round,line cap=round,fill=fillColor] (212.03,180.30) circle (  1.49);
\definecolor{drawColor}{RGB}{0,0,0}
\definecolor{fillColor}{RGB}{0,0,0}

\path[draw=drawColor,line width= 0.4pt,line join=round,line cap=round,fill=fillColor] (212.04,170.07) circle (  1.49);
\definecolor{drawColor}{RGB}{255,0,0}
\definecolor{fillColor}{RGB}{255,0,0}

\path[draw=drawColor,line width= 0.4pt,line join=round,line cap=round,fill=fillColor] (212.49,180.08) circle (  1.49);
\definecolor{drawColor}{RGB}{0,0,0}
\definecolor{fillColor}{RGB}{0,0,0}

\path[draw=drawColor,line width= 0.4pt,line join=round,line cap=round,fill=fillColor] (212.50,170.53) circle (  1.49);
\definecolor{drawColor}{RGB}{255,0,0}
\definecolor{fillColor}{RGB}{255,0,0}

\path[draw=drawColor,line width= 0.4pt,line join=round,line cap=round,fill=fillColor] (212.99,180.10) circle (  1.49);
\definecolor{drawColor}{RGB}{0,0,0}
\definecolor{fillColor}{RGB}{0,0,0}

\path[draw=drawColor,line width= 0.4pt,line join=round,line cap=round,fill=fillColor] (213.01,170.35) circle (  1.49);
\definecolor{drawColor}{RGB}{255,0,0}
\definecolor{fillColor}{RGB}{255,0,0}

\path[draw=drawColor,line width= 0.4pt,line join=round,line cap=round,fill=fillColor] (213.45,179.94) circle (  1.49);
\definecolor{drawColor}{RGB}{0,0,0}
\definecolor{fillColor}{RGB}{0,0,0}

\path[draw=drawColor,line width= 0.4pt,line join=round,line cap=round,fill=fillColor] (213.47,166.80) circle (  1.49);
\definecolor{drawColor}{RGB}{255,0,0}
\definecolor{fillColor}{RGB}{255,0,0}

\path[draw=drawColor,line width= 0.4pt,line join=round,line cap=round,fill=fillColor] (213.91,180.26) circle (  1.49);
\definecolor{drawColor}{RGB}{0,0,0}
\definecolor{fillColor}{RGB}{0,0,0}

\path[draw=drawColor,line width= 0.4pt,line join=round,line cap=round,fill=fillColor] (213.93,170.49) circle (  1.49);
\definecolor{drawColor}{RGB}{255,0,0}
\definecolor{fillColor}{RGB}{255,0,0}

\path[draw=drawColor,line width= 0.4pt,line join=round,line cap=round,fill=fillColor] (214.43,179.75) circle (  1.49);
\definecolor{drawColor}{RGB}{0,0,0}
\definecolor{fillColor}{RGB}{0,0,0}

\path[draw=drawColor,line width= 0.4pt,line join=round,line cap=round,fill=fillColor] (214.45,170.57) circle (  1.49);
\definecolor{drawColor}{RGB}{255,0,0}
\definecolor{fillColor}{RGB}{255,0,0}

\path[draw=drawColor,line width= 0.4pt,line join=round,line cap=round,fill=fillColor] (214.89,179.70) circle (  1.49);
\definecolor{drawColor}{RGB}{0,0,0}
\definecolor{fillColor}{RGB}{0,0,0}

\path[draw=drawColor,line width= 0.4pt,line join=round,line cap=round,fill=fillColor] (214.91,144.98) circle (  1.49);
\definecolor{drawColor}{RGB}{255,0,0}
\definecolor{fillColor}{RGB}{255,0,0}

\path[draw=drawColor,line width= 0.4pt,line join=round,line cap=round,fill=fillColor] (215.33,179.28) circle (  1.49);
\definecolor{drawColor}{RGB}{0,0,0}
\definecolor{fillColor}{RGB}{0,0,0}

\path[draw=drawColor,line width= 0.4pt,line join=round,line cap=round,fill=fillColor] (215.35,170.01) circle (  1.49);
\definecolor{drawColor}{RGB}{255,0,0}
\definecolor{fillColor}{RGB}{255,0,0}

\path[draw=drawColor,line width= 0.4pt,line join=round,line cap=round,fill=fillColor] (215.78,179.33) circle (  1.49);
\definecolor{drawColor}{RGB}{0,0,0}
\definecolor{fillColor}{RGB}{0,0,0}

\path[draw=drawColor,line width= 0.4pt,line join=round,line cap=round,fill=fillColor] (215.79,166.90) circle (  1.49);
\definecolor{drawColor}{RGB}{255,0,0}
\definecolor{fillColor}{RGB}{255,0,0}

\path[draw=drawColor,line width= 0.4pt,line join=round,line cap=round,fill=fillColor] (216.20,179.20) circle (  1.49);
\definecolor{drawColor}{RGB}{0,0,0}
\definecolor{fillColor}{RGB}{0,0,0}

\path[draw=drawColor,line width= 0.4pt,line join=round,line cap=round,fill=fillColor] (216.22,168.97) circle (  1.49);
\definecolor{drawColor}{RGB}{255,0,0}
\definecolor{fillColor}{RGB}{255,0,0}

\path[draw=drawColor,line width= 0.4pt,line join=round,line cap=round,fill=fillColor] (216.63,178.92) circle (  1.49);
\definecolor{drawColor}{RGB}{0,0,0}
\definecolor{fillColor}{RGB}{0,0,0}

\path[draw=drawColor,line width= 0.4pt,line join=round,line cap=round,fill=fillColor] (216.66, 88.67) circle (  1.49);
\definecolor{drawColor}{RGB}{255,0,0}
\definecolor{fillColor}{RGB}{255,0,0}

\path[draw=drawColor,line width= 0.4pt,line join=round,line cap=round,fill=fillColor] (217.13,178.57) circle (  1.49);
\definecolor{drawColor}{RGB}{0,0,0}
\definecolor{fillColor}{RGB}{0,0,0}

\path[draw=drawColor,line width= 0.4pt,line join=round,line cap=round,fill=fillColor] (217.17, 77.42) circle (  1.49);
\definecolor{drawColor}{RGB}{255,0,0}
\definecolor{fillColor}{RGB}{255,0,0}

\path[draw=drawColor,line width= 0.4pt,line join=round,line cap=round,fill=fillColor] (217.66,178.59) circle (  1.49);
\definecolor{drawColor}{RGB}{0,0,0}
\definecolor{fillColor}{RGB}{0,0,0}

\path[draw=drawColor,line width= 0.4pt,line join=round,line cap=round,fill=fillColor] (217.67,169.17) circle (  1.49);
\definecolor{drawColor}{RGB}{255,0,0}
\definecolor{fillColor}{RGB}{255,0,0}

\path[draw=drawColor,line width= 0.4pt,line join=round,line cap=round,fill=fillColor] (218.13,178.43) circle (  1.49);
\definecolor{drawColor}{RGB}{0,0,0}
\definecolor{fillColor}{RGB}{0,0,0}

\path[draw=drawColor,line width= 0.4pt,line join=round,line cap=round,fill=fillColor] (218.15,155.37) circle (  1.49);
\definecolor{drawColor}{RGB}{255,0,0}
\definecolor{fillColor}{RGB}{255,0,0}

\path[draw=drawColor,line width= 0.4pt,line join=round,line cap=round,fill=fillColor] (218.59,178.19) circle (  1.49);
\definecolor{drawColor}{RGB}{0,0,0}
\definecolor{fillColor}{RGB}{0,0,0}

\path[draw=drawColor,line width= 0.4pt,line join=round,line cap=round,fill=fillColor] (218.61,158.70) circle (  1.49);
\definecolor{drawColor}{RGB}{255,0,0}
\definecolor{fillColor}{RGB}{255,0,0}

\path[draw=drawColor,line width= 0.4pt,line join=round,line cap=round,fill=fillColor] (219.11,178.49) circle (  1.49);
\definecolor{drawColor}{RGB}{0,0,0}
\definecolor{fillColor}{RGB}{0,0,0}

\path[draw=drawColor,line width= 0.4pt,line join=round,line cap=round,fill=fillColor] (219.13,160.13) circle (  1.49);
\definecolor{drawColor}{RGB}{255,0,0}
\definecolor{fillColor}{RGB}{255,0,0}

\path[draw=drawColor,line width= 0.4pt,line join=round,line cap=round,fill=fillColor] (219.56,178.60) circle (  1.49);
\definecolor{drawColor}{RGB}{0,0,0}
\definecolor{fillColor}{RGB}{0,0,0}

\path[draw=drawColor,line width= 0.4pt,line join=round,line cap=round,fill=fillColor] (219.57,168.97) circle (  1.49);
\definecolor{drawColor}{RGB}{255,0,0}
\definecolor{fillColor}{RGB}{255,0,0}

\path[draw=drawColor,line width= 0.4pt,line join=round,line cap=round,fill=fillColor] (219.98,177.83) circle (  1.49);
\definecolor{drawColor}{RGB}{0,0,0}
\definecolor{fillColor}{RGB}{0,0,0}

\path[draw=drawColor,line width= 0.4pt,line join=round,line cap=round,fill=fillColor] (220.00,164.93) circle (  1.49);
\definecolor{drawColor}{RGB}{255,0,0}
\definecolor{fillColor}{RGB}{255,0,0}

\path[draw=drawColor,line width= 0.4pt,line join=round,line cap=round,fill=fillColor] (220.51,178.43) circle (  1.49);
\definecolor{drawColor}{RGB}{0,0,0}
\definecolor{fillColor}{RGB}{0,0,0}

\path[draw=drawColor,line width= 0.4pt,line join=round,line cap=round,fill=fillColor] (220.54,153.65) circle (  1.49);
\definecolor{drawColor}{RGB}{255,0,0}
\definecolor{fillColor}{RGB}{255,0,0}

\path[draw=drawColor,line width= 0.4pt,line join=round,line cap=round,fill=fillColor] (220.98,178.12) circle (  1.49);
\definecolor{drawColor}{RGB}{0,0,0}
\definecolor{fillColor}{RGB}{0,0,0}

\path[draw=drawColor,line width= 0.4pt,line join=round,line cap=round,fill=fillColor] (221.03,168.81) circle (  1.49);
\definecolor{drawColor}{RGB}{255,0,0}
\definecolor{fillColor}{RGB}{255,0,0}

\path[draw=drawColor,line width= 0.4pt,line join=round,line cap=round,fill=fillColor] (221.54,178.04) circle (  1.49);
\definecolor{drawColor}{RGB}{0,0,0}
\definecolor{fillColor}{RGB}{0,0,0}

\path[draw=drawColor,line width= 0.4pt,line join=round,line cap=round,fill=fillColor] (221.55,168.67) circle (  1.49);
\definecolor{drawColor}{RGB}{255,0,0}
\definecolor{fillColor}{RGB}{255,0,0}

\path[draw=drawColor,line width= 0.4pt,line join=round,line cap=round,fill=fillColor] (221.98,177.99) circle (  1.49);
\definecolor{drawColor}{RGB}{0,0,0}
\definecolor{fillColor}{RGB}{0,0,0}

\path[draw=drawColor,line width= 0.4pt,line join=round,line cap=round,fill=fillColor] (222.01, 86.34) circle (  1.49);
\definecolor{drawColor}{RGB}{255,0,0}
\definecolor{fillColor}{RGB}{255,0,0}

\path[draw=drawColor,line width= 0.4pt,line join=round,line cap=round,fill=fillColor] (222.44,177.88) circle (  1.49);
\definecolor{drawColor}{RGB}{0,0,0}
\definecolor{fillColor}{RGB}{0,0,0}

\path[draw=drawColor,line width= 0.4pt,line join=round,line cap=round,fill=fillColor] (222.47,168.47) circle (  1.49);
\definecolor{drawColor}{RGB}{255,0,0}
\definecolor{fillColor}{RGB}{255,0,0}

\path[draw=drawColor,line width= 0.4pt,line join=round,line cap=round,fill=fillColor] (222.88,177.52) circle (  1.49);
\definecolor{drawColor}{RGB}{0,0,0}
\definecolor{fillColor}{RGB}{0,0,0}

\path[draw=drawColor,line width= 0.4pt,line join=round,line cap=round,fill=fillColor] (222.90,168.14) circle (  1.49);
\definecolor{drawColor}{RGB}{255,0,0}
\definecolor{fillColor}{RGB}{255,0,0}

\path[draw=drawColor,line width= 0.4pt,line join=round,line cap=round,fill=fillColor] (223.50,177.39) circle (  1.49);
\definecolor{drawColor}{RGB}{0,0,0}
\definecolor{fillColor}{RGB}{0,0,0}

\path[draw=drawColor,line width= 0.4pt,line join=round,line cap=round,fill=fillColor] (223.52,168.11) circle (  1.49);
\definecolor{drawColor}{RGB}{255,0,0}
\definecolor{fillColor}{RGB}{255,0,0}

\path[draw=drawColor,line width= 0.4pt,line join=round,line cap=round,fill=fillColor] (224.07,177.24) circle (  1.49);
\definecolor{drawColor}{RGB}{0,0,0}
\definecolor{fillColor}{RGB}{0,0,0}

\path[draw=drawColor,line width= 0.4pt,line join=round,line cap=round,fill=fillColor] (224.09,166.30) circle (  1.49);
\definecolor{drawColor}{RGB}{255,0,0}
\definecolor{fillColor}{RGB}{255,0,0}

\path[draw=drawColor,line width= 0.4pt,line join=round,line cap=round,fill=fillColor] (224.57,176.71) circle (  1.49);
\definecolor{drawColor}{RGB}{0,0,0}
\definecolor{fillColor}{RGB}{0,0,0}

\path[draw=drawColor,line width= 0.4pt,line join=round,line cap=round,fill=fillColor] (224.58,167.17) circle (  1.49);
\definecolor{drawColor}{RGB}{255,0,0}
\definecolor{fillColor}{RGB}{255,0,0}

\path[draw=drawColor,line width= 0.4pt,line join=round,line cap=round,fill=fillColor] (225.11,176.52) circle (  1.49);
\definecolor{drawColor}{RGB}{0,0,0}
\definecolor{fillColor}{RGB}{0,0,0}

\path[draw=drawColor,line width= 0.4pt,line join=round,line cap=round,fill=fillColor] (225.12,167.24) circle (  1.49);
\definecolor{drawColor}{RGB}{255,0,0}
\definecolor{fillColor}{RGB}{255,0,0}

\path[draw=drawColor,line width= 0.4pt,line join=round,line cap=round,fill=fillColor] (225.58,176.53) circle (  1.49);
\definecolor{drawColor}{RGB}{0,0,0}
\definecolor{fillColor}{RGB}{0,0,0}

\path[draw=drawColor,line width= 0.4pt,line join=round,line cap=round,fill=fillColor] (225.60,167.14) circle (  1.49);
\definecolor{drawColor}{RGB}{255,0,0}
\definecolor{fillColor}{RGB}{255,0,0}

\path[draw=drawColor,line width= 0.4pt,line join=round,line cap=round,fill=fillColor] (226.02,176.42) circle (  1.49);
\definecolor{drawColor}{RGB}{0,0,0}
\definecolor{fillColor}{RGB}{0,0,0}

\path[draw=drawColor,line width= 0.4pt,line join=round,line cap=round,fill=fillColor] (226.04,165.89) circle (  1.49);
\definecolor{drawColor}{RGB}{255,0,0}
\definecolor{fillColor}{RGB}{255,0,0}

\path[draw=drawColor,line width= 0.4pt,line join=round,line cap=round,fill=fillColor] (226.48,176.29) circle (  1.49);
\definecolor{drawColor}{RGB}{0,0,0}
\definecolor{fillColor}{RGB}{0,0,0}

\path[draw=drawColor,line width= 0.4pt,line join=round,line cap=round,fill=fillColor] (226.51, 68.27) circle (  1.49);
\definecolor{drawColor}{RGB}{255,0,0}
\definecolor{fillColor}{RGB}{255,0,0}

\path[draw=drawColor,line width= 0.4pt,line join=round,line cap=round,fill=fillColor] (226.94,176.08) circle (  1.49);
\definecolor{drawColor}{RGB}{0,0,0}
\definecolor{fillColor}{RGB}{0,0,0}

\path[draw=drawColor,line width= 0.4pt,line join=round,line cap=round,fill=fillColor] (226.96,166.76) circle (  1.49);
\definecolor{drawColor}{RGB}{255,0,0}
\definecolor{fillColor}{RGB}{255,0,0}

\path[draw=drawColor,line width= 0.4pt,line join=round,line cap=round,fill=fillColor] (227.41,176.02) circle (  1.49);
\definecolor{drawColor}{RGB}{0,0,0}
\definecolor{fillColor}{RGB}{0,0,0}

\path[draw=drawColor,line width= 0.4pt,line join=round,line cap=round,fill=fillColor] (227.43,166.41) circle (  1.49);
\definecolor{drawColor}{RGB}{255,0,0}
\definecolor{fillColor}{RGB}{255,0,0}

\path[draw=drawColor,line width= 0.4pt,line join=round,line cap=round,fill=fillColor] (227.82,175.85) circle (  1.49);
\definecolor{drawColor}{RGB}{0,0,0}
\definecolor{fillColor}{RGB}{0,0,0}

\path[draw=drawColor,line width= 0.4pt,line join=round,line cap=round,fill=fillColor] (227.84, 82.59) circle (  1.49);
\definecolor{drawColor}{RGB}{255,0,0}
\definecolor{fillColor}{RGB}{255,0,0}

\path[draw=drawColor,line width= 0.4pt,line join=round,line cap=round,fill=fillColor] (228.43,175.35) circle (  1.49);
\definecolor{drawColor}{RGB}{0,0,0}
\definecolor{fillColor}{RGB}{0,0,0}

\path[draw=drawColor,line width= 0.4pt,line join=round,line cap=round,fill=fillColor] (228.45,166.16) circle (  1.49);
\definecolor{drawColor}{RGB}{255,0,0}
\definecolor{fillColor}{RGB}{255,0,0}

\path[draw=drawColor,line width= 0.4pt,line join=round,line cap=round,fill=fillColor] (229.03,175.16) circle (  1.49);
\definecolor{drawColor}{RGB}{0,0,0}
\definecolor{fillColor}{RGB}{0,0,0}

\path[draw=drawColor,line width= 0.4pt,line join=round,line cap=round,fill=fillColor] (229.05,166.42) circle (  1.49);
\definecolor{drawColor}{RGB}{255,0,0}
\definecolor{fillColor}{RGB}{255,0,0}

\path[draw=drawColor,line width= 0.4pt,line join=round,line cap=round,fill=fillColor] (229.46,175.31) circle (  1.49);
\definecolor{drawColor}{RGB}{0,0,0}
\definecolor{fillColor}{RGB}{0,0,0}

\path[draw=drawColor,line width= 0.4pt,line join=round,line cap=round,fill=fillColor] (229.48,166.17) circle (  1.49);
\definecolor{drawColor}{RGB}{255,0,0}
\definecolor{fillColor}{RGB}{255,0,0}

\path[draw=drawColor,line width= 0.4pt,line join=round,line cap=round,fill=fillColor] (229.89,175.25) circle (  1.49);
\definecolor{drawColor}{RGB}{0,0,0}
\definecolor{fillColor}{RGB}{0,0,0}

\path[draw=drawColor,line width= 0.4pt,line join=round,line cap=round,fill=fillColor] (229.90,166.11) circle (  1.49);
\definecolor{drawColor}{RGB}{255,0,0}
\definecolor{fillColor}{RGB}{255,0,0}

\path[draw=drawColor,line width= 0.4pt,line join=round,line cap=round,fill=fillColor] (230.31,174.86) circle (  1.49);
\definecolor{drawColor}{RGB}{0,0,0}
\definecolor{fillColor}{RGB}{0,0,0}

\path[draw=drawColor,line width= 0.4pt,line join=round,line cap=round,fill=fillColor] (230.33,166.26) circle (  1.49);
\definecolor{drawColor}{RGB}{255,0,0}
\definecolor{fillColor}{RGB}{255,0,0}

\path[draw=drawColor,line width= 0.4pt,line join=round,line cap=round,fill=fillColor] (230.74,174.72) circle (  1.49);
\definecolor{drawColor}{RGB}{0,0,0}
\definecolor{fillColor}{RGB}{0,0,0}

\path[draw=drawColor,line width= 0.4pt,line join=round,line cap=round,fill=fillColor] (230.75,165.90) circle (  1.49);
\definecolor{drawColor}{RGB}{255,0,0}
\definecolor{fillColor}{RGB}{255,0,0}

\path[draw=drawColor,line width= 0.4pt,line join=round,line cap=round,fill=fillColor] (231.16,174.66) circle (  1.49);
\definecolor{drawColor}{RGB}{0,0,0}
\definecolor{fillColor}{RGB}{0,0,0}

\path[draw=drawColor,line width= 0.4pt,line join=round,line cap=round,fill=fillColor] (231.20,166.11) circle (  1.49);
\end{scope}
\begin{scope}
\path[clip] (  0.00,  0.00) rectangle (722.70,289.08);
\definecolor{drawColor}{RGB}{0,0,0}

\path[draw=drawColor,line width= 0.4pt,line join=round,line cap=round] ( 54.39, 47.52) -- (231.18, 47.52);

\path[draw=drawColor,line width= 0.4pt,line join=round,line cap=round] ( 54.39, 47.52) -- ( 54.39, 43.56);

\path[draw=drawColor,line width= 0.4pt,line join=round,line cap=round] ( 83.85, 47.52) -- ( 83.85, 43.56);

\path[draw=drawColor,line width= 0.4pt,line join=round,line cap=round] (113.32, 47.52) -- (113.32, 43.56);

\path[draw=drawColor,line width= 0.4pt,line join=round,line cap=round] (142.78, 47.52) -- (142.78, 43.56);

\path[draw=drawColor,line width= 0.4pt,line join=round,line cap=round] (172.25, 47.52) -- (172.25, 43.56);

\path[draw=drawColor,line width= 0.4pt,line join=round,line cap=round] (201.71, 47.52) -- (201.71, 43.56);

\path[draw=drawColor,line width= 0.4pt,line join=round,line cap=round] (231.18, 47.52) -- (231.18, 43.56);

\node[text=drawColor,anchor=base,inner sep=0pt, outer sep=0pt, scale=  0.99] at ( 54.39, 33.26) {11:00};

\node[text=drawColor,anchor=base,inner sep=0pt, outer sep=0pt, scale=  0.99] at (113.32, 33.26) {12:00};

\node[text=drawColor,anchor=base,inner sep=0pt, outer sep=0pt, scale=  0.99] at (172.25, 33.26) {13:00};

\node[text=drawColor,anchor=base,inner sep=0pt, outer sep=0pt, scale=  0.99] at (231.18, 33.26) {14:00};

\path[draw=drawColor,line width= 0.4pt,line join=round,line cap=round] ( 47.52, 89.90) -- ( 47.52,230.77);

\path[draw=drawColor,line width= 0.4pt,line join=round,line cap=round] ( 47.52, 89.90) -- ( 43.56, 89.90);

\path[draw=drawColor,line width= 0.4pt,line join=round,line cap=round] ( 47.52,136.86) -- ( 43.56,136.86);

\path[draw=drawColor,line width= 0.4pt,line join=round,line cap=round] ( 47.52,183.82) -- ( 43.56,183.82);

\path[draw=drawColor,line width= 0.4pt,line join=round,line cap=round] ( 47.52,230.77) -- ( 43.56,230.77);

\node[text=drawColor,rotate= 90.00,anchor=base,inner sep=0pt, outer sep=0pt, scale=  0.99] at ( 38.02, 89.90) {500};

\node[text=drawColor,rotate= 90.00,anchor=base,inner sep=0pt, outer sep=0pt, scale=  0.99] at ( 38.02,136.86) {1000};

\node[text=drawColor,rotate= 90.00,anchor=base,inner sep=0pt, outer sep=0pt, scale=  0.99] at ( 38.02,183.82) {1500};

\node[text=drawColor,rotate= 90.00,anchor=base,inner sep=0pt, outer sep=0pt, scale=  0.99] at ( 38.02,230.77) {2000};

\path[draw=drawColor,line width= 0.4pt,line join=round,line cap=round] ( 47.52, 47.52) --
	(238.26, 47.52) --
	(238.26,241.56) --
	( 47.52,241.56) --
	( 47.52, 47.52);
\end{scope}
\begin{scope}
\path[clip] (  7.92,  7.92) rectangle (246.18,281.16);
\definecolor{drawColor}{RGB}{0,0,0}

\node[text=drawColor,anchor=base,inner sep=0pt, outer sep=0pt, scale=  1.32] at (142.89,256.75) {\bfseries PAR transmis};

\node[text=drawColor,anchor=base,inner sep=0pt, outer sep=0pt, scale=  0.99] at (142.89, 17.42) {Temps UTC};

\node[text=drawColor,rotate= 90.00,anchor=base,inner sep=0pt, outer sep=0pt, scale=  0.99] at ( 22.18,144.54) {PAR (µmol.m$^{-2}$.s$^{-1}$)};
\end{scope}
\begin{scope}
\path[clip] ( 47.52, 47.52) rectangle (238.26,241.56);
\definecolor{drawColor}{RGB}{0,255,0}
\definecolor{fillColor}{RGB}{0,255,0}

\path[draw=drawColor,line width= 0.4pt,line join=round,line cap=round,fill=fillColor] ( 54.73, 92.24) circle (  1.49);

\path[draw=drawColor,line width= 0.4pt,line join=round,line cap=round,fill=fillColor] ( 55.19, 93.58) circle (  1.49);

\path[draw=drawColor,line width= 0.4pt,line join=round,line cap=round,fill=fillColor] ( 55.63, 95.07) circle (  1.49);

\path[draw=drawColor,line width= 0.4pt,line join=round,line cap=round,fill=fillColor] ( 56.09, 95.07) circle (  1.49);

\path[draw=drawColor,line width= 0.4pt,line join=round,line cap=round,fill=fillColor] ( 56.53, 95.04) circle (  1.49);

\path[draw=drawColor,line width= 0.4pt,line join=round,line cap=round,fill=fillColor] ( 56.99, 94.32) circle (  1.49);

\path[draw=drawColor,line width= 0.4pt,line join=round,line cap=round,fill=fillColor] ( 57.43, 93.60) circle (  1.49);

\path[draw=drawColor,line width= 0.4pt,line join=round,line cap=round,fill=fillColor] ( 57.91, 93.18) circle (  1.49);

\path[draw=drawColor,line width= 0.4pt,line join=round,line cap=round,fill=fillColor] ( 58.37, 92.85) circle (  1.49);

\path[draw=drawColor,line width= 0.4pt,line join=round,line cap=round,fill=fillColor] ( 58.82, 92.67) circle (  1.49);

\path[draw=drawColor,line width= 0.4pt,line join=round,line cap=round,fill=fillColor] ( 59.27, 92.80) circle (  1.49);

\path[draw=drawColor,line width= 0.4pt,line join=round,line cap=round,fill=fillColor] ( 59.74, 93.51) circle (  1.49);

\path[draw=drawColor,line width= 0.4pt,line join=round,line cap=round,fill=fillColor] ( 60.18, 94.60) circle (  1.49);

\path[draw=drawColor,line width= 0.4pt,line join=round,line cap=round,fill=fillColor] ( 60.64, 95.41) circle (  1.49);

\path[draw=drawColor,line width= 0.4pt,line join=round,line cap=round,fill=fillColor] ( 61.10, 95.84) circle (  1.49);

\path[draw=drawColor,line width= 0.4pt,line join=round,line cap=round,fill=fillColor] ( 61.56, 95.63) circle (  1.49);

\path[draw=drawColor,line width= 0.4pt,line join=round,line cap=round,fill=fillColor] ( 62.07, 95.26) circle (  1.49);

\path[draw=drawColor,line width= 0.4pt,line join=round,line cap=round,fill=fillColor] ( 62.52, 94.23) circle (  1.49);

\path[draw=drawColor,line width= 0.4pt,line join=round,line cap=round,fill=fillColor] ( 62.98, 95.09) circle (  1.49);

\path[draw=drawColor,line width= 0.4pt,line join=round,line cap=round,fill=fillColor] ( 63.42, 95.27) circle (  1.49);

\path[draw=drawColor,line width= 0.4pt,line join=round,line cap=round,fill=fillColor] ( 63.88, 96.17) circle (  1.49);

\path[draw=drawColor,line width= 0.4pt,line join=round,line cap=round,fill=fillColor] ( 64.34, 95.74) circle (  1.49);

\path[draw=drawColor,line width= 0.4pt,line join=round,line cap=round,fill=fillColor] ( 64.80, 95.02) circle (  1.49);

\path[draw=drawColor,line width= 0.4pt,line join=round,line cap=round,fill=fillColor] ( 65.27, 95.40) circle (  1.49);

\path[draw=drawColor,line width= 0.4pt,line join=round,line cap=round,fill=fillColor] ( 65.73, 95.19) circle (  1.49);

\path[draw=drawColor,line width= 0.4pt,line join=round,line cap=round,fill=fillColor] ( 66.17, 94.89) circle (  1.49);

\path[draw=drawColor,line width= 0.4pt,line join=round,line cap=round,fill=fillColor] ( 66.65, 93.92) circle (  1.49);

\path[draw=drawColor,line width= 0.4pt,line join=round,line cap=round,fill=fillColor] ( 67.11, 93.92) circle (  1.49);

\path[draw=drawColor,line width= 0.4pt,line join=round,line cap=round,fill=fillColor] ( 67.57, 94.58) circle (  1.49);

\path[draw=drawColor,line width= 0.4pt,line join=round,line cap=round,fill=fillColor] ( 68.04, 94.62) circle (  1.49);

\path[draw=drawColor,line width= 0.4pt,line join=round,line cap=round,fill=fillColor] ( 68.50, 94.77) circle (  1.49);

\path[draw=drawColor,line width= 0.4pt,line join=round,line cap=round,fill=fillColor] ( 68.96, 95.15) circle (  1.49);

\path[draw=drawColor,line width= 0.4pt,line join=round,line cap=round,fill=fillColor] ( 69.42, 95.38) circle (  1.49);

\path[draw=drawColor,line width= 0.4pt,line join=round,line cap=round,fill=fillColor] ( 69.87, 95.45) circle (  1.49);

\path[draw=drawColor,line width= 0.4pt,line join=round,line cap=round,fill=fillColor] ( 70.33, 96.06) circle (  1.49);

\path[draw=drawColor,line width= 0.4pt,line join=round,line cap=round,fill=fillColor] ( 70.79, 96.10) circle (  1.49);

\path[draw=drawColor,line width= 0.4pt,line join=round,line cap=round,fill=fillColor] ( 71.25, 97.36) circle (  1.49);

\path[draw=drawColor,line width= 0.4pt,line join=round,line cap=round,fill=fillColor] ( 71.71, 97.30) circle (  1.49);

\path[draw=drawColor,line width= 0.4pt,line join=round,line cap=round,fill=fillColor] ( 72.17, 98.64) circle (  1.49);

\path[draw=drawColor,line width= 0.4pt,line join=round,line cap=round,fill=fillColor] ( 72.62, 98.99) circle (  1.49);

\path[draw=drawColor,line width= 0.4pt,line join=round,line cap=round,fill=fillColor] ( 73.07, 98.96) circle (  1.49);

\path[draw=drawColor,line width= 0.4pt,line join=round,line cap=round,fill=fillColor] ( 73.52, 99.64) circle (  1.49);

\path[draw=drawColor,line width= 0.4pt,line join=round,line cap=round,fill=fillColor] ( 73.97, 98.43) circle (  1.49);

\path[draw=drawColor,line width= 0.4pt,line join=round,line cap=round,fill=fillColor] ( 74.41, 98.14) circle (  1.49);

\path[draw=drawColor,line width= 0.4pt,line join=round,line cap=round,fill=fillColor] ( 74.85, 97.55) circle (  1.49);

\path[draw=drawColor,line width= 0.4pt,line join=round,line cap=round,fill=fillColor] ( 75.31, 96.90) circle (  1.49);

\path[draw=drawColor,line width= 0.4pt,line join=round,line cap=round,fill=fillColor] ( 75.77, 95.85) circle (  1.49);

\path[draw=drawColor,line width= 0.4pt,line join=round,line cap=round,fill=fillColor] ( 76.22, 94.20) circle (  1.49);

\path[draw=drawColor,line width= 0.4pt,line join=round,line cap=round,fill=fillColor] ( 76.67, 93.01) circle (  1.49);

\path[draw=drawColor,line width= 0.4pt,line join=round,line cap=round,fill=fillColor] ( 77.13, 91.74) circle (  1.49);

\path[draw=drawColor,line width= 0.4pt,line join=round,line cap=round,fill=fillColor] ( 77.58, 89.22) circle (  1.49);

\path[draw=drawColor,line width= 0.4pt,line join=round,line cap=round,fill=fillColor] ( 78.09, 93.20) circle (  1.49);

\path[draw=drawColor,line width= 0.4pt,line join=round,line cap=round,fill=fillColor] ( 78.55, 84.87) circle (  1.49);

\path[draw=drawColor,line width= 0.4pt,line join=round,line cap=round,fill=fillColor] ( 78.99, 82.84) circle (  1.49);

\path[draw=drawColor,line width= 0.4pt,line join=round,line cap=round,fill=fillColor] ( 79.47, 80.90) circle (  1.49);

\path[draw=drawColor,line width= 0.4pt,line join=round,line cap=round,fill=fillColor] ( 79.94, 77.97) circle (  1.49);

\path[draw=drawColor,line width= 0.4pt,line join=round,line cap=round,fill=fillColor] ( 80.42, 78.16) circle (  1.49);

\path[draw=drawColor,line width= 0.4pt,line join=round,line cap=round,fill=fillColor] ( 80.91, 77.80) circle (  1.49);

\path[draw=drawColor,line width= 0.4pt,line join=round,line cap=round,fill=fillColor] ( 81.41, 76.56) circle (  1.49);

\path[draw=drawColor,line width= 0.4pt,line join=round,line cap=round,fill=fillColor] ( 81.89, 74.45) circle (  1.49);

\path[draw=drawColor,line width= 0.4pt,line join=round,line cap=round,fill=fillColor] ( 82.36, 73.57) circle (  1.49);

\path[draw=drawColor,line width= 0.4pt,line join=round,line cap=round,fill=fillColor] ( 82.85, 73.07) circle (  1.49);

\path[draw=drawColor,line width= 0.4pt,line join=round,line cap=round,fill=fillColor] ( 83.31, 73.19) circle (  1.49);

\path[draw=drawColor,line width= 0.4pt,line join=round,line cap=round,fill=fillColor] ( 83.79, 72.73) circle (  1.49);

\path[draw=drawColor,line width= 0.4pt,line join=round,line cap=round,fill=fillColor] ( 84.25, 71.11) circle (  1.49);

\path[draw=drawColor,line width= 0.4pt,line join=round,line cap=round,fill=fillColor] ( 84.74, 69.65) circle (  1.49);

\path[draw=drawColor,line width= 0.4pt,line join=round,line cap=round,fill=fillColor] ( 85.23, 67.89) circle (  1.49);

\path[draw=drawColor,line width= 0.4pt,line join=round,line cap=round,fill=fillColor] ( 85.72, 66.84) circle (  1.49);

\path[draw=drawColor,line width= 0.4pt,line join=round,line cap=round,fill=fillColor] ( 86.21, 67.14) circle (  1.49);

\path[draw=drawColor,line width= 0.4pt,line join=round,line cap=round,fill=fillColor] ( 86.70, 67.11) circle (  1.49);

\path[draw=drawColor,line width= 0.4pt,line join=round,line cap=round,fill=fillColor] ( 87.26, 68.11) circle (  1.49);

\path[draw=drawColor,line width= 0.4pt,line join=round,line cap=round,fill=fillColor] ( 87.75, 68.41) circle (  1.49);

\path[draw=drawColor,line width= 0.4pt,line join=round,line cap=round,fill=fillColor] ( 88.26, 68.50) circle (  1.49);

\path[draw=drawColor,line width= 0.4pt,line join=round,line cap=round,fill=fillColor] ( 88.76, 68.75) circle (  1.49);

\path[draw=drawColor,line width= 0.4pt,line join=round,line cap=round,fill=fillColor] ( 89.24, 68.69) circle (  1.49);

\path[draw=drawColor,line width= 0.4pt,line join=round,line cap=round,fill=fillColor] ( 89.73, 68.38) circle (  1.49);

\path[draw=drawColor,line width= 0.4pt,line join=round,line cap=round,fill=fillColor] ( 90.20, 68.84) circle (  1.49);

\path[draw=drawColor,line width= 0.4pt,line join=round,line cap=round,fill=fillColor] ( 90.70, 69.80) circle (  1.49);

\path[draw=drawColor,line width= 0.4pt,line join=round,line cap=round,fill=fillColor] ( 91.15, 71.05) circle (  1.49);

\path[draw=drawColor,line width= 0.4pt,line join=round,line cap=round,fill=fillColor] ( 91.63, 71.17) circle (  1.49);

\path[draw=drawColor,line width= 0.4pt,line join=round,line cap=round,fill=fillColor] ( 92.10, 71.22) circle (  1.49);

\path[draw=drawColor,line width= 0.4pt,line join=round,line cap=round,fill=fillColor] ( 92.58, 71.08) circle (  1.49);

\path[draw=drawColor,line width= 0.4pt,line join=round,line cap=round,fill=fillColor] ( 93.07, 71.44) circle (  1.49);

\path[draw=drawColor,line width= 0.4pt,line join=round,line cap=round,fill=fillColor] ( 93.54, 73.03) circle (  1.49);

\path[draw=drawColor,line width= 0.4pt,line join=round,line cap=round,fill=fillColor] ( 94.05, 74.33) circle (  1.49);

\path[draw=drawColor,line width= 0.4pt,line join=round,line cap=round,fill=fillColor] ( 94.53, 72.75) circle (  1.49);

\path[draw=drawColor,line width= 0.4pt,line join=round,line cap=round,fill=fillColor] ( 95.00, 72.47) circle (  1.49);

\path[draw=drawColor,line width= 0.4pt,line join=round,line cap=round,fill=fillColor] ( 95.49, 72.01) circle (  1.49);

\path[draw=drawColor,line width= 0.4pt,line join=round,line cap=round,fill=fillColor] ( 95.97, 71.55) circle (  1.49);

\path[draw=drawColor,line width= 0.4pt,line join=round,line cap=round,fill=fillColor] ( 96.44, 70.41) circle (  1.49);

\path[draw=drawColor,line width= 0.4pt,line join=round,line cap=round,fill=fillColor] ( 96.93, 68.52) circle (  1.49);

\path[draw=drawColor,line width= 0.4pt,line join=round,line cap=round,fill=fillColor] ( 97.41, 66.28) circle (  1.49);

\path[draw=drawColor,line width= 0.4pt,line join=round,line cap=round,fill=fillColor] ( 97.88, 65.13) circle (  1.49);

\path[draw=drawColor,line width= 0.4pt,line join=round,line cap=round,fill=fillColor] ( 98.37, 63.70) circle (  1.49);

\path[draw=drawColor,line width= 0.4pt,line join=round,line cap=round,fill=fillColor] ( 98.83, 64.94) circle (  1.49);

\path[draw=drawColor,line width= 0.4pt,line join=round,line cap=round,fill=fillColor] ( 99.32, 63.49) circle (  1.49);

\path[draw=drawColor,line width= 0.4pt,line join=round,line cap=round,fill=fillColor] ( 99.81, 64.24) circle (  1.49);

\path[draw=drawColor,line width= 0.4pt,line join=round,line cap=round,fill=fillColor] (100.29, 64.33) circle (  1.49);

\path[draw=drawColor,line width= 0.4pt,line join=round,line cap=round,fill=fillColor] (100.80, 61.54) circle (  1.49);

\path[draw=drawColor,line width= 0.4pt,line join=round,line cap=round,fill=fillColor] (101.34, 60.01) circle (  1.49);

\path[draw=drawColor,line width= 0.4pt,line join=round,line cap=round,fill=fillColor] (101.86, 59.38) circle (  1.49);

\path[draw=drawColor,line width= 0.4pt,line join=round,line cap=round,fill=fillColor] (102.37, 59.40) circle (  1.49);

\path[draw=drawColor,line width= 0.4pt,line join=round,line cap=round,fill=fillColor] (102.84, 58.85) circle (  1.49);

\path[draw=drawColor,line width= 0.4pt,line join=round,line cap=round,fill=fillColor] (103.32, 59.00) circle (  1.49);

\path[draw=drawColor,line width= 0.4pt,line join=round,line cap=round,fill=fillColor] (103.79, 59.17) circle (  1.49);

\path[draw=drawColor,line width= 0.4pt,line join=round,line cap=round,fill=fillColor] (104.28, 59.58) circle (  1.49);

\path[draw=drawColor,line width= 0.4pt,line join=round,line cap=round,fill=fillColor] (104.77, 59.88) circle (  1.49);

\path[draw=drawColor,line width= 0.4pt,line join=round,line cap=round,fill=fillColor] (105.25, 59.36) circle (  1.49);

\path[draw=drawColor,line width= 0.4pt,line join=round,line cap=round,fill=fillColor] (105.72, 58.72) circle (  1.49);

\path[draw=drawColor,line width= 0.4pt,line join=round,line cap=round,fill=fillColor] (106.33, 56.61) circle (  1.49);

\path[draw=drawColor,line width= 0.4pt,line join=round,line cap=round,fill=fillColor] (106.84, 55.27) circle (  1.49);

\path[draw=drawColor,line width= 0.4pt,line join=round,line cap=round,fill=fillColor] (107.38, 54.75) circle (  1.49);

\path[draw=drawColor,line width= 0.4pt,line join=round,line cap=round,fill=fillColor] (107.87, 54.96) circle (  1.49);

\path[draw=drawColor,line width= 0.4pt,line join=round,line cap=round,fill=fillColor] (108.55, 54.68) circle (  1.49);

\path[draw=drawColor,line width= 0.4pt,line join=round,line cap=round,fill=fillColor] (109.10, 54.50) circle (  1.49);

\path[draw=drawColor,line width= 0.4pt,line join=round,line cap=round,fill=fillColor] (109.60, 54.15) circle (  1.49);

\path[draw=drawColor,line width= 0.4pt,line join=round,line cap=round,fill=fillColor] (110.09, 54.65) circle (  1.49);

\path[draw=drawColor,line width= 0.4pt,line join=round,line cap=round,fill=fillColor] (110.60, 55.18) circle (  1.49);

\path[draw=drawColor,line width= 0.4pt,line join=round,line cap=round,fill=fillColor] (111.11, 54.59) circle (  1.49);

\path[draw=drawColor,line width= 0.4pt,line join=round,line cap=round,fill=fillColor] (111.62, 54.01) circle (  1.49);

\path[draw=drawColor,line width= 0.4pt,line join=round,line cap=round,fill=fillColor] (112.16, 53.69) circle (  1.49);

\path[draw=drawColor,line width= 0.4pt,line join=round,line cap=round,fill=fillColor] (112.66, 53.65) circle (  1.49);

\path[draw=drawColor,line width= 0.4pt,line join=round,line cap=round,fill=fillColor] (113.17, 53.78) circle (  1.49);

\path[draw=drawColor,line width= 0.4pt,line join=round,line cap=round,fill=fillColor] (113.68, 53.89) circle (  1.49);

\path[draw=drawColor,line width= 0.4pt,line join=round,line cap=round,fill=fillColor] (114.17, 54.19) circle (  1.49);

\path[draw=drawColor,line width= 0.4pt,line join=round,line cap=round,fill=fillColor] (114.69, 54.44) circle (  1.49);

\path[draw=drawColor,line width= 0.4pt,line join=round,line cap=round,fill=fillColor] (115.18, 54.91) circle (  1.49);

\path[draw=drawColor,line width= 0.4pt,line join=round,line cap=round,fill=fillColor] (115.69, 55.89) circle (  1.49);

\path[draw=drawColor,line width= 0.4pt,line join=round,line cap=round,fill=fillColor] (116.18, 56.10) circle (  1.49);

\path[draw=drawColor,line width= 0.4pt,line join=round,line cap=round,fill=fillColor] (116.69, 55.48) circle (  1.49);

\path[draw=drawColor,line width= 0.4pt,line join=round,line cap=round,fill=fillColor] (117.18, 54.94) circle (  1.49);

\path[draw=drawColor,line width= 0.4pt,line join=round,line cap=round,fill=fillColor] (117.74, 54.68) circle (  1.49);

\path[draw=drawColor,line width= 0.4pt,line join=round,line cap=round,fill=fillColor] (118.26, 54.17) circle (  1.49);

\path[draw=drawColor,line width= 0.4pt,line join=round,line cap=round,fill=fillColor] (118.75, 54.27) circle (  1.49);

\path[draw=drawColor,line width= 0.4pt,line join=round,line cap=round,fill=fillColor] (119.38, 54.06) circle (  1.49);

\path[draw=drawColor,line width= 0.4pt,line join=round,line cap=round,fill=fillColor] (119.87, 53.75) circle (  1.49);

\path[draw=drawColor,line width= 0.4pt,line join=round,line cap=round,fill=fillColor] (120.42, 53.52) circle (  1.49);

\path[draw=drawColor,line width= 0.4pt,line join=round,line cap=round,fill=fillColor] (121.11, 53.53) circle (  1.49);

\path[draw=drawColor,line width= 0.4pt,line join=round,line cap=round,fill=fillColor] (121.73, 53.58) circle (  1.49);

\path[draw=drawColor,line width= 0.4pt,line join=round,line cap=round,fill=fillColor] (122.26, 53.37) circle (  1.49);

\path[draw=drawColor,line width= 0.4pt,line join=round,line cap=round,fill=fillColor] (122.76, 53.49) circle (  1.49);

\path[draw=drawColor,line width= 0.4pt,line join=round,line cap=round,fill=fillColor] (123.50, 53.57) circle (  1.49);

\path[draw=drawColor,line width= 0.4pt,line join=round,line cap=round,fill=fillColor] (124.01, 53.52) circle (  1.49);

\path[draw=drawColor,line width= 0.4pt,line join=round,line cap=round,fill=fillColor] (124.50, 53.57) circle (  1.49);

\path[draw=drawColor,line width= 0.4pt,line join=round,line cap=round,fill=fillColor] (125.07, 53.63) circle (  1.49);

\path[draw=drawColor,line width= 0.4pt,line join=round,line cap=round,fill=fillColor] (125.56, 53.47) circle (  1.49);

\path[draw=drawColor,line width= 0.4pt,line join=round,line cap=round,fill=fillColor] (126.15, 53.45) circle (  1.49);

\path[draw=drawColor,line width= 0.4pt,line join=round,line cap=round,fill=fillColor] (126.73, 53.34) circle (  1.49);

\path[draw=drawColor,line width= 0.4pt,line join=round,line cap=round,fill=fillColor] (127.23, 53.07) circle (  1.49);

\path[draw=drawColor,line width= 0.4pt,line join=round,line cap=round,fill=fillColor] (127.74, 52.97) circle (  1.49);

\path[draw=drawColor,line width= 0.4pt,line join=round,line cap=round,fill=fillColor] (128.23, 53.25) circle (  1.49);

\path[draw=drawColor,line width= 0.4pt,line join=round,line cap=round,fill=fillColor] (128.74, 52.99) circle (  1.49);

\path[draw=drawColor,line width= 0.4pt,line join=round,line cap=round,fill=fillColor] (129.30, 52.75) circle (  1.49);

\path[draw=drawColor,line width= 0.4pt,line join=round,line cap=round,fill=fillColor] (129.87, 52.68) circle (  1.49);

\path[draw=drawColor,line width= 0.4pt,line join=round,line cap=round,fill=fillColor] (130.34, 52.63) circle (  1.49);

\path[draw=drawColor,line width= 0.4pt,line join=round,line cap=round,fill=fillColor] (130.83, 52.58) circle (  1.49);

\path[draw=drawColor,line width= 0.4pt,line join=round,line cap=round,fill=fillColor] (131.31, 52.60) circle (  1.49);

\path[draw=drawColor,line width= 0.4pt,line join=round,line cap=round,fill=fillColor] (131.78, 52.57) circle (  1.49);

\path[draw=drawColor,line width= 0.4pt,line join=round,line cap=round,fill=fillColor] (132.42, 52.58) circle (  1.49);

\path[draw=drawColor,line width= 0.4pt,line join=round,line cap=round,fill=fillColor] (132.91, 52.30) circle (  1.49);

\path[draw=drawColor,line width= 0.4pt,line join=round,line cap=round,fill=fillColor] (133.49, 52.23) circle (  1.49);

\path[draw=drawColor,line width= 0.4pt,line join=round,line cap=round,fill=fillColor] (134.01, 52.40) circle (  1.49);

\path[draw=drawColor,line width= 0.4pt,line join=round,line cap=round,fill=fillColor] (134.52, 52.30) circle (  1.49);

\path[draw=drawColor,line width= 0.4pt,line join=round,line cap=round,fill=fillColor] (135.02, 52.25) circle (  1.49);

\path[draw=drawColor,line width= 0.4pt,line join=round,line cap=round,fill=fillColor] (135.53, 52.23) circle (  1.49);

\path[draw=drawColor,line width= 0.4pt,line join=round,line cap=round,fill=fillColor] (136.09, 52.18) circle (  1.49);

\path[draw=drawColor,line width= 0.4pt,line join=round,line cap=round,fill=fillColor] (136.61, 52.17) circle (  1.49);

\path[draw=drawColor,line width= 0.4pt,line join=round,line cap=round,fill=fillColor] (137.10, 52.10) circle (  1.49);

\path[draw=drawColor,line width= 0.4pt,line join=round,line cap=round,fill=fillColor] (137.66, 52.02) circle (  1.49);

\path[draw=drawColor,line width= 0.4pt,line join=round,line cap=round,fill=fillColor] (138.17, 52.12) circle (  1.49);

\path[draw=drawColor,line width= 0.4pt,line join=round,line cap=round,fill=fillColor] (138.66, 52.29) circle (  1.49);

\path[draw=drawColor,line width= 0.4pt,line join=round,line cap=round,fill=fillColor] (139.15, 52.25) circle (  1.49);

\path[draw=drawColor,line width= 0.4pt,line join=round,line cap=round,fill=fillColor] (139.64, 52.25) circle (  1.49);

\path[draw=drawColor,line width= 0.4pt,line join=round,line cap=round,fill=fillColor] (140.13, 52.26) circle (  1.49);

\path[draw=drawColor,line width= 0.4pt,line join=round,line cap=round,fill=fillColor] (140.66, 52.30) circle (  1.49);

\path[draw=drawColor,line width= 0.4pt,line join=round,line cap=round,fill=fillColor] (141.15, 52.36) circle (  1.49);

\path[draw=drawColor,line width= 0.4pt,line join=round,line cap=round,fill=fillColor] (141.74, 52.26) circle (  1.49);

\path[draw=drawColor,line width= 0.4pt,line join=round,line cap=round,fill=fillColor] (142.28, 52.34) circle (  1.49);

\path[draw=drawColor,line width= 0.4pt,line join=round,line cap=round,fill=fillColor] (142.77, 52.21) circle (  1.49);

\path[draw=drawColor,line width= 0.4pt,line join=round,line cap=round,fill=fillColor] (143.24, 52.39) circle (  1.49);

\path[draw=drawColor,line width= 0.4pt,line join=round,line cap=round,fill=fillColor] (143.75, 52.31) circle (  1.49);

\path[draw=drawColor,line width= 0.4pt,line join=round,line cap=round,fill=fillColor] (144.27, 52.26) circle (  1.49);

\path[draw=drawColor,line width= 0.4pt,line join=round,line cap=round,fill=fillColor] (144.78, 52.32) circle (  1.49);

\path[draw=drawColor,line width= 0.4pt,line join=round,line cap=round,fill=fillColor] (145.27, 52.38) circle (  1.49);

\path[draw=drawColor,line width= 0.4pt,line join=round,line cap=round,fill=fillColor] (145.80, 52.55) circle (  1.49);

\path[draw=drawColor,line width= 0.4pt,line join=round,line cap=round,fill=fillColor] (146.27, 52.65) circle (  1.49);

\path[draw=drawColor,line width= 0.4pt,line join=round,line cap=round,fill=fillColor] (146.79, 53.40) circle (  1.49);

\path[draw=drawColor,line width= 0.4pt,line join=round,line cap=round,fill=fillColor] (147.30, 52.67) circle (  1.49);

\path[draw=drawColor,line width= 0.4pt,line join=round,line cap=round,fill=fillColor] (147.76, 52.52) circle (  1.49);

\path[draw=drawColor,line width= 0.4pt,line join=round,line cap=round,fill=fillColor] (148.28, 52.63) circle (  1.49);

\path[draw=drawColor,line width= 0.4pt,line join=round,line cap=round,fill=fillColor] (148.76, 52.50) circle (  1.49);

\path[draw=drawColor,line width= 0.4pt,line join=round,line cap=round,fill=fillColor] (149.28, 63.15) circle (  1.49);

\path[draw=drawColor,line width= 0.4pt,line join=round,line cap=round,fill=fillColor] (149.84, 52.74) circle (  1.49);

\path[draw=drawColor,line width= 0.4pt,line join=round,line cap=round,fill=fillColor] (150.35, 77.21) circle (  1.49);

\path[draw=drawColor,line width= 0.4pt,line join=round,line cap=round,fill=fillColor] (150.82, 56.04) circle (  1.49);

\path[draw=drawColor,line width= 0.4pt,line join=round,line cap=round,fill=fillColor] (151.34, 52.98) circle (  1.49);

\path[draw=drawColor,line width= 0.4pt,line join=round,line cap=round,fill=fillColor] (151.87, 57.61) circle (  1.49);

\path[draw=drawColor,line width= 0.4pt,line join=round,line cap=round,fill=fillColor] (152.38, 52.91) circle (  1.49);

\path[draw=drawColor,line width= 0.4pt,line join=round,line cap=round,fill=fillColor] (152.87, 52.77) circle (  1.49);

\path[draw=drawColor,line width= 0.4pt,line join=round,line cap=round,fill=fillColor] (153.37, 52.81) circle (  1.49);

\path[draw=drawColor,line width= 0.4pt,line join=round,line cap=round,fill=fillColor] (153.87, 52.72) circle (  1.49);

\path[draw=drawColor,line width= 0.4pt,line join=round,line cap=round,fill=fillColor] (154.41, 52.64) circle (  1.49);

\path[draw=drawColor,line width= 0.4pt,line join=round,line cap=round,fill=fillColor] (154.93, 52.57) circle (  1.49);

\path[draw=drawColor,line width= 0.4pt,line join=round,line cap=round,fill=fillColor] (155.40, 52.52) circle (  1.49);

\path[draw=drawColor,line width= 0.4pt,line join=round,line cap=round,fill=fillColor] (155.91, 52.48) circle (  1.49);

\path[draw=drawColor,line width= 0.4pt,line join=round,line cap=round,fill=fillColor] (156.42, 52.46) circle (  1.49);

\path[draw=drawColor,line width= 0.4pt,line join=round,line cap=round,fill=fillColor] (156.93, 52.44) circle (  1.49);

\path[draw=drawColor,line width= 0.4pt,line join=round,line cap=round,fill=fillColor] (157.43, 52.40) circle (  1.49);

\path[draw=drawColor,line width= 0.4pt,line join=round,line cap=round,fill=fillColor] (157.94, 52.49) circle (  1.49);

\path[draw=drawColor,line width= 0.4pt,line join=round,line cap=round,fill=fillColor] (158.43, 52.43) circle (  1.49);

\path[draw=drawColor,line width= 0.4pt,line join=round,line cap=round,fill=fillColor] (158.97, 52.39) circle (  1.49);

\path[draw=drawColor,line width= 0.4pt,line join=round,line cap=round,fill=fillColor] (159.53, 52.79) circle (  1.49);

\path[draw=drawColor,line width= 0.4pt,line join=round,line cap=round,fill=fillColor] (160.05, 52.31) circle (  1.49);

\path[draw=drawColor,line width= 0.4pt,line join=round,line cap=round,fill=fillColor] (160.53, 52.54) circle (  1.49);

\path[draw=drawColor,line width= 0.4pt,line join=round,line cap=round,fill=fillColor] (161.04, 52.29) circle (  1.49);

\path[draw=drawColor,line width= 0.4pt,line join=round,line cap=round,fill=fillColor] (161.56, 52.27) circle (  1.49);

\path[draw=drawColor,line width= 0.4pt,line join=round,line cap=round,fill=fillColor] (162.03, 52.38) circle (  1.49);

\path[draw=drawColor,line width= 0.4pt,line join=round,line cap=round,fill=fillColor] (162.75, 52.47) circle (  1.49);

\path[draw=drawColor,line width= 0.4pt,line join=round,line cap=round,fill=fillColor] (163.26, 52.44) circle (  1.49);

\path[draw=drawColor,line width= 0.4pt,line join=round,line cap=round,fill=fillColor] (163.75, 52.29) circle (  1.49);

\path[draw=drawColor,line width= 0.4pt,line join=round,line cap=round,fill=fillColor] (164.24, 52.30) circle (  1.49);

\path[draw=drawColor,line width= 0.4pt,line join=round,line cap=round,fill=fillColor] (164.77, 52.28) circle (  1.49);

\path[draw=drawColor,line width= 0.4pt,line join=round,line cap=round,fill=fillColor] (165.24, 52.45) circle (  1.49);

\path[draw=drawColor,line width= 0.4pt,line join=round,line cap=round,fill=fillColor] (165.80, 52.47) circle (  1.49);

\path[draw=drawColor,line width= 0.4pt,line join=round,line cap=round,fill=fillColor] (166.31, 52.44) circle (  1.49);

\path[draw=drawColor,line width= 0.4pt,line join=round,line cap=round,fill=fillColor] (166.83, 52.46) circle (  1.49);

\path[draw=drawColor,line width= 0.4pt,line join=round,line cap=round,fill=fillColor] (167.32, 52.42) circle (  1.49);

\path[draw=drawColor,line width= 0.4pt,line join=round,line cap=round,fill=fillColor] (167.83, 52.19) circle (  1.49);

\path[draw=drawColor,line width= 0.4pt,line join=round,line cap=round,fill=fillColor] (168.34, 52.15) circle (  1.49);

\path[draw=drawColor,line width= 0.4pt,line join=round,line cap=round,fill=fillColor] (168.97, 51.95) circle (  1.49);

\path[draw=drawColor,line width= 0.4pt,line join=round,line cap=round,fill=fillColor] (169.47, 54.77) circle (  1.49);

\path[draw=drawColor,line width= 0.4pt,line join=round,line cap=round,fill=fillColor] (169.94, 52.28) circle (  1.49);

\path[draw=drawColor,line width= 0.4pt,line join=round,line cap=round,fill=fillColor] (170.51, 52.33) circle (  1.49);

\path[draw=drawColor,line width= 0.4pt,line join=round,line cap=round,fill=fillColor] (171.02, 52.26) circle (  1.49);

\path[draw=drawColor,line width= 0.4pt,line join=round,line cap=round,fill=fillColor] (171.56, 52.34) circle (  1.49);

\path[draw=drawColor,line width= 0.4pt,line join=round,line cap=round,fill=fillColor] (172.04, 52.24) circle (  1.49);

\path[draw=drawColor,line width= 0.4pt,line join=round,line cap=round,fill=fillColor] (172.56, 52.22) circle (  1.49);

\path[draw=drawColor,line width= 0.4pt,line join=round,line cap=round,fill=fillColor] (173.07, 52.24) circle (  1.49);

\path[draw=drawColor,line width= 0.4pt,line join=round,line cap=round,fill=fillColor] (173.64,125.41) circle (  1.49);

\path[draw=drawColor,line width= 0.4pt,line join=round,line cap=round,fill=fillColor] (174.21, 52.32) circle (  1.49);

\path[draw=drawColor,line width= 0.4pt,line join=round,line cap=round,fill=fillColor] (174.72, 52.27) circle (  1.49);

\path[draw=drawColor,line width= 0.4pt,line join=round,line cap=round,fill=fillColor] (175.21, 54.76) circle (  1.49);

\path[draw=drawColor,line width= 0.4pt,line join=round,line cap=round,fill=fillColor] (175.75, 72.11) circle (  1.49);

\path[draw=drawColor,line width= 0.4pt,line join=round,line cap=round,fill=fillColor] (176.28, 53.90) circle (  1.49);

\path[draw=drawColor,line width= 0.4pt,line join=round,line cap=round,fill=fillColor] (176.75, 65.59) circle (  1.49);

\path[draw=drawColor,line width= 0.4pt,line join=round,line cap=round,fill=fillColor] (177.23, 57.17) circle (  1.49);

\path[draw=drawColor,line width= 0.4pt,line join=round,line cap=round,fill=fillColor] (177.70, 99.78) circle (  1.49);

\path[draw=drawColor,line width= 0.4pt,line join=round,line cap=round,fill=fillColor] (178.19, 52.40) circle (  1.49);

\path[draw=drawColor,line width= 0.4pt,line join=round,line cap=round,fill=fillColor] (178.67, 52.37) circle (  1.49);

\path[draw=drawColor,line width= 0.4pt,line join=round,line cap=round,fill=fillColor] (179.19, 54.59) circle (  1.49);

\path[draw=drawColor,line width= 0.4pt,line join=round,line cap=round,fill=fillColor] (179.80, 52.91) circle (  1.49);

\path[draw=drawColor,line width= 0.4pt,line join=round,line cap=round,fill=fillColor] (180.37, 52.38) circle (  1.49);

\path[draw=drawColor,line width= 0.4pt,line join=round,line cap=round,fill=fillColor] (180.89, 52.96) circle (  1.49);

\path[draw=drawColor,line width= 0.4pt,line join=round,line cap=round,fill=fillColor] (181.38,196.50) circle (  1.49);

\path[draw=drawColor,line width= 0.4pt,line join=round,line cap=round,fill=fillColor] (181.86,195.64) circle (  1.49);

\path[draw=drawColor,line width= 0.4pt,line join=round,line cap=round,fill=fillColor] (182.30,174.73) circle (  1.49);

\path[draw=drawColor,line width= 0.4pt,line join=round,line cap=round,fill=fillColor] (182.77,138.87) circle (  1.49);

\path[draw=drawColor,line width= 0.4pt,line join=round,line cap=round,fill=fillColor] (183.23,133.46) circle (  1.49);

\path[draw=drawColor,line width= 0.4pt,line join=round,line cap=round,fill=fillColor] (183.71,183.86) circle (  1.49);

\path[draw=drawColor,line width= 0.4pt,line join=round,line cap=round,fill=fillColor] (184.23,183.90) circle (  1.49);

\path[draw=drawColor,line width= 0.4pt,line join=round,line cap=round,fill=fillColor] (184.80, 52.45) circle (  1.49);

\path[draw=drawColor,line width= 0.4pt,line join=round,line cap=round,fill=fillColor] (185.39, 52.28) circle (  1.49);

\path[draw=drawColor,line width= 0.4pt,line join=round,line cap=round,fill=fillColor] (185.85, 88.72) circle (  1.49);

\path[draw=drawColor,line width= 0.4pt,line join=round,line cap=round,fill=fillColor] (186.33,197.01) circle (  1.49);

\path[draw=drawColor,line width= 0.4pt,line join=round,line cap=round,fill=fillColor] (186.78, 53.17) circle (  1.49);

\path[draw=drawColor,line width= 0.4pt,line join=round,line cap=round,fill=fillColor] (187.28,195.71) circle (  1.49);

\path[draw=drawColor,line width= 0.4pt,line join=round,line cap=round,fill=fillColor] (187.83, 52.37) circle (  1.49);

\path[draw=drawColor,line width= 0.4pt,line join=round,line cap=round,fill=fillColor] (188.29,190.53) circle (  1.49);

\path[draw=drawColor,line width= 0.4pt,line join=round,line cap=round,fill=fillColor] (188.77,196.09) circle (  1.49);

\path[draw=drawColor,line width= 0.4pt,line join=round,line cap=round,fill=fillColor] (189.24,195.69) circle (  1.49);

\path[draw=drawColor,line width= 0.4pt,line join=round,line cap=round,fill=fillColor] (189.72,196.37) circle (  1.49);

\path[draw=drawColor,line width= 0.4pt,line join=round,line cap=round,fill=fillColor] (190.19,195.14) circle (  1.49);

\path[draw=drawColor,line width= 0.4pt,line join=round,line cap=round,fill=fillColor] (190.66,193.84) circle (  1.49);

\path[draw=drawColor,line width= 0.4pt,line join=round,line cap=round,fill=fillColor] (191.12,195.46) circle (  1.49);

\path[draw=drawColor,line width= 0.4pt,line join=round,line cap=round,fill=fillColor] (191.61,179.51) circle (  1.49);

\path[draw=drawColor,line width= 0.4pt,line join=round,line cap=round,fill=fillColor] (192.09, 95.29) circle (  1.49);

\path[draw=drawColor,line width= 0.4pt,line join=round,line cap=round,fill=fillColor] (192.56,195.30) circle (  1.49);

\path[draw=drawColor,line width= 0.4pt,line join=round,line cap=round,fill=fillColor] (193.09,195.48) circle (  1.49);

\path[draw=drawColor,line width= 0.4pt,line join=round,line cap=round,fill=fillColor] (193.59,178.74) circle (  1.49);

\path[draw=drawColor,line width= 0.4pt,line join=round,line cap=round,fill=fillColor] (194.07,194.49) circle (  1.49);

\path[draw=drawColor,line width= 0.4pt,line join=round,line cap=round,fill=fillColor] (194.56,152.34) circle (  1.49);

\path[draw=drawColor,line width= 0.4pt,line join=round,line cap=round,fill=fillColor] (195.02,193.34) circle (  1.49);

\path[draw=drawColor,line width= 0.4pt,line join=round,line cap=round,fill=fillColor] (195.49,160.79) circle (  1.49);

\path[draw=drawColor,line width= 0.4pt,line join=round,line cap=round,fill=fillColor] (195.97,193.41) circle (  1.49);

\path[draw=drawColor,line width= 0.4pt,line join=round,line cap=round,fill=fillColor] (196.41,184.60) circle (  1.49);

\path[draw=drawColor,line width= 0.4pt,line join=round,line cap=round,fill=fillColor] (196.88,193.44) circle (  1.49);

\path[draw=drawColor,line width= 0.4pt,line join=round,line cap=round,fill=fillColor] (197.36,192.39) circle (  1.49);

\path[draw=drawColor,line width= 0.4pt,line join=round,line cap=round,fill=fillColor] (197.82,192.67) circle (  1.49);

\path[draw=drawColor,line width= 0.4pt,line join=round,line cap=round,fill=fillColor] (198.26,192.89) circle (  1.49);

\path[draw=drawColor,line width= 0.4pt,line join=round,line cap=round,fill=fillColor] (198.73,192.89) circle (  1.49);

\path[draw=drawColor,line width= 0.4pt,line join=round,line cap=round,fill=fillColor] (199.21,192.87) circle (  1.49);

\path[draw=drawColor,line width= 0.4pt,line join=round,line cap=round,fill=fillColor] (199.68,192.72) circle (  1.49);

\path[draw=drawColor,line width= 0.4pt,line join=round,line cap=round,fill=fillColor] (200.14,192.90) circle (  1.49);

\path[draw=drawColor,line width= 0.4pt,line join=round,line cap=round,fill=fillColor] (200.60,193.73) circle (  1.49);

\path[draw=drawColor,line width= 0.4pt,line join=round,line cap=round,fill=fillColor] (201.08,193.17) circle (  1.49);

\path[draw=drawColor,line width= 0.4pt,line join=round,line cap=round,fill=fillColor] (201.53,193.26) circle (  1.49);

\path[draw=drawColor,line width= 0.4pt,line join=round,line cap=round,fill=fillColor] (201.99,192.91) circle (  1.49);

\path[draw=drawColor,line width= 0.4pt,line join=round,line cap=round,fill=fillColor] (202.45,192.61) circle (  1.49);

\path[draw=drawColor,line width= 0.4pt,line join=round,line cap=round,fill=fillColor] (202.93,192.70) circle (  1.49);

\path[draw=drawColor,line width= 0.4pt,line join=round,line cap=round,fill=fillColor] (203.40,192.18) circle (  1.49);

\path[draw=drawColor,line width= 0.4pt,line join=round,line cap=round,fill=fillColor] (203.96,192.02) circle (  1.49);

\path[draw=drawColor,line width= 0.4pt,line join=round,line cap=round,fill=fillColor] (204.45,192.05) circle (  1.49);

\path[draw=drawColor,line width= 0.4pt,line join=round,line cap=round,fill=fillColor] (204.91,191.91) circle (  1.49);

\path[draw=drawColor,line width= 0.4pt,line join=round,line cap=round,fill=fillColor] (205.33,191.90) circle (  1.49);

\path[draw=drawColor,line width= 0.4pt,line join=round,line cap=round,fill=fillColor] (205.89,192.13) circle (  1.49);

\path[draw=drawColor,line width= 0.4pt,line join=round,line cap=round,fill=fillColor] (206.38,191.94) circle (  1.49);

\path[draw=drawColor,line width= 0.4pt,line join=round,line cap=round,fill=fillColor] (206.87,191.76) circle (  1.49);

\path[draw=drawColor,line width= 0.4pt,line join=round,line cap=round,fill=fillColor] (207.38,191.70) circle (  1.49);

\path[draw=drawColor,line width= 0.4pt,line join=round,line cap=round,fill=fillColor] (207.80,191.73) circle (  1.49);

\path[draw=drawColor,line width= 0.4pt,line join=round,line cap=round,fill=fillColor] (208.31,191.51) circle (  1.49);

\path[draw=drawColor,line width= 0.4pt,line join=round,line cap=round,fill=fillColor] (208.80,191.26) circle (  1.49);

\path[draw=drawColor,line width= 0.4pt,line join=round,line cap=round,fill=fillColor] (209.23,190.91) circle (  1.49);

\path[draw=drawColor,line width= 0.4pt,line join=round,line cap=round,fill=fillColor] (209.74,190.97) circle (  1.49);

\path[draw=drawColor,line width= 0.4pt,line join=round,line cap=round,fill=fillColor] (210.19,190.76) circle (  1.49);

\path[draw=drawColor,line width= 0.4pt,line join=round,line cap=round,fill=fillColor] (210.68,190.63) circle (  1.49);

\path[draw=drawColor,line width= 0.4pt,line join=round,line cap=round,fill=fillColor] (211.21,190.83) circle (  1.49);

\path[draw=drawColor,line width= 0.4pt,line join=round,line cap=round,fill=fillColor] (211.67,190.53) circle (  1.49);

\path[draw=drawColor,line width= 0.4pt,line join=round,line cap=round,fill=fillColor] (212.17,190.27) circle (  1.49);

\path[draw=drawColor,line width= 0.4pt,line join=round,line cap=round,fill=fillColor] (212.63,190.09) circle (  1.49);

\path[draw=drawColor,line width= 0.4pt,line join=round,line cap=round,fill=fillColor] (213.12,190.06) circle (  1.49);

\path[draw=drawColor,line width= 0.4pt,line join=round,line cap=round,fill=fillColor] (213.60,189.95) circle (  1.49);

\path[draw=drawColor,line width= 0.4pt,line join=round,line cap=round,fill=fillColor] (214.04,190.20) circle (  1.49);

\path[draw=drawColor,line width= 0.4pt,line join=round,line cap=round,fill=fillColor] (214.56,189.53) circle (  1.49);

\path[draw=drawColor,line width= 0.4pt,line join=round,line cap=round,fill=fillColor] (215.02,189.59) circle (  1.49);

\path[draw=drawColor,line width= 0.4pt,line join=round,line cap=round,fill=fillColor] (215.48,189.28) circle (  1.49);

\path[draw=drawColor,line width= 0.4pt,line join=round,line cap=round,fill=fillColor] (215.91,189.55) circle (  1.49);

\path[draw=drawColor,line width= 0.4pt,line join=round,line cap=round,fill=fillColor] (216.33,189.25) circle (  1.49);

\path[draw=drawColor,line width= 0.4pt,line join=round,line cap=round,fill=fillColor] (216.79,189.33) circle (  1.49);

\path[draw=drawColor,line width= 0.4pt,line join=round,line cap=round,fill=fillColor] (217.30,188.71) circle (  1.49);

\path[draw=drawColor,line width= 0.4pt,line join=round,line cap=round,fill=fillColor] (217.81,188.79) circle (  1.49);

\path[draw=drawColor,line width= 0.4pt,line join=round,line cap=round,fill=fillColor] (218.26,188.77) circle (  1.49);

\path[draw=drawColor,line width= 0.4pt,line join=round,line cap=round,fill=fillColor] (218.72,188.82) circle (  1.49);

\path[draw=drawColor,line width= 0.4pt,line join=round,line cap=round,fill=fillColor] (219.25,189.01) circle (  1.49);

\path[draw=drawColor,line width= 0.4pt,line join=round,line cap=round,fill=fillColor] (219.69,188.61) circle (  1.49);

\path[draw=drawColor,line width= 0.4pt,line join=round,line cap=round,fill=fillColor] (220.13,188.61) circle (  1.49);

\path[draw=drawColor,line width= 0.4pt,line join=round,line cap=round,fill=fillColor] (220.65,188.58) circle (  1.49);

\path[draw=drawColor,line width= 0.4pt,line join=round,line cap=round,fill=fillColor] (221.16,188.79) circle (  1.49);

\path[draw=drawColor,line width= 0.4pt,line join=round,line cap=round,fill=fillColor] (221.68,188.58) circle (  1.49);

\path[draw=drawColor,line width= 0.4pt,line join=round,line cap=round,fill=fillColor] (222.13,188.45) circle (  1.49);

\path[draw=drawColor,line width= 0.4pt,line join=round,line cap=round,fill=fillColor] (222.59,188.36) circle (  1.49);

\path[draw=drawColor,line width= 0.4pt,line join=round,line cap=round,fill=fillColor] (223.06,187.91) circle (  1.49);

\path[draw=drawColor,line width= 0.4pt,line join=round,line cap=round,fill=fillColor] (223.65,187.86) circle (  1.49);

\path[draw=drawColor,line width= 0.4pt,line join=round,line cap=round,fill=fillColor] (224.22,187.48) circle (  1.49);

\path[draw=drawColor,line width= 0.4pt,line join=round,line cap=round,fill=fillColor] (224.71,187.25) circle (  1.49);

\path[draw=drawColor,line width= 0.4pt,line join=round,line cap=round,fill=fillColor] (225.27,187.13) circle (  1.49);

\path[draw=drawColor,line width= 0.4pt,line join=round,line cap=round,fill=fillColor] (225.71,187.07) circle (  1.49);

\path[draw=drawColor,line width= 0.4pt,line join=round,line cap=round,fill=fillColor] (226.17,186.83) circle (  1.49);

\path[draw=drawColor,line width= 0.4pt,line join=round,line cap=round,fill=fillColor] (226.63,186.24) circle (  1.49);

\path[draw=drawColor,line width= 0.4pt,line join=round,line cap=round,fill=fillColor] (227.12,186.45) circle (  1.49);

\path[draw=drawColor,line width= 0.4pt,line join=round,line cap=round,fill=fillColor] (227.55,186.33) circle (  1.49);

\path[draw=drawColor,line width= 0.4pt,line join=round,line cap=round,fill=fillColor] (227.99,186.28) circle (  1.49);

\path[draw=drawColor,line width= 0.4pt,line join=round,line cap=round,fill=fillColor] (228.58,186.39) circle (  1.49);

\path[draw=drawColor,line width= 0.4pt,line join=round,line cap=round,fill=fillColor] (229.18,185.96) circle (  1.49);

\path[draw=drawColor,line width= 0.4pt,line join=round,line cap=round,fill=fillColor] (229.61,185.19) circle (  1.49);

\path[draw=drawColor,line width= 0.4pt,line join=round,line cap=round,fill=fillColor] (230.03,185.82) circle (  1.49);

\path[draw=drawColor,line width= 0.4pt,line join=round,line cap=round,fill=fillColor] (230.44,185.90) circle (  1.49);

\path[draw=drawColor,line width= 0.4pt,line join=round,line cap=round,fill=fillColor] (230.87,185.63) circle (  1.49);
\definecolor{drawColor}{RGB}{0,0,255}
\definecolor{fillColor}{RGB}{0,0,255}

\path[draw=drawColor,line width= 0.4pt,line join=round,line cap=round,fill=fillColor] ( 54.81, 79.34) circle (  1.49);

\path[draw=drawColor,line width= 0.4pt,line join=round,line cap=round,fill=fillColor] ( 55.26, 80.64) circle (  1.49);

\path[draw=drawColor,line width= 0.4pt,line join=round,line cap=round,fill=fillColor] ( 55.71, 81.52) circle (  1.49);

\path[draw=drawColor,line width= 0.4pt,line join=round,line cap=round,fill=fillColor] ( 56.16, 81.92) circle (  1.49);

\path[draw=drawColor,line width= 0.4pt,line join=round,line cap=round,fill=fillColor] ( 56.61, 82.50) circle (  1.49);

\path[draw=drawColor,line width= 0.4pt,line join=round,line cap=round,fill=fillColor] ( 57.06, 82.60) circle (  1.49);

\path[draw=drawColor,line width= 0.4pt,line join=round,line cap=round,fill=fillColor] ( 57.51, 81.63) circle (  1.49);

\path[draw=drawColor,line width= 0.4pt,line join=round,line cap=round,fill=fillColor] ( 57.99, 80.97) circle (  1.49);

\path[draw=drawColor,line width= 0.4pt,line join=round,line cap=round,fill=fillColor] ( 58.45, 80.48) circle (  1.49);

\path[draw=drawColor,line width= 0.4pt,line join=round,line cap=round,fill=fillColor] ( 58.89, 78.69) circle (  1.49);

\path[draw=drawColor,line width= 0.4pt,line join=round,line cap=round,fill=fillColor] ( 59.35, 80.17) circle (  1.49);

\path[draw=drawColor,line width= 0.4pt,line join=round,line cap=round,fill=fillColor] ( 59.82, 79.02) circle (  1.49);

\path[draw=drawColor,line width= 0.4pt,line join=round,line cap=round,fill=fillColor] ( 60.26, 81.81) circle (  1.49);

\path[draw=drawColor,line width= 0.4pt,line join=round,line cap=round,fill=fillColor] ( 60.72, 82.49) circle (  1.49);

\path[draw=drawColor,line width= 0.4pt,line join=round,line cap=round,fill=fillColor] ( 61.18, 82.51) circle (  1.49);

\path[draw=drawColor,line width= 0.4pt,line join=round,line cap=round,fill=fillColor] ( 61.64, 82.10) circle (  1.49);

\path[draw=drawColor,line width= 0.4pt,line join=round,line cap=round,fill=fillColor] ( 62.15, 82.40) circle (  1.49);

\path[draw=drawColor,line width= 0.4pt,line join=round,line cap=round,fill=fillColor] ( 62.61, 82.86) circle (  1.49);

\path[draw=drawColor,line width= 0.4pt,line join=round,line cap=round,fill=fillColor] ( 63.06, 83.22) circle (  1.49);

\path[draw=drawColor,line width= 0.4pt,line join=round,line cap=round,fill=fillColor] ( 63.51, 83.69) circle (  1.49);

\path[draw=drawColor,line width= 0.4pt,line join=round,line cap=round,fill=fillColor] ( 63.96, 83.19) circle (  1.49);

\path[draw=drawColor,line width= 0.4pt,line join=round,line cap=round,fill=fillColor] ( 64.42, 83.36) circle (  1.49);

\path[draw=drawColor,line width= 0.4pt,line join=round,line cap=round,fill=fillColor] ( 64.88, 82.13) circle (  1.49);

\path[draw=drawColor,line width= 0.4pt,line join=round,line cap=round,fill=fillColor] ( 65.36, 82.11) circle (  1.49);

\path[draw=drawColor,line width= 0.4pt,line join=round,line cap=round,fill=fillColor] ( 65.81, 82.14) circle (  1.49);

\path[draw=drawColor,line width= 0.4pt,line join=round,line cap=round,fill=fillColor] ( 66.26, 82.18) circle (  1.49);

\path[draw=drawColor,line width= 0.4pt,line join=round,line cap=round,fill=fillColor] ( 66.73, 82.03) circle (  1.49);

\path[draw=drawColor,line width= 0.4pt,line join=round,line cap=round,fill=fillColor] ( 67.19, 82.46) circle (  1.49);

\path[draw=drawColor,line width= 0.4pt,line join=round,line cap=round,fill=fillColor] ( 67.65, 82.42) circle (  1.49);

\path[draw=drawColor,line width= 0.4pt,line join=round,line cap=round,fill=fillColor] ( 68.12, 82.42) circle (  1.49);

\path[draw=drawColor,line width= 0.4pt,line join=round,line cap=round,fill=fillColor] ( 68.58, 82.62) circle (  1.49);

\path[draw=drawColor,line width= 0.4pt,line join=round,line cap=round,fill=fillColor] ( 69.04, 82.53) circle (  1.49);

\path[draw=drawColor,line width= 0.4pt,line join=round,line cap=round,fill=fillColor] ( 69.50, 82.91) circle (  1.49);

\path[draw=drawColor,line width= 0.4pt,line join=round,line cap=round,fill=fillColor] ( 69.96, 82.84) circle (  1.49);

\path[draw=drawColor,line width= 0.4pt,line join=round,line cap=round,fill=fillColor] ( 70.41, 82.50) circle (  1.49);

\path[draw=drawColor,line width= 0.4pt,line join=round,line cap=round,fill=fillColor] ( 70.87, 83.85) circle (  1.49);

\path[draw=drawColor,line width= 0.4pt,line join=round,line cap=round,fill=fillColor] ( 71.33, 83.82) circle (  1.49);

\path[draw=drawColor,line width= 0.4pt,line join=round,line cap=round,fill=fillColor] ( 71.79, 84.17) circle (  1.49);

\path[draw=drawColor,line width= 0.4pt,line join=round,line cap=round,fill=fillColor] ( 72.25, 85.72) circle (  1.49);

\path[draw=drawColor,line width= 0.4pt,line join=round,line cap=round,fill=fillColor] ( 72.69, 85.08) circle (  1.49);

\path[draw=drawColor,line width= 0.4pt,line join=round,line cap=round,fill=fillColor] ( 73.15, 83.91) circle (  1.49);

\path[draw=drawColor,line width= 0.4pt,line join=round,line cap=round,fill=fillColor] ( 73.61, 85.32) circle (  1.49);

\path[draw=drawColor,line width= 0.4pt,line join=round,line cap=round,fill=fillColor] ( 74.05, 85.06) circle (  1.49);

\path[draw=drawColor,line width= 0.4pt,line join=round,line cap=round,fill=fillColor] ( 74.49, 85.76) circle (  1.49);

\path[draw=drawColor,line width= 0.4pt,line join=round,line cap=round,fill=fillColor] ( 74.93, 85.23) circle (  1.49);

\path[draw=drawColor,line width= 0.4pt,line join=round,line cap=round,fill=fillColor] ( 75.39, 84.97) circle (  1.49);

\path[draw=drawColor,line width= 0.4pt,line join=round,line cap=round,fill=fillColor] ( 75.85, 83.35) circle (  1.49);

\path[draw=drawColor,line width= 0.4pt,line join=round,line cap=round,fill=fillColor] ( 76.31, 81.76) circle (  1.49);

\path[draw=drawColor,line width= 0.4pt,line join=round,line cap=round,fill=fillColor] ( 76.75, 80.52) circle (  1.49);

\path[draw=drawColor,line width= 0.4pt,line join=round,line cap=round,fill=fillColor] ( 77.21, 80.12) circle (  1.49);

\path[draw=drawColor,line width= 0.4pt,line join=round,line cap=round,fill=fillColor] ( 77.67, 78.02) circle (  1.49);

\path[draw=drawColor,line width= 0.4pt,line join=round,line cap=round,fill=fillColor] ( 78.17, 76.11) circle (  1.49);

\path[draw=drawColor,line width= 0.4pt,line join=round,line cap=round,fill=fillColor] ( 78.61, 74.24) circle (  1.49);

\path[draw=drawColor,line width= 0.4pt,line join=round,line cap=round,fill=fillColor] ( 79.07, 72.64) circle (  1.49);

\path[draw=drawColor,line width= 0.4pt,line join=round,line cap=round,fill=fillColor] ( 79.55, 70.56) circle (  1.49);

\path[draw=drawColor,line width= 0.4pt,line join=round,line cap=round,fill=fillColor] ( 80.02, 70.43) circle (  1.49);

\path[draw=drawColor,line width= 0.4pt,line join=round,line cap=round,fill=fillColor] ( 80.50, 70.42) circle (  1.49);

\path[draw=drawColor,line width= 0.4pt,line join=round,line cap=round,fill=fillColor] ( 80.99, 69.80) circle (  1.49);

\path[draw=drawColor,line width= 0.4pt,line join=round,line cap=round,fill=fillColor] ( 81.50, 69.36) circle (  1.49);

\path[draw=drawColor,line width= 0.4pt,line join=round,line cap=round,fill=fillColor] ( 81.97, 68.63) circle (  1.49);

\path[draw=drawColor,line width= 0.4pt,line join=round,line cap=round,fill=fillColor] ( 82.45, 68.44) circle (  1.49);

\path[draw=drawColor,line width= 0.4pt,line join=round,line cap=round,fill=fillColor] ( 82.92, 68.22) circle (  1.49);

\path[draw=drawColor,line width= 0.4pt,line join=round,line cap=round,fill=fillColor] ( 83.39, 67.97) circle (  1.49);

\path[draw=drawColor,line width= 0.4pt,line join=round,line cap=round,fill=fillColor] ( 83.87, 67.28) circle (  1.49);

\path[draw=drawColor,line width= 0.4pt,line join=round,line cap=round,fill=fillColor] ( 84.33, 66.45) circle (  1.49);

\path[draw=drawColor,line width= 0.4pt,line join=round,line cap=round,fill=fillColor] ( 84.80, 65.27) circle (  1.49);

\path[draw=drawColor,line width= 0.4pt,line join=round,line cap=round,fill=fillColor] ( 85.31, 64.32) circle (  1.49);

\path[draw=drawColor,line width= 0.4pt,line join=round,line cap=round,fill=fillColor] ( 85.80, 63.28) circle (  1.49);

\path[draw=drawColor,line width= 0.4pt,line join=round,line cap=round,fill=fillColor] ( 86.29, 62.11) circle (  1.49);

\path[draw=drawColor,line width= 0.4pt,line join=round,line cap=round,fill=fillColor] ( 86.78, 61.95) circle (  1.49);

\path[draw=drawColor,line width= 0.4pt,line join=round,line cap=round,fill=fillColor] ( 87.34, 61.63) circle (  1.49);

\path[draw=drawColor,line width= 0.4pt,line join=round,line cap=round,fill=fillColor] ( 87.83, 61.23) circle (  1.49);

\path[draw=drawColor,line width= 0.4pt,line join=round,line cap=round,fill=fillColor] ( 88.34, 61.61) circle (  1.49);

\path[draw=drawColor,line width= 0.4pt,line join=round,line cap=round,fill=fillColor] ( 88.85, 60.83) circle (  1.49);

\path[draw=drawColor,line width= 0.4pt,line join=round,line cap=round,fill=fillColor] ( 89.30, 61.33) circle (  1.49);

\path[draw=drawColor,line width= 0.4pt,line join=round,line cap=round,fill=fillColor] ( 89.81, 61.82) circle (  1.49);

\path[draw=drawColor,line width= 0.4pt,line join=round,line cap=round,fill=fillColor] ( 90.29, 62.04) circle (  1.49);

\path[draw=drawColor,line width= 0.4pt,line join=round,line cap=round,fill=fillColor] ( 90.78, 62.22) circle (  1.49);

\path[draw=drawColor,line width= 0.4pt,line join=round,line cap=round,fill=fillColor] ( 91.24, 61.99) circle (  1.49);

\path[draw=drawColor,line width= 0.4pt,line join=round,line cap=round,fill=fillColor] ( 91.71, 62.83) circle (  1.49);

\path[draw=drawColor,line width= 0.4pt,line join=round,line cap=round,fill=fillColor] ( 92.19, 63.11) circle (  1.49);

\path[draw=drawColor,line width= 0.4pt,line join=round,line cap=round,fill=fillColor] ( 92.66, 63.71) circle (  1.49);

\path[draw=drawColor,line width= 0.4pt,line join=round,line cap=round,fill=fillColor] ( 93.15, 65.08) circle (  1.49);

\path[draw=drawColor,line width= 0.4pt,line join=round,line cap=round,fill=fillColor] ( 93.63, 65.19) circle (  1.49);

\path[draw=drawColor,line width= 0.4pt,line join=round,line cap=round,fill=fillColor] ( 94.12, 65.33) circle (  1.49);

\path[draw=drawColor,line width= 0.4pt,line join=round,line cap=round,fill=fillColor] ( 94.59, 64.30) circle (  1.49);

\path[draw=drawColor,line width= 0.4pt,line join=round,line cap=round,fill=fillColor] ( 95.07, 63.21) circle (  1.49);

\path[draw=drawColor,line width= 0.4pt,line join=round,line cap=round,fill=fillColor] ( 95.57, 62.57) circle (  1.49);

\path[draw=drawColor,line width= 0.4pt,line join=round,line cap=round,fill=fillColor] ( 96.05, 62.58) circle (  1.49);

\path[draw=drawColor,line width= 0.4pt,line join=round,line cap=round,fill=fillColor] ( 96.54, 61.49) circle (  1.49);

\path[draw=drawColor,line width= 0.4pt,line join=round,line cap=round,fill=fillColor] ( 97.01, 60.70) circle (  1.49);

\path[draw=drawColor,line width= 0.4pt,line join=round,line cap=round,fill=fillColor] ( 97.49, 59.29) circle (  1.49);

\path[draw=drawColor,line width= 0.4pt,line join=round,line cap=round,fill=fillColor] ( 97.96, 58.91) circle (  1.49);

\path[draw=drawColor,line width= 0.4pt,line join=round,line cap=round,fill=fillColor] ( 98.44, 58.04) circle (  1.49);

\path[draw=drawColor,line width= 0.4pt,line join=round,line cap=round,fill=fillColor] ( 98.91, 59.63) circle (  1.49);

\path[draw=drawColor,line width= 0.4pt,line join=round,line cap=round,fill=fillColor] ( 99.40, 58.11) circle (  1.49);

\path[draw=drawColor,line width= 0.4pt,line join=round,line cap=round,fill=fillColor] ( 99.88, 58.27) circle (  1.49);

\path[draw=drawColor,line width= 0.4pt,line join=round,line cap=round,fill=fillColor] (100.37, 58.31) circle (  1.49);

\path[draw=drawColor,line width= 0.4pt,line join=round,line cap=round,fill=fillColor] (100.89, 56.54) circle (  1.49);

\path[draw=drawColor,line width= 0.4pt,line join=round,line cap=round,fill=fillColor] (101.43, 55.07) circle (  1.49);

\path[draw=drawColor,line width= 0.4pt,line join=round,line cap=round,fill=fillColor] (101.94, 55.04) circle (  1.49);

\path[draw=drawColor,line width= 0.4pt,line join=round,line cap=round,fill=fillColor] (102.43, 55.27) circle (  1.49);

\path[draw=drawColor,line width= 0.4pt,line join=round,line cap=round,fill=fillColor] (102.91, 55.63) circle (  1.49);

\path[draw=drawColor,line width= 0.4pt,line join=round,line cap=round,fill=fillColor] (103.40, 56.58) circle (  1.49);

\path[draw=drawColor,line width= 0.4pt,line join=round,line cap=round,fill=fillColor] (103.87, 55.27) circle (  1.49);

\path[draw=drawColor,line width= 0.4pt,line join=round,line cap=round,fill=fillColor] (104.36, 55.90) circle (  1.49);

\path[draw=drawColor,line width= 0.4pt,line join=round,line cap=round,fill=fillColor] (104.84, 55.36) circle (  1.49);

\path[draw=drawColor,line width= 0.4pt,line join=round,line cap=round,fill=fillColor] (105.33, 54.92) circle (  1.49);

\path[draw=drawColor,line width= 0.4pt,line join=round,line cap=round,fill=fillColor] (105.80, 54.12) circle (  1.49);

\path[draw=drawColor,line width= 0.4pt,line join=round,line cap=round,fill=fillColor] (106.41, 52.81) circle (  1.49);

\path[draw=drawColor,line width= 0.4pt,line join=round,line cap=round,fill=fillColor] (106.92, 52.20) circle (  1.49);

\path[draw=drawColor,line width= 0.4pt,line join=round,line cap=round,fill=fillColor] (107.46, 52.12) circle (  1.49);

\path[draw=drawColor,line width= 0.4pt,line join=round,line cap=round,fill=fillColor] (107.95, 52.34) circle (  1.49);

\path[draw=drawColor,line width= 0.4pt,line join=round,line cap=round,fill=fillColor] (108.64, 52.16) circle (  1.49);

\path[draw=drawColor,line width= 0.4pt,line join=round,line cap=round,fill=fillColor] (109.18, 51.96) circle (  1.49);

\path[draw=drawColor,line width= 0.4pt,line join=round,line cap=round,fill=fillColor] (109.68, 51.87) circle (  1.49);

\path[draw=drawColor,line width= 0.4pt,line join=round,line cap=round,fill=fillColor] (110.19, 52.03) circle (  1.49);

\path[draw=drawColor,line width= 0.4pt,line join=round,line cap=round,fill=fillColor] (110.68, 52.34) circle (  1.49);

\path[draw=drawColor,line width= 0.4pt,line join=round,line cap=round,fill=fillColor] (111.21, 51.92) circle (  1.49);

\path[draw=drawColor,line width= 0.4pt,line join=round,line cap=round,fill=fillColor] (111.70, 51.37) circle (  1.49);

\path[draw=drawColor,line width= 0.4pt,line join=round,line cap=round,fill=fillColor] (112.24, 51.13) circle (  1.49);

\path[draw=drawColor,line width= 0.4pt,line join=round,line cap=round,fill=fillColor] (112.75, 51.04) circle (  1.49);

\path[draw=drawColor,line width= 0.4pt,line join=round,line cap=round,fill=fillColor] (113.25, 51.02) circle (  1.49);

\path[draw=drawColor,line width= 0.4pt,line join=round,line cap=round,fill=fillColor] (113.76, 51.15) circle (  1.49);

\path[draw=drawColor,line width= 0.4pt,line join=round,line cap=round,fill=fillColor] (114.25, 51.20) circle (  1.49);

\path[draw=drawColor,line width= 0.4pt,line join=round,line cap=round,fill=fillColor] (114.78, 51.14) circle (  1.49);

\path[draw=drawColor,line width= 0.4pt,line join=round,line cap=round,fill=fillColor] (115.27, 51.34) circle (  1.49);

\path[draw=drawColor,line width= 0.4pt,line join=round,line cap=round,fill=fillColor] (115.77, 51.70) circle (  1.49);

\path[draw=drawColor,line width= 0.4pt,line join=round,line cap=round,fill=fillColor] (116.26, 51.94) circle (  1.49);

\path[draw=drawColor,line width= 0.4pt,line join=round,line cap=round,fill=fillColor] (116.77, 51.90) circle (  1.49);

\path[draw=drawColor,line width= 0.4pt,line join=round,line cap=round,fill=fillColor] (117.26, 51.82) circle (  1.49);

\path[draw=drawColor,line width= 0.4pt,line join=round,line cap=round,fill=fillColor] (117.82, 51.63) circle (  1.49);

\path[draw=drawColor,line width= 0.4pt,line join=round,line cap=round,fill=fillColor] (118.34, 51.31) circle (  1.49);

\path[draw=drawColor,line width= 0.4pt,line join=round,line cap=round,fill=fillColor] (118.83, 51.33) circle (  1.49);

\path[draw=drawColor,line width= 0.4pt,line join=round,line cap=round,fill=fillColor] (119.46, 51.23) circle (  1.49);

\path[draw=drawColor,line width= 0.4pt,line join=round,line cap=round,fill=fillColor] (119.95, 50.98) circle (  1.49);

\path[draw=drawColor,line width= 0.4pt,line join=round,line cap=round,fill=fillColor] (120.52, 50.94) circle (  1.49);

\path[draw=drawColor,line width= 0.4pt,line join=round,line cap=round,fill=fillColor] (121.19, 50.95) circle (  1.49);

\path[draw=drawColor,line width= 0.4pt,line join=round,line cap=round,fill=fillColor] (121.81, 50.80) circle (  1.49);

\path[draw=drawColor,line width= 0.4pt,line join=round,line cap=round,fill=fillColor] (122.34, 50.67) circle (  1.49);

\path[draw=drawColor,line width= 0.4pt,line join=round,line cap=round,fill=fillColor] (122.86, 50.57) circle (  1.49);

\path[draw=drawColor,line width= 0.4pt,line join=round,line cap=round,fill=fillColor] (123.60, 50.72) circle (  1.49);

\path[draw=drawColor,line width= 0.4pt,line join=round,line cap=round,fill=fillColor] (124.09, 50.75) circle (  1.49);

\path[draw=drawColor,line width= 0.4pt,line join=round,line cap=round,fill=fillColor] (124.60, 50.85) circle (  1.49);

\path[draw=drawColor,line width= 0.4pt,line join=round,line cap=round,fill=fillColor] (125.15, 50.97) circle (  1.49);

\path[draw=drawColor,line width= 0.4pt,line join=round,line cap=round,fill=fillColor] (125.64, 50.78) circle (  1.49);

\path[draw=drawColor,line width= 0.4pt,line join=round,line cap=round,fill=fillColor] (126.25, 50.85) circle (  1.49);

\path[draw=drawColor,line width= 0.4pt,line join=round,line cap=round,fill=fillColor] (126.81, 50.78) circle (  1.49);

\path[draw=drawColor,line width= 0.4pt,line join=round,line cap=round,fill=fillColor] (127.31, 50.70) circle (  1.49);

\path[draw=drawColor,line width= 0.4pt,line join=round,line cap=round,fill=fillColor] (127.82, 50.59) circle (  1.49);

\path[draw=drawColor,line width= 0.4pt,line join=round,line cap=round,fill=fillColor] (128.31, 50.72) circle (  1.49);

\path[draw=drawColor,line width= 0.4pt,line join=round,line cap=round,fill=fillColor] (128.82, 50.57) circle (  1.49);

\path[draw=drawColor,line width= 0.4pt,line join=round,line cap=round,fill=fillColor] (129.39, 50.40) circle (  1.49);

\path[draw=drawColor,line width= 0.4pt,line join=round,line cap=round,fill=fillColor] (129.95, 50.37) circle (  1.49);

\path[draw=drawColor,line width= 0.4pt,line join=round,line cap=round,fill=fillColor] (130.42, 50.29) circle (  1.49);

\path[draw=drawColor,line width= 0.4pt,line join=round,line cap=round,fill=fillColor] (130.92, 50.22) circle (  1.49);

\path[draw=drawColor,line width= 0.4pt,line join=round,line cap=round,fill=fillColor] (131.39, 50.25) circle (  1.49);

\path[draw=drawColor,line width= 0.4pt,line join=round,line cap=round,fill=fillColor] (131.87, 50.18) circle (  1.49);

\path[draw=drawColor,line width= 0.4pt,line join=round,line cap=round,fill=fillColor] (132.50, 50.16) circle (  1.49);

\path[draw=drawColor,line width= 0.4pt,line join=round,line cap=round,fill=fillColor] (133.01, 50.05) circle (  1.49);

\path[draw=drawColor,line width= 0.4pt,line join=round,line cap=round,fill=fillColor] (133.57, 49.97) circle (  1.49);

\path[draw=drawColor,line width= 0.4pt,line join=round,line cap=round,fill=fillColor] (134.09, 50.08) circle (  1.49);

\path[draw=drawColor,line width= 0.4pt,line join=round,line cap=round,fill=fillColor] (134.60, 50.09) circle (  1.49);

\path[draw=drawColor,line width= 0.4pt,line join=round,line cap=round,fill=fillColor] (135.12, 50.09) circle (  1.49);

\path[draw=drawColor,line width= 0.4pt,line join=round,line cap=round,fill=fillColor] (135.61, 50.11) circle (  1.49);

\path[draw=drawColor,line width= 0.4pt,line join=round,line cap=round,fill=fillColor] (136.19, 50.00) circle (  1.49);

\path[draw=drawColor,line width= 0.4pt,line join=round,line cap=round,fill=fillColor] (136.69, 49.97) circle (  1.49);

\path[draw=drawColor,line width= 0.4pt,line join=round,line cap=round,fill=fillColor] (137.20, 49.89) circle (  1.49);

\path[draw=drawColor,line width= 0.4pt,line join=round,line cap=round,fill=fillColor] (137.74, 49.94) circle (  1.49);

\path[draw=drawColor,line width= 0.4pt,line join=round,line cap=round,fill=fillColor] (138.25, 49.94) circle (  1.49);

\path[draw=drawColor,line width= 0.4pt,line join=round,line cap=round,fill=fillColor] (138.74, 50.37) circle (  1.49);

\path[draw=drawColor,line width= 0.4pt,line join=round,line cap=round,fill=fillColor] (139.23, 49.95) circle (  1.49);

\path[draw=drawColor,line width= 0.4pt,line join=round,line cap=round,fill=fillColor] (139.72, 50.04) circle (  1.49);

\path[draw=drawColor,line width= 0.4pt,line join=round,line cap=round,fill=fillColor] (140.23, 49.97) circle (  1.49);

\path[draw=drawColor,line width= 0.4pt,line join=round,line cap=round,fill=fillColor] (140.74, 50.03) circle (  1.49);

\path[draw=drawColor,line width= 0.4pt,line join=round,line cap=round,fill=fillColor] (141.23, 50.05) circle (  1.49);

\path[draw=drawColor,line width= 0.4pt,line join=round,line cap=round,fill=fillColor] (141.85, 50.06) circle (  1.49);

\path[draw=drawColor,line width= 0.4pt,line join=round,line cap=round,fill=fillColor] (142.36, 49.95) circle (  1.49);

\path[draw=drawColor,line width= 0.4pt,line join=round,line cap=round,fill=fillColor] (142.85, 49.99) circle (  1.49);

\path[draw=drawColor,line width= 0.4pt,line join=round,line cap=round,fill=fillColor] (143.34, 50.05) circle (  1.49);

\path[draw=drawColor,line width= 0.4pt,line join=round,line cap=round,fill=fillColor] (143.83, 50.08) circle (  1.49);

\path[draw=drawColor,line width= 0.4pt,line join=round,line cap=round,fill=fillColor] (144.36, 49.94) circle (  1.49);

\path[draw=drawColor,line width= 0.4pt,line join=round,line cap=round,fill=fillColor] (144.86, 50.05) circle (  1.49);

\path[draw=drawColor,line width= 0.4pt,line join=round,line cap=round,fill=fillColor] (145.35, 50.08) circle (  1.49);

\path[draw=drawColor,line width= 0.4pt,line join=round,line cap=round,fill=fillColor] (145.88, 50.18) circle (  1.49);

\path[draw=drawColor,line width= 0.4pt,line join=round,line cap=round,fill=fillColor] (146.37, 50.12) circle (  1.49);

\path[draw=drawColor,line width= 0.4pt,line join=round,line cap=round,fill=fillColor] (146.88, 50.36) circle (  1.49);

\path[draw=drawColor,line width= 0.4pt,line join=round,line cap=round,fill=fillColor] (147.38, 50.24) circle (  1.49);

\path[draw=drawColor,line width= 0.4pt,line join=round,line cap=round,fill=fillColor] (147.86, 50.18) circle (  1.49);

\path[draw=drawColor,line width= 0.4pt,line join=round,line cap=round,fill=fillColor] (148.38, 50.22) circle (  1.49);

\path[draw=drawColor,line width= 0.4pt,line join=round,line cap=round,fill=fillColor] (148.84, 50.25) circle (  1.49);

\path[draw=drawColor,line width= 0.4pt,line join=round,line cap=round,fill=fillColor] (149.36, 50.30) circle (  1.49);

\path[draw=drawColor,line width= 0.4pt,line join=round,line cap=round,fill=fillColor] (149.92, 50.33) circle (  1.49);

\path[draw=drawColor,line width= 0.4pt,line join=round,line cap=round,fill=fillColor] (150.43, 50.65) circle (  1.49);

\path[draw=drawColor,line width= 0.4pt,line join=round,line cap=round,fill=fillColor] (150.90, 50.57) circle (  1.49);

\path[draw=drawColor,line width= 0.4pt,line join=round,line cap=round,fill=fillColor] (151.44, 50.54) circle (  1.49);

\path[draw=drawColor,line width= 0.4pt,line join=round,line cap=round,fill=fillColor] (151.97, 50.44) circle (  1.49);

\path[draw=drawColor,line width= 0.4pt,line join=round,line cap=round,fill=fillColor] (152.47, 50.38) circle (  1.49);

\path[draw=drawColor,line width= 0.4pt,line join=round,line cap=round,fill=fillColor] (152.95, 50.31) circle (  1.49);

\path[draw=drawColor,line width= 0.4pt,line join=round,line cap=round,fill=fillColor] (153.46, 50.31) circle (  1.49);

\path[draw=drawColor,line width= 0.4pt,line join=round,line cap=round,fill=fillColor] (153.95, 50.33) circle (  1.49);

\path[draw=drawColor,line width= 0.4pt,line join=round,line cap=round,fill=fillColor] (154.49, 50.15) circle (  1.49);

\path[draw=drawColor,line width= 0.4pt,line join=round,line cap=round,fill=fillColor] (155.01, 50.16) circle (  1.49);

\path[draw=drawColor,line width= 0.4pt,line join=round,line cap=round,fill=fillColor] (155.50, 50.13) circle (  1.49);

\path[draw=drawColor,line width= 0.4pt,line join=round,line cap=round,fill=fillColor] (155.99, 50.16) circle (  1.49);

\path[draw=drawColor,line width= 0.4pt,line join=round,line cap=round,fill=fillColor] (156.50, 50.15) circle (  1.49);

\path[draw=drawColor,line width= 0.4pt,line join=round,line cap=round,fill=fillColor] (157.01, 50.14) circle (  1.49);

\path[draw=drawColor,line width= 0.4pt,line join=round,line cap=round,fill=fillColor] (157.52, 50.13) circle (  1.49);

\path[draw=drawColor,line width= 0.4pt,line join=round,line cap=round,fill=fillColor] (158.02, 50.15) circle (  1.49);

\path[draw=drawColor,line width= 0.4pt,line join=round,line cap=round,fill=fillColor] (158.51, 50.21) circle (  1.49);

\path[draw=drawColor,line width= 0.4pt,line join=round,line cap=round,fill=fillColor] (159.09, 50.24) circle (  1.49);

\path[draw=drawColor,line width= 0.4pt,line join=round,line cap=round,fill=fillColor] (159.61, 50.50) circle (  1.49);

\path[draw=drawColor,line width= 0.4pt,line join=round,line cap=round,fill=fillColor] (160.14, 50.25) circle (  1.49);

\path[draw=drawColor,line width= 0.4pt,line join=round,line cap=round,fill=fillColor] (160.61, 50.08) circle (  1.49);

\path[draw=drawColor,line width= 0.4pt,line join=round,line cap=round,fill=fillColor] (161.12, 50.14) circle (  1.49);

\path[draw=drawColor,line width= 0.4pt,line join=round,line cap=round,fill=fillColor] (161.64, 50.12) circle (  1.49);

\path[draw=drawColor,line width= 0.4pt,line join=round,line cap=round,fill=fillColor] (162.15, 50.27) circle (  1.49);

\path[draw=drawColor,line width= 0.4pt,line join=round,line cap=round,fill=fillColor] (162.84, 50.07) circle (  1.49);

\path[draw=drawColor,line width= 0.4pt,line join=round,line cap=round,fill=fillColor] (163.34, 50.12) circle (  1.49);

\path[draw=drawColor,line width= 0.4pt,line join=round,line cap=round,fill=fillColor] (163.83, 50.09) circle (  1.49);

\path[draw=drawColor,line width= 0.4pt,line join=round,line cap=round,fill=fillColor] (164.33, 49.74) circle (  1.49);

\path[draw=drawColor,line width= 0.4pt,line join=round,line cap=round,fill=fillColor] (164.85, 49.87) circle (  1.49);

\path[draw=drawColor,line width= 0.4pt,line join=round,line cap=round,fill=fillColor] (165.34, 50.03) circle (  1.49);

\path[draw=drawColor,line width= 0.4pt,line join=round,line cap=round,fill=fillColor] (165.90, 50.18) circle (  1.49);

\path[draw=drawColor,line width= 0.4pt,line join=round,line cap=round,fill=fillColor] (166.40, 50.17) circle (  1.49);

\path[draw=drawColor,line width= 0.4pt,line join=round,line cap=round,fill=fillColor] (166.91, 50.09) circle (  1.49);

\path[draw=drawColor,line width= 0.4pt,line join=round,line cap=round,fill=fillColor] (167.40, 50.12) circle (  1.49);

\path[draw=drawColor,line width= 0.4pt,line join=round,line cap=round,fill=fillColor] (167.91, 50.00) circle (  1.49);

\path[draw=drawColor,line width= 0.4pt,line join=round,line cap=round,fill=fillColor] (168.42, 49.98) circle (  1.49);

\path[draw=drawColor,line width= 0.4pt,line join=round,line cap=round,fill=fillColor] (169.06, 50.02) circle (  1.49);

\path[draw=drawColor,line width= 0.4pt,line join=round,line cap=round,fill=fillColor] (169.55, 50.18) circle (  1.49);

\path[draw=drawColor,line width= 0.4pt,line join=round,line cap=round,fill=fillColor] (170.02, 51.76) circle (  1.49);

\path[draw=drawColor,line width= 0.4pt,line join=round,line cap=round,fill=fillColor] (170.61, 50.23) circle (  1.49);

\path[draw=drawColor,line width= 0.4pt,line join=round,line cap=round,fill=fillColor] (171.10, 50.07) circle (  1.49);

\path[draw=drawColor,line width= 0.4pt,line join=round,line cap=round,fill=fillColor] (171.66, 55.40) circle (  1.49);

\path[draw=drawColor,line width= 0.4pt,line join=round,line cap=round,fill=fillColor] (172.12, 50.04) circle (  1.49);

\path[draw=drawColor,line width= 0.4pt,line join=round,line cap=round,fill=fillColor] (172.64, 50.15) circle (  1.49);

\path[draw=drawColor,line width= 0.4pt,line join=round,line cap=round,fill=fillColor] (173.15, 50.08) circle (  1.49);

\path[draw=drawColor,line width= 0.4pt,line join=round,line cap=round,fill=fillColor] (173.72, 50.10) circle (  1.49);

\path[draw=drawColor,line width= 0.4pt,line join=round,line cap=round,fill=fillColor] (174.31, 50.20) circle (  1.49);

\path[draw=drawColor,line width= 0.4pt,line join=round,line cap=round,fill=fillColor] (174.82, 50.03) circle (  1.49);

\path[draw=drawColor,line width= 0.4pt,line join=round,line cap=round,fill=fillColor] (175.29, 50.02) circle (  1.49);

\path[draw=drawColor,line width= 0.4pt,line join=round,line cap=round,fill=fillColor] (175.87, 50.08) circle (  1.49);

\path[draw=drawColor,line width= 0.4pt,line join=round,line cap=round,fill=fillColor] (176.36, 50.04) circle (  1.49);

\path[draw=drawColor,line width= 0.4pt,line join=round,line cap=round,fill=fillColor] (176.83, 49.97) circle (  1.49);

\path[draw=drawColor,line width= 0.4pt,line join=round,line cap=round,fill=fillColor] (177.31, 50.01) circle (  1.49);

\path[draw=drawColor,line width= 0.4pt,line join=round,line cap=round,fill=fillColor] (177.78, 50.09) circle (  1.49);

\path[draw=drawColor,line width= 0.4pt,line join=round,line cap=round,fill=fillColor] (178.27, 50.12) circle (  1.49);

\path[draw=drawColor,line width= 0.4pt,line join=round,line cap=round,fill=fillColor] (178.75, 50.24) circle (  1.49);

\path[draw=drawColor,line width= 0.4pt,line join=round,line cap=round,fill=fillColor] (179.27, 50.26) circle (  1.49);

\path[draw=drawColor,line width= 0.4pt,line join=round,line cap=round,fill=fillColor] (179.88, 50.23) circle (  1.49);

\path[draw=drawColor,line width= 0.4pt,line join=round,line cap=round,fill=fillColor] (180.45, 50.21) circle (  1.49);

\path[draw=drawColor,line width= 0.4pt,line join=round,line cap=round,fill=fillColor] (180.97, 50.09) circle (  1.49);

\path[draw=drawColor,line width= 0.4pt,line join=round,line cap=round,fill=fillColor] (181.48, 50.11) circle (  1.49);

\path[draw=drawColor,line width= 0.4pt,line join=round,line cap=round,fill=fillColor] (181.94, 50.22) circle (  1.49);

\path[draw=drawColor,line width= 0.4pt,line join=round,line cap=round,fill=fillColor] (182.38, 50.12) circle (  1.49);

\path[draw=drawColor,line width= 0.4pt,line join=round,line cap=round,fill=fillColor] (182.86, 50.15) circle (  1.49);

\path[draw=drawColor,line width= 0.4pt,line join=round,line cap=round,fill=fillColor] (183.31, 50.06) circle (  1.49);

\path[draw=drawColor,line width= 0.4pt,line join=round,line cap=round,fill=fillColor] (183.79, 50.07) circle (  1.49);

\path[draw=drawColor,line width= 0.4pt,line join=round,line cap=round,fill=fillColor] (184.33, 50.51) circle (  1.49);

\path[draw=drawColor,line width= 0.4pt,line join=round,line cap=round,fill=fillColor] (184.89, 50.15) circle (  1.49);

\path[draw=drawColor,line width= 0.4pt,line join=round,line cap=round,fill=fillColor] (185.48, 50.13) circle (  1.49);

\path[draw=drawColor,line width= 0.4pt,line join=round,line cap=round,fill=fillColor] (185.93, 50.17) circle (  1.49);

\path[draw=drawColor,line width= 0.4pt,line join=round,line cap=round,fill=fillColor] (186.41, 50.17) circle (  1.49);

\path[draw=drawColor,line width= 0.4pt,line join=round,line cap=round,fill=fillColor] (186.87, 50.28) circle (  1.49);

\path[draw=drawColor,line width= 0.4pt,line join=round,line cap=round,fill=fillColor] (187.36, 50.03) circle (  1.49);

\path[draw=drawColor,line width= 0.4pt,line join=round,line cap=round,fill=fillColor] (187.91, 50.09) circle (  1.49);

\path[draw=drawColor,line width= 0.4pt,line join=round,line cap=round,fill=fillColor] (188.39, 50.09) circle (  1.49);

\path[draw=drawColor,line width= 0.4pt,line join=round,line cap=round,fill=fillColor] (188.86, 50.12) circle (  1.49);

\path[draw=drawColor,line width= 0.4pt,line join=round,line cap=round,fill=fillColor] (189.32, 50.09) circle (  1.49);

\path[draw=drawColor,line width= 0.4pt,line join=round,line cap=round,fill=fillColor] (189.80, 50.09) circle (  1.49);

\path[draw=drawColor,line width= 0.4pt,line join=round,line cap=round,fill=fillColor] (190.27, 50.09) circle (  1.49);

\path[draw=drawColor,line width= 0.4pt,line join=round,line cap=round,fill=fillColor] (190.75, 50.16) circle (  1.49);

\path[draw=drawColor,line width= 0.4pt,line join=round,line cap=round,fill=fillColor] (191.22, 50.34) circle (  1.49);

\path[draw=drawColor,line width= 0.4pt,line join=round,line cap=round,fill=fillColor] (191.71, 50.25) circle (  1.49);

\path[draw=drawColor,line width= 0.4pt,line join=round,line cap=round,fill=fillColor] (192.17, 50.24) circle (  1.49);

\path[draw=drawColor,line width= 0.4pt,line join=round,line cap=round,fill=fillColor] (192.65, 50.35) circle (  1.49);

\path[draw=drawColor,line width= 0.4pt,line join=round,line cap=round,fill=fillColor] (193.20,177.59) circle (  1.49);

\path[draw=drawColor,line width= 0.4pt,line join=round,line cap=round,fill=fillColor] (193.68, 50.47) circle (  1.49);

\path[draw=drawColor,line width= 0.4pt,line join=round,line cap=round,fill=fillColor] (194.15, 50.51) circle (  1.49);

\path[draw=drawColor,line width= 0.4pt,line join=round,line cap=round,fill=fillColor] (194.64, 50.39) circle (  1.49);

\path[draw=drawColor,line width= 0.4pt,line join=round,line cap=round,fill=fillColor] (195.12, 50.40) circle (  1.49);

\path[draw=drawColor,line width= 0.4pt,line join=round,line cap=round,fill=fillColor] (195.58, 50.45) circle (  1.49);

\path[draw=drawColor,line width= 0.4pt,line join=round,line cap=round,fill=fillColor] (196.03,175.86) circle (  1.49);

\path[draw=drawColor,line width= 0.4pt,line join=round,line cap=round,fill=fillColor] (196.51, 50.77) circle (  1.49);

\path[draw=drawColor,line width= 0.4pt,line join=round,line cap=round,fill=fillColor] (196.97, 50.50) circle (  1.49);

\path[draw=drawColor,line width= 0.4pt,line join=round,line cap=round,fill=fillColor] (197.44, 50.48) circle (  1.49);

\path[draw=drawColor,line width= 0.4pt,line join=round,line cap=round,fill=fillColor] (197.90, 50.59) circle (  1.49);

\path[draw=drawColor,line width= 0.4pt,line join=round,line cap=round,fill=fillColor] (198.34, 71.00) circle (  1.49);

\path[draw=drawColor,line width= 0.4pt,line join=round,line cap=round,fill=fillColor] (198.83, 50.56) circle (  1.49);

\path[draw=drawColor,line width= 0.4pt,line join=round,line cap=round,fill=fillColor] (199.29, 51.64) circle (  1.49);

\path[draw=drawColor,line width= 0.4pt,line join=round,line cap=round,fill=fillColor] (199.77, 53.32) circle (  1.49);

\path[draw=drawColor,line width= 0.4pt,line join=round,line cap=round,fill=fillColor] (200.21,127.20) circle (  1.49);

\path[draw=drawColor,line width= 0.4pt,line join=round,line cap=round,fill=fillColor] (200.67,167.49) circle (  1.49);

\path[draw=drawColor,line width= 0.4pt,line join=round,line cap=round,fill=fillColor] (201.16, 98.25) circle (  1.49);

\path[draw=drawColor,line width= 0.4pt,line join=round,line cap=round,fill=fillColor] (201.62,172.15) circle (  1.49);

\path[draw=drawColor,line width= 0.4pt,line join=round,line cap=round,fill=fillColor] (202.06,175.02) circle (  1.49);

\path[draw=drawColor,line width= 0.4pt,line join=round,line cap=round,fill=fillColor] (202.55,130.28) circle (  1.49);

\path[draw=drawColor,line width= 0.4pt,line join=round,line cap=round,fill=fillColor] (203.01, 90.43) circle (  1.49);

\path[draw=drawColor,line width= 0.4pt,line join=round,line cap=round,fill=fillColor] (203.53,174.57) circle (  1.49);

\path[draw=drawColor,line width= 0.4pt,line join=round,line cap=round,fill=fillColor] (204.02,173.97) circle (  1.49);

\path[draw=drawColor,line width= 0.4pt,line join=round,line cap=round,fill=fillColor] (204.53, 50.52) circle (  1.49);

\path[draw=drawColor,line width= 0.4pt,line join=round,line cap=round,fill=fillColor] (204.97,173.95) circle (  1.49);

\path[draw=drawColor,line width= 0.4pt,line join=round,line cap=round,fill=fillColor] (205.41, 50.32) circle (  1.49);

\path[draw=drawColor,line width= 0.4pt,line join=round,line cap=round,fill=fillColor] (205.99, 50.89) circle (  1.49);

\path[draw=drawColor,line width= 0.4pt,line join=round,line cap=round,fill=fillColor] (206.46, 50.38) circle (  1.49);

\path[draw=drawColor,line width= 0.4pt,line join=round,line cap=round,fill=fillColor] (206.97, 50.29) circle (  1.49);

\path[draw=drawColor,line width= 0.4pt,line join=round,line cap=round,fill=fillColor] (207.44,173.59) circle (  1.49);

\path[draw=drawColor,line width= 0.4pt,line join=round,line cap=round,fill=fillColor] (207.92,173.38) circle (  1.49);

\path[draw=drawColor,line width= 0.4pt,line join=round,line cap=round,fill=fillColor] (208.39,173.73) circle (  1.49);

\path[draw=drawColor,line width= 0.4pt,line join=round,line cap=round,fill=fillColor] (208.87,173.34) circle (  1.49);

\path[draw=drawColor,line width= 0.4pt,line join=round,line cap=round,fill=fillColor] (209.29,173.17) circle (  1.49);

\path[draw=drawColor,line width= 0.4pt,line join=round,line cap=round,fill=fillColor] (209.82,172.92) circle (  1.49);

\path[draw=drawColor,line width= 0.4pt,line join=round,line cap=round,fill=fillColor] (210.26,172.73) circle (  1.49);

\path[draw=drawColor,line width= 0.4pt,line join=round,line cap=round,fill=fillColor] (210.75,172.89) circle (  1.49);

\path[draw=drawColor,line width= 0.4pt,line join=round,line cap=round,fill=fillColor] (211.27,173.00) circle (  1.49);

\path[draw=drawColor,line width= 0.4pt,line join=round,line cap=round,fill=fillColor] (211.75,172.74) circle (  1.49);

\path[draw=drawColor,line width= 0.4pt,line join=round,line cap=round,fill=fillColor] (212.24,172.69) circle (  1.49);

\path[draw=drawColor,line width= 0.4pt,line join=round,line cap=round,fill=fillColor] (212.70,164.20) circle (  1.49);

\path[draw=drawColor,line width= 0.4pt,line join=round,line cap=round,fill=fillColor] (213.21,172.11) circle (  1.49);

\path[draw=drawColor,line width= 0.4pt,line join=round,line cap=round,fill=fillColor] (213.66,171.95) circle (  1.49);

\path[draw=drawColor,line width= 0.4pt,line join=round,line cap=round,fill=fillColor] (214.11,171.95) circle (  1.49);

\path[draw=drawColor,line width= 0.4pt,line join=round,line cap=round,fill=fillColor] (214.65,171.93) circle (  1.49);

\path[draw=drawColor,line width= 0.4pt,line join=round,line cap=round,fill=fillColor] (215.09,172.08) circle (  1.49);

\path[draw=drawColor,line width= 0.4pt,line join=round,line cap=round,fill=fillColor] (215.55,171.73) circle (  1.49);

\path[draw=drawColor,line width= 0.4pt,line join=round,line cap=round,fill=fillColor] (215.99,172.41) circle (  1.49);

\path[draw=drawColor,line width= 0.4pt,line join=round,line cap=round,fill=fillColor] (216.40,171.56) circle (  1.49);

\path[draw=drawColor,line width= 0.4pt,line join=round,line cap=round,fill=fillColor] (216.90,171.78) circle (  1.49);

\path[draw=drawColor,line width= 0.4pt,line join=round,line cap=round,fill=fillColor] (217.40,171.01) circle (  1.49);

\path[draw=drawColor,line width= 0.4pt,line join=round,line cap=round,fill=fillColor] (217.89,167.67) circle (  1.49);

\path[draw=drawColor,line width= 0.4pt,line join=round,line cap=round,fill=fillColor] (218.35, 97.63) circle (  1.49);

\path[draw=drawColor,line width= 0.4pt,line join=round,line cap=round,fill=fillColor] (218.80, 50.80) circle (  1.49);

\path[draw=drawColor,line width= 0.4pt,line join=round,line cap=round,fill=fillColor] (219.33,171.52) circle (  1.49);

\path[draw=drawColor,line width= 0.4pt,line join=round,line cap=round,fill=fillColor] (219.77,170.85) circle (  1.49);

\path[draw=drawColor,line width= 0.4pt,line join=round,line cap=round,fill=fillColor] (220.21,170.96) circle (  1.49);

\path[draw=drawColor,line width= 0.4pt,line join=round,line cap=round,fill=fillColor] (220.74,158.39) circle (  1.49);

\path[draw=drawColor,line width= 0.4pt,line join=round,line cap=round,fill=fillColor] (221.23,171.06) circle (  1.49);

\path[draw=drawColor,line width= 0.4pt,line join=round,line cap=round,fill=fillColor] (221.77,170.68) circle (  1.49);

\path[draw=drawColor,line width= 0.4pt,line join=round,line cap=round,fill=fillColor] (222.21,170.71) circle (  1.49);

\path[draw=drawColor,line width= 0.4pt,line join=round,line cap=round,fill=fillColor] (222.67, 75.57) circle (  1.49);

\path[draw=drawColor,line width= 0.4pt,line join=round,line cap=round,fill=fillColor] (223.14,126.82) circle (  1.49);

\path[draw=drawColor,line width= 0.4pt,line join=round,line cap=round,fill=fillColor] (223.71,170.17) circle (  1.49);

\path[draw=drawColor,line width= 0.4pt,line join=round,line cap=round,fill=fillColor] (224.29,169.73) circle (  1.49);

\path[draw=drawColor,line width= 0.4pt,line join=round,line cap=round,fill=fillColor] (224.78,169.08) circle (  1.49);

\path[draw=drawColor,line width= 0.4pt,line join=round,line cap=round,fill=fillColor] (225.34,169.55) circle (  1.49);

\path[draw=drawColor,line width= 0.4pt,line join=round,line cap=round,fill=fillColor] (225.79,169.41) circle (  1.49);

\path[draw=drawColor,line width= 0.4pt,line join=round,line cap=round,fill=fillColor] (226.25,169.15) circle (  1.49);

\path[draw=drawColor,line width= 0.4pt,line join=round,line cap=round,fill=fillColor] (226.69,168.86) circle (  1.49);

\path[draw=drawColor,line width= 0.4pt,line join=round,line cap=round,fill=fillColor] (227.19,168.82) circle (  1.49);

\path[draw=drawColor,line width= 0.4pt,line join=round,line cap=round,fill=fillColor] (227.61,168.55) circle (  1.49);

\path[draw=drawColor,line width= 0.4pt,line join=round,line cap=round,fill=fillColor] (228.05,168.58) circle (  1.49);

\path[draw=drawColor,line width= 0.4pt,line join=round,line cap=round,fill=fillColor] (228.64,168.22) circle (  1.49);

\path[draw=drawColor,line width= 0.4pt,line join=round,line cap=round,fill=fillColor] (229.26,168.29) circle (  1.49);

\path[draw=drawColor,line width= 0.4pt,line join=round,line cap=round,fill=fillColor] (229.69,168.27) circle (  1.49);

\path[draw=drawColor,line width= 0.4pt,line join=round,line cap=round,fill=fillColor] (230.10,168.21) circle (  1.49);

\path[draw=drawColor,line width= 0.4pt,line join=round,line cap=round,fill=fillColor] (230.52,167.93) circle (  1.49);

\path[draw=drawColor,line width= 0.4pt,line join=round,line cap=round,fill=fillColor] (230.95,167.79) circle (  1.49);
\definecolor{drawColor}{RGB}{0,0,0}
\definecolor{fillColor}{RGB}{255,255,255}

\path[draw=drawColor,line width= 0.4pt,line join=round,line cap=round,fill=fillColor] (159.69,241.56) rectangle (238.26,182.16);
\definecolor{fillColor}{RGB}{0,0,0}

\path[draw=drawColor,line width= 0.4pt,line join=round,line cap=round,fill=fillColor] (168.60,232.65) rectangle (175.72,226.71);
\definecolor{fillColor}{RGB}{255,0,0}

\path[draw=drawColor,line width= 0.4pt,line join=round,line cap=round,fill=fillColor] (168.60,220.77) rectangle (175.72,214.83);
\definecolor{fillColor}{RGB}{0,255,0}

\path[draw=drawColor,line width= 0.4pt,line join=round,line cap=round,fill=fillColor] (168.60,208.89) rectangle (175.72,202.95);
\definecolor{fillColor}{RGB}{0,0,255}

\path[draw=drawColor,line width= 0.4pt,line join=round,line cap=round,fill=fillColor] (168.60,197.01) rectangle (175.72,191.07);

\node[text=drawColor,anchor=base west,inner sep=0pt, outer sep=0pt, scale=  0.99] at (184.63,226.27) {Incident};

\node[text=drawColor,anchor=base west,inner sep=0pt, outer sep=0pt, scale=  0.99] at (184.63,214.39) {F. excelsior};

\node[text=drawColor,anchor=base west,inner sep=0pt, outer sep=0pt, scale=  0.99] at (184.63,202.51) {A. cordata};

\node[text=drawColor,anchor=base west,inner sep=0pt, outer sep=0pt, scale=  0.99] at (184.63,190.63) {M. alba};
\end{scope}
\begin{scope}
\path[clip] (285.78, 47.52) rectangle (476.52,241.56);
\definecolor{drawColor}{RGB}{255,0,0}
\definecolor{fillColor}{RGB}{255,0,0}

\path[draw=drawColor,line width= 0.4pt,line join=round,line cap=round,fill=fillColor] (292.84, 94.81) circle (  1.49);
\definecolor{drawColor}{RGB}{0,0,0}
\definecolor{fillColor}{RGB}{0,0,0}

\path[draw=drawColor,line width= 0.4pt,line join=round,line cap=round,fill=fillColor] (292.86,109.81) circle (  1.49);
\definecolor{drawColor}{RGB}{255,0,0}
\definecolor{fillColor}{RGB}{255,0,0}

\path[draw=drawColor,line width= 0.4pt,line join=round,line cap=round,fill=fillColor] (293.30, 97.79) circle (  1.49);
\definecolor{drawColor}{RGB}{0,0,0}
\definecolor{fillColor}{RGB}{0,0,0}

\path[draw=drawColor,line width= 0.4pt,line join=round,line cap=round,fill=fillColor] (293.32,115.33) circle (  1.49);
\definecolor{drawColor}{RGB}{255,0,0}
\definecolor{fillColor}{RGB}{255,0,0}

\path[draw=drawColor,line width= 0.4pt,line join=round,line cap=round,fill=fillColor] (293.74, 99.22) circle (  1.49);
\definecolor{drawColor}{RGB}{0,0,0}
\definecolor{fillColor}{RGB}{0,0,0}

\path[draw=drawColor,line width= 0.4pt,line join=round,line cap=round,fill=fillColor] (293.76,129.66) circle (  1.49);
\definecolor{drawColor}{RGB}{255,0,0}
\definecolor{fillColor}{RGB}{255,0,0}

\path[draw=drawColor,line width= 0.4pt,line join=round,line cap=round,fill=fillColor] (294.19,100.70) circle (  1.49);
\definecolor{drawColor}{RGB}{0,0,0}
\definecolor{fillColor}{RGB}{0,0,0}

\path[draw=drawColor,line width= 0.4pt,line join=round,line cap=round,fill=fillColor] (294.20,119.95) circle (  1.49);
\definecolor{drawColor}{RGB}{255,0,0}
\definecolor{fillColor}{RGB}{255,0,0}

\path[draw=drawColor,line width= 0.4pt,line join=round,line cap=round,fill=fillColor] (294.65,101.11) circle (  1.49);
\definecolor{drawColor}{RGB}{0,0,0}
\definecolor{fillColor}{RGB}{0,0,0}

\path[draw=drawColor,line width= 0.4pt,line join=round,line cap=round,fill=fillColor] (294.66,114.63) circle (  1.49);
\definecolor{drawColor}{RGB}{255,0,0}
\definecolor{fillColor}{RGB}{255,0,0}

\path[draw=drawColor,line width= 0.4pt,line join=round,line cap=round,fill=fillColor] (295.10, 99.39) circle (  1.49);
\definecolor{drawColor}{RGB}{0,0,0}
\definecolor{fillColor}{RGB}{0,0,0}

\path[draw=drawColor,line width= 0.4pt,line join=round,line cap=round,fill=fillColor] (295.12,108.84) circle (  1.49);
\definecolor{drawColor}{RGB}{255,0,0}
\definecolor{fillColor}{RGB}{255,0,0}

\path[draw=drawColor,line width= 0.4pt,line join=round,line cap=round,fill=fillColor] (295.55, 96.94) circle (  1.49);
\definecolor{drawColor}{RGB}{0,0,0}
\definecolor{fillColor}{RGB}{0,0,0}

\path[draw=drawColor,line width= 0.4pt,line join=round,line cap=round,fill=fillColor] (295.56,105.56) circle (  1.49);
\definecolor{drawColor}{RGB}{255,0,0}
\definecolor{fillColor}{RGB}{255,0,0}

\path[draw=drawColor,line width= 0.4pt,line join=round,line cap=round,fill=fillColor] (296.02, 95.50) circle (  1.49);
\definecolor{drawColor}{RGB}{0,0,0}
\definecolor{fillColor}{RGB}{0,0,0}

\path[draw=drawColor,line width= 0.4pt,line join=round,line cap=round,fill=fillColor] (296.04,104.19) circle (  1.49);
\definecolor{drawColor}{RGB}{255,0,0}
\definecolor{fillColor}{RGB}{255,0,0}

\path[draw=drawColor,line width= 0.4pt,line join=round,line cap=round,fill=fillColor] (296.48, 94.84) circle (  1.49);
\definecolor{drawColor}{RGB}{0,0,0}
\definecolor{fillColor}{RGB}{0,0,0}

\path[draw=drawColor,line width= 0.4pt,line join=round,line cap=round,fill=fillColor] (296.49,104.77) circle (  1.49);
\definecolor{drawColor}{RGB}{255,0,0}
\definecolor{fillColor}{RGB}{255,0,0}

\path[draw=drawColor,line width= 0.4pt,line join=round,line cap=round,fill=fillColor] (296.92, 95.07) circle (  1.49);
\definecolor{drawColor}{RGB}{0,0,0}
\definecolor{fillColor}{RGB}{0,0,0}

\path[draw=drawColor,line width= 0.4pt,line join=round,line cap=round,fill=fillColor] (296.95,106.31) circle (  1.49);
\definecolor{drawColor}{RGB}{255,0,0}
\definecolor{fillColor}{RGB}{255,0,0}

\path[draw=drawColor,line width= 0.4pt,line join=round,line cap=round,fill=fillColor] (297.38, 96.02) circle (  1.49);
\definecolor{drawColor}{RGB}{0,0,0}
\definecolor{fillColor}{RGB}{0,0,0}

\path[draw=drawColor,line width= 0.4pt,line join=round,line cap=round,fill=fillColor] (297.40,108.46) circle (  1.49);
\definecolor{drawColor}{RGB}{255,0,0}
\definecolor{fillColor}{RGB}{255,0,0}

\path[draw=drawColor,line width= 0.4pt,line join=round,line cap=round,fill=fillColor] (297.82, 96.90) circle (  1.49);
\definecolor{drawColor}{RGB}{0,0,0}
\definecolor{fillColor}{RGB}{0,0,0}

\path[draw=drawColor,line width= 0.4pt,line join=round,line cap=round,fill=fillColor] (297.84,110.52) circle (  1.49);
\definecolor{drawColor}{RGB}{255,0,0}
\definecolor{fillColor}{RGB}{255,0,0}

\path[draw=drawColor,line width= 0.4pt,line join=round,line cap=round,fill=fillColor] (298.30, 98.28) circle (  1.49);
\definecolor{drawColor}{RGB}{0,0,0}
\definecolor{fillColor}{RGB}{0,0,0}

\path[draw=drawColor,line width= 0.4pt,line join=round,line cap=round,fill=fillColor] (298.31,110.63) circle (  1.49);
\definecolor{drawColor}{RGB}{255,0,0}
\definecolor{fillColor}{RGB}{255,0,0}

\path[draw=drawColor,line width= 0.4pt,line join=round,line cap=round,fill=fillColor] (298.75, 99.00) circle (  1.49);
\definecolor{drawColor}{RGB}{0,0,0}
\definecolor{fillColor}{RGB}{0,0,0}

\path[draw=drawColor,line width= 0.4pt,line join=round,line cap=round,fill=fillColor] (298.77,110.67) circle (  1.49);
\definecolor{drawColor}{RGB}{255,0,0}
\definecolor{fillColor}{RGB}{255,0,0}

\path[draw=drawColor,line width= 0.4pt,line join=round,line cap=round,fill=fillColor] (299.21, 99.11) circle (  1.49);
\definecolor{drawColor}{RGB}{0,0,0}
\definecolor{fillColor}{RGB}{0,0,0}

\path[draw=drawColor,line width= 0.4pt,line join=round,line cap=round,fill=fillColor] (299.23,112.46) circle (  1.49);
\definecolor{drawColor}{RGB}{255,0,0}
\definecolor{fillColor}{RGB}{255,0,0}

\path[draw=drawColor,line width= 0.4pt,line join=round,line cap=round,fill=fillColor] (299.67, 99.19) circle (  1.49);
\definecolor{drawColor}{RGB}{0,0,0}
\definecolor{fillColor}{RGB}{0,0,0}

\path[draw=drawColor,line width= 0.4pt,line join=round,line cap=round,fill=fillColor] (299.69,106.53) circle (  1.49);
\definecolor{drawColor}{RGB}{255,0,0}
\definecolor{fillColor}{RGB}{255,0,0}

\path[draw=drawColor,line width= 0.4pt,line join=round,line cap=round,fill=fillColor] (300.18, 99.61) circle (  1.49);
\definecolor{drawColor}{RGB}{0,0,0}
\definecolor{fillColor}{RGB}{0,0,0}

\path[draw=drawColor,line width= 0.4pt,line join=round,line cap=round,fill=fillColor] (300.19,116.28) circle (  1.49);
\definecolor{drawColor}{RGB}{255,0,0}
\definecolor{fillColor}{RGB}{255,0,0}

\path[draw=drawColor,line width= 0.4pt,line join=round,line cap=round,fill=fillColor] (300.64,100.42) circle (  1.49);
\definecolor{drawColor}{RGB}{0,0,0}
\definecolor{fillColor}{RGB}{0,0,0}

\path[draw=drawColor,line width= 0.4pt,line join=round,line cap=round,fill=fillColor] (300.65,118.74) circle (  1.49);
\definecolor{drawColor}{RGB}{255,0,0}
\definecolor{fillColor}{RGB}{255,0,0}

\path[draw=drawColor,line width= 0.4pt,line join=round,line cap=round,fill=fillColor] (301.09,101.18) circle (  1.49);
\definecolor{drawColor}{RGB}{0,0,0}
\definecolor{fillColor}{RGB}{0,0,0}

\path[draw=drawColor,line width= 0.4pt,line join=round,line cap=round,fill=fillColor] (301.11,123.14) circle (  1.49);
\definecolor{drawColor}{RGB}{255,0,0}
\definecolor{fillColor}{RGB}{255,0,0}

\path[draw=drawColor,line width= 0.4pt,line join=round,line cap=round,fill=fillColor] (301.54,101.72) circle (  1.49);
\definecolor{drawColor}{RGB}{0,0,0}
\definecolor{fillColor}{RGB}{0,0,0}

\path[draw=drawColor,line width= 0.4pt,line join=round,line cap=round,fill=fillColor] (301.55,119.60) circle (  1.49);
\definecolor{drawColor}{RGB}{255,0,0}
\definecolor{fillColor}{RGB}{255,0,0}

\path[draw=drawColor,line width= 0.4pt,line join=round,line cap=round,fill=fillColor] (302.00,102.18) circle (  1.49);
\definecolor{drawColor}{RGB}{0,0,0}
\definecolor{fillColor}{RGB}{0,0,0}

\path[draw=drawColor,line width= 0.4pt,line join=round,line cap=round,fill=fillColor] (302.01,114.79) circle (  1.49);
\definecolor{drawColor}{RGB}{255,0,0}
\definecolor{fillColor}{RGB}{255,0,0}

\path[draw=drawColor,line width= 0.4pt,line join=round,line cap=round,fill=fillColor] (302.45,100.60) circle (  1.49);
\definecolor{drawColor}{RGB}{0,0,0}
\definecolor{fillColor}{RGB}{0,0,0}

\path[draw=drawColor,line width= 0.4pt,line join=round,line cap=round,fill=fillColor] (302.47,113.67) circle (  1.49);
\definecolor{drawColor}{RGB}{255,0,0}
\definecolor{fillColor}{RGB}{255,0,0}

\path[draw=drawColor,line width= 0.4pt,line join=round,line cap=round,fill=fillColor] (302.91, 99.78) circle (  1.49);
\definecolor{drawColor}{RGB}{0,0,0}
\definecolor{fillColor}{RGB}{0,0,0}

\path[draw=drawColor,line width= 0.4pt,line join=round,line cap=round,fill=fillColor] (302.93,115.25) circle (  1.49);
\definecolor{drawColor}{RGB}{255,0,0}
\definecolor{fillColor}{RGB}{255,0,0}

\path[draw=drawColor,line width= 0.4pt,line join=round,line cap=round,fill=fillColor] (303.39, 99.82) circle (  1.49);
\definecolor{drawColor}{RGB}{0,0,0}
\definecolor{fillColor}{RGB}{0,0,0}

\path[draw=drawColor,line width= 0.4pt,line join=round,line cap=round,fill=fillColor] (303.40,118.34) circle (  1.49);
\definecolor{drawColor}{RGB}{255,0,0}
\definecolor{fillColor}{RGB}{255,0,0}

\path[draw=drawColor,line width= 0.4pt,line join=round,line cap=round,fill=fillColor] (303.84,100.31) circle (  1.49);
\definecolor{drawColor}{RGB}{0,0,0}
\definecolor{fillColor}{RGB}{0,0,0}

\path[draw=drawColor,line width= 0.4pt,line join=round,line cap=round,fill=fillColor] (303.86,117.95) circle (  1.49);
\definecolor{drawColor}{RGB}{255,0,0}
\definecolor{fillColor}{RGB}{255,0,0}

\path[draw=drawColor,line width= 0.4pt,line join=round,line cap=round,fill=fillColor] (304.29,100.69) circle (  1.49);
\definecolor{drawColor}{RGB}{0,0,0}
\definecolor{fillColor}{RGB}{0,0,0}

\path[draw=drawColor,line width= 0.4pt,line join=round,line cap=round,fill=fillColor] (304.30,117.38) circle (  1.49);
\definecolor{drawColor}{RGB}{255,0,0}
\definecolor{fillColor}{RGB}{255,0,0}

\path[draw=drawColor,line width= 0.4pt,line join=round,line cap=round,fill=fillColor] (304.75, 99.50) circle (  1.49);
\definecolor{drawColor}{RGB}{0,0,0}
\definecolor{fillColor}{RGB}{0,0,0}

\path[draw=drawColor,line width= 0.4pt,line join=round,line cap=round,fill=fillColor] (304.78,116.21) circle (  1.49);
\definecolor{drawColor}{RGB}{255,0,0}
\definecolor{fillColor}{RGB}{255,0,0}

\path[draw=drawColor,line width= 0.4pt,line join=round,line cap=round,fill=fillColor] (305.20, 99.19) circle (  1.49);
\definecolor{drawColor}{RGB}{0,0,0}
\definecolor{fillColor}{RGB}{0,0,0}

\path[draw=drawColor,line width= 0.4pt,line join=round,line cap=round,fill=fillColor] (305.22,112.52) circle (  1.49);
\definecolor{drawColor}{RGB}{255,0,0}
\definecolor{fillColor}{RGB}{255,0,0}

\path[draw=drawColor,line width= 0.4pt,line join=round,line cap=round,fill=fillColor] (305.66, 99.77) circle (  1.49);
\definecolor{drawColor}{RGB}{0,0,0}
\definecolor{fillColor}{RGB}{0,0,0}

\path[draw=drawColor,line width= 0.4pt,line join=round,line cap=round,fill=fillColor] (305.68,110.74) circle (  1.49);
\definecolor{drawColor}{RGB}{255,0,0}
\definecolor{fillColor}{RGB}{255,0,0}

\path[draw=drawColor,line width= 0.4pt,line join=round,line cap=round,fill=fillColor] (306.15, 98.95) circle (  1.49);
\definecolor{drawColor}{RGB}{0,0,0}
\definecolor{fillColor}{RGB}{0,0,0}

\path[draw=drawColor,line width= 0.4pt,line join=round,line cap=round,fill=fillColor] (306.17,109.41) circle (  1.49);
\definecolor{drawColor}{RGB}{255,0,0}
\definecolor{fillColor}{RGB}{255,0,0}

\path[draw=drawColor,line width= 0.4pt,line join=round,line cap=round,fill=fillColor] (306.61, 98.81) circle (  1.49);
\definecolor{drawColor}{RGB}{0,0,0}
\definecolor{fillColor}{RGB}{0,0,0}

\path[draw=drawColor,line width= 0.4pt,line join=round,line cap=round,fill=fillColor] (306.63,108.80) circle (  1.49);
\definecolor{drawColor}{RGB}{255,0,0}
\definecolor{fillColor}{RGB}{255,0,0}

\path[draw=drawColor,line width= 0.4pt,line join=round,line cap=round,fill=fillColor] (307.07, 98.41) circle (  1.49);
\definecolor{drawColor}{RGB}{0,0,0}
\definecolor{fillColor}{RGB}{0,0,0}

\path[draw=drawColor,line width= 0.4pt,line join=round,line cap=round,fill=fillColor] (307.09,108.86) circle (  1.49);
\definecolor{drawColor}{RGB}{255,0,0}
\definecolor{fillColor}{RGB}{255,0,0}

\path[draw=drawColor,line width= 0.4pt,line join=round,line cap=round,fill=fillColor] (307.53, 98.35) circle (  1.49);
\definecolor{drawColor}{RGB}{0,0,0}
\definecolor{fillColor}{RGB}{0,0,0}

\path[draw=drawColor,line width= 0.4pt,line join=round,line cap=round,fill=fillColor] (307.54,103.61) circle (  1.49);
\definecolor{drawColor}{RGB}{255,0,0}
\definecolor{fillColor}{RGB}{255,0,0}

\path[draw=drawColor,line width= 0.4pt,line join=round,line cap=round,fill=fillColor] (307.99, 98.99) circle (  1.49);
\definecolor{drawColor}{RGB}{0,0,0}
\definecolor{fillColor}{RGB}{0,0,0}

\path[draw=drawColor,line width= 0.4pt,line join=round,line cap=round,fill=fillColor] (308.00,111.33) circle (  1.49);
\definecolor{drawColor}{RGB}{255,0,0}
\definecolor{fillColor}{RGB}{255,0,0}

\path[draw=drawColor,line width= 0.4pt,line join=round,line cap=round,fill=fillColor] (308.44, 99.52) circle (  1.49);
\definecolor{drawColor}{RGB}{0,0,0}
\definecolor{fillColor}{RGB}{0,0,0}

\path[draw=drawColor,line width= 0.4pt,line join=round,line cap=round,fill=fillColor] (308.46,115.42) circle (  1.49);
\definecolor{drawColor}{RGB}{255,0,0}
\definecolor{fillColor}{RGB}{255,0,0}

\path[draw=drawColor,line width= 0.4pt,line join=round,line cap=round,fill=fillColor] (308.90,100.46) circle (  1.49);
\definecolor{drawColor}{RGB}{0,0,0}
\definecolor{fillColor}{RGB}{0,0,0}

\path[draw=drawColor,line width= 0.4pt,line join=round,line cap=round,fill=fillColor] (308.92,119.28) circle (  1.49);
\definecolor{drawColor}{RGB}{255,0,0}
\definecolor{fillColor}{RGB}{255,0,0}

\path[draw=drawColor,line width= 0.4pt,line join=round,line cap=round,fill=fillColor] (309.36,102.69) circle (  1.49);
\definecolor{drawColor}{RGB}{0,0,0}
\definecolor{fillColor}{RGB}{0,0,0}

\path[draw=drawColor,line width= 0.4pt,line join=round,line cap=round,fill=fillColor] (309.38,125.60) circle (  1.49);
\definecolor{drawColor}{RGB}{255,0,0}
\definecolor{fillColor}{RGB}{255,0,0}

\path[draw=drawColor,line width= 0.4pt,line join=round,line cap=round,fill=fillColor] (309.82,102.57) circle (  1.49);
\definecolor{drawColor}{RGB}{0,0,0}
\definecolor{fillColor}{RGB}{0,0,0}

\path[draw=drawColor,line width= 0.4pt,line join=round,line cap=round,fill=fillColor] (309.84,125.25) circle (  1.49);
\definecolor{drawColor}{RGB}{255,0,0}
\definecolor{fillColor}{RGB}{255,0,0}

\path[draw=drawColor,line width= 0.4pt,line join=round,line cap=round,fill=fillColor] (310.28,104.90) circle (  1.49);
\definecolor{drawColor}{RGB}{0,0,0}
\definecolor{fillColor}{RGB}{0,0,0}

\path[draw=drawColor,line width= 0.4pt,line join=round,line cap=round,fill=fillColor] (310.29,128.42) circle (  1.49);
\definecolor{drawColor}{RGB}{255,0,0}
\definecolor{fillColor}{RGB}{255,0,0}

\path[draw=drawColor,line width= 0.4pt,line join=round,line cap=round,fill=fillColor] (310.72,105.58) circle (  1.49);
\definecolor{drawColor}{RGB}{0,0,0}
\definecolor{fillColor}{RGB}{0,0,0}

\path[draw=drawColor,line width= 0.4pt,line join=round,line cap=round,fill=fillColor] (310.75,128.36) circle (  1.49);
\definecolor{drawColor}{RGB}{255,0,0}
\definecolor{fillColor}{RGB}{255,0,0}

\path[draw=drawColor,line width= 0.4pt,line join=round,line cap=round,fill=fillColor] (311.18,106.68) circle (  1.49);
\definecolor{drawColor}{RGB}{0,0,0}
\definecolor{fillColor}{RGB}{0,0,0}

\path[draw=drawColor,line width= 0.4pt,line join=round,line cap=round,fill=fillColor] (311.19,130.89) circle (  1.49);
\definecolor{drawColor}{RGB}{255,0,0}
\definecolor{fillColor}{RGB}{255,0,0}

\path[draw=drawColor,line width= 0.4pt,line join=round,line cap=round,fill=fillColor] (311.62,107.64) circle (  1.49);
\definecolor{drawColor}{RGB}{0,0,0}
\definecolor{fillColor}{RGB}{0,0,0}

\path[draw=drawColor,line width= 0.4pt,line join=round,line cap=round,fill=fillColor] (311.64,144.84) circle (  1.49);
\definecolor{drawColor}{RGB}{255,0,0}
\definecolor{fillColor}{RGB}{255,0,0}

\path[draw=drawColor,line width= 0.4pt,line join=round,line cap=round,fill=fillColor] (312.08,107.96) circle (  1.49);
\definecolor{drawColor}{RGB}{0,0,0}
\definecolor{fillColor}{RGB}{0,0,0}

\path[draw=drawColor,line width= 0.4pt,line join=round,line cap=round,fill=fillColor] (312.10,154.24) circle (  1.49);
\definecolor{drawColor}{RGB}{255,0,0}
\definecolor{fillColor}{RGB}{255,0,0}

\path[draw=drawColor,line width= 0.4pt,line join=round,line cap=round,fill=fillColor] (312.52,108.90) circle (  1.49);
\definecolor{drawColor}{RGB}{0,0,0}
\definecolor{fillColor}{RGB}{0,0,0}

\path[draw=drawColor,line width= 0.4pt,line join=round,line cap=round,fill=fillColor] (312.54,174.32) circle (  1.49);
\definecolor{drawColor}{RGB}{255,0,0}
\definecolor{fillColor}{RGB}{255,0,0}

\path[draw=drawColor,line width= 0.4pt,line join=round,line cap=round,fill=fillColor] (312.96,107.88) circle (  1.49);
\definecolor{drawColor}{RGB}{0,0,0}
\definecolor{fillColor}{RGB}{0,0,0}

\path[draw=drawColor,line width= 0.4pt,line join=round,line cap=round,fill=fillColor] (312.98,142.21) circle (  1.49);
\definecolor{drawColor}{RGB}{255,0,0}
\definecolor{fillColor}{RGB}{255,0,0}

\path[draw=drawColor,line width= 0.4pt,line join=round,line cap=round,fill=fillColor] (313.42,105.98) circle (  1.49);
\definecolor{drawColor}{RGB}{0,0,0}
\definecolor{fillColor}{RGB}{0,0,0}

\path[draw=drawColor,line width= 0.4pt,line join=round,line cap=round,fill=fillColor] (313.44,136.50) circle (  1.49);
\definecolor{drawColor}{RGB}{255,0,0}
\definecolor{fillColor}{RGB}{255,0,0}

\path[draw=drawColor,line width= 0.4pt,line join=round,line cap=round,fill=fillColor] (313.86,104.53) circle (  1.49);
\definecolor{drawColor}{RGB}{0,0,0}
\definecolor{fillColor}{RGB}{0,0,0}

\path[draw=drawColor,line width= 0.4pt,line join=round,line cap=round,fill=fillColor] (313.88,132.30) circle (  1.49);
\definecolor{drawColor}{RGB}{255,0,0}
\definecolor{fillColor}{RGB}{255,0,0}

\path[draw=drawColor,line width= 0.4pt,line join=round,line cap=round,fill=fillColor] (314.32,102.46) circle (  1.49);
\definecolor{drawColor}{RGB}{0,0,0}
\definecolor{fillColor}{RGB}{0,0,0}

\path[draw=drawColor,line width= 0.4pt,line join=round,line cap=round,fill=fillColor] (314.34,155.40) circle (  1.49);
\definecolor{drawColor}{RGB}{255,0,0}
\definecolor{fillColor}{RGB}{255,0,0}

\path[draw=drawColor,line width= 0.4pt,line join=round,line cap=round,fill=fillColor] (314.78,101.56) circle (  1.49);
\definecolor{drawColor}{RGB}{0,0,0}
\definecolor{fillColor}{RGB}{0,0,0}

\path[draw=drawColor,line width= 0.4pt,line join=round,line cap=round,fill=fillColor] (314.80,232.23) circle (  1.49);
\definecolor{drawColor}{RGB}{255,0,0}
\definecolor{fillColor}{RGB}{255,0,0}

\path[draw=drawColor,line width= 0.4pt,line join=round,line cap=round,fill=fillColor] (315.22, 99.89) circle (  1.49);
\definecolor{drawColor}{RGB}{0,0,0}
\definecolor{fillColor}{RGB}{0,0,0}

\path[draw=drawColor,line width= 0.4pt,line join=round,line cap=round,fill=fillColor] (315.24,159.14) circle (  1.49);
\definecolor{drawColor}{RGB}{255,0,0}
\definecolor{fillColor}{RGB}{255,0,0}

\path[draw=drawColor,line width= 0.4pt,line join=round,line cap=round,fill=fillColor] (315.70, 98.30) circle (  1.49);
\definecolor{drawColor}{RGB}{0,0,0}
\definecolor{fillColor}{RGB}{0,0,0}

\path[draw=drawColor,line width= 0.4pt,line join=round,line cap=round,fill=fillColor] (315.71,234.37) circle (  1.49);
\definecolor{drawColor}{RGB}{255,0,0}
\definecolor{fillColor}{RGB}{255,0,0}

\path[draw=drawColor,line width= 0.4pt,line join=round,line cap=round,fill=fillColor] (316.20, 96.82) circle (  1.49);
\definecolor{drawColor}{RGB}{0,0,0}
\definecolor{fillColor}{RGB}{0,0,0}

\path[draw=drawColor,line width= 0.4pt,line join=round,line cap=round,fill=fillColor] (316.24,231.53) circle (  1.49);
\definecolor{drawColor}{RGB}{255,0,0}
\definecolor{fillColor}{RGB}{255,0,0}

\path[draw=drawColor,line width= 0.4pt,line join=round,line cap=round,fill=fillColor] (316.66, 95.12) circle (  1.49);
\definecolor{drawColor}{RGB}{0,0,0}
\definecolor{fillColor}{RGB}{0,0,0}

\path[draw=drawColor,line width= 0.4pt,line join=round,line cap=round,fill=fillColor] (316.68,212.25) circle (  1.49);
\definecolor{drawColor}{RGB}{255,0,0}
\definecolor{fillColor}{RGB}{255,0,0}

\path[draw=drawColor,line width= 0.4pt,line join=round,line cap=round,fill=fillColor] (317.10, 93.88) circle (  1.49);
\definecolor{drawColor}{RGB}{0,0,0}
\definecolor{fillColor}{RGB}{0,0,0}

\path[draw=drawColor,line width= 0.4pt,line join=round,line cap=round,fill=fillColor] (317.12,168.39) circle (  1.49);
\definecolor{drawColor}{RGB}{255,0,0}
\definecolor{fillColor}{RGB}{255,0,0}

\path[draw=drawColor,line width= 0.4pt,line join=round,line cap=round,fill=fillColor] (317.56, 90.12) circle (  1.49);
\definecolor{drawColor}{RGB}{0,0,0}
\definecolor{fillColor}{RGB}{0,0,0}

\path[draw=drawColor,line width= 0.4pt,line join=round,line cap=round,fill=fillColor] (317.58,141.18) circle (  1.49);
\definecolor{drawColor}{RGB}{255,0,0}
\definecolor{fillColor}{RGB}{255,0,0}

\path[draw=drawColor,line width= 0.4pt,line join=round,line cap=round,fill=fillColor] (318.04, 89.21) circle (  1.49);
\definecolor{drawColor}{RGB}{0,0,0}
\definecolor{fillColor}{RGB}{0,0,0}

\path[draw=drawColor,line width= 0.4pt,line join=round,line cap=round,fill=fillColor] (318.05,109.46) circle (  1.49);
\definecolor{drawColor}{RGB}{255,0,0}
\definecolor{fillColor}{RGB}{255,0,0}

\path[draw=drawColor,line width= 0.4pt,line join=round,line cap=round,fill=fillColor] (318.51, 89.35) circle (  1.49);
\definecolor{drawColor}{RGB}{0,0,0}
\definecolor{fillColor}{RGB}{0,0,0}

\path[draw=drawColor,line width= 0.4pt,line join=round,line cap=round,fill=fillColor] (318.53,108.35) circle (  1.49);
\definecolor{drawColor}{RGB}{255,0,0}
\definecolor{fillColor}{RGB}{255,0,0}

\path[draw=drawColor,line width= 0.4pt,line join=round,line cap=round,fill=fillColor] (319.02, 88.69) circle (  1.49);
\definecolor{drawColor}{RGB}{0,0,0}
\definecolor{fillColor}{RGB}{0,0,0}

\path[draw=drawColor,line width= 0.4pt,line join=round,line cap=round,fill=fillColor] (319.04,110.02) circle (  1.49);
\definecolor{drawColor}{RGB}{255,0,0}
\definecolor{fillColor}{RGB}{255,0,0}

\path[draw=drawColor,line width= 0.4pt,line join=round,line cap=round,fill=fillColor] (319.51, 87.12) circle (  1.49);
\definecolor{drawColor}{RGB}{0,0,0}
\definecolor{fillColor}{RGB}{0,0,0}

\path[draw=drawColor,line width= 0.4pt,line join=round,line cap=round,fill=fillColor] (319.53,118.81) circle (  1.49);
\definecolor{drawColor}{RGB}{255,0,0}
\definecolor{fillColor}{RGB}{255,0,0}

\path[draw=drawColor,line width= 0.4pt,line join=round,line cap=round,fill=fillColor] (319.99, 84.53) circle (  1.49);
\definecolor{drawColor}{RGB}{0,0,0}
\definecolor{fillColor}{RGB}{0,0,0}

\path[draw=drawColor,line width= 0.4pt,line join=round,line cap=round,fill=fillColor] (320.00,117.51) circle (  1.49);
\definecolor{drawColor}{RGB}{255,0,0}
\definecolor{fillColor}{RGB}{255,0,0}

\path[draw=drawColor,line width= 0.4pt,line join=round,line cap=round,fill=fillColor] (320.46, 82.06) circle (  1.49);
\definecolor{drawColor}{RGB}{0,0,0}
\definecolor{fillColor}{RGB}{0,0,0}

\path[draw=drawColor,line width= 0.4pt,line join=round,line cap=round,fill=fillColor] (320.48,142.23) circle (  1.49);
\definecolor{drawColor}{RGB}{255,0,0}
\definecolor{fillColor}{RGB}{255,0,0}

\path[draw=drawColor,line width= 0.4pt,line join=round,line cap=round,fill=fillColor] (320.95, 80.06) circle (  1.49);
\definecolor{drawColor}{RGB}{0,0,0}
\definecolor{fillColor}{RGB}{0,0,0}

\path[draw=drawColor,line width= 0.4pt,line join=round,line cap=round,fill=fillColor] (320.97,206.49) circle (  1.49);
\definecolor{drawColor}{RGB}{255,0,0}
\definecolor{fillColor}{RGB}{255,0,0}

\path[draw=drawColor,line width= 0.4pt,line join=round,line cap=round,fill=fillColor] (321.43, 80.41) circle (  1.49);
\definecolor{drawColor}{RGB}{0,0,0}
\definecolor{fillColor}{RGB}{0,0,0}

\path[draw=drawColor,line width= 0.4pt,line join=round,line cap=round,fill=fillColor] (321.44,216.46) circle (  1.49);
\definecolor{drawColor}{RGB}{255,0,0}
\definecolor{fillColor}{RGB}{255,0,0}

\path[draw=drawColor,line width= 0.4pt,line join=round,line cap=round,fill=fillColor] (321.88, 79.35) circle (  1.49);
\definecolor{drawColor}{RGB}{0,0,0}
\definecolor{fillColor}{RGB}{0,0,0}

\path[draw=drawColor,line width= 0.4pt,line join=round,line cap=round,fill=fillColor] (321.90,207.30) circle (  1.49);
\definecolor{drawColor}{RGB}{255,0,0}
\definecolor{fillColor}{RGB}{255,0,0}

\path[draw=drawColor,line width= 0.4pt,line join=round,line cap=round,fill=fillColor] (322.36, 77.09) circle (  1.49);
\definecolor{drawColor}{RGB}{0,0,0}
\definecolor{fillColor}{RGB}{0,0,0}

\path[draw=drawColor,line width= 0.4pt,line join=round,line cap=round,fill=fillColor] (322.38,199.75) circle (  1.49);
\definecolor{drawColor}{RGB}{255,0,0}
\definecolor{fillColor}{RGB}{255,0,0}

\path[draw=drawColor,line width= 0.4pt,line join=round,line cap=round,fill=fillColor] (322.83, 77.05) circle (  1.49);
\definecolor{drawColor}{RGB}{0,0,0}
\definecolor{fillColor}{RGB}{0,0,0}

\path[draw=drawColor,line width= 0.4pt,line join=round,line cap=round,fill=fillColor] (322.85,197.16) circle (  1.49);
\definecolor{drawColor}{RGB}{255,0,0}
\definecolor{fillColor}{RGB}{255,0,0}

\path[draw=drawColor,line width= 0.4pt,line join=round,line cap=round,fill=fillColor] (323.31, 73.82) circle (  1.49);
\definecolor{drawColor}{RGB}{0,0,0}
\definecolor{fillColor}{RGB}{0,0,0}

\path[draw=drawColor,line width= 0.4pt,line join=round,line cap=round,fill=fillColor] (323.32,197.30) circle (  1.49);
\definecolor{drawColor}{RGB}{255,0,0}
\definecolor{fillColor}{RGB}{255,0,0}

\path[draw=drawColor,line width= 0.4pt,line join=round,line cap=round,fill=fillColor] (323.83, 71.75) circle (  1.49);
\definecolor{drawColor}{RGB}{0,0,0}
\definecolor{fillColor}{RGB}{0,0,0}

\path[draw=drawColor,line width= 0.4pt,line join=round,line cap=round,fill=fillColor] (323.85,190.16) circle (  1.49);
\definecolor{drawColor}{RGB}{255,0,0}
\definecolor{fillColor}{RGB}{255,0,0}

\path[draw=drawColor,line width= 0.4pt,line join=round,line cap=round,fill=fillColor] (324.32, 70.19) circle (  1.49);
\definecolor{drawColor}{RGB}{0,0,0}
\definecolor{fillColor}{RGB}{0,0,0}

\path[draw=drawColor,line width= 0.4pt,line join=round,line cap=round,fill=fillColor] (324.34,196.06) circle (  1.49);
\definecolor{drawColor}{RGB}{255,0,0}
\definecolor{fillColor}{RGB}{255,0,0}

\path[draw=drawColor,line width= 0.4pt,line join=round,line cap=round,fill=fillColor] (324.80, 69.44) circle (  1.49);
\definecolor{drawColor}{RGB}{0,0,0}
\definecolor{fillColor}{RGB}{0,0,0}

\path[draw=drawColor,line width= 0.4pt,line join=round,line cap=round,fill=fillColor] (324.81,182.13) circle (  1.49);
\definecolor{drawColor}{RGB}{255,0,0}
\definecolor{fillColor}{RGB}{255,0,0}

\path[draw=drawColor,line width= 0.4pt,line join=round,line cap=round,fill=fillColor] (325.35, 69.73) circle (  1.49);
\definecolor{drawColor}{RGB}{0,0,0}
\definecolor{fillColor}{RGB}{0,0,0}

\path[draw=drawColor,line width= 0.4pt,line join=round,line cap=round,fill=fillColor] (325.37,199.85) circle (  1.49);
\definecolor{drawColor}{RGB}{255,0,0}
\definecolor{fillColor}{RGB}{255,0,0}

\path[draw=drawColor,line width= 0.4pt,line join=round,line cap=round,fill=fillColor] (325.85, 70.57) circle (  1.49);
\definecolor{drawColor}{RGB}{0,0,0}
\definecolor{fillColor}{RGB}{0,0,0}

\path[draw=drawColor,line width= 0.4pt,line join=round,line cap=round,fill=fillColor] (325.86,205.18) circle (  1.49);
\definecolor{drawColor}{RGB}{255,0,0}
\definecolor{fillColor}{RGB}{255,0,0}

\path[draw=drawColor,line width= 0.4pt,line join=round,line cap=round,fill=fillColor] (326.34, 71.46) circle (  1.49);
\definecolor{drawColor}{RGB}{0,0,0}
\definecolor{fillColor}{RGB}{0,0,0}

\path[draw=drawColor,line width= 0.4pt,line join=round,line cap=round,fill=fillColor] (326.35,203.42) circle (  1.49);
\definecolor{drawColor}{RGB}{255,0,0}
\definecolor{fillColor}{RGB}{255,0,0}

\path[draw=drawColor,line width= 0.4pt,line join=round,line cap=round,fill=fillColor] (326.86, 71.19) circle (  1.49);
\definecolor{drawColor}{RGB}{0,0,0}
\definecolor{fillColor}{RGB}{0,0,0}

\path[draw=drawColor,line width= 0.4pt,line join=round,line cap=round,fill=fillColor] (326.88,203.51) circle (  1.49);
\definecolor{drawColor}{RGB}{255,0,0}
\definecolor{fillColor}{RGB}{255,0,0}

\path[draw=drawColor,line width= 0.4pt,line join=round,line cap=round,fill=fillColor] (327.34, 71.83) circle (  1.49);
\definecolor{drawColor}{RGB}{0,0,0}
\definecolor{fillColor}{RGB}{0,0,0}

\path[draw=drawColor,line width= 0.4pt,line join=round,line cap=round,fill=fillColor] (327.35,131.31) circle (  1.49);
\definecolor{drawColor}{RGB}{255,0,0}
\definecolor{fillColor}{RGB}{255,0,0}

\path[draw=drawColor,line width= 0.4pt,line join=round,line cap=round,fill=fillColor] (327.81, 72.32) circle (  1.49);
\definecolor{drawColor}{RGB}{0,0,0}
\definecolor{fillColor}{RGB}{0,0,0}

\path[draw=drawColor,line width= 0.4pt,line join=round,line cap=round,fill=fillColor] (327.84,177.74) circle (  1.49);
\definecolor{drawColor}{RGB}{255,0,0}
\definecolor{fillColor}{RGB}{255,0,0}

\path[draw=drawColor,line width= 0.4pt,line join=round,line cap=round,fill=fillColor] (328.30, 72.59) circle (  1.49);
\definecolor{drawColor}{RGB}{0,0,0}
\definecolor{fillColor}{RGB}{0,0,0}

\path[draw=drawColor,line width= 0.4pt,line join=round,line cap=round,fill=fillColor] (328.33,186.30) circle (  1.49);
\definecolor{drawColor}{RGB}{255,0,0}
\definecolor{fillColor}{RGB}{255,0,0}

\path[draw=drawColor,line width= 0.4pt,line join=round,line cap=round,fill=fillColor] (328.79, 73.32) circle (  1.49);
\definecolor{drawColor}{RGB}{0,0,0}
\definecolor{fillColor}{RGB}{0,0,0}

\path[draw=drawColor,line width= 0.4pt,line join=round,line cap=round,fill=fillColor] (328.81,186.84) circle (  1.49);
\definecolor{drawColor}{RGB}{255,0,0}
\definecolor{fillColor}{RGB}{255,0,0}

\path[draw=drawColor,line width= 0.4pt,line join=round,line cap=round,fill=fillColor] (329.27, 73.99) circle (  1.49);
\definecolor{drawColor}{RGB}{0,0,0}
\definecolor{fillColor}{RGB}{0,0,0}

\path[draw=drawColor,line width= 0.4pt,line join=round,line cap=round,fill=fillColor] (329.28,207.18) circle (  1.49);
\definecolor{drawColor}{RGB}{255,0,0}
\definecolor{fillColor}{RGB}{255,0,0}

\path[draw=drawColor,line width= 0.4pt,line join=round,line cap=round,fill=fillColor] (329.74, 76.69) circle (  1.49);
\definecolor{drawColor}{RGB}{0,0,0}
\definecolor{fillColor}{RGB}{0,0,0}

\path[draw=drawColor,line width= 0.4pt,line join=round,line cap=round,fill=fillColor] (329.76,209.25) circle (  1.49);
\definecolor{drawColor}{RGB}{255,0,0}
\definecolor{fillColor}{RGB}{255,0,0}

\path[draw=drawColor,line width= 0.4pt,line join=round,line cap=round,fill=fillColor] (330.20, 78.57) circle (  1.49);
\definecolor{drawColor}{RGB}{0,0,0}
\definecolor{fillColor}{RGB}{0,0,0}

\path[draw=drawColor,line width= 0.4pt,line join=round,line cap=round,fill=fillColor] (330.22,198.50) circle (  1.49);
\definecolor{drawColor}{RGB}{255,0,0}
\definecolor{fillColor}{RGB}{255,0,0}

\path[draw=drawColor,line width= 0.4pt,line join=round,line cap=round,fill=fillColor] (330.67, 78.60) circle (  1.49);
\definecolor{drawColor}{RGB}{0,0,0}
\definecolor{fillColor}{RGB}{0,0,0}

\path[draw=drawColor,line width= 0.4pt,line join=round,line cap=round,fill=fillColor] (330.69,212.82) circle (  1.49);
\definecolor{drawColor}{RGB}{255,0,0}
\definecolor{fillColor}{RGB}{255,0,0}

\path[draw=drawColor,line width= 0.4pt,line join=round,line cap=round,fill=fillColor] (331.17, 77.77) circle (  1.49);
\definecolor{drawColor}{RGB}{0,0,0}
\definecolor{fillColor}{RGB}{0,0,0}

\path[draw=drawColor,line width= 0.4pt,line join=round,line cap=round,fill=fillColor] (331.18,127.54) circle (  1.49);
\definecolor{drawColor}{RGB}{255,0,0}
\definecolor{fillColor}{RGB}{255,0,0}

\path[draw=drawColor,line width= 0.4pt,line join=round,line cap=round,fill=fillColor] (331.64, 79.01) circle (  1.49);
\definecolor{drawColor}{RGB}{0,0,0}
\definecolor{fillColor}{RGB}{0,0,0}

\path[draw=drawColor,line width= 0.4pt,line join=round,line cap=round,fill=fillColor] (331.66,197.44) circle (  1.49);
\definecolor{drawColor}{RGB}{255,0,0}
\definecolor{fillColor}{RGB}{255,0,0}

\path[draw=drawColor,line width= 0.4pt,line join=round,line cap=round,fill=fillColor] (332.12, 80.82) circle (  1.49);
\definecolor{drawColor}{RGB}{0,0,0}
\definecolor{fillColor}{RGB}{0,0,0}

\path[draw=drawColor,line width= 0.4pt,line join=round,line cap=round,fill=fillColor] (332.13,209.94) circle (  1.49);
\definecolor{drawColor}{RGB}{255,0,0}
\definecolor{fillColor}{RGB}{255,0,0}

\path[draw=drawColor,line width= 0.4pt,line join=round,line cap=round,fill=fillColor] (332.62, 81.35) circle (  1.49);
\definecolor{drawColor}{RGB}{0,0,0}
\definecolor{fillColor}{RGB}{0,0,0}

\path[draw=drawColor,line width= 0.4pt,line join=round,line cap=round,fill=fillColor] (332.64,209.40) circle (  1.49);
\definecolor{drawColor}{RGB}{255,0,0}
\definecolor{fillColor}{RGB}{255,0,0}

\path[draw=drawColor,line width= 0.4pt,line join=round,line cap=round,fill=fillColor] (333.10, 78.04) circle (  1.49);
\definecolor{drawColor}{RGB}{0,0,0}
\definecolor{fillColor}{RGB}{0,0,0}

\path[draw=drawColor,line width= 0.4pt,line join=round,line cap=round,fill=fillColor] (333.11,209.04) circle (  1.49);
\definecolor{drawColor}{RGB}{255,0,0}
\definecolor{fillColor}{RGB}{255,0,0}

\path[draw=drawColor,line width= 0.4pt,line join=round,line cap=round,fill=fillColor] (333.59, 77.11) circle (  1.49);
\definecolor{drawColor}{RGB}{0,0,0}
\definecolor{fillColor}{RGB}{0,0,0}

\path[draw=drawColor,line width= 0.4pt,line join=round,line cap=round,fill=fillColor] (333.62,205.81) circle (  1.49);
\definecolor{drawColor}{RGB}{255,0,0}
\definecolor{fillColor}{RGB}{255,0,0}

\path[draw=drawColor,line width= 0.4pt,line join=round,line cap=round,fill=fillColor] (334.06, 76.12) circle (  1.49);
\definecolor{drawColor}{RGB}{0,0,0}
\definecolor{fillColor}{RGB}{0,0,0}

\path[draw=drawColor,line width= 0.4pt,line join=round,line cap=round,fill=fillColor] (334.08,200.42) circle (  1.49);
\definecolor{drawColor}{RGB}{255,0,0}
\definecolor{fillColor}{RGB}{255,0,0}

\path[draw=drawColor,line width= 0.4pt,line join=round,line cap=round,fill=fillColor] (334.55, 75.15) circle (  1.49);
\definecolor{drawColor}{RGB}{0,0,0}
\definecolor{fillColor}{RGB}{0,0,0}

\path[draw=drawColor,line width= 0.4pt,line join=round,line cap=round,fill=fillColor] (334.57,202.80) circle (  1.49);
\definecolor{drawColor}{RGB}{255,0,0}
\definecolor{fillColor}{RGB}{255,0,0}

\path[draw=drawColor,line width= 0.4pt,line join=round,line cap=round,fill=fillColor] (335.03, 74.54) circle (  1.49);
\definecolor{drawColor}{RGB}{0,0,0}
\definecolor{fillColor}{RGB}{0,0,0}

\path[draw=drawColor,line width= 0.4pt,line join=round,line cap=round,fill=fillColor] (335.05,202.80) circle (  1.49);
\definecolor{drawColor}{RGB}{255,0,0}
\definecolor{fillColor}{RGB}{255,0,0}

\path[draw=drawColor,line width= 0.4pt,line join=round,line cap=round,fill=fillColor] (335.52, 74.08) circle (  1.49);
\definecolor{drawColor}{RGB}{0,0,0}
\definecolor{fillColor}{RGB}{0,0,0}

\path[draw=drawColor,line width= 0.4pt,line join=round,line cap=round,fill=fillColor] (335.54,193.68) circle (  1.49);
\definecolor{drawColor}{RGB}{255,0,0}
\definecolor{fillColor}{RGB}{255,0,0}

\path[draw=drawColor,line width= 0.4pt,line join=round,line cap=round,fill=fillColor] (335.98, 72.14) circle (  1.49);
\definecolor{drawColor}{RGB}{0,0,0}
\definecolor{fillColor}{RGB}{0,0,0}

\path[draw=drawColor,line width= 0.4pt,line join=round,line cap=round,fill=fillColor] (335.99,198.29) circle (  1.49);
\definecolor{drawColor}{RGB}{255,0,0}
\definecolor{fillColor}{RGB}{255,0,0}

\path[draw=drawColor,line width= 0.4pt,line join=round,line cap=round,fill=fillColor] (336.47, 71.11) circle (  1.49);
\definecolor{drawColor}{RGB}{0,0,0}
\definecolor{fillColor}{RGB}{0,0,0}

\path[draw=drawColor,line width= 0.4pt,line join=round,line cap=round,fill=fillColor] (336.49,171.16) circle (  1.49);
\definecolor{drawColor}{RGB}{255,0,0}
\definecolor{fillColor}{RGB}{255,0,0}

\path[draw=drawColor,line width= 0.4pt,line join=round,line cap=round,fill=fillColor] (336.94, 69.82) circle (  1.49);
\definecolor{drawColor}{RGB}{0,0,0}
\definecolor{fillColor}{RGB}{0,0,0}

\path[draw=drawColor,line width= 0.4pt,line join=round,line cap=round,fill=fillColor] (336.96,177.48) circle (  1.49);
\definecolor{drawColor}{RGB}{255,0,0}
\definecolor{fillColor}{RGB}{255,0,0}

\path[draw=drawColor,line width= 0.4pt,line join=round,line cap=round,fill=fillColor] (337.42, 70.09) circle (  1.49);
\definecolor{drawColor}{RGB}{0,0,0}
\definecolor{fillColor}{RGB}{0,0,0}

\path[draw=drawColor,line width= 0.4pt,line join=round,line cap=round,fill=fillColor] (337.44,150.88) circle (  1.49);
\definecolor{drawColor}{RGB}{255,0,0}
\definecolor{fillColor}{RGB}{255,0,0}

\path[draw=drawColor,line width= 0.4pt,line join=round,line cap=round,fill=fillColor] (337.91, 69.83) circle (  1.49);
\definecolor{drawColor}{RGB}{0,0,0}
\definecolor{fillColor}{RGB}{0,0,0}

\path[draw=drawColor,line width= 0.4pt,line join=round,line cap=round,fill=fillColor] (337.93,167.91) circle (  1.49);
\definecolor{drawColor}{RGB}{255,0,0}
\definecolor{fillColor}{RGB}{255,0,0}

\path[draw=drawColor,line width= 0.4pt,line join=round,line cap=round,fill=fillColor] (338.38, 70.83) circle (  1.49);
\definecolor{drawColor}{RGB}{0,0,0}
\definecolor{fillColor}{RGB}{0,0,0}

\path[draw=drawColor,line width= 0.4pt,line join=round,line cap=round,fill=fillColor] (338.40,189.37) circle (  1.49);
\definecolor{drawColor}{RGB}{255,0,0}
\definecolor{fillColor}{RGB}{255,0,0}

\path[draw=drawColor,line width= 0.4pt,line join=round,line cap=round,fill=fillColor] (338.86, 69.65) circle (  1.49);
\definecolor{drawColor}{RGB}{0,0,0}
\definecolor{fillColor}{RGB}{0,0,0}

\path[draw=drawColor,line width= 0.4pt,line join=round,line cap=round,fill=fillColor] (338.88,180.77) circle (  1.49);
\definecolor{drawColor}{RGB}{255,0,0}
\definecolor{fillColor}{RGB}{255,0,0}

\path[draw=drawColor,line width= 0.4pt,line join=round,line cap=round,fill=fillColor] (339.42, 66.75) circle (  1.49);
\definecolor{drawColor}{RGB}{0,0,0}
\definecolor{fillColor}{RGB}{0,0,0}

\path[draw=drawColor,line width= 0.4pt,line join=round,line cap=round,fill=fillColor] (339.43, 96.24) circle (  1.49);
\definecolor{drawColor}{RGB}{255,0,0}
\definecolor{fillColor}{RGB}{255,0,0}

\path[draw=drawColor,line width= 0.4pt,line join=round,line cap=round,fill=fillColor] (339.96, 65.01) circle (  1.49);
\definecolor{drawColor}{RGB}{0,0,0}
\definecolor{fillColor}{RGB}{0,0,0}

\path[draw=drawColor,line width= 0.4pt,line join=round,line cap=round,fill=fillColor] (339.97,122.34) circle (  1.49);
\definecolor{drawColor}{RGB}{255,0,0}
\definecolor{fillColor}{RGB}{255,0,0}

\path[draw=drawColor,line width= 0.4pt,line join=round,line cap=round,fill=fillColor] (340.45, 64.18) circle (  1.49);
\definecolor{drawColor}{RGB}{0,0,0}
\definecolor{fillColor}{RGB}{0,0,0}

\path[draw=drawColor,line width= 0.4pt,line join=round,line cap=round,fill=fillColor] (340.46,200.35) circle (  1.49);
\definecolor{drawColor}{RGB}{255,0,0}
\definecolor{fillColor}{RGB}{255,0,0}

\path[draw=drawColor,line width= 0.4pt,line join=round,line cap=round,fill=fillColor] (340.94, 71.56) circle (  1.49);
\definecolor{drawColor}{RGB}{0,0,0}
\definecolor{fillColor}{RGB}{0,0,0}

\path[draw=drawColor,line width= 0.4pt,line join=round,line cap=round,fill=fillColor] (340.95,181.66) circle (  1.49);
\definecolor{drawColor}{RGB}{255,0,0}
\definecolor{fillColor}{RGB}{255,0,0}

\path[draw=drawColor,line width= 0.4pt,line join=round,line cap=round,fill=fillColor] (341.41, 64.88) circle (  1.49);
\definecolor{drawColor}{RGB}{0,0,0}
\definecolor{fillColor}{RGB}{0,0,0}

\path[draw=drawColor,line width= 0.4pt,line join=round,line cap=round,fill=fillColor] (341.43,193.30) circle (  1.49);
\definecolor{drawColor}{RGB}{255,0,0}
\definecolor{fillColor}{RGB}{255,0,0}

\path[draw=drawColor,line width= 0.4pt,line join=round,line cap=round,fill=fillColor] (341.89, 63.78) circle (  1.49);
\definecolor{drawColor}{RGB}{0,0,0}
\definecolor{fillColor}{RGB}{0,0,0}

\path[draw=drawColor,line width= 0.4pt,line join=round,line cap=round,fill=fillColor] (341.90,195.32) circle (  1.49);
\definecolor{drawColor}{RGB}{255,0,0}
\definecolor{fillColor}{RGB}{255,0,0}

\path[draw=drawColor,line width= 0.4pt,line join=round,line cap=round,fill=fillColor] (342.38, 63.93) circle (  1.49);
\definecolor{drawColor}{RGB}{0,0,0}
\definecolor{fillColor}{RGB}{0,0,0}

\path[draw=drawColor,line width= 0.4pt,line join=round,line cap=round,fill=fillColor] (342.40,194.18) circle (  1.49);
\definecolor{drawColor}{RGB}{255,0,0}
\definecolor{fillColor}{RGB}{255,0,0}

\path[draw=drawColor,line width= 0.4pt,line join=round,line cap=round,fill=fillColor] (342.87, 64.24) circle (  1.49);
\definecolor{drawColor}{RGB}{0,0,0}
\definecolor{fillColor}{RGB}{0,0,0}

\path[draw=drawColor,line width= 0.4pt,line join=round,line cap=round,fill=fillColor] (342.89,194.23) circle (  1.49);
\definecolor{drawColor}{RGB}{255,0,0}
\definecolor{fillColor}{RGB}{255,0,0}

\path[draw=drawColor,line width= 0.4pt,line join=round,line cap=round,fill=fillColor] (343.34, 67.52) circle (  1.49);
\definecolor{drawColor}{RGB}{0,0,0}
\definecolor{fillColor}{RGB}{0,0,0}

\path[draw=drawColor,line width= 0.4pt,line join=round,line cap=round,fill=fillColor] (343.36,190.57) circle (  1.49);
\definecolor{drawColor}{RGB}{255,0,0}
\definecolor{fillColor}{RGB}{255,0,0}

\path[draw=drawColor,line width= 0.4pt,line join=round,line cap=round,fill=fillColor] (343.82, 65.45) circle (  1.49);
\definecolor{drawColor}{RGB}{0,0,0}
\definecolor{fillColor}{RGB}{0,0,0}

\path[draw=drawColor,line width= 0.4pt,line join=round,line cap=round,fill=fillColor] (343.84,186.05) circle (  1.49);
\definecolor{drawColor}{RGB}{255,0,0}
\definecolor{fillColor}{RGB}{255,0,0}

\path[draw=drawColor,line width= 0.4pt,line join=round,line cap=round,fill=fillColor] (344.41,100.86) circle (  1.49);
\definecolor{drawColor}{RGB}{0,0,0}
\definecolor{fillColor}{RGB}{0,0,0}

\path[draw=drawColor,line width= 0.4pt,line join=round,line cap=round,fill=fillColor] (344.42,145.33) circle (  1.49);
\definecolor{drawColor}{RGB}{255,0,0}
\definecolor{fillColor}{RGB}{255,0,0}

\path[draw=drawColor,line width= 0.4pt,line join=round,line cap=round,fill=fillColor] (344.92, 59.16) circle (  1.49);
\definecolor{drawColor}{RGB}{0,0,0}
\definecolor{fillColor}{RGB}{0,0,0}

\path[draw=drawColor,line width= 0.4pt,line join=round,line cap=round,fill=fillColor] (344.93,193.01) circle (  1.49);
\definecolor{drawColor}{RGB}{255,0,0}
\definecolor{fillColor}{RGB}{255,0,0}

\path[draw=drawColor,line width= 0.4pt,line join=round,line cap=round,fill=fillColor] (345.42, 64.20) circle (  1.49);
\definecolor{drawColor}{RGB}{0,0,0}
\definecolor{fillColor}{RGB}{0,0,0}

\path[draw=drawColor,line width= 0.4pt,line join=round,line cap=round,fill=fillColor] (345.44,182.36) circle (  1.49);
\definecolor{drawColor}{RGB}{255,0,0}
\definecolor{fillColor}{RGB}{255,0,0}

\path[draw=drawColor,line width= 0.4pt,line join=round,line cap=round,fill=fillColor] (345.96, 76.26) circle (  1.49);
\definecolor{drawColor}{RGB}{0,0,0}
\definecolor{fillColor}{RGB}{0,0,0}

\path[draw=drawColor,line width= 0.4pt,line join=round,line cap=round,fill=fillColor] (345.98,181.50) circle (  1.49);
\definecolor{drawColor}{RGB}{255,0,0}
\definecolor{fillColor}{RGB}{255,0,0}

\path[draw=drawColor,line width= 0.4pt,line join=round,line cap=round,fill=fillColor] (346.65, 59.56) circle (  1.49);
\definecolor{drawColor}{RGB}{0,0,0}
\definecolor{fillColor}{RGB}{0,0,0}

\path[draw=drawColor,line width= 0.4pt,line join=round,line cap=round,fill=fillColor] (346.67,108.23) circle (  1.49);
\definecolor{drawColor}{RGB}{255,0,0}
\definecolor{fillColor}{RGB}{255,0,0}

\path[draw=drawColor,line width= 0.4pt,line join=round,line cap=round,fill=fillColor] (347.19, 59.48) circle (  1.49);
\definecolor{drawColor}{RGB}{0,0,0}
\definecolor{fillColor}{RGB}{0,0,0}

\path[draw=drawColor,line width= 0.4pt,line join=round,line cap=round,fill=fillColor] (347.21,191.29) circle (  1.49);
\definecolor{drawColor}{RGB}{255,0,0}
\definecolor{fillColor}{RGB}{255,0,0}

\path[draw=drawColor,line width= 0.4pt,line join=round,line cap=round,fill=fillColor] (347.68, 58.93) circle (  1.49);
\definecolor{drawColor}{RGB}{0,0,0}
\definecolor{fillColor}{RGB}{0,0,0}

\path[draw=drawColor,line width= 0.4pt,line join=round,line cap=round,fill=fillColor] (347.70,188.10) circle (  1.49);
\definecolor{drawColor}{RGB}{255,0,0}
\definecolor{fillColor}{RGB}{255,0,0}

\path[draw=drawColor,line width= 0.4pt,line join=round,line cap=round,fill=fillColor] (348.19, 58.41) circle (  1.49);
\definecolor{drawColor}{RGB}{0,0,0}
\definecolor{fillColor}{RGB}{0,0,0}

\path[draw=drawColor,line width= 0.4pt,line join=round,line cap=round,fill=fillColor] (348.21,188.99) circle (  1.49);
\definecolor{drawColor}{RGB}{255,0,0}
\definecolor{fillColor}{RGB}{255,0,0}

\path[draw=drawColor,line width= 0.4pt,line join=round,line cap=round,fill=fillColor] (348.71, 58.03) circle (  1.49);
\definecolor{drawColor}{RGB}{0,0,0}
\definecolor{fillColor}{RGB}{0,0,0}

\path[draw=drawColor,line width= 0.4pt,line join=round,line cap=round,fill=fillColor] (348.73,189.09) circle (  1.49);
\definecolor{drawColor}{RGB}{255,0,0}
\definecolor{fillColor}{RGB}{255,0,0}

\path[draw=drawColor,line width= 0.4pt,line join=round,line cap=round,fill=fillColor] (349.20, 56.86) circle (  1.49);
\definecolor{drawColor}{RGB}{0,0,0}
\definecolor{fillColor}{RGB}{0,0,0}

\path[draw=drawColor,line width= 0.4pt,line join=round,line cap=round,fill=fillColor] (349.22,184.19) circle (  1.49);
\definecolor{drawColor}{RGB}{255,0,0}
\definecolor{fillColor}{RGB}{255,0,0}

\path[draw=drawColor,line width= 0.4pt,line join=round,line cap=round,fill=fillColor] (349.71, 55.91) circle (  1.49);
\definecolor{drawColor}{RGB}{0,0,0}
\definecolor{fillColor}{RGB}{0,0,0}

\path[draw=drawColor,line width= 0.4pt,line join=round,line cap=round,fill=fillColor] (349.73,187.73) circle (  1.49);
\definecolor{drawColor}{RGB}{255,0,0}
\definecolor{fillColor}{RGB}{255,0,0}

\path[draw=drawColor,line width= 0.4pt,line join=round,line cap=round,fill=fillColor] (350.25, 59.14) circle (  1.49);
\definecolor{drawColor}{RGB}{0,0,0}
\definecolor{fillColor}{RGB}{0,0,0}

\path[draw=drawColor,line width= 0.4pt,line join=round,line cap=round,fill=fillColor] (350.27,187.78) circle (  1.49);
\definecolor{drawColor}{RGB}{255,0,0}
\definecolor{fillColor}{RGB}{255,0,0}

\path[draw=drawColor,line width= 0.4pt,line join=round,line cap=round,fill=fillColor] (350.74, 60.11) circle (  1.49);
\definecolor{drawColor}{RGB}{0,0,0}
\definecolor{fillColor}{RGB}{0,0,0}

\path[draw=drawColor,line width= 0.4pt,line join=round,line cap=round,fill=fillColor] (350.78,160.08) circle (  1.49);
\definecolor{drawColor}{RGB}{255,0,0}
\definecolor{fillColor}{RGB}{255,0,0}

\path[draw=drawColor,line width= 0.4pt,line join=round,line cap=round,fill=fillColor] (351.25, 63.55) circle (  1.49);
\definecolor{drawColor}{RGB}{0,0,0}
\definecolor{fillColor}{RGB}{0,0,0}

\path[draw=drawColor,line width= 0.4pt,line join=round,line cap=round,fill=fillColor] (351.27,187.10) circle (  1.49);
\definecolor{drawColor}{RGB}{255,0,0}
\definecolor{fillColor}{RGB}{255,0,0}

\path[draw=drawColor,line width= 0.4pt,line join=round,line cap=round,fill=fillColor] (351.76, 88.45) circle (  1.49);
\definecolor{drawColor}{RGB}{0,0,0}
\definecolor{fillColor}{RGB}{0,0,0}

\path[draw=drawColor,line width= 0.4pt,line join=round,line cap=round,fill=fillColor] (351.77,187.04) circle (  1.49);
\definecolor{drawColor}{RGB}{255,0,0}
\definecolor{fillColor}{RGB}{255,0,0}

\path[draw=drawColor,line width= 0.4pt,line join=round,line cap=round,fill=fillColor] (352.27, 56.47) circle (  1.49);
\definecolor{drawColor}{RGB}{0,0,0}
\definecolor{fillColor}{RGB}{0,0,0}

\path[draw=drawColor,line width= 0.4pt,line join=round,line cap=round,fill=fillColor] (352.28,185.25) circle (  1.49);
\definecolor{drawColor}{RGB}{255,0,0}
\definecolor{fillColor}{RGB}{255,0,0}

\path[draw=drawColor,line width= 0.4pt,line join=round,line cap=round,fill=fillColor] (352.77, 59.12) circle (  1.49);
\definecolor{drawColor}{RGB}{0,0,0}
\definecolor{fillColor}{RGB}{0,0,0}

\path[draw=drawColor,line width= 0.4pt,line join=round,line cap=round,fill=fillColor] (352.79,187.54) circle (  1.49);
\definecolor{drawColor}{RGB}{255,0,0}
\definecolor{fillColor}{RGB}{255,0,0}

\path[draw=drawColor,line width= 0.4pt,line join=round,line cap=round,fill=fillColor] (353.28, 59.64) circle (  1.49);
\definecolor{drawColor}{RGB}{0,0,0}
\definecolor{fillColor}{RGB}{0,0,0}

\path[draw=drawColor,line width= 0.4pt,line join=round,line cap=round,fill=fillColor] (353.30,186.79) circle (  1.49);
\definecolor{drawColor}{RGB}{255,0,0}
\definecolor{fillColor}{RGB}{255,0,0}

\path[draw=drawColor,line width= 0.4pt,line join=round,line cap=round,fill=fillColor] (353.79, 69.38) circle (  1.49);
\definecolor{drawColor}{RGB}{0,0,0}
\definecolor{fillColor}{RGB}{0,0,0}

\path[draw=drawColor,line width= 0.4pt,line join=round,line cap=round,fill=fillColor] (353.80,186.20) circle (  1.49);
\definecolor{drawColor}{RGB}{255,0,0}
\definecolor{fillColor}{RGB}{255,0,0}

\path[draw=drawColor,line width= 0.4pt,line join=round,line cap=round,fill=fillColor] (354.28, 67.90) circle (  1.49);
\definecolor{drawColor}{RGB}{0,0,0}
\definecolor{fillColor}{RGB}{0,0,0}

\path[draw=drawColor,line width= 0.4pt,line join=round,line cap=round,fill=fillColor] (354.30,185.46) circle (  1.49);
\definecolor{drawColor}{RGB}{255,0,0}
\definecolor{fillColor}{RGB}{255,0,0}

\path[draw=drawColor,line width= 0.4pt,line join=round,line cap=round,fill=fillColor] (354.77, 59.02) circle (  1.49);
\definecolor{drawColor}{RGB}{0,0,0}
\definecolor{fillColor}{RGB}{0,0,0}

\path[draw=drawColor,line width= 0.4pt,line join=round,line cap=round,fill=fillColor] (354.79,158.17) circle (  1.49);
\definecolor{drawColor}{RGB}{255,0,0}
\definecolor{fillColor}{RGB}{255,0,0}

\path[draw=drawColor,line width= 0.4pt,line join=round,line cap=round,fill=fillColor] (355.28, 92.11) circle (  1.49);
\definecolor{drawColor}{RGB}{0,0,0}
\definecolor{fillColor}{RGB}{0,0,0}

\path[draw=drawColor,line width= 0.4pt,line join=round,line cap=round,fill=fillColor] (355.29,187.16) circle (  1.49);
\definecolor{drawColor}{RGB}{255,0,0}
\definecolor{fillColor}{RGB}{255,0,0}

\path[draw=drawColor,line width= 0.4pt,line join=round,line cap=round,fill=fillColor] (355.82, 57.88) circle (  1.49);
\definecolor{drawColor}{RGB}{0,0,0}
\definecolor{fillColor}{RGB}{0,0,0}

\path[draw=drawColor,line width= 0.4pt,line join=round,line cap=round,fill=fillColor] (355.85,186.24) circle (  1.49);
\definecolor{drawColor}{RGB}{255,0,0}
\definecolor{fillColor}{RGB}{255,0,0}

\path[draw=drawColor,line width= 0.4pt,line join=round,line cap=round,fill=fillColor] (356.33, 57.19) circle (  1.49);
\definecolor{drawColor}{RGB}{0,0,0}
\definecolor{fillColor}{RGB}{0,0,0}

\path[draw=drawColor,line width= 0.4pt,line join=round,line cap=round,fill=fillColor] (356.34,184.42) circle (  1.49);
\definecolor{drawColor}{RGB}{255,0,0}
\definecolor{fillColor}{RGB}{255,0,0}

\path[draw=drawColor,line width= 0.4pt,line join=round,line cap=round,fill=fillColor] (356.85,104.71) circle (  1.49);
\definecolor{drawColor}{RGB}{0,0,0}
\definecolor{fillColor}{RGB}{0,0,0}

\path[draw=drawColor,line width= 0.4pt,line join=round,line cap=round,fill=fillColor] (356.87,185.14) circle (  1.49);
\definecolor{drawColor}{RGB}{255,0,0}
\definecolor{fillColor}{RGB}{255,0,0}

\path[draw=drawColor,line width= 0.4pt,line join=round,line cap=round,fill=fillColor] (357.46, 89.79) circle (  1.49);
\definecolor{drawColor}{RGB}{0,0,0}
\definecolor{fillColor}{RGB}{0,0,0}

\path[draw=drawColor,line width= 0.4pt,line join=round,line cap=round,fill=fillColor] (357.47,185.70) circle (  1.49);
\definecolor{drawColor}{RGB}{255,0,0}
\definecolor{fillColor}{RGB}{255,0,0}

\path[draw=drawColor,line width= 0.4pt,line join=round,line cap=round,fill=fillColor] (357.96,143.05) circle (  1.49);
\definecolor{drawColor}{RGB}{0,0,0}
\definecolor{fillColor}{RGB}{0,0,0}

\path[draw=drawColor,line width= 0.4pt,line join=round,line cap=round,fill=fillColor] (357.98,187.46) circle (  1.49);
\definecolor{drawColor}{RGB}{255,0,0}
\definecolor{fillColor}{RGB}{255,0,0}

\path[draw=drawColor,line width= 0.4pt,line join=round,line cap=round,fill=fillColor] (358.45, 70.28) circle (  1.49);
\definecolor{drawColor}{RGB}{0,0,0}
\definecolor{fillColor}{RGB}{0,0,0}

\path[draw=drawColor,line width= 0.4pt,line join=round,line cap=round,fill=fillColor] (358.49, 89.32) circle (  1.49);
\definecolor{drawColor}{RGB}{255,0,0}
\definecolor{fillColor}{RGB}{255,0,0}

\path[draw=drawColor,line width= 0.4pt,line join=round,line cap=round,fill=fillColor] (359.11, 55.67) circle (  1.49);
\definecolor{drawColor}{RGB}{0,0,0}
\definecolor{fillColor}{RGB}{0,0,0}

\path[draw=drawColor,line width= 0.4pt,line join=round,line cap=round,fill=fillColor] (359.12,150.72) circle (  1.49);
\definecolor{drawColor}{RGB}{255,0,0}
\definecolor{fillColor}{RGB}{255,0,0}

\path[draw=drawColor,line width= 0.4pt,line join=round,line cap=round,fill=fillColor] (359.83, 84.83) circle (  1.49);
\definecolor{drawColor}{RGB}{0,0,0}
\definecolor{fillColor}{RGB}{0,0,0}

\path[draw=drawColor,line width= 0.4pt,line join=round,line cap=round,fill=fillColor] (359.85,148.65) circle (  1.49);
\definecolor{drawColor}{RGB}{255,0,0}
\definecolor{fillColor}{RGB}{255,0,0}

\path[draw=drawColor,line width= 0.4pt,line join=round,line cap=round,fill=fillColor] (360.35, 69.32) circle (  1.49);
\definecolor{drawColor}{RGB}{0,0,0}
\definecolor{fillColor}{RGB}{0,0,0}

\path[draw=drawColor,line width= 0.4pt,line join=round,line cap=round,fill=fillColor] (360.37,151.09) circle (  1.49);
\definecolor{drawColor}{RGB}{255,0,0}
\definecolor{fillColor}{RGB}{255,0,0}

\path[draw=drawColor,line width= 0.4pt,line join=round,line cap=round,fill=fillColor] (360.86,101.36) circle (  1.49);
\definecolor{drawColor}{RGB}{0,0,0}
\definecolor{fillColor}{RGB}{0,0,0}

\path[draw=drawColor,line width= 0.4pt,line join=round,line cap=round,fill=fillColor] (360.88,149.75) circle (  1.49);
\definecolor{drawColor}{RGB}{255,0,0}
\definecolor{fillColor}{RGB}{255,0,0}

\path[draw=drawColor,line width= 0.4pt,line join=round,line cap=round,fill=fillColor] (361.51,113.73) circle (  1.49);
\definecolor{drawColor}{RGB}{0,0,0}
\definecolor{fillColor}{RGB}{0,0,0}

\path[draw=drawColor,line width= 0.4pt,line join=round,line cap=round,fill=fillColor] (361.53,150.44) circle (  1.49);
\definecolor{drawColor}{RGB}{255,0,0}
\definecolor{fillColor}{RGB}{255,0,0}

\path[draw=drawColor,line width= 0.4pt,line join=round,line cap=round,fill=fillColor] (362.10,122.36) circle (  1.49);
\definecolor{drawColor}{RGB}{0,0,0}
\definecolor{fillColor}{RGB}{0,0,0}

\path[draw=drawColor,line width= 0.4pt,line join=round,line cap=round,fill=fillColor] (362.12,185.41) circle (  1.49);
\definecolor{drawColor}{RGB}{255,0,0}
\definecolor{fillColor}{RGB}{255,0,0}

\path[draw=drawColor,line width= 0.4pt,line join=round,line cap=round,fill=fillColor] (362.60, 58.44) circle (  1.49);
\definecolor{drawColor}{RGB}{0,0,0}
\definecolor{fillColor}{RGB}{0,0,0}

\path[draw=drawColor,line width= 0.4pt,line join=round,line cap=round,fill=fillColor] (362.61,184.13) circle (  1.49);
\definecolor{drawColor}{RGB}{255,0,0}
\definecolor{fillColor}{RGB}{255,0,0}

\path[draw=drawColor,line width= 0.4pt,line join=round,line cap=round,fill=fillColor] (363.14,180.65) circle (  1.49);
\definecolor{drawColor}{RGB}{0,0,0}
\definecolor{fillColor}{RGB}{0,0,0}

\path[draw=drawColor,line width= 0.4pt,line join=round,line cap=round,fill=fillColor] (363.15,183.18) circle (  1.49);
\definecolor{drawColor}{RGB}{255,0,0}
\definecolor{fillColor}{RGB}{255,0,0}

\path[draw=drawColor,line width= 0.4pt,line join=round,line cap=round,fill=fillColor] (363.66,189.17) circle (  1.49);
\definecolor{drawColor}{RGB}{0,0,0}
\definecolor{fillColor}{RGB}{0,0,0}

\path[draw=drawColor,line width= 0.4pt,line join=round,line cap=round,fill=fillColor] (363.68,181.45) circle (  1.49);
\definecolor{drawColor}{RGB}{255,0,0}
\definecolor{fillColor}{RGB}{255,0,0}

\path[draw=drawColor,line width= 0.4pt,line join=round,line cap=round,fill=fillColor] (364.25,192.23) circle (  1.49);
\definecolor{drawColor}{RGB}{0,0,0}
\definecolor{fillColor}{RGB}{0,0,0}

\path[draw=drawColor,line width= 0.4pt,line join=round,line cap=round,fill=fillColor] (364.26,184.06) circle (  1.49);
\definecolor{drawColor}{RGB}{255,0,0}
\definecolor{fillColor}{RGB}{255,0,0}

\path[draw=drawColor,line width= 0.4pt,line join=round,line cap=round,fill=fillColor] (364.82,194.71) circle (  1.49);
\definecolor{drawColor}{RGB}{0,0,0}
\definecolor{fillColor}{RGB}{0,0,0}

\path[draw=drawColor,line width= 0.4pt,line join=round,line cap=round,fill=fillColor] (364.84,185.61) circle (  1.49);
\definecolor{drawColor}{RGB}{255,0,0}
\definecolor{fillColor}{RGB}{255,0,0}

\path[draw=drawColor,line width= 0.4pt,line join=round,line cap=round,fill=fillColor] (365.31,192.84) circle (  1.49);
\definecolor{drawColor}{RGB}{0,0,0}
\definecolor{fillColor}{RGB}{0,0,0}

\path[draw=drawColor,line width= 0.4pt,line join=round,line cap=round,fill=fillColor] (365.33,183.88) circle (  1.49);
\definecolor{drawColor}{RGB}{255,0,0}
\definecolor{fillColor}{RGB}{255,0,0}

\path[draw=drawColor,line width= 0.4pt,line join=round,line cap=round,fill=fillColor] (365.82,195.32) circle (  1.49);
\definecolor{drawColor}{RGB}{0,0,0}
\definecolor{fillColor}{RGB}{0,0,0}

\path[draw=drawColor,line width= 0.4pt,line join=round,line cap=round,fill=fillColor] (365.84,158.12) circle (  1.49);
\definecolor{drawColor}{RGB}{255,0,0}
\definecolor{fillColor}{RGB}{255,0,0}

\path[draw=drawColor,line width= 0.4pt,line join=round,line cap=round,fill=fillColor] (366.33,195.09) circle (  1.49);
\definecolor{drawColor}{RGB}{0,0,0}
\definecolor{fillColor}{RGB}{0,0,0}

\path[draw=drawColor,line width= 0.4pt,line join=round,line cap=round,fill=fillColor] (366.34,185.55) circle (  1.49);
\definecolor{drawColor}{RGB}{255,0,0}
\definecolor{fillColor}{RGB}{255,0,0}

\path[draw=drawColor,line width= 0.4pt,line join=round,line cap=round,fill=fillColor] (366.82,195.65) circle (  1.49);
\definecolor{drawColor}{RGB}{0,0,0}
\definecolor{fillColor}{RGB}{0,0,0}

\path[draw=drawColor,line width= 0.4pt,line join=round,line cap=round,fill=fillColor] (366.83,185.49) circle (  1.49);
\definecolor{drawColor}{RGB}{255,0,0}
\definecolor{fillColor}{RGB}{255,0,0}

\path[draw=drawColor,line width= 0.4pt,line join=round,line cap=round,fill=fillColor] (367.38,196.27) circle (  1.49);
\definecolor{drawColor}{RGB}{0,0,0}
\definecolor{fillColor}{RGB}{0,0,0}

\path[draw=drawColor,line width= 0.4pt,line join=round,line cap=round,fill=fillColor] (367.39, 68.78) circle (  1.49);
\definecolor{drawColor}{RGB}{255,0,0}
\definecolor{fillColor}{RGB}{255,0,0}

\path[draw=drawColor,line width= 0.4pt,line join=round,line cap=round,fill=fillColor] (367.93,196.30) circle (  1.49);
\definecolor{drawColor}{RGB}{0,0,0}
\definecolor{fillColor}{RGB}{0,0,0}

\path[draw=drawColor,line width= 0.4pt,line join=round,line cap=round,fill=fillColor] (367.95,185.95) circle (  1.49);
\definecolor{drawColor}{RGB}{255,0,0}
\definecolor{fillColor}{RGB}{255,0,0}

\path[draw=drawColor,line width= 0.4pt,line join=round,line cap=round,fill=fillColor] (368.44,197.82) circle (  1.49);
\definecolor{drawColor}{RGB}{0,0,0}
\definecolor{fillColor}{RGB}{0,0,0}

\path[draw=drawColor,line width= 0.4pt,line join=round,line cap=round,fill=fillColor] (368.46,186.37) circle (  1.49);
\definecolor{drawColor}{RGB}{255,0,0}
\definecolor{fillColor}{RGB}{255,0,0}

\path[draw=drawColor,line width= 0.4pt,line join=round,line cap=round,fill=fillColor] (368.91,198.06) circle (  1.49);
\definecolor{drawColor}{RGB}{0,0,0}
\definecolor{fillColor}{RGB}{0,0,0}

\path[draw=drawColor,line width= 0.4pt,line join=round,line cap=round,fill=fillColor] (368.93,152.68) circle (  1.49);
\definecolor{drawColor}{RGB}{255,0,0}
\definecolor{fillColor}{RGB}{255,0,0}

\path[draw=drawColor,line width= 0.4pt,line join=round,line cap=round,fill=fillColor] (369.40,195.30) circle (  1.49);
\definecolor{drawColor}{RGB}{0,0,0}
\definecolor{fillColor}{RGB}{0,0,0}

\path[draw=drawColor,line width= 0.4pt,line join=round,line cap=round,fill=fillColor] (369.42,185.31) circle (  1.49);
\definecolor{drawColor}{RGB}{255,0,0}
\definecolor{fillColor}{RGB}{255,0,0}

\path[draw=drawColor,line width= 0.4pt,line join=round,line cap=round,fill=fillColor] (369.88,195.20) circle (  1.49);
\definecolor{drawColor}{RGB}{0,0,0}
\definecolor{fillColor}{RGB}{0,0,0}

\path[draw=drawColor,line width= 0.4pt,line join=round,line cap=round,fill=fillColor] (369.90,185.27) circle (  1.49);
\definecolor{drawColor}{RGB}{255,0,0}
\definecolor{fillColor}{RGB}{255,0,0}

\path[draw=drawColor,line width= 0.4pt,line join=round,line cap=round,fill=fillColor] (370.45,119.96) circle (  1.49);
\definecolor{drawColor}{RGB}{0,0,0}
\definecolor{fillColor}{RGB}{0,0,0}

\path[draw=drawColor,line width= 0.4pt,line join=round,line cap=round,fill=fillColor] (370.47,172.53) circle (  1.49);
\definecolor{drawColor}{RGB}{255,0,0}
\definecolor{fillColor}{RGB}{255,0,0}

\path[draw=drawColor,line width= 0.4pt,line join=round,line cap=round,fill=fillColor] (371.01,151.46) circle (  1.49);
\definecolor{drawColor}{RGB}{0,0,0}
\definecolor{fillColor}{RGB}{0,0,0}

\path[draw=drawColor,line width= 0.4pt,line join=round,line cap=round,fill=fillColor] (371.03,185.87) circle (  1.49);
\definecolor{drawColor}{RGB}{255,0,0}
\definecolor{fillColor}{RGB}{255,0,0}

\path[draw=drawColor,line width= 0.4pt,line join=round,line cap=round,fill=fillColor] (371.53,160.72) circle (  1.49);
\definecolor{drawColor}{RGB}{0,0,0}
\definecolor{fillColor}{RGB}{0,0,0}

\path[draw=drawColor,line width= 0.4pt,line join=round,line cap=round,fill=fillColor] (371.55,185.47) circle (  1.49);
\definecolor{drawColor}{RGB}{255,0,0}
\definecolor{fillColor}{RGB}{255,0,0}

\path[draw=drawColor,line width= 0.4pt,line join=round,line cap=round,fill=fillColor] (372.06,185.89) circle (  1.49);
\definecolor{drawColor}{RGB}{0,0,0}
\definecolor{fillColor}{RGB}{0,0,0}

\path[draw=drawColor,line width= 0.4pt,line join=round,line cap=round,fill=fillColor] (372.07,184.68) circle (  1.49);
\definecolor{drawColor}{RGB}{255,0,0}
\definecolor{fillColor}{RGB}{255,0,0}

\path[draw=drawColor,line width= 0.4pt,line join=round,line cap=round,fill=fillColor] (372.60,193.96) circle (  1.49);
\definecolor{drawColor}{RGB}{0,0,0}
\definecolor{fillColor}{RGB}{0,0,0}

\path[draw=drawColor,line width= 0.4pt,line join=round,line cap=round,fill=fillColor] (372.61,184.85) circle (  1.49);
\definecolor{drawColor}{RGB}{255,0,0}
\definecolor{fillColor}{RGB}{255,0,0}

\path[draw=drawColor,line width= 0.4pt,line join=round,line cap=round,fill=fillColor] (373.10,177.58) circle (  1.49);
\definecolor{drawColor}{RGB}{0,0,0}
\definecolor{fillColor}{RGB}{0,0,0}

\path[draw=drawColor,line width= 0.4pt,line join=round,line cap=round,fill=fillColor] (373.12,184.38) circle (  1.49);
\definecolor{drawColor}{RGB}{255,0,0}
\definecolor{fillColor}{RGB}{255,0,0}

\path[draw=drawColor,line width= 0.4pt,line join=round,line cap=round,fill=fillColor] (373.61,197.76) circle (  1.49);
\definecolor{drawColor}{RGB}{0,0,0}
\definecolor{fillColor}{RGB}{0,0,0}

\path[draw=drawColor,line width= 0.4pt,line join=round,line cap=round,fill=fillColor] (373.63,185.85) circle (  1.49);
\definecolor{drawColor}{RGB}{255,0,0}
\definecolor{fillColor}{RGB}{255,0,0}

\path[draw=drawColor,line width= 0.4pt,line join=round,line cap=round,fill=fillColor] (374.17,139.10) circle (  1.49);
\definecolor{drawColor}{RGB}{0,0,0}
\definecolor{fillColor}{RGB}{0,0,0}

\path[draw=drawColor,line width= 0.4pt,line join=round,line cap=round,fill=fillColor] (374.18,180.15) circle (  1.49);
\definecolor{drawColor}{RGB}{255,0,0}
\definecolor{fillColor}{RGB}{255,0,0}

\path[draw=drawColor,line width= 0.4pt,line join=round,line cap=round,fill=fillColor] (374.69,126.68) circle (  1.49);
\definecolor{drawColor}{RGB}{0,0,0}
\definecolor{fillColor}{RGB}{0,0,0}

\path[draw=drawColor,line width= 0.4pt,line join=round,line cap=round,fill=fillColor] (374.71,182.92) circle (  1.49);
\definecolor{drawColor}{RGB}{255,0,0}
\definecolor{fillColor}{RGB}{255,0,0}

\path[draw=drawColor,line width= 0.4pt,line join=round,line cap=round,fill=fillColor] (375.18,180.25) circle (  1.49);
\definecolor{drawColor}{RGB}{0,0,0}
\definecolor{fillColor}{RGB}{0,0,0}

\path[draw=drawColor,line width= 0.4pt,line join=round,line cap=round,fill=fillColor] (375.20,174.32) circle (  1.49);
\definecolor{drawColor}{RGB}{255,0,0}
\definecolor{fillColor}{RGB}{255,0,0}

\path[draw=drawColor,line width= 0.4pt,line join=round,line cap=round,fill=fillColor] (375.74,185.13) circle (  1.49);
\definecolor{drawColor}{RGB}{0,0,0}
\definecolor{fillColor}{RGB}{0,0,0}

\path[draw=drawColor,line width= 0.4pt,line join=round,line cap=round,fill=fillColor] (375.77, 61.94) circle (  1.49);
\definecolor{drawColor}{RGB}{255,0,0}
\definecolor{fillColor}{RGB}{255,0,0}

\path[draw=drawColor,line width= 0.4pt,line join=round,line cap=round,fill=fillColor] (376.28,195.56) circle (  1.49);
\definecolor{drawColor}{RGB}{0,0,0}
\definecolor{fillColor}{RGB}{0,0,0}

\path[draw=drawColor,line width= 0.4pt,line join=round,line cap=round,fill=fillColor] (376.30,185.76) circle (  1.49);
\definecolor{drawColor}{RGB}{255,0,0}
\definecolor{fillColor}{RGB}{255,0,0}

\path[draw=drawColor,line width= 0.4pt,line join=round,line cap=round,fill=fillColor] (376.75,191.57) circle (  1.49);
\definecolor{drawColor}{RGB}{0,0,0}
\definecolor{fillColor}{RGB}{0,0,0}

\path[draw=drawColor,line width= 0.4pt,line join=round,line cap=round,fill=fillColor] (376.77,185.69) circle (  1.49);
\definecolor{drawColor}{RGB}{255,0,0}
\definecolor{fillColor}{RGB}{255,0,0}

\path[draw=drawColor,line width= 0.4pt,line join=round,line cap=round,fill=fillColor] (377.25, 68.39) circle (  1.49);
\definecolor{drawColor}{RGB}{0,0,0}
\definecolor{fillColor}{RGB}{0,0,0}

\path[draw=drawColor,line width= 0.4pt,line join=round,line cap=round,fill=fillColor] (377.26,184.62) circle (  1.49);
\definecolor{drawColor}{RGB}{255,0,0}
\definecolor{fillColor}{RGB}{255,0,0}

\path[draw=drawColor,line width= 0.4pt,line join=round,line cap=round,fill=fillColor] (377.75,196.94) circle (  1.49);
\definecolor{drawColor}{RGB}{0,0,0}
\definecolor{fillColor}{RGB}{0,0,0}

\path[draw=drawColor,line width= 0.4pt,line join=round,line cap=round,fill=fillColor] (377.77,184.65) circle (  1.49);
\definecolor{drawColor}{RGB}{255,0,0}
\definecolor{fillColor}{RGB}{255,0,0}

\path[draw=drawColor,line width= 0.4pt,line join=round,line cap=round,fill=fillColor] (378.23,197.27) circle (  1.49);
\definecolor{drawColor}{RGB}{0,0,0}
\definecolor{fillColor}{RGB}{0,0,0}

\path[draw=drawColor,line width= 0.4pt,line join=round,line cap=round,fill=fillColor] (378.24,132.62) circle (  1.49);
\definecolor{drawColor}{RGB}{255,0,0}
\definecolor{fillColor}{RGB}{255,0,0}

\path[draw=drawColor,line width= 0.4pt,line join=round,line cap=round,fill=fillColor] (378.75,197.24) circle (  1.49);
\definecolor{drawColor}{RGB}{0,0,0}
\definecolor{fillColor}{RGB}{0,0,0}

\path[draw=drawColor,line width= 0.4pt,line join=round,line cap=round,fill=fillColor] (378.77,184.84) circle (  1.49);
\definecolor{drawColor}{RGB}{255,0,0}
\definecolor{fillColor}{RGB}{255,0,0}

\path[draw=drawColor,line width= 0.4pt,line join=round,line cap=round,fill=fillColor] (379.24,197.22) circle (  1.49);
\definecolor{drawColor}{RGB}{0,0,0}
\definecolor{fillColor}{RGB}{0,0,0}

\path[draw=drawColor,line width= 0.4pt,line join=round,line cap=round,fill=fillColor] (379.26,185.07) circle (  1.49);
\definecolor{drawColor}{RGB}{255,0,0}
\definecolor{fillColor}{RGB}{255,0,0}

\path[draw=drawColor,line width= 0.4pt,line join=round,line cap=round,fill=fillColor] (379.77,196.51) circle (  1.49);
\definecolor{drawColor}{RGB}{0,0,0}
\definecolor{fillColor}{RGB}{0,0,0}

\path[draw=drawColor,line width= 0.4pt,line join=round,line cap=round,fill=fillColor] (379.78,185.01) circle (  1.49);
\definecolor{drawColor}{RGB}{255,0,0}
\definecolor{fillColor}{RGB}{255,0,0}

\path[draw=drawColor,line width= 0.4pt,line join=round,line cap=round,fill=fillColor] (380.37,195.90) circle (  1.49);
\definecolor{drawColor}{RGB}{0,0,0}
\definecolor{fillColor}{RGB}{0,0,0}

\path[draw=drawColor,line width= 0.4pt,line join=round,line cap=round,fill=fillColor] (380.39,184.42) circle (  1.49);
\definecolor{drawColor}{RGB}{255,0,0}
\definecolor{fillColor}{RGB}{255,0,0}

\path[draw=drawColor,line width= 0.4pt,line join=round,line cap=round,fill=fillColor] (380.86,196.02) circle (  1.49);
\definecolor{drawColor}{RGB}{0,0,0}
\definecolor{fillColor}{RGB}{0,0,0}

\path[draw=drawColor,line width= 0.4pt,line join=round,line cap=round,fill=fillColor] (380.88,183.60) circle (  1.49);
\definecolor{drawColor}{RGB}{255,0,0}
\definecolor{fillColor}{RGB}{255,0,0}

\path[draw=drawColor,line width= 0.4pt,line join=round,line cap=round,fill=fillColor] (381.35,195.59) circle (  1.49);
\definecolor{drawColor}{RGB}{0,0,0}
\definecolor{fillColor}{RGB}{0,0,0}

\path[draw=drawColor,line width= 0.4pt,line join=round,line cap=round,fill=fillColor] (381.37,184.14) circle (  1.49);
\definecolor{drawColor}{RGB}{255,0,0}
\definecolor{fillColor}{RGB}{255,0,0}

\path[draw=drawColor,line width= 0.4pt,line join=round,line cap=round,fill=fillColor] (381.83,195.56) circle (  1.49);
\definecolor{drawColor}{RGB}{0,0,0}
\definecolor{fillColor}{RGB}{0,0,0}

\path[draw=drawColor,line width= 0.4pt,line join=round,line cap=round,fill=fillColor] (381.85,184.18) circle (  1.49);
\definecolor{drawColor}{RGB}{255,0,0}
\definecolor{fillColor}{RGB}{255,0,0}

\path[draw=drawColor,line width= 0.4pt,line join=round,line cap=round,fill=fillColor] (382.35,106.24) circle (  1.49);
\definecolor{drawColor}{RGB}{0,0,0}
\definecolor{fillColor}{RGB}{0,0,0}

\path[draw=drawColor,line width= 0.4pt,line join=round,line cap=round,fill=fillColor] (382.37,183.74) circle (  1.49);
\definecolor{drawColor}{RGB}{255,0,0}
\definecolor{fillColor}{RGB}{255,0,0}

\path[draw=drawColor,line width= 0.4pt,line join=round,line cap=round,fill=fillColor] (382.86,195.67) circle (  1.49);
\definecolor{drawColor}{RGB}{0,0,0}
\definecolor{fillColor}{RGB}{0,0,0}

\path[draw=drawColor,line width= 0.4pt,line join=round,line cap=round,fill=fillColor] (382.88,182.98) circle (  1.49);
\definecolor{drawColor}{RGB}{255,0,0}
\definecolor{fillColor}{RGB}{255,0,0}

\path[draw=drawColor,line width= 0.4pt,line join=round,line cap=round,fill=fillColor] (383.37,195.80) circle (  1.49);
\definecolor{drawColor}{RGB}{0,0,0}
\definecolor{fillColor}{RGB}{0,0,0}

\path[draw=drawColor,line width= 0.4pt,line join=round,line cap=round,fill=fillColor] (383.38,182.04) circle (  1.49);
\definecolor{drawColor}{RGB}{255,0,0}
\definecolor{fillColor}{RGB}{255,0,0}

\path[draw=drawColor,line width= 0.4pt,line join=round,line cap=round,fill=fillColor] (383.86,196.54) circle (  1.49);
\definecolor{drawColor}{RGB}{0,0,0}
\definecolor{fillColor}{RGB}{0,0,0}

\path[draw=drawColor,line width= 0.4pt,line join=round,line cap=round,fill=fillColor] (383.88,176.51) circle (  1.49);
\definecolor{drawColor}{RGB}{255,0,0}
\definecolor{fillColor}{RGB}{255,0,0}

\path[draw=drawColor,line width= 0.4pt,line join=round,line cap=round,fill=fillColor] (384.37,173.46) circle (  1.49);
\definecolor{drawColor}{RGB}{0,0,0}
\definecolor{fillColor}{RGB}{0,0,0}

\path[draw=drawColor,line width= 0.4pt,line join=round,line cap=round,fill=fillColor] (384.38,109.39) circle (  1.49);
\definecolor{drawColor}{RGB}{255,0,0}
\definecolor{fillColor}{RGB}{255,0,0}

\path[draw=drawColor,line width= 0.4pt,line join=round,line cap=round,fill=fillColor] (384.86,196.28) circle (  1.49);
\definecolor{drawColor}{RGB}{0,0,0}
\definecolor{fillColor}{RGB}{0,0,0}

\path[draw=drawColor,line width= 0.4pt,line join=round,line cap=round,fill=fillColor] (384.87,184.41) circle (  1.49);
\definecolor{drawColor}{RGB}{255,0,0}
\definecolor{fillColor}{RGB}{255,0,0}

\path[draw=drawColor,line width= 0.4pt,line join=round,line cap=round,fill=fillColor] (385.37,195.37) circle (  1.49);
\definecolor{drawColor}{RGB}{0,0,0}
\definecolor{fillColor}{RGB}{0,0,0}

\path[draw=drawColor,line width= 0.4pt,line join=round,line cap=round,fill=fillColor] (385.38,164.94) circle (  1.49);
\definecolor{drawColor}{RGB}{255,0,0}
\definecolor{fillColor}{RGB}{255,0,0}

\path[draw=drawColor,line width= 0.4pt,line join=round,line cap=round,fill=fillColor] (385.87,195.56) circle (  1.49);
\definecolor{drawColor}{RGB}{0,0,0}
\definecolor{fillColor}{RGB}{0,0,0}

\path[draw=drawColor,line width= 0.4pt,line join=round,line cap=round,fill=fillColor] (385.89,184.07) circle (  1.49);
\definecolor{drawColor}{RGB}{255,0,0}
\definecolor{fillColor}{RGB}{255,0,0}

\path[draw=drawColor,line width= 0.4pt,line join=round,line cap=round,fill=fillColor] (386.35,194.53) circle (  1.49);
\definecolor{drawColor}{RGB}{0,0,0}
\definecolor{fillColor}{RGB}{0,0,0}

\path[draw=drawColor,line width= 0.4pt,line join=round,line cap=round,fill=fillColor] (386.36,180.88) circle (  1.49);
\definecolor{drawColor}{RGB}{255,0,0}
\definecolor{fillColor}{RGB}{255,0,0}

\path[draw=drawColor,line width= 0.4pt,line join=round,line cap=round,fill=fillColor] (386.87,178.54) circle (  1.49);
\definecolor{drawColor}{RGB}{0,0,0}
\definecolor{fillColor}{RGB}{0,0,0}

\path[draw=drawColor,line width= 0.4pt,line join=round,line cap=round,fill=fillColor] (386.89,182.57) circle (  1.49);
\definecolor{drawColor}{RGB}{255,0,0}
\definecolor{fillColor}{RGB}{255,0,0}

\path[draw=drawColor,line width= 0.4pt,line join=round,line cap=round,fill=fillColor] (387.36,184.00) circle (  1.49);
\definecolor{drawColor}{RGB}{0,0,0}
\definecolor{fillColor}{RGB}{0,0,0}

\path[draw=drawColor,line width= 0.4pt,line join=round,line cap=round,fill=fillColor] (387.38,182.59) circle (  1.49);
\definecolor{drawColor}{RGB}{255,0,0}
\definecolor{fillColor}{RGB}{255,0,0}

\path[draw=drawColor,line width= 0.4pt,line join=round,line cap=round,fill=fillColor] (387.95,197.06) circle (  1.49);
\definecolor{drawColor}{RGB}{0,0,0}
\definecolor{fillColor}{RGB}{0,0,0}

\path[draw=drawColor,line width= 0.4pt,line join=round,line cap=round,fill=fillColor] (387.97,185.57) circle (  1.49);
\definecolor{drawColor}{RGB}{255,0,0}
\definecolor{fillColor}{RGB}{255,0,0}

\path[draw=drawColor,line width= 0.4pt,line join=round,line cap=round,fill=fillColor] (388.48,198.29) circle (  1.49);
\definecolor{drawColor}{RGB}{0,0,0}
\definecolor{fillColor}{RGB}{0,0,0}

\path[draw=drawColor,line width= 0.4pt,line join=round,line cap=round,fill=fillColor] (388.49,186.26) circle (  1.49);
\definecolor{drawColor}{RGB}{255,0,0}
\definecolor{fillColor}{RGB}{255,0,0}

\path[draw=drawColor,line width= 0.4pt,line join=round,line cap=round,fill=fillColor] (388.93,197.84) circle (  1.49);
\definecolor{drawColor}{RGB}{0,0,0}
\definecolor{fillColor}{RGB}{0,0,0}

\path[draw=drawColor,line width= 0.4pt,line join=round,line cap=round,fill=fillColor] (388.95,184.93) circle (  1.49);
\definecolor{drawColor}{RGB}{255,0,0}
\definecolor{fillColor}{RGB}{255,0,0}

\path[draw=drawColor,line width= 0.4pt,line join=round,line cap=round,fill=fillColor] (389.41,196.63) circle (  1.49);
\definecolor{drawColor}{RGB}{0,0,0}
\definecolor{fillColor}{RGB}{0,0,0}

\path[draw=drawColor,line width= 0.4pt,line join=round,line cap=round,fill=fillColor] (389.42,185.11) circle (  1.49);
\definecolor{drawColor}{RGB}{255,0,0}
\definecolor{fillColor}{RGB}{255,0,0}

\path[draw=drawColor,line width= 0.4pt,line join=round,line cap=round,fill=fillColor] (389.95,195.68) circle (  1.49);
\definecolor{drawColor}{RGB}{0,0,0}
\definecolor{fillColor}{RGB}{0,0,0}

\path[draw=drawColor,line width= 0.4pt,line join=round,line cap=round,fill=fillColor] (389.97,184.00) circle (  1.49);
\definecolor{drawColor}{RGB}{255,0,0}
\definecolor{fillColor}{RGB}{255,0,0}

\path[draw=drawColor,line width= 0.4pt,line join=round,line cap=round,fill=fillColor] (390.47,195.54) circle (  1.49);
\definecolor{drawColor}{RGB}{0,0,0}
\definecolor{fillColor}{RGB}{0,0,0}

\path[draw=drawColor,line width= 0.4pt,line join=round,line cap=round,fill=fillColor] (390.49,180.42) circle (  1.49);
\definecolor{drawColor}{RGB}{255,0,0}
\definecolor{fillColor}{RGB}{255,0,0}

\path[draw=drawColor,line width= 0.4pt,line join=round,line cap=round,fill=fillColor] (390.96,196.13) circle (  1.49);
\definecolor{drawColor}{RGB}{0,0,0}
\definecolor{fillColor}{RGB}{0,0,0}

\path[draw=drawColor,line width= 0.4pt,line join=round,line cap=round,fill=fillColor] (391.00,184.23) circle (  1.49);
\definecolor{drawColor}{RGB}{255,0,0}
\definecolor{fillColor}{RGB}{255,0,0}

\path[draw=drawColor,line width= 0.4pt,line join=round,line cap=round,fill=fillColor] (391.45,193.57) circle (  1.49);
\definecolor{drawColor}{RGB}{0,0,0}
\definecolor{fillColor}{RGB}{0,0,0}

\path[draw=drawColor,line width= 0.4pt,line join=round,line cap=round,fill=fillColor] (391.47,178.11) circle (  1.49);
\definecolor{drawColor}{RGB}{255,0,0}
\definecolor{fillColor}{RGB}{255,0,0}

\path[draw=drawColor,line width= 0.4pt,line join=round,line cap=round,fill=fillColor] (391.96,194.17) circle (  1.49);
\definecolor{drawColor}{RGB}{0,0,0}
\definecolor{fillColor}{RGB}{0,0,0}

\path[draw=drawColor,line width= 0.4pt,line join=round,line cap=round,fill=fillColor] (391.98,182.93) circle (  1.49);
\definecolor{drawColor}{RGB}{255,0,0}
\definecolor{fillColor}{RGB}{255,0,0}

\path[draw=drawColor,line width= 0.4pt,line join=round,line cap=round,fill=fillColor] (392.47,194.66) circle (  1.49);
\definecolor{drawColor}{RGB}{0,0,0}
\definecolor{fillColor}{RGB}{0,0,0}

\path[draw=drawColor,line width= 0.4pt,line join=round,line cap=round,fill=fillColor] (392.49,182.76) circle (  1.49);
\definecolor{drawColor}{RGB}{255,0,0}
\definecolor{fillColor}{RGB}{255,0,0}

\path[draw=drawColor,line width= 0.4pt,line join=round,line cap=round,fill=fillColor] (392.99,194.31) circle (  1.49);
\definecolor{drawColor}{RGB}{0,0,0}
\definecolor{fillColor}{RGB}{0,0,0}

\path[draw=drawColor,line width= 0.4pt,line join=round,line cap=round,fill=fillColor] (393.01,182.30) circle (  1.49);
\definecolor{drawColor}{RGB}{255,0,0}
\definecolor{fillColor}{RGB}{255,0,0}

\path[draw=drawColor,line width= 0.4pt,line join=round,line cap=round,fill=fillColor] (393.50,194.36) circle (  1.49);
\definecolor{drawColor}{RGB}{0,0,0}
\definecolor{fillColor}{RGB}{0,0,0}

\path[draw=drawColor,line width= 0.4pt,line join=round,line cap=round,fill=fillColor] (393.52,183.05) circle (  1.49);
\definecolor{drawColor}{RGB}{255,0,0}
\definecolor{fillColor}{RGB}{255,0,0}

\path[draw=drawColor,line width= 0.4pt,line join=round,line cap=round,fill=fillColor] (394.01,193.35) circle (  1.49);
\definecolor{drawColor}{RGB}{0,0,0}
\definecolor{fillColor}{RGB}{0,0,0}

\path[draw=drawColor,line width= 0.4pt,line join=round,line cap=round,fill=fillColor] (394.02,182.05) circle (  1.49);
\definecolor{drawColor}{RGB}{255,0,0}
\definecolor{fillColor}{RGB}{255,0,0}

\path[draw=drawColor,line width= 0.4pt,line join=round,line cap=round,fill=fillColor] (394.50,193.67) circle (  1.49);
\definecolor{drawColor}{RGB}{0,0,0}
\definecolor{fillColor}{RGB}{0,0,0}

\path[draw=drawColor,line width= 0.4pt,line join=round,line cap=round,fill=fillColor] (394.52,181.54) circle (  1.49);
\definecolor{drawColor}{RGB}{255,0,0}
\definecolor{fillColor}{RGB}{255,0,0}

\path[draw=drawColor,line width= 0.4pt,line join=round,line cap=round,fill=fillColor] (395.01,193.71) circle (  1.49);
\definecolor{drawColor}{RGB}{0,0,0}
\definecolor{fillColor}{RGB}{0,0,0}

\path[draw=drawColor,line width= 0.4pt,line join=round,line cap=round,fill=fillColor] (395.02,179.49) circle (  1.49);
\definecolor{drawColor}{RGB}{255,0,0}
\definecolor{fillColor}{RGB}{255,0,0}

\path[draw=drawColor,line width= 0.4pt,line join=round,line cap=round,fill=fillColor] (395.51,194.21) circle (  1.49);
\definecolor{drawColor}{RGB}{0,0,0}
\definecolor{fillColor}{RGB}{0,0,0}

\path[draw=drawColor,line width= 0.4pt,line join=round,line cap=round,fill=fillColor] (395.53,182.39) circle (  1.49);
\definecolor{drawColor}{RGB}{255,0,0}
\definecolor{fillColor}{RGB}{255,0,0}

\path[draw=drawColor,line width= 0.4pt,line join=round,line cap=round,fill=fillColor] (396.02,194.81) circle (  1.49);
\definecolor{drawColor}{RGB}{0,0,0}
\definecolor{fillColor}{RGB}{0,0,0}

\path[draw=drawColor,line width= 0.4pt,line join=round,line cap=round,fill=fillColor] (396.04,125.52) circle (  1.49);
\definecolor{drawColor}{RGB}{255,0,0}
\definecolor{fillColor}{RGB}{255,0,0}

\path[draw=drawColor,line width= 0.4pt,line join=round,line cap=round,fill=fillColor] (396.51,195.28) circle (  1.49);
\definecolor{drawColor}{RGB}{0,0,0}
\definecolor{fillColor}{RGB}{0,0,0}

\path[draw=drawColor,line width= 0.4pt,line join=round,line cap=round,fill=fillColor] (396.55,182.84) circle (  1.49);
\definecolor{drawColor}{RGB}{255,0,0}
\definecolor{fillColor}{RGB}{255,0,0}

\path[draw=drawColor,line width= 0.4pt,line join=round,line cap=round,fill=fillColor] (397.04,195.32) circle (  1.49);
\definecolor{drawColor}{RGB}{0,0,0}
\definecolor{fillColor}{RGB}{0,0,0}

\path[draw=drawColor,line width= 0.4pt,line join=round,line cap=round,fill=fillColor] (397.07, 75.28) circle (  1.49);
\definecolor{drawColor}{RGB}{255,0,0}
\definecolor{fillColor}{RGB}{255,0,0}

\path[draw=drawColor,line width= 0.4pt,line join=round,line cap=round,fill=fillColor] (397.59,194.69) circle (  1.49);
\definecolor{drawColor}{RGB}{0,0,0}
\definecolor{fillColor}{RGB}{0,0,0}

\path[draw=drawColor,line width= 0.4pt,line join=round,line cap=round,fill=fillColor] (397.61,182.53) circle (  1.49);
\definecolor{drawColor}{RGB}{255,0,0}
\definecolor{fillColor}{RGB}{255,0,0}

\path[draw=drawColor,line width= 0.4pt,line join=round,line cap=round,fill=fillColor] (398.12,194.71) circle (  1.49);
\definecolor{drawColor}{RGB}{0,0,0}
\definecolor{fillColor}{RGB}{0,0,0}

\path[draw=drawColor,line width= 0.4pt,line join=round,line cap=round,fill=fillColor] (398.13,182.66) circle (  1.49);
\definecolor{drawColor}{RGB}{255,0,0}
\definecolor{fillColor}{RGB}{255,0,0}

\path[draw=drawColor,line width= 0.4pt,line join=round,line cap=round,fill=fillColor] (398.64,195.65) circle (  1.49);
\definecolor{drawColor}{RGB}{0,0,0}
\definecolor{fillColor}{RGB}{0,0,0}

\path[draw=drawColor,line width= 0.4pt,line join=round,line cap=round,fill=fillColor] (398.66,183.03) circle (  1.49);
\definecolor{drawColor}{RGB}{255,0,0}
\definecolor{fillColor}{RGB}{255,0,0}

\path[draw=drawColor,line width= 0.4pt,line join=round,line cap=round,fill=fillColor] (399.12,195.63) circle (  1.49);
\definecolor{drawColor}{RGB}{0,0,0}
\definecolor{fillColor}{RGB}{0,0,0}

\path[draw=drawColor,line width= 0.4pt,line join=round,line cap=round,fill=fillColor] (399.13,182.41) circle (  1.49);
\definecolor{drawColor}{RGB}{255,0,0}
\definecolor{fillColor}{RGB}{255,0,0}

\path[draw=drawColor,line width= 0.4pt,line join=round,line cap=round,fill=fillColor] (399.66,195.13) circle (  1.49);
\definecolor{drawColor}{RGB}{0,0,0}
\definecolor{fillColor}{RGB}{0,0,0}

\path[draw=drawColor,line width= 0.4pt,line join=round,line cap=round,fill=fillColor] (399.67,182.45) circle (  1.49);
\definecolor{drawColor}{RGB}{255,0,0}
\definecolor{fillColor}{RGB}{255,0,0}

\path[draw=drawColor,line width= 0.4pt,line join=round,line cap=round,fill=fillColor] (400.15,194.98) circle (  1.49);
\definecolor{drawColor}{RGB}{0,0,0}
\definecolor{fillColor}{RGB}{0,0,0}

\path[draw=drawColor,line width= 0.4pt,line join=round,line cap=round,fill=fillColor] (400.16,182.47) circle (  1.49);
\definecolor{drawColor}{RGB}{255,0,0}
\definecolor{fillColor}{RGB}{255,0,0}

\path[draw=drawColor,line width= 0.4pt,line join=round,line cap=round,fill=fillColor] (400.67,195.58) circle (  1.49);
\definecolor{drawColor}{RGB}{0,0,0}
\definecolor{fillColor}{RGB}{0,0,0}

\path[draw=drawColor,line width= 0.4pt,line join=round,line cap=round,fill=fillColor] (400.70,182.71) circle (  1.49);
\definecolor{drawColor}{RGB}{255,0,0}
\definecolor{fillColor}{RGB}{255,0,0}

\path[draw=drawColor,line width= 0.4pt,line join=round,line cap=round,fill=fillColor] (401.37,194.85) circle (  1.49);
\definecolor{drawColor}{RGB}{0,0,0}
\definecolor{fillColor}{RGB}{0,0,0}

\path[draw=drawColor,line width= 0.4pt,line join=round,line cap=round,fill=fillColor] (401.39,175.47) circle (  1.49);
\definecolor{drawColor}{RGB}{255,0,0}
\definecolor{fillColor}{RGB}{255,0,0}

\path[draw=drawColor,line width= 0.4pt,line join=round,line cap=round,fill=fillColor] (401.85,194.20) circle (  1.49);
\definecolor{drawColor}{RGB}{0,0,0}
\definecolor{fillColor}{RGB}{0,0,0}

\path[draw=drawColor,line width= 0.4pt,line join=round,line cap=round,fill=fillColor] (401.87,181.19) circle (  1.49);
\definecolor{drawColor}{RGB}{255,0,0}
\definecolor{fillColor}{RGB}{255,0,0}

\path[draw=drawColor,line width= 0.4pt,line join=round,line cap=round,fill=fillColor] (402.36,193.15) circle (  1.49);
\definecolor{drawColor}{RGB}{0,0,0}
\definecolor{fillColor}{RGB}{0,0,0}

\path[draw=drawColor,line width= 0.4pt,line join=round,line cap=round,fill=fillColor] (402.37,179.87) circle (  1.49);
\definecolor{drawColor}{RGB}{255,0,0}
\definecolor{fillColor}{RGB}{255,0,0}

\path[draw=drawColor,line width= 0.4pt,line join=round,line cap=round,fill=fillColor] (402.83,192.53) circle (  1.49);
\definecolor{drawColor}{RGB}{0,0,0}
\definecolor{fillColor}{RGB}{0,0,0}

\path[draw=drawColor,line width= 0.4pt,line join=round,line cap=round,fill=fillColor] (402.85,180.11) circle (  1.49);
\definecolor{drawColor}{RGB}{255,0,0}
\definecolor{fillColor}{RGB}{255,0,0}

\path[draw=drawColor,line width= 0.4pt,line join=round,line cap=round,fill=fillColor] (403.36,192.84) circle (  1.49);
\definecolor{drawColor}{RGB}{0,0,0}
\definecolor{fillColor}{RGB}{0,0,0}

\path[draw=drawColor,line width= 0.4pt,line join=round,line cap=round,fill=fillColor] (403.37,176.29) circle (  1.49);
\definecolor{drawColor}{RGB}{255,0,0}
\definecolor{fillColor}{RGB}{255,0,0}

\path[draw=drawColor,line width= 0.4pt,line join=round,line cap=round,fill=fillColor] (403.85,193.91) circle (  1.49);
\definecolor{drawColor}{RGB}{0,0,0}
\definecolor{fillColor}{RGB}{0,0,0}

\path[draw=drawColor,line width= 0.4pt,line join=round,line cap=round,fill=fillColor] (403.86,180.14) circle (  1.49);
\definecolor{drawColor}{RGB}{255,0,0}
\definecolor{fillColor}{RGB}{255,0,0}

\path[draw=drawColor,line width= 0.4pt,line join=round,line cap=round,fill=fillColor] (404.40,193.57) circle (  1.49);
\definecolor{drawColor}{RGB}{0,0,0}
\definecolor{fillColor}{RGB}{0,0,0}

\path[draw=drawColor,line width= 0.4pt,line join=round,line cap=round,fill=fillColor] (404.42,181.62) circle (  1.49);
\definecolor{drawColor}{RGB}{255,0,0}
\definecolor{fillColor}{RGB}{255,0,0}

\path[draw=drawColor,line width= 0.4pt,line join=round,line cap=round,fill=fillColor] (404.91,193.73) circle (  1.49);
\definecolor{drawColor}{RGB}{0,0,0}
\definecolor{fillColor}{RGB}{0,0,0}

\path[draw=drawColor,line width= 0.4pt,line join=round,line cap=round,fill=fillColor] (404.93,181.66) circle (  1.49);
\definecolor{drawColor}{RGB}{255,0,0}
\definecolor{fillColor}{RGB}{255,0,0}

\path[draw=drawColor,line width= 0.4pt,line join=round,line cap=round,fill=fillColor] (405.42,193.50) circle (  1.49);
\definecolor{drawColor}{RGB}{0,0,0}
\definecolor{fillColor}{RGB}{0,0,0}

\path[draw=drawColor,line width= 0.4pt,line join=round,line cap=round,fill=fillColor] (405.43,181.16) circle (  1.49);
\definecolor{drawColor}{RGB}{255,0,0}
\definecolor{fillColor}{RGB}{255,0,0}

\path[draw=drawColor,line width= 0.4pt,line join=round,line cap=round,fill=fillColor] (405.93,193.61) circle (  1.49);
\definecolor{drawColor}{RGB}{0,0,0}
\definecolor{fillColor}{RGB}{0,0,0}

\path[draw=drawColor,line width= 0.4pt,line join=round,line cap=round,fill=fillColor] (405.94,181.31) circle (  1.49);
\definecolor{drawColor}{RGB}{255,0,0}
\definecolor{fillColor}{RGB}{255,0,0}

\path[draw=drawColor,line width= 0.4pt,line join=round,line cap=round,fill=fillColor] (406.42,193.76) circle (  1.49);
\definecolor{drawColor}{RGB}{0,0,0}
\definecolor{fillColor}{RGB}{0,0,0}

\path[draw=drawColor,line width= 0.4pt,line join=round,line cap=round,fill=fillColor] (406.43,180.01) circle (  1.49);
\definecolor{drawColor}{RGB}{255,0,0}
\definecolor{fillColor}{RGB}{255,0,0}

\path[draw=drawColor,line width= 0.4pt,line join=round,line cap=round,fill=fillColor] (407.07,193.67) circle (  1.49);
\definecolor{drawColor}{RGB}{0,0,0}
\definecolor{fillColor}{RGB}{0,0,0}

\path[draw=drawColor,line width= 0.4pt,line join=round,line cap=round,fill=fillColor] (407.09,176.08) circle (  1.49);
\definecolor{drawColor}{RGB}{255,0,0}
\definecolor{fillColor}{RGB}{255,0,0}

\path[draw=drawColor,line width= 0.4pt,line join=round,line cap=round,fill=fillColor] (407.56,194.26) circle (  1.49);
\definecolor{drawColor}{RGB}{0,0,0}
\definecolor{fillColor}{RGB}{0,0,0}

\path[draw=drawColor,line width= 0.4pt,line join=round,line cap=round,fill=fillColor] (407.58,180.96) circle (  1.49);
\definecolor{drawColor}{RGB}{255,0,0}
\definecolor{fillColor}{RGB}{255,0,0}

\path[draw=drawColor,line width= 0.4pt,line join=round,line cap=round,fill=fillColor] (408.05,193.65) circle (  1.49);
\definecolor{drawColor}{RGB}{0,0,0}
\definecolor{fillColor}{RGB}{0,0,0}

\path[draw=drawColor,line width= 0.4pt,line join=round,line cap=round,fill=fillColor] (408.07,180.94) circle (  1.49);
\definecolor{drawColor}{RGB}{255,0,0}
\definecolor{fillColor}{RGB}{255,0,0}

\path[draw=drawColor,line width= 0.4pt,line join=round,line cap=round,fill=fillColor] (408.61,193.70) circle (  1.49);
\definecolor{drawColor}{RGB}{0,0,0}
\definecolor{fillColor}{RGB}{0,0,0}

\path[draw=drawColor,line width= 0.4pt,line join=round,line cap=round,fill=fillColor] (408.63,181.26) circle (  1.49);
\definecolor{drawColor}{RGB}{255,0,0}
\definecolor{fillColor}{RGB}{255,0,0}

\path[draw=drawColor,line width= 0.4pt,line join=round,line cap=round,fill=fillColor] (409.12,193.88) circle (  1.49);
\definecolor{drawColor}{RGB}{0,0,0}
\definecolor{fillColor}{RGB}{0,0,0}

\path[draw=drawColor,line width= 0.4pt,line join=round,line cap=round,fill=fillColor] (409.13,181.41) circle (  1.49);
\definecolor{drawColor}{RGB}{255,0,0}
\definecolor{fillColor}{RGB}{255,0,0}

\path[draw=drawColor,line width= 0.4pt,line join=round,line cap=round,fill=fillColor] (409.66,194.48) circle (  1.49);
\definecolor{drawColor}{RGB}{0,0,0}
\definecolor{fillColor}{RGB}{0,0,0}

\path[draw=drawColor,line width= 0.4pt,line join=round,line cap=round,fill=fillColor] (409.67,181.50) circle (  1.49);
\definecolor{drawColor}{RGB}{255,0,0}
\definecolor{fillColor}{RGB}{255,0,0}

\path[draw=drawColor,line width= 0.4pt,line join=round,line cap=round,fill=fillColor] (410.15,194.14) circle (  1.49);
\definecolor{drawColor}{RGB}{0,0,0}
\definecolor{fillColor}{RGB}{0,0,0}

\path[draw=drawColor,line width= 0.4pt,line join=round,line cap=round,fill=fillColor] (410.17,181.01) circle (  1.49);
\definecolor{drawColor}{RGB}{255,0,0}
\definecolor{fillColor}{RGB}{255,0,0}

\path[draw=drawColor,line width= 0.4pt,line join=round,line cap=round,fill=fillColor] (410.62,193.61) circle (  1.49);
\definecolor{drawColor}{RGB}{0,0,0}
\definecolor{fillColor}{RGB}{0,0,0}

\path[draw=drawColor,line width= 0.4pt,line join=round,line cap=round,fill=fillColor] (410.64,180.65) circle (  1.49);
\definecolor{drawColor}{RGB}{255,0,0}
\definecolor{fillColor}{RGB}{255,0,0}

\path[draw=drawColor,line width= 0.4pt,line join=round,line cap=round,fill=fillColor] (411.15,192.97) circle (  1.49);
\definecolor{drawColor}{RGB}{0,0,0}
\definecolor{fillColor}{RGB}{0,0,0}

\path[draw=drawColor,line width= 0.4pt,line join=round,line cap=round,fill=fillColor] (411.16,180.20) circle (  1.49);
\definecolor{drawColor}{RGB}{255,0,0}
\definecolor{fillColor}{RGB}{255,0,0}

\path[draw=drawColor,line width= 0.4pt,line join=round,line cap=round,fill=fillColor] (411.65,192.80) circle (  1.49);
\definecolor{drawColor}{RGB}{0,0,0}
\definecolor{fillColor}{RGB}{0,0,0}

\path[draw=drawColor,line width= 0.4pt,line join=round,line cap=round,fill=fillColor] (411.67,180.05) circle (  1.49);
\definecolor{drawColor}{RGB}{255,0,0}
\definecolor{fillColor}{RGB}{255,0,0}

\path[draw=drawColor,line width= 0.4pt,line join=round,line cap=round,fill=fillColor] (412.23,193.06) circle (  1.49);
\definecolor{drawColor}{RGB}{0,0,0}
\definecolor{fillColor}{RGB}{0,0,0}

\path[draw=drawColor,line width= 0.4pt,line join=round,line cap=round,fill=fillColor] (412.24,180.02) circle (  1.49);
\definecolor{drawColor}{RGB}{255,0,0}
\definecolor{fillColor}{RGB}{255,0,0}

\path[draw=drawColor,line width= 0.4pt,line join=round,line cap=round,fill=fillColor] (412.82,192.42) circle (  1.49);
\definecolor{drawColor}{RGB}{0,0,0}
\definecolor{fillColor}{RGB}{0,0,0}

\path[draw=drawColor,line width= 0.4pt,line join=round,line cap=round,fill=fillColor] (412.83,179.93) circle (  1.49);
\definecolor{drawColor}{RGB}{255,0,0}
\definecolor{fillColor}{RGB}{255,0,0}

\path[draw=drawColor,line width= 0.4pt,line join=round,line cap=round,fill=fillColor] (413.32,191.91) circle (  1.49);
\definecolor{drawColor}{RGB}{0,0,0}
\definecolor{fillColor}{RGB}{0,0,0}

\path[draw=drawColor,line width= 0.4pt,line join=round,line cap=round,fill=fillColor] (413.34,178.66) circle (  1.49);
\definecolor{drawColor}{RGB}{255,0,0}
\definecolor{fillColor}{RGB}{255,0,0}

\path[draw=drawColor,line width= 0.4pt,line join=round,line cap=round,fill=fillColor] (413.80,191.74) circle (  1.49);
\definecolor{drawColor}{RGB}{0,0,0}
\definecolor{fillColor}{RGB}{0,0,0}

\path[draw=drawColor,line width= 0.4pt,line join=round,line cap=round,fill=fillColor] (413.82,179.23) circle (  1.49);
\definecolor{drawColor}{RGB}{255,0,0}
\definecolor{fillColor}{RGB}{255,0,0}

\path[draw=drawColor,line width= 0.4pt,line join=round,line cap=round,fill=fillColor] (414.36,192.04) circle (  1.49);
\definecolor{drawColor}{RGB}{0,0,0}
\definecolor{fillColor}{RGB}{0,0,0}

\path[draw=drawColor,line width= 0.4pt,line join=round,line cap=round,fill=fillColor] (414.37,179.53) circle (  1.49);
\definecolor{drawColor}{RGB}{255,0,0}
\definecolor{fillColor}{RGB}{255,0,0}

\path[draw=drawColor,line width= 0.4pt,line join=round,line cap=round,fill=fillColor] (414.86,192.00) circle (  1.49);
\definecolor{drawColor}{RGB}{0,0,0}
\definecolor{fillColor}{RGB}{0,0,0}

\path[draw=drawColor,line width= 0.4pt,line join=round,line cap=round,fill=fillColor] (414.88,179.58) circle (  1.49);
\definecolor{drawColor}{RGB}{255,0,0}
\definecolor{fillColor}{RGB}{255,0,0}

\path[draw=drawColor,line width= 0.4pt,line join=round,line cap=round,fill=fillColor] (415.34,192.14) circle (  1.49);
\definecolor{drawColor}{RGB}{0,0,0}
\definecolor{fillColor}{RGB}{0,0,0}

\path[draw=drawColor,line width= 0.4pt,line join=round,line cap=round,fill=fillColor] (415.35,179.67) circle (  1.49);
\definecolor{drawColor}{RGB}{255,0,0}
\definecolor{fillColor}{RGB}{255,0,0}

\path[draw=drawColor,line width= 0.4pt,line join=round,line cap=round,fill=fillColor] (415.81,192.48) circle (  1.49);
\definecolor{drawColor}{RGB}{0,0,0}
\definecolor{fillColor}{RGB}{0,0,0}

\path[draw=drawColor,line width= 0.4pt,line join=round,line cap=round,fill=fillColor] (415.83,179.96) circle (  1.49);
\definecolor{drawColor}{RGB}{255,0,0}
\definecolor{fillColor}{RGB}{255,0,0}

\path[draw=drawColor,line width= 0.4pt,line join=round,line cap=round,fill=fillColor] (416.29,192.28) circle (  1.49);
\definecolor{drawColor}{RGB}{0,0,0}
\definecolor{fillColor}{RGB}{0,0,0}

\path[draw=drawColor,line width= 0.4pt,line join=round,line cap=round,fill=fillColor] (416.30,179.50) circle (  1.49);
\definecolor{drawColor}{RGB}{255,0,0}
\definecolor{fillColor}{RGB}{255,0,0}

\path[draw=drawColor,line width= 0.4pt,line join=round,line cap=round,fill=fillColor] (416.76,192.40) circle (  1.49);
\definecolor{drawColor}{RGB}{0,0,0}
\definecolor{fillColor}{RGB}{0,0,0}

\path[draw=drawColor,line width= 0.4pt,line join=round,line cap=round,fill=fillColor] (416.78,180.13) circle (  1.49);
\definecolor{drawColor}{RGB}{255,0,0}
\definecolor{fillColor}{RGB}{255,0,0}

\path[draw=drawColor,line width= 0.4pt,line join=round,line cap=round,fill=fillColor] (417.29,192.36) circle (  1.49);
\definecolor{drawColor}{RGB}{0,0,0}
\definecolor{fillColor}{RGB}{0,0,0}

\path[draw=drawColor,line width= 0.4pt,line join=round,line cap=round,fill=fillColor] (417.30,170.55) circle (  1.49);
\definecolor{drawColor}{RGB}{255,0,0}
\definecolor{fillColor}{RGB}{255,0,0}

\path[draw=drawColor,line width= 0.4pt,line join=round,line cap=round,fill=fillColor] (417.76,191.45) circle (  1.49);
\definecolor{drawColor}{RGB}{0,0,0}
\definecolor{fillColor}{RGB}{0,0,0}

\path[draw=drawColor,line width= 0.4pt,line join=round,line cap=round,fill=fillColor] (417.78,177.17) circle (  1.49);
\definecolor{drawColor}{RGB}{255,0,0}
\definecolor{fillColor}{RGB}{255,0,0}

\path[draw=drawColor,line width= 0.4pt,line join=round,line cap=round,fill=fillColor] (418.38,191.30) circle (  1.49);
\definecolor{drawColor}{RGB}{0,0,0}
\definecolor{fillColor}{RGB}{0,0,0}

\path[draw=drawColor,line width= 0.4pt,line join=round,line cap=round,fill=fillColor] (418.40,165.95) circle (  1.49);
\definecolor{drawColor}{RGB}{255,0,0}
\definecolor{fillColor}{RGB}{255,0,0}

\path[draw=drawColor,line width= 0.4pt,line join=round,line cap=round,fill=fillColor] (418.96,190.81) circle (  1.49);
\definecolor{drawColor}{RGB}{0,0,0}
\definecolor{fillColor}{RGB}{0,0,0}

\path[draw=drawColor,line width= 0.4pt,line join=round,line cap=round,fill=fillColor] (418.97,178.49) circle (  1.49);
\definecolor{drawColor}{RGB}{255,0,0}
\definecolor{fillColor}{RGB}{255,0,0}

\path[draw=drawColor,line width= 0.4pt,line join=round,line cap=round,fill=fillColor] (419.51,189.91) circle (  1.49);
\definecolor{drawColor}{RGB}{0,0,0}
\definecolor{fillColor}{RGB}{0,0,0}

\path[draw=drawColor,line width= 0.4pt,line join=round,line cap=round,fill=fillColor] (419.53,177.91) circle (  1.49);
\definecolor{drawColor}{RGB}{255,0,0}
\definecolor{fillColor}{RGB}{255,0,0}

\path[draw=drawColor,line width= 0.4pt,line join=round,line cap=round,fill=fillColor] (419.97,190.43) circle (  1.49);
\definecolor{drawColor}{RGB}{0,0,0}
\definecolor{fillColor}{RGB}{0,0,0}

\path[draw=drawColor,line width= 0.4pt,line join=round,line cap=round,fill=fillColor] (419.99,178.15) circle (  1.49);
\definecolor{drawColor}{RGB}{255,0,0}
\definecolor{fillColor}{RGB}{255,0,0}

\path[draw=drawColor,line width= 0.4pt,line join=round,line cap=round,fill=fillColor] (420.43,191.25) circle (  1.49);
\definecolor{drawColor}{RGB}{0,0,0}
\definecolor{fillColor}{RGB}{0,0,0}

\path[draw=drawColor,line width= 0.4pt,line join=round,line cap=round,fill=fillColor] (420.45,178.86) circle (  1.49);
\definecolor{drawColor}{RGB}{255,0,0}
\definecolor{fillColor}{RGB}{255,0,0}

\path[draw=drawColor,line width= 0.4pt,line join=round,line cap=round,fill=fillColor] (420.89,191.43) circle (  1.49);
\definecolor{drawColor}{RGB}{0,0,0}
\definecolor{fillColor}{RGB}{0,0,0}

\path[draw=drawColor,line width= 0.4pt,line join=round,line cap=round,fill=fillColor] (420.90,179.00) circle (  1.49);
\definecolor{drawColor}{RGB}{255,0,0}
\definecolor{fillColor}{RGB}{255,0,0}

\path[draw=drawColor,line width= 0.4pt,line join=round,line cap=round,fill=fillColor] (421.36,191.03) circle (  1.49);
\definecolor{drawColor}{RGB}{0,0,0}
\definecolor{fillColor}{RGB}{0,0,0}

\path[draw=drawColor,line width= 0.4pt,line join=round,line cap=round,fill=fillColor] (421.38,178.66) circle (  1.49);
\definecolor{drawColor}{RGB}{255,0,0}
\definecolor{fillColor}{RGB}{255,0,0}

\path[draw=drawColor,line width= 0.4pt,line join=round,line cap=round,fill=fillColor] (421.82,191.16) circle (  1.49);
\definecolor{drawColor}{RGB}{0,0,0}
\definecolor{fillColor}{RGB}{0,0,0}

\path[draw=drawColor,line width= 0.4pt,line join=round,line cap=round,fill=fillColor] (421.84,178.54) circle (  1.49);
\definecolor{drawColor}{RGB}{255,0,0}
\definecolor{fillColor}{RGB}{255,0,0}

\path[draw=drawColor,line width= 0.4pt,line join=round,line cap=round,fill=fillColor] (422.29,190.64) circle (  1.49);
\definecolor{drawColor}{RGB}{0,0,0}
\definecolor{fillColor}{RGB}{0,0,0}

\path[draw=drawColor,line width= 0.4pt,line join=round,line cap=round,fill=fillColor] (422.31,178.39) circle (  1.49);
\definecolor{drawColor}{RGB}{255,0,0}
\definecolor{fillColor}{RGB}{255,0,0}

\path[draw=drawColor,line width= 0.4pt,line join=round,line cap=round,fill=fillColor] (422.90,190.75) circle (  1.49);
\definecolor{drawColor}{RGB}{0,0,0}
\definecolor{fillColor}{RGB}{0,0,0}

\path[draw=drawColor,line width= 0.4pt,line join=round,line cap=round,fill=fillColor] (422.92,178.49) circle (  1.49);
\definecolor{drawColor}{RGB}{255,0,0}
\definecolor{fillColor}{RGB}{255,0,0}

\path[draw=drawColor,line width= 0.4pt,line join=round,line cap=round,fill=fillColor] (423.42,190.92) circle (  1.49);
\definecolor{drawColor}{RGB}{0,0,0}
\definecolor{fillColor}{RGB}{0,0,0}

\path[draw=drawColor,line width= 0.4pt,line join=round,line cap=round,fill=fillColor] (423.44,178.32) circle (  1.49);
\definecolor{drawColor}{RGB}{255,0,0}
\definecolor{fillColor}{RGB}{255,0,0}

\path[draw=drawColor,line width= 0.4pt,line join=round,line cap=round,fill=fillColor] (423.98,190.70) circle (  1.49);
\definecolor{drawColor}{RGB}{0,0,0}
\definecolor{fillColor}{RGB}{0,0,0}

\path[draw=drawColor,line width= 0.4pt,line join=round,line cap=round,fill=fillColor] (424.00,178.53) circle (  1.49);
\definecolor{drawColor}{RGB}{255,0,0}
\definecolor{fillColor}{RGB}{255,0,0}

\path[draw=drawColor,line width= 0.4pt,line join=round,line cap=round,fill=fillColor] (424.44,191.28) circle (  1.49);
\definecolor{drawColor}{RGB}{0,0,0}
\definecolor{fillColor}{RGB}{0,0,0}

\path[draw=drawColor,line width= 0.4pt,line join=round,line cap=round,fill=fillColor] (424.47,178.56) circle (  1.49);
\definecolor{drawColor}{RGB}{255,0,0}
\definecolor{fillColor}{RGB}{255,0,0}

\path[draw=drawColor,line width= 0.4pt,line join=round,line cap=round,fill=fillColor] (424.88,190.59) circle (  1.49);
\definecolor{drawColor}{RGB}{0,0,0}
\definecolor{fillColor}{RGB}{0,0,0}

\path[draw=drawColor,line width= 0.4pt,line join=round,line cap=round,fill=fillColor] (424.91,178.34) circle (  1.49);
\definecolor{drawColor}{RGB}{255,0,0}
\definecolor{fillColor}{RGB}{255,0,0}

\path[draw=drawColor,line width= 0.4pt,line join=round,line cap=round,fill=fillColor] (425.37,190.24) circle (  1.49);
\definecolor{drawColor}{RGB}{0,0,0}
\definecolor{fillColor}{RGB}{0,0,0}

\path[draw=drawColor,line width= 0.4pt,line join=round,line cap=round,fill=fillColor] (425.39,177.44) circle (  1.49);
\definecolor{drawColor}{RGB}{255,0,0}
\definecolor{fillColor}{RGB}{255,0,0}

\path[draw=drawColor,line width= 0.4pt,line join=round,line cap=round,fill=fillColor] (425.93,189.65) circle (  1.49);
\definecolor{drawColor}{RGB}{0,0,0}
\definecolor{fillColor}{RGB}{0,0,0}

\path[draw=drawColor,line width= 0.4pt,line join=round,line cap=round,fill=fillColor] (425.95,177.40) circle (  1.49);
\definecolor{drawColor}{RGB}{255,0,0}
\definecolor{fillColor}{RGB}{255,0,0}

\path[draw=drawColor,line width= 0.4pt,line join=round,line cap=round,fill=fillColor] (426.42,189.70) circle (  1.49);
\definecolor{drawColor}{RGB}{0,0,0}
\definecolor{fillColor}{RGB}{0,0,0}

\path[draw=drawColor,line width= 0.4pt,line join=round,line cap=round,fill=fillColor] (426.44,173.04) circle (  1.49);
\definecolor{drawColor}{RGB}{255,0,0}
\definecolor{fillColor}{RGB}{255,0,0}

\path[draw=drawColor,line width= 0.4pt,line join=round,line cap=round,fill=fillColor] (426.89,189.69) circle (  1.49);
\definecolor{drawColor}{RGB}{0,0,0}
\definecolor{fillColor}{RGB}{0,0,0}

\path[draw=drawColor,line width= 0.4pt,line join=round,line cap=round,fill=fillColor] (426.91,122.23) circle (  1.49);
\definecolor{drawColor}{RGB}{255,0,0}
\definecolor{fillColor}{RGB}{255,0,0}

\path[draw=drawColor,line width= 0.4pt,line join=round,line cap=round,fill=fillColor] (427.37,189.81) circle (  1.49);
\definecolor{drawColor}{RGB}{0,0,0}
\definecolor{fillColor}{RGB}{0,0,0}

\path[draw=drawColor,line width= 0.4pt,line join=round,line cap=round,fill=fillColor] (427.39,177.50) circle (  1.49);
\definecolor{drawColor}{RGB}{255,0,0}
\definecolor{fillColor}{RGB}{255,0,0}

\path[draw=drawColor,line width= 0.4pt,line join=round,line cap=round,fill=fillColor] (427.83,189.52) circle (  1.49);
\definecolor{drawColor}{RGB}{0,0,0}
\definecolor{fillColor}{RGB}{0,0,0}

\path[draw=drawColor,line width= 0.4pt,line join=round,line cap=round,fill=fillColor] (427.86,177.48) circle (  1.49);
\definecolor{drawColor}{RGB}{255,0,0}
\definecolor{fillColor}{RGB}{255,0,0}

\path[draw=drawColor,line width= 0.4pt,line join=round,line cap=round,fill=fillColor] (428.30,189.12) circle (  1.49);
\definecolor{drawColor}{RGB}{0,0,0}
\definecolor{fillColor}{RGB}{0,0,0}

\path[draw=drawColor,line width= 0.4pt,line join=round,line cap=round,fill=fillColor] (428.32,174.54) circle (  1.49);
\definecolor{drawColor}{RGB}{255,0,0}
\definecolor{fillColor}{RGB}{255,0,0}

\path[draw=drawColor,line width= 0.4pt,line join=round,line cap=round,fill=fillColor] (428.78,188.93) circle (  1.49);
\definecolor{drawColor}{RGB}{0,0,0}
\definecolor{fillColor}{RGB}{0,0,0}

\path[draw=drawColor,line width= 0.4pt,line join=round,line cap=round,fill=fillColor] (428.79,176.65) circle (  1.49);
\definecolor{drawColor}{RGB}{255,0,0}
\definecolor{fillColor}{RGB}{255,0,0}

\path[draw=drawColor,line width= 0.4pt,line join=round,line cap=round,fill=fillColor] (429.25,189.27) circle (  1.49);
\definecolor{drawColor}{RGB}{0,0,0}
\definecolor{fillColor}{RGB}{0,0,0}

\path[draw=drawColor,line width= 0.4pt,line join=round,line cap=round,fill=fillColor] (429.27,167.14) circle (  1.49);
\definecolor{drawColor}{RGB}{255,0,0}
\definecolor{fillColor}{RGB}{255,0,0}

\path[draw=drawColor,line width= 0.4pt,line join=round,line cap=round,fill=fillColor] (429.74,189.34) circle (  1.49);
\definecolor{drawColor}{RGB}{0,0,0}
\definecolor{fillColor}{RGB}{0,0,0}

\path[draw=drawColor,line width= 0.4pt,line join=round,line cap=round,fill=fillColor] (429.76,157.90) circle (  1.49);
\definecolor{drawColor}{RGB}{255,0,0}
\definecolor{fillColor}{RGB}{255,0,0}

\path[draw=drawColor,line width= 0.4pt,line join=round,line cap=round,fill=fillColor] (430.20,188.96) circle (  1.49);
\definecolor{drawColor}{RGB}{0,0,0}
\definecolor{fillColor}{RGB}{0,0,0}

\path[draw=drawColor,line width= 0.4pt,line join=round,line cap=round,fill=fillColor] (430.23,172.15) circle (  1.49);
\definecolor{drawColor}{RGB}{255,0,0}
\definecolor{fillColor}{RGB}{255,0,0}

\path[draw=drawColor,line width= 0.4pt,line join=round,line cap=round,fill=fillColor] (430.68,188.51) circle (  1.49);
\definecolor{drawColor}{RGB}{0,0,0}
\definecolor{fillColor}{RGB}{0,0,0}

\path[draw=drawColor,line width= 0.4pt,line join=round,line cap=round,fill=fillColor] (430.69,176.35) circle (  1.49);
\definecolor{drawColor}{RGB}{255,0,0}
\definecolor{fillColor}{RGB}{255,0,0}

\path[draw=drawColor,line width= 0.4pt,line join=round,line cap=round,fill=fillColor] (431.20,189.13) circle (  1.49);
\definecolor{drawColor}{RGB}{0,0,0}
\definecolor{fillColor}{RGB}{0,0,0}

\path[draw=drawColor,line width= 0.4pt,line join=round,line cap=round,fill=fillColor] (431.22,166.26) circle (  1.49);
\definecolor{drawColor}{RGB}{255,0,0}
\definecolor{fillColor}{RGB}{255,0,0}

\path[draw=drawColor,line width= 0.4pt,line join=round,line cap=round,fill=fillColor] (431.71,188.70) circle (  1.49);
\definecolor{drawColor}{RGB}{0,0,0}
\definecolor{fillColor}{RGB}{0,0,0}

\path[draw=drawColor,line width= 0.4pt,line join=round,line cap=round,fill=fillColor] (431.72,171.75) circle (  1.49);
\definecolor{drawColor}{RGB}{255,0,0}
\definecolor{fillColor}{RGB}{255,0,0}

\path[draw=drawColor,line width= 0.4pt,line join=round,line cap=round,fill=fillColor] (432.18,187.62) circle (  1.49);
\definecolor{drawColor}{RGB}{0,0,0}
\definecolor{fillColor}{RGB}{0,0,0}

\path[draw=drawColor,line width= 0.4pt,line join=round,line cap=round,fill=fillColor] (432.20,175.90) circle (  1.49);
\definecolor{drawColor}{RGB}{255,0,0}
\definecolor{fillColor}{RGB}{255,0,0}

\path[draw=drawColor,line width= 0.4pt,line join=round,line cap=round,fill=fillColor] (432.67,187.50) circle (  1.49);
\definecolor{drawColor}{RGB}{0,0,0}
\definecolor{fillColor}{RGB}{0,0,0}

\path[draw=drawColor,line width= 0.4pt,line join=round,line cap=round,fill=fillColor] (432.69,175.25) circle (  1.49);
\definecolor{drawColor}{RGB}{255,0,0}
\definecolor{fillColor}{RGB}{255,0,0}

\path[draw=drawColor,line width= 0.4pt,line join=round,line cap=round,fill=fillColor] (433.15,186.92) circle (  1.49);
\definecolor{drawColor}{RGB}{0,0,0}
\definecolor{fillColor}{RGB}{0,0,0}

\path[draw=drawColor,line width= 0.4pt,line join=round,line cap=round,fill=fillColor] (433.16,174.53) circle (  1.49);
\definecolor{drawColor}{RGB}{255,0,0}
\definecolor{fillColor}{RGB}{255,0,0}

\path[draw=drawColor,line width= 0.4pt,line join=round,line cap=round,fill=fillColor] (433.62,186.69) circle (  1.49);
\definecolor{drawColor}{RGB}{0,0,0}
\definecolor{fillColor}{RGB}{0,0,0}

\path[draw=drawColor,line width= 0.4pt,line join=round,line cap=round,fill=fillColor] (433.64,174.87) circle (  1.49);
\definecolor{drawColor}{RGB}{255,0,0}
\definecolor{fillColor}{RGB}{255,0,0}

\path[draw=drawColor,line width= 0.4pt,line join=round,line cap=round,fill=fillColor] (434.08,186.90) circle (  1.49);
\definecolor{drawColor}{RGB}{0,0,0}
\definecolor{fillColor}{RGB}{0,0,0}

\path[draw=drawColor,line width= 0.4pt,line join=round,line cap=round,fill=fillColor] (434.10,174.72) circle (  1.49);
\definecolor{drawColor}{RGB}{255,0,0}
\definecolor{fillColor}{RGB}{255,0,0}

\path[draw=drawColor,line width= 0.4pt,line join=round,line cap=round,fill=fillColor] (434.52,186.96) circle (  1.49);
\definecolor{drawColor}{RGB}{0,0,0}
\definecolor{fillColor}{RGB}{0,0,0}

\path[draw=drawColor,line width= 0.4pt,line join=round,line cap=round,fill=fillColor] (434.54,174.87) circle (  1.49);
\definecolor{drawColor}{RGB}{255,0,0}
\definecolor{fillColor}{RGB}{255,0,0}

\path[draw=drawColor,line width= 0.4pt,line join=round,line cap=round,fill=fillColor] (435.01,186.51) circle (  1.49);
\definecolor{drawColor}{RGB}{0,0,0}
\definecolor{fillColor}{RGB}{0,0,0}

\path[draw=drawColor,line width= 0.4pt,line join=round,line cap=round,fill=fillColor] (435.03,173.36) circle (  1.49);
\definecolor{drawColor}{RGB}{255,0,0}
\definecolor{fillColor}{RGB}{255,0,0}

\path[draw=drawColor,line width= 0.4pt,line join=round,line cap=round,fill=fillColor] (435.47,185.95) circle (  1.49);
\definecolor{drawColor}{RGB}{0,0,0}
\definecolor{fillColor}{RGB}{0,0,0}

\path[draw=drawColor,line width= 0.4pt,line join=round,line cap=round,fill=fillColor] (435.49,170.60) circle (  1.49);
\definecolor{drawColor}{RGB}{255,0,0}
\definecolor{fillColor}{RGB}{255,0,0}

\path[draw=drawColor,line width= 0.4pt,line join=round,line cap=round,fill=fillColor] (435.93,185.63) circle (  1.49);
\definecolor{drawColor}{RGB}{0,0,0}
\definecolor{fillColor}{RGB}{0,0,0}

\path[draw=drawColor,line width= 0.4pt,line join=round,line cap=round,fill=fillColor] (435.95,173.88) circle (  1.49);
\definecolor{drawColor}{RGB}{255,0,0}
\definecolor{fillColor}{RGB}{255,0,0}

\path[draw=drawColor,line width= 0.4pt,line join=round,line cap=round,fill=fillColor] (436.39,185.84) circle (  1.49);
\definecolor{drawColor}{RGB}{0,0,0}
\definecolor{fillColor}{RGB}{0,0,0}

\path[draw=drawColor,line width= 0.4pt,line join=round,line cap=round,fill=fillColor] (436.41,173.87) circle (  1.49);
\definecolor{drawColor}{RGB}{255,0,0}
\definecolor{fillColor}{RGB}{255,0,0}

\path[draw=drawColor,line width= 0.4pt,line join=round,line cap=round,fill=fillColor] (436.85,185.84) circle (  1.49);
\definecolor{drawColor}{RGB}{0,0,0}
\definecolor{fillColor}{RGB}{0,0,0}

\path[draw=drawColor,line width= 0.4pt,line join=round,line cap=round,fill=fillColor] (436.86,174.21) circle (  1.49);
\definecolor{drawColor}{RGB}{255,0,0}
\definecolor{fillColor}{RGB}{255,0,0}

\path[draw=drawColor,line width= 0.4pt,line join=round,line cap=round,fill=fillColor] (437.32,185.91) circle (  1.49);
\definecolor{drawColor}{RGB}{0,0,0}
\definecolor{fillColor}{RGB}{0,0,0}

\path[draw=drawColor,line width= 0.4pt,line join=round,line cap=round,fill=fillColor] (437.34,172.53) circle (  1.49);
\definecolor{drawColor}{RGB}{255,0,0}
\definecolor{fillColor}{RGB}{255,0,0}

\path[draw=drawColor,line width= 0.4pt,line join=round,line cap=round,fill=fillColor] (437.80,186.08) circle (  1.49);
\definecolor{drawColor}{RGB}{0,0,0}
\definecolor{fillColor}{RGB}{0,0,0}

\path[draw=drawColor,line width= 0.4pt,line join=round,line cap=round,fill=fillColor] (437.81,174.39) circle (  1.49);
\definecolor{drawColor}{RGB}{255,0,0}
\definecolor{fillColor}{RGB}{255,0,0}

\path[draw=drawColor,line width= 0.4pt,line join=round,line cap=round,fill=fillColor] (438.27,186.00) circle (  1.49);
\definecolor{drawColor}{RGB}{0,0,0}
\definecolor{fillColor}{RGB}{0,0,0}

\path[draw=drawColor,line width= 0.4pt,line join=round,line cap=round,fill=fillColor] (438.29,174.63) circle (  1.49);
\definecolor{drawColor}{RGB}{255,0,0}
\definecolor{fillColor}{RGB}{255,0,0}

\path[draw=drawColor,line width= 0.4pt,line join=round,line cap=round,fill=fillColor] (438.73,186.56) circle (  1.49);
\definecolor{drawColor}{RGB}{0,0,0}
\definecolor{fillColor}{RGB}{0,0,0}

\path[draw=drawColor,line width= 0.4pt,line join=round,line cap=round,fill=fillColor] (438.75,174.59) circle (  1.49);
\definecolor{drawColor}{RGB}{255,0,0}
\definecolor{fillColor}{RGB}{255,0,0}

\path[draw=drawColor,line width= 0.4pt,line join=round,line cap=round,fill=fillColor] (439.20,186.62) circle (  1.49);
\definecolor{drawColor}{RGB}{0,0,0}
\definecolor{fillColor}{RGB}{0,0,0}

\path[draw=drawColor,line width= 0.4pt,line join=round,line cap=round,fill=fillColor] (439.22,174.78) circle (  1.49);
\definecolor{drawColor}{RGB}{255,0,0}
\definecolor{fillColor}{RGB}{255,0,0}

\path[draw=drawColor,line width= 0.4pt,line join=round,line cap=round,fill=fillColor] (439.66,186.76) circle (  1.49);
\definecolor{drawColor}{RGB}{0,0,0}
\definecolor{fillColor}{RGB}{0,0,0}

\path[draw=drawColor,line width= 0.4pt,line join=round,line cap=round,fill=fillColor] (439.68,174.45) circle (  1.49);
\definecolor{drawColor}{RGB}{255,0,0}
\definecolor{fillColor}{RGB}{255,0,0}

\path[draw=drawColor,line width= 0.4pt,line join=round,line cap=round,fill=fillColor] (440.10,186.21) circle (  1.49);
\definecolor{drawColor}{RGB}{0,0,0}
\definecolor{fillColor}{RGB}{0,0,0}

\path[draw=drawColor,line width= 0.4pt,line join=round,line cap=round,fill=fillColor] (440.12,166.20) circle (  1.49);
\definecolor{drawColor}{RGB}{255,0,0}
\definecolor{fillColor}{RGB}{255,0,0}

\path[draw=drawColor,line width= 0.4pt,line join=round,line cap=round,fill=fillColor] (440.58,185.89) circle (  1.49);
\definecolor{drawColor}{RGB}{0,0,0}
\definecolor{fillColor}{RGB}{0,0,0}

\path[draw=drawColor,line width= 0.4pt,line join=round,line cap=round,fill=fillColor] (440.60,173.75) circle (  1.49);
\definecolor{drawColor}{RGB}{255,0,0}
\definecolor{fillColor}{RGB}{255,0,0}

\path[draw=drawColor,line width= 0.4pt,line join=round,line cap=round,fill=fillColor] (441.05,185.53) circle (  1.49);
\definecolor{drawColor}{RGB}{0,0,0}
\definecolor{fillColor}{RGB}{0,0,0}

\path[draw=drawColor,line width= 0.4pt,line join=round,line cap=round,fill=fillColor] (441.07,173.83) circle (  1.49);
\definecolor{drawColor}{RGB}{255,0,0}
\definecolor{fillColor}{RGB}{255,0,0}

\path[draw=drawColor,line width= 0.4pt,line join=round,line cap=round,fill=fillColor] (441.53,185.46) circle (  1.49);
\definecolor{drawColor}{RGB}{0,0,0}
\definecolor{fillColor}{RGB}{0,0,0}

\path[draw=drawColor,line width= 0.4pt,line join=round,line cap=round,fill=fillColor] (441.55,173.09) circle (  1.49);
\definecolor{drawColor}{RGB}{255,0,0}
\definecolor{fillColor}{RGB}{255,0,0}

\path[draw=drawColor,line width= 0.4pt,line join=round,line cap=round,fill=fillColor] (442.07,184.85) circle (  1.49);
\definecolor{drawColor}{RGB}{0,0,0}
\definecolor{fillColor}{RGB}{0,0,0}

\path[draw=drawColor,line width= 0.4pt,line join=round,line cap=round,fill=fillColor] (442.09,172.89) circle (  1.49);
\definecolor{drawColor}{RGB}{255,0,0}
\definecolor{fillColor}{RGB}{255,0,0}

\path[draw=drawColor,line width= 0.4pt,line join=round,line cap=round,fill=fillColor] (442.56,184.49) circle (  1.49);
\definecolor{drawColor}{RGB}{0,0,0}
\definecolor{fillColor}{RGB}{0,0,0}

\path[draw=drawColor,line width= 0.4pt,line join=round,line cap=round,fill=fillColor] (442.58,172.83) circle (  1.49);
\definecolor{drawColor}{RGB}{255,0,0}
\definecolor{fillColor}{RGB}{255,0,0}

\path[draw=drawColor,line width= 0.4pt,line join=round,line cap=round,fill=fillColor] (443.02,184.96) circle (  1.49);
\definecolor{drawColor}{RGB}{0,0,0}
\definecolor{fillColor}{RGB}{0,0,0}

\path[draw=drawColor,line width= 0.4pt,line join=round,line cap=round,fill=fillColor] (443.04,169.92) circle (  1.49);
\definecolor{drawColor}{RGB}{255,0,0}
\definecolor{fillColor}{RGB}{255,0,0}

\path[draw=drawColor,line width= 0.4pt,line join=round,line cap=round,fill=fillColor] (443.44,184.68) circle (  1.49);
\definecolor{drawColor}{RGB}{0,0,0}
\definecolor{fillColor}{RGB}{0,0,0}

\path[draw=drawColor,line width= 0.4pt,line join=round,line cap=round,fill=fillColor] (443.46,169.66) circle (  1.49);
\definecolor{drawColor}{RGB}{255,0,0}
\definecolor{fillColor}{RGB}{255,0,0}

\path[draw=drawColor,line width= 0.4pt,line join=round,line cap=round,fill=fillColor] (443.98,184.58) circle (  1.49);
\definecolor{drawColor}{RGB}{0,0,0}
\definecolor{fillColor}{RGB}{0,0,0}

\path[draw=drawColor,line width= 0.4pt,line join=round,line cap=round,fill=fillColor] (444.02,129.59) circle (  1.49);
\definecolor{drawColor}{RGB}{255,0,0}
\definecolor{fillColor}{RGB}{255,0,0}

\path[draw=drawColor,line width= 0.4pt,line join=round,line cap=round,fill=fillColor] (444.49,184.87) circle (  1.49);
\definecolor{drawColor}{RGB}{0,0,0}
\definecolor{fillColor}{RGB}{0,0,0}

\path[draw=drawColor,line width= 0.4pt,line join=round,line cap=round,fill=fillColor] (444.51,173.11) circle (  1.49);
\definecolor{drawColor}{RGB}{255,0,0}
\definecolor{fillColor}{RGB}{255,0,0}

\path[draw=drawColor,line width= 0.4pt,line join=round,line cap=round,fill=fillColor] (444.98,184.92) circle (  1.49);
\definecolor{drawColor}{RGB}{0,0,0}
\definecolor{fillColor}{RGB}{0,0,0}

\path[draw=drawColor,line width= 0.4pt,line join=round,line cap=round,fill=fillColor] (445.00,145.12) circle (  1.49);
\definecolor{drawColor}{RGB}{255,0,0}
\definecolor{fillColor}{RGB}{255,0,0}

\path[draw=drawColor,line width= 0.4pt,line join=round,line cap=round,fill=fillColor] (445.47,184.53) circle (  1.49);
\definecolor{drawColor}{RGB}{0,0,0}
\definecolor{fillColor}{RGB}{0,0,0}

\path[draw=drawColor,line width= 0.4pt,line join=round,line cap=round,fill=fillColor] (445.49,169.84) circle (  1.49);
\definecolor{drawColor}{RGB}{255,0,0}
\definecolor{fillColor}{RGB}{255,0,0}

\path[draw=drawColor,line width= 0.4pt,line join=round,line cap=round,fill=fillColor] (445.92,184.00) circle (  1.49);
\definecolor{drawColor}{RGB}{0,0,0}
\definecolor{fillColor}{RGB}{0,0,0}

\path[draw=drawColor,line width= 0.4pt,line join=round,line cap=round,fill=fillColor] (445.93,172.24) circle (  1.49);
\definecolor{drawColor}{RGB}{255,0,0}
\definecolor{fillColor}{RGB}{255,0,0}

\path[draw=drawColor,line width= 0.4pt,line join=round,line cap=round,fill=fillColor] (446.41,184.25) circle (  1.49);
\definecolor{drawColor}{RGB}{0,0,0}
\definecolor{fillColor}{RGB}{0,0,0}

\path[draw=drawColor,line width= 0.4pt,line join=round,line cap=round,fill=fillColor] (446.46, 54.71) circle (  1.49);
\definecolor{drawColor}{RGB}{255,0,0}
\definecolor{fillColor}{RGB}{255,0,0}

\path[draw=drawColor,line width= 0.4pt,line join=round,line cap=round,fill=fillColor] (446.91,183.83) circle (  1.49);
\definecolor{drawColor}{RGB}{0,0,0}
\definecolor{fillColor}{RGB}{0,0,0}

\path[draw=drawColor,line width= 0.4pt,line join=round,line cap=round,fill=fillColor] (446.93,141.21) circle (  1.49);
\definecolor{drawColor}{RGB}{255,0,0}
\definecolor{fillColor}{RGB}{255,0,0}

\path[draw=drawColor,line width= 0.4pt,line join=round,line cap=round,fill=fillColor] (447.36,183.33) circle (  1.49);
\definecolor{drawColor}{RGB}{0,0,0}
\definecolor{fillColor}{RGB}{0,0,0}

\path[draw=drawColor,line width= 0.4pt,line join=round,line cap=round,fill=fillColor] (447.37,171.93) circle (  1.49);
\definecolor{drawColor}{RGB}{255,0,0}
\definecolor{fillColor}{RGB}{255,0,0}

\path[draw=drawColor,line width= 0.4pt,line join=round,line cap=round,fill=fillColor] (447.85,184.10) circle (  1.49);
\definecolor{drawColor}{RGB}{0,0,0}
\definecolor{fillColor}{RGB}{0,0,0}

\path[draw=drawColor,line width= 0.4pt,line join=round,line cap=round,fill=fillColor] (447.86,171.73) circle (  1.49);
\definecolor{drawColor}{RGB}{255,0,0}
\definecolor{fillColor}{RGB}{255,0,0}

\path[draw=drawColor,line width= 0.4pt,line join=round,line cap=round,fill=fillColor] (448.31,183.40) circle (  1.49);
\definecolor{drawColor}{RGB}{0,0,0}
\definecolor{fillColor}{RGB}{0,0,0}

\path[draw=drawColor,line width= 0.4pt,line join=round,line cap=round,fill=fillColor] (448.32,170.23) circle (  1.49);
\definecolor{drawColor}{RGB}{255,0,0}
\definecolor{fillColor}{RGB}{255,0,0}

\path[draw=drawColor,line width= 0.4pt,line join=round,line cap=round,fill=fillColor] (448.80,182.98) circle (  1.49);
\definecolor{drawColor}{RGB}{0,0,0}
\definecolor{fillColor}{RGB}{0,0,0}

\path[draw=drawColor,line width= 0.4pt,line join=round,line cap=round,fill=fillColor] (448.81,171.25) circle (  1.49);
\definecolor{drawColor}{RGB}{255,0,0}
\definecolor{fillColor}{RGB}{255,0,0}

\path[draw=drawColor,line width= 0.4pt,line join=round,line cap=round,fill=fillColor] (449.32,183.36) circle (  1.49);
\definecolor{drawColor}{RGB}{0,0,0}
\definecolor{fillColor}{RGB}{0,0,0}

\path[draw=drawColor,line width= 0.4pt,line join=round,line cap=round,fill=fillColor] (449.34,171.23) circle (  1.49);
\definecolor{drawColor}{RGB}{255,0,0}
\definecolor{fillColor}{RGB}{255,0,0}

\path[draw=drawColor,line width= 0.4pt,line join=round,line cap=round,fill=fillColor] (449.80,183.42) circle (  1.49);
\definecolor{drawColor}{RGB}{0,0,0}
\definecolor{fillColor}{RGB}{0,0,0}

\path[draw=drawColor,line width= 0.4pt,line join=round,line cap=round,fill=fillColor] (449.81,168.89) circle (  1.49);
\definecolor{drawColor}{RGB}{255,0,0}
\definecolor{fillColor}{RGB}{255,0,0}

\path[draw=drawColor,line width= 0.4pt,line join=round,line cap=round,fill=fillColor] (450.29,182.84) circle (  1.49);
\definecolor{drawColor}{RGB}{0,0,0}
\definecolor{fillColor}{RGB}{0,0,0}

\path[draw=drawColor,line width= 0.4pt,line join=round,line cap=round,fill=fillColor] (450.30,168.64) circle (  1.49);
\definecolor{drawColor}{RGB}{255,0,0}
\definecolor{fillColor}{RGB}{255,0,0}

\path[draw=drawColor,line width= 0.4pt,line join=round,line cap=round,fill=fillColor] (450.75,182.48) circle (  1.49);
\definecolor{drawColor}{RGB}{0,0,0}
\definecolor{fillColor}{RGB}{0,0,0}

\path[draw=drawColor,line width= 0.4pt,line join=round,line cap=round,fill=fillColor] (450.76,170.67) circle (  1.49);
\definecolor{drawColor}{RGB}{255,0,0}
\definecolor{fillColor}{RGB}{255,0,0}

\path[draw=drawColor,line width= 0.4pt,line join=round,line cap=round,fill=fillColor] (451.25,182.62) circle (  1.49);
\definecolor{drawColor}{RGB}{0,0,0}
\definecolor{fillColor}{RGB}{0,0,0}

\path[draw=drawColor,line width= 0.4pt,line join=round,line cap=round,fill=fillColor] (451.27,170.61) circle (  1.49);
\definecolor{drawColor}{RGB}{255,0,0}
\definecolor{fillColor}{RGB}{255,0,0}

\path[draw=drawColor,line width= 0.4pt,line join=round,line cap=round,fill=fillColor] (451.71,182.32) circle (  1.49);
\definecolor{drawColor}{RGB}{0,0,0}
\definecolor{fillColor}{RGB}{0,0,0}

\path[draw=drawColor,line width= 0.4pt,line join=round,line cap=round,fill=fillColor] (451.73,165.83) circle (  1.49);
\definecolor{drawColor}{RGB}{255,0,0}
\definecolor{fillColor}{RGB}{255,0,0}

\path[draw=drawColor,line width= 0.4pt,line join=round,line cap=round,fill=fillColor] (452.17,182.73) circle (  1.49);
\definecolor{drawColor}{RGB}{0,0,0}
\definecolor{fillColor}{RGB}{0,0,0}

\path[draw=drawColor,line width= 0.4pt,line join=round,line cap=round,fill=fillColor] (452.19,170.75) circle (  1.49);
\definecolor{drawColor}{RGB}{255,0,0}
\definecolor{fillColor}{RGB}{255,0,0}

\path[draw=drawColor,line width= 0.4pt,line join=round,line cap=round,fill=fillColor] (452.69,182.22) circle (  1.49);
\definecolor{drawColor}{RGB}{0,0,0}
\definecolor{fillColor}{RGB}{0,0,0}

\path[draw=drawColor,line width= 0.4pt,line join=round,line cap=round,fill=fillColor] (452.71,170.83) circle (  1.49);
\definecolor{drawColor}{RGB}{255,0,0}
\definecolor{fillColor}{RGB}{255,0,0}

\path[draw=drawColor,line width= 0.4pt,line join=round,line cap=round,fill=fillColor] (453.15,182.07) circle (  1.49);
\definecolor{drawColor}{RGB}{0,0,0}
\definecolor{fillColor}{RGB}{0,0,0}

\path[draw=drawColor,line width= 0.4pt,line join=round,line cap=round,fill=fillColor] (453.17,142.75) circle (  1.49);
\definecolor{drawColor}{RGB}{255,0,0}
\definecolor{fillColor}{RGB}{255,0,0}

\path[draw=drawColor,line width= 0.4pt,line join=round,line cap=round,fill=fillColor] (453.59,181.61) circle (  1.49);
\definecolor{drawColor}{RGB}{0,0,0}
\definecolor{fillColor}{RGB}{0,0,0}

\path[draw=drawColor,line width= 0.4pt,line join=round,line cap=round,fill=fillColor] (453.61,170.13) circle (  1.49);
\definecolor{drawColor}{RGB}{255,0,0}
\definecolor{fillColor}{RGB}{255,0,0}

\path[draw=drawColor,line width= 0.4pt,line join=round,line cap=round,fill=fillColor] (454.04,181.81) circle (  1.49);
\definecolor{drawColor}{RGB}{0,0,0}
\definecolor{fillColor}{RGB}{0,0,0}

\path[draw=drawColor,line width= 0.4pt,line join=round,line cap=round,fill=fillColor] (454.05,165.41) circle (  1.49);
\definecolor{drawColor}{RGB}{255,0,0}
\definecolor{fillColor}{RGB}{255,0,0}

\path[draw=drawColor,line width= 0.4pt,line join=round,line cap=round,fill=fillColor] (454.46,181.81) circle (  1.49);
\definecolor{drawColor}{RGB}{0,0,0}
\definecolor{fillColor}{RGB}{0,0,0}

\path[draw=drawColor,line width= 0.4pt,line join=round,line cap=round,fill=fillColor] (454.48,167.37) circle (  1.49);
\definecolor{drawColor}{RGB}{255,0,0}
\definecolor{fillColor}{RGB}{255,0,0}

\path[draw=drawColor,line width= 0.4pt,line join=round,line cap=round,fill=fillColor] (454.89,181.44) circle (  1.49);
\definecolor{drawColor}{RGB}{0,0,0}
\definecolor{fillColor}{RGB}{0,0,0}

\path[draw=drawColor,line width= 0.4pt,line join=round,line cap=round,fill=fillColor] (454.92, 85.47) circle (  1.49);
\definecolor{drawColor}{RGB}{255,0,0}
\definecolor{fillColor}{RGB}{255,0,0}

\path[draw=drawColor,line width= 0.4pt,line join=round,line cap=round,fill=fillColor] (455.39,180.82) circle (  1.49);
\definecolor{drawColor}{RGB}{0,0,0}
\definecolor{fillColor}{RGB}{0,0,0}

\path[draw=drawColor,line width= 0.4pt,line join=round,line cap=round,fill=fillColor] (455.43, 74.90) circle (  1.49);
\definecolor{drawColor}{RGB}{255,0,0}
\definecolor{fillColor}{RGB}{255,0,0}

\path[draw=drawColor,line width= 0.4pt,line join=round,line cap=round,fill=fillColor] (455.92,180.86) circle (  1.49);
\definecolor{drawColor}{RGB}{0,0,0}
\definecolor{fillColor}{RGB}{0,0,0}

\path[draw=drawColor,line width= 0.4pt,line join=round,line cap=round,fill=fillColor] (455.93,169.26) circle (  1.49);
\definecolor{drawColor}{RGB}{255,0,0}
\definecolor{fillColor}{RGB}{255,0,0}

\path[draw=drawColor,line width= 0.4pt,line join=round,line cap=round,fill=fillColor] (456.39,180.71) circle (  1.49);
\definecolor{drawColor}{RGB}{0,0,0}
\definecolor{fillColor}{RGB}{0,0,0}

\path[draw=drawColor,line width= 0.4pt,line join=round,line cap=round,fill=fillColor] (456.41,152.12) circle (  1.49);
\definecolor{drawColor}{RGB}{255,0,0}
\definecolor{fillColor}{RGB}{255,0,0}

\path[draw=drawColor,line width= 0.4pt,line join=round,line cap=round,fill=fillColor] (456.85,180.48) circle (  1.49);
\definecolor{drawColor}{RGB}{0,0,0}
\definecolor{fillColor}{RGB}{0,0,0}

\path[draw=drawColor,line width= 0.4pt,line join=round,line cap=round,fill=fillColor] (456.87,155.44) circle (  1.49);
\definecolor{drawColor}{RGB}{255,0,0}
\definecolor{fillColor}{RGB}{255,0,0}

\path[draw=drawColor,line width= 0.4pt,line join=round,line cap=round,fill=fillColor] (457.37,180.83) circle (  1.49);
\definecolor{drawColor}{RGB}{0,0,0}
\definecolor{fillColor}{RGB}{0,0,0}

\path[draw=drawColor,line width= 0.4pt,line join=round,line cap=round,fill=fillColor] (457.39,156.82) circle (  1.49);
\definecolor{drawColor}{RGB}{255,0,0}
\definecolor{fillColor}{RGB}{255,0,0}

\path[draw=drawColor,line width= 0.4pt,line join=round,line cap=round,fill=fillColor] (457.82,181.29) circle (  1.49);
\definecolor{drawColor}{RGB}{0,0,0}
\definecolor{fillColor}{RGB}{0,0,0}

\path[draw=drawColor,line width= 0.4pt,line join=round,line cap=round,fill=fillColor] (457.83,169.03) circle (  1.49);
\definecolor{drawColor}{RGB}{255,0,0}
\definecolor{fillColor}{RGB}{255,0,0}

\path[draw=drawColor,line width= 0.4pt,line join=round,line cap=round,fill=fillColor] (458.24,180.20) circle (  1.49);
\definecolor{drawColor}{RGB}{0,0,0}
\definecolor{fillColor}{RGB}{0,0,0}

\path[draw=drawColor,line width= 0.4pt,line join=round,line cap=round,fill=fillColor] (458.26,162.42) circle (  1.49);
\definecolor{drawColor}{RGB}{255,0,0}
\definecolor{fillColor}{RGB}{255,0,0}

\path[draw=drawColor,line width= 0.4pt,line join=round,line cap=round,fill=fillColor] (458.77,180.80) circle (  1.49);
\definecolor{drawColor}{RGB}{0,0,0}
\definecolor{fillColor}{RGB}{0,0,0}

\path[draw=drawColor,line width= 0.4pt,line join=round,line cap=round,fill=fillColor] (458.80,150.05) circle (  1.49);
\definecolor{drawColor}{RGB}{255,0,0}
\definecolor{fillColor}{RGB}{255,0,0}

\path[draw=drawColor,line width= 0.4pt,line join=round,line cap=round,fill=fillColor] (459.24,180.56) circle (  1.49);
\definecolor{drawColor}{RGB}{0,0,0}
\definecolor{fillColor}{RGB}{0,0,0}

\path[draw=drawColor,line width= 0.4pt,line join=round,line cap=round,fill=fillColor] (459.29,168.77) circle (  1.49);
\definecolor{drawColor}{RGB}{255,0,0}
\definecolor{fillColor}{RGB}{255,0,0}

\path[draw=drawColor,line width= 0.4pt,line join=round,line cap=round,fill=fillColor] (459.80,180.40) circle (  1.49);
\definecolor{drawColor}{RGB}{0,0,0}
\definecolor{fillColor}{RGB}{0,0,0}

\path[draw=drawColor,line width= 0.4pt,line join=round,line cap=round,fill=fillColor] (459.81,168.71) circle (  1.49);
\definecolor{drawColor}{RGB}{255,0,0}
\definecolor{fillColor}{RGB}{255,0,0}

\path[draw=drawColor,line width= 0.4pt,line join=round,line cap=round,fill=fillColor] (460.24,180.39) circle (  1.49);
\definecolor{drawColor}{RGB}{0,0,0}
\definecolor{fillColor}{RGB}{0,0,0}

\path[draw=drawColor,line width= 0.4pt,line join=round,line cap=round,fill=fillColor] (460.27, 82.13) circle (  1.49);
\definecolor{drawColor}{RGB}{255,0,0}
\definecolor{fillColor}{RGB}{255,0,0}

\path[draw=drawColor,line width= 0.4pt,line join=round,line cap=round,fill=fillColor] (460.70,180.21) circle (  1.49);
\definecolor{drawColor}{RGB}{0,0,0}
\definecolor{fillColor}{RGB}{0,0,0}

\path[draw=drawColor,line width= 0.4pt,line join=round,line cap=round,fill=fillColor] (460.73,168.54) circle (  1.49);
\definecolor{drawColor}{RGB}{255,0,0}
\definecolor{fillColor}{RGB}{255,0,0}

\path[draw=drawColor,line width= 0.4pt,line join=round,line cap=round,fill=fillColor] (461.14,179.72) circle (  1.49);
\definecolor{drawColor}{RGB}{0,0,0}
\definecolor{fillColor}{RGB}{0,0,0}

\path[draw=drawColor,line width= 0.4pt,line join=round,line cap=round,fill=fillColor] (461.16,168.04) circle (  1.49);
\definecolor{drawColor}{RGB}{255,0,0}
\definecolor{fillColor}{RGB}{255,0,0}

\path[draw=drawColor,line width= 0.4pt,line join=round,line cap=round,fill=fillColor] (461.76,179.60) circle (  1.49);
\definecolor{drawColor}{RGB}{0,0,0}
\definecolor{fillColor}{RGB}{0,0,0}

\path[draw=drawColor,line width= 0.4pt,line join=round,line cap=round,fill=fillColor] (461.78,168.09) circle (  1.49);
\definecolor{drawColor}{RGB}{255,0,0}
\definecolor{fillColor}{RGB}{255,0,0}

\path[draw=drawColor,line width= 0.4pt,line join=round,line cap=round,fill=fillColor] (462.33,179.50) circle (  1.49);
\definecolor{drawColor}{RGB}{0,0,0}
\definecolor{fillColor}{RGB}{0,0,0}

\path[draw=drawColor,line width= 0.4pt,line join=round,line cap=round,fill=fillColor] (462.35,163.88) circle (  1.49);
\definecolor{drawColor}{RGB}{255,0,0}
\definecolor{fillColor}{RGB}{255,0,0}

\path[draw=drawColor,line width= 0.4pt,line join=round,line cap=round,fill=fillColor] (462.83,178.68) circle (  1.49);
\definecolor{drawColor}{RGB}{0,0,0}
\definecolor{fillColor}{RGB}{0,0,0}

\path[draw=drawColor,line width= 0.4pt,line join=round,line cap=round,fill=fillColor] (462.84,166.89) circle (  1.49);
\definecolor{drawColor}{RGB}{255,0,0}
\definecolor{fillColor}{RGB}{255,0,0}

\path[draw=drawColor,line width= 0.4pt,line join=round,line cap=round,fill=fillColor] (463.37,178.63) circle (  1.49);
\definecolor{drawColor}{RGB}{0,0,0}
\definecolor{fillColor}{RGB}{0,0,0}

\path[draw=drawColor,line width= 0.4pt,line join=round,line cap=round,fill=fillColor] (463.38,166.96) circle (  1.49);
\definecolor{drawColor}{RGB}{255,0,0}
\definecolor{fillColor}{RGB}{255,0,0}

\path[draw=drawColor,line width= 0.4pt,line join=round,line cap=round,fill=fillColor] (463.84,178.66) circle (  1.49);
\definecolor{drawColor}{RGB}{0,0,0}
\definecolor{fillColor}{RGB}{0,0,0}

\path[draw=drawColor,line width= 0.4pt,line join=round,line cap=round,fill=fillColor] (463.86,166.83) circle (  1.49);
\definecolor{drawColor}{RGB}{255,0,0}
\definecolor{fillColor}{RGB}{255,0,0}

\path[draw=drawColor,line width= 0.4pt,line join=round,line cap=round,fill=fillColor] (464.28,178.49) circle (  1.49);
\definecolor{drawColor}{RGB}{0,0,0}
\definecolor{fillColor}{RGB}{0,0,0}

\path[draw=drawColor,line width= 0.4pt,line join=round,line cap=round,fill=fillColor] (464.30,163.23) circle (  1.49);
\definecolor{drawColor}{RGB}{255,0,0}
\definecolor{fillColor}{RGB}{255,0,0}

\path[draw=drawColor,line width= 0.4pt,line join=round,line cap=round,fill=fillColor] (464.74,178.35) circle (  1.49);
\definecolor{drawColor}{RGB}{0,0,0}
\definecolor{fillColor}{RGB}{0,0,0}

\path[draw=drawColor,line width= 0.4pt,line join=round,line cap=round,fill=fillColor] (464.77, 64.64) circle (  1.49);
\definecolor{drawColor}{RGB}{255,0,0}
\definecolor{fillColor}{RGB}{255,0,0}

\path[draw=drawColor,line width= 0.4pt,line join=round,line cap=round,fill=fillColor] (465.20,178.13) circle (  1.49);
\definecolor{drawColor}{RGB}{0,0,0}
\definecolor{fillColor}{RGB}{0,0,0}

\path[draw=drawColor,line width= 0.4pt,line join=round,line cap=round,fill=fillColor] (465.22,166.51) circle (  1.49);
\definecolor{drawColor}{RGB}{255,0,0}
\definecolor{fillColor}{RGB}{255,0,0}

\path[draw=drawColor,line width= 0.4pt,line join=round,line cap=round,fill=fillColor] (465.67,178.07) circle (  1.49);
\definecolor{drawColor}{RGB}{0,0,0}
\definecolor{fillColor}{RGB}{0,0,0}

\path[draw=drawColor,line width= 0.4pt,line join=round,line cap=round,fill=fillColor] (465.69,166.21) circle (  1.49);
\definecolor{drawColor}{RGB}{255,0,0}
\definecolor{fillColor}{RGB}{255,0,0}

\path[draw=drawColor,line width= 0.4pt,line join=round,line cap=round,fill=fillColor] (466.08,177.84) circle (  1.49);
\definecolor{drawColor}{RGB}{0,0,0}
\definecolor{fillColor}{RGB}{0,0,0}

\path[draw=drawColor,line width= 0.4pt,line join=round,line cap=round,fill=fillColor] (466.10, 78.47) circle (  1.49);
\definecolor{drawColor}{RGB}{255,0,0}
\definecolor{fillColor}{RGB}{255,0,0}

\path[draw=drawColor,line width= 0.4pt,line join=round,line cap=round,fill=fillColor] (466.69,177.18) circle (  1.49);
\definecolor{drawColor}{RGB}{0,0,0}
\definecolor{fillColor}{RGB}{0,0,0}

\path[draw=drawColor,line width= 0.4pt,line join=round,line cap=round,fill=fillColor] (466.71,164.71) circle (  1.49);
\definecolor{drawColor}{RGB}{255,0,0}
\definecolor{fillColor}{RGB}{255,0,0}

\path[draw=drawColor,line width= 0.4pt,line join=round,line cap=round,fill=fillColor] (467.29,177.11) circle (  1.49);
\definecolor{drawColor}{RGB}{0,0,0}
\definecolor{fillColor}{RGB}{0,0,0}

\path[draw=drawColor,line width= 0.4pt,line join=round,line cap=round,fill=fillColor] (467.31,166.11) circle (  1.49);
\definecolor{drawColor}{RGB}{255,0,0}
\definecolor{fillColor}{RGB}{255,0,0}

\path[draw=drawColor,line width= 0.4pt,line join=round,line cap=round,fill=fillColor] (467.72,177.29) circle (  1.49);
\definecolor{drawColor}{RGB}{0,0,0}
\definecolor{fillColor}{RGB}{0,0,0}

\path[draw=drawColor,line width= 0.4pt,line join=round,line cap=round,fill=fillColor] (467.74,165.94) circle (  1.49);
\definecolor{drawColor}{RGB}{255,0,0}
\definecolor{fillColor}{RGB}{255,0,0}

\path[draw=drawColor,line width= 0.4pt,line join=round,line cap=round,fill=fillColor] (468.15,177.25) circle (  1.49);
\definecolor{drawColor}{RGB}{0,0,0}
\definecolor{fillColor}{RGB}{0,0,0}

\path[draw=drawColor,line width= 0.4pt,line join=round,line cap=round,fill=fillColor] (468.16,165.85) circle (  1.49);
\definecolor{drawColor}{RGB}{255,0,0}
\definecolor{fillColor}{RGB}{255,0,0}

\path[draw=drawColor,line width= 0.4pt,line join=round,line cap=round,fill=fillColor] (468.57,176.89) circle (  1.49);
\definecolor{drawColor}{RGB}{0,0,0}
\definecolor{fillColor}{RGB}{0,0,0}

\path[draw=drawColor,line width= 0.4pt,line join=round,line cap=round,fill=fillColor] (468.59,166.00) circle (  1.49);
\definecolor{drawColor}{RGB}{255,0,0}
\definecolor{fillColor}{RGB}{255,0,0}

\path[draw=drawColor,line width= 0.4pt,line join=round,line cap=round,fill=fillColor] (469.00,176.61) circle (  1.49);
\definecolor{drawColor}{RGB}{0,0,0}
\definecolor{fillColor}{RGB}{0,0,0}

\path[draw=drawColor,line width= 0.4pt,line join=round,line cap=round,fill=fillColor] (469.01,165.56) circle (  1.49);
\definecolor{drawColor}{RGB}{255,0,0}
\definecolor{fillColor}{RGB}{255,0,0}

\path[draw=drawColor,line width= 0.4pt,line join=round,line cap=round,fill=fillColor] (469.42,176.59) circle (  1.49);
\definecolor{drawColor}{RGB}{0,0,0}
\definecolor{fillColor}{RGB}{0,0,0}

\path[draw=drawColor,line width= 0.4pt,line join=round,line cap=round,fill=fillColor] (469.46,165.78) circle (  1.49);
\end{scope}
\begin{scope}
\path[clip] (  0.00,  0.00) rectangle (722.70,289.08);
\definecolor{drawColor}{RGB}{0,0,0}

\path[draw=drawColor,line width= 0.4pt,line join=round,line cap=round] (292.65, 47.52) -- (469.44, 47.52);

\path[draw=drawColor,line width= 0.4pt,line join=round,line cap=round] (292.65, 47.52) -- (292.65, 43.56);

\path[draw=drawColor,line width= 0.4pt,line join=round,line cap=round] (322.11, 47.52) -- (322.11, 43.56);

\path[draw=drawColor,line width= 0.4pt,line join=round,line cap=round] (351.58, 47.52) -- (351.58, 43.56);

\path[draw=drawColor,line width= 0.4pt,line join=round,line cap=round] (381.04, 47.52) -- (381.04, 43.56);

\path[draw=drawColor,line width= 0.4pt,line join=round,line cap=round] (410.51, 47.52) -- (410.51, 43.56);

\path[draw=drawColor,line width= 0.4pt,line join=round,line cap=round] (439.97, 47.52) -- (439.97, 43.56);

\path[draw=drawColor,line width= 0.4pt,line join=round,line cap=round] (469.44, 47.52) -- (469.44, 43.56);

\node[text=drawColor,anchor=base,inner sep=0pt, outer sep=0pt, scale=  0.99] at (292.65, 33.26) {11:00};

\node[text=drawColor,anchor=base,inner sep=0pt, outer sep=0pt, scale=  0.99] at (351.58, 33.26) {12:00};

\node[text=drawColor,anchor=base,inner sep=0pt, outer sep=0pt, scale=  0.99] at (410.51, 33.26) {13:00};

\node[text=drawColor,anchor=base,inner sep=0pt, outer sep=0pt, scale=  0.99] at (469.44, 33.26) {14:00};

\path[draw=drawColor,line width= 0.4pt,line join=round,line cap=round] (285.78, 71.42) -- (285.78,222.22);

\path[draw=drawColor,line width= 0.4pt,line join=round,line cap=round] (285.78, 71.42) -- (281.82, 71.42);

\path[draw=drawColor,line width= 0.4pt,line join=round,line cap=round] (285.78,109.12) -- (281.82,109.12);

\path[draw=drawColor,line width= 0.4pt,line join=round,line cap=round] (285.78,146.82) -- (281.82,146.82);

\path[draw=drawColor,line width= 0.4pt,line join=round,line cap=round] (285.78,184.52) -- (281.82,184.52);

\path[draw=drawColor,line width= 0.4pt,line join=round,line cap=round] (285.78,222.22) -- (281.82,222.22);

\node[text=drawColor,rotate= 90.00,anchor=base,inner sep=0pt, outer sep=0pt, scale=  0.99] at (276.28, 71.42) {100};

\node[text=drawColor,rotate= 90.00,anchor=base,inner sep=0pt, outer sep=0pt, scale=  0.99] at (276.28,109.12) {200};

\node[text=drawColor,rotate= 90.00,anchor=base,inner sep=0pt, outer sep=0pt, scale=  0.99] at (276.28,146.82) {300};

\node[text=drawColor,rotate= 90.00,anchor=base,inner sep=0pt, outer sep=0pt, scale=  0.99] at (276.28,184.52) {400};

\node[text=drawColor,rotate= 90.00,anchor=base,inner sep=0pt, outer sep=0pt, scale=  0.99] at (276.28,222.22) {500};

\path[draw=drawColor,line width= 0.4pt,line join=round,line cap=round] (285.78, 47.52) --
	(476.52, 47.52) --
	(476.52,241.56) --
	(285.78,241.56) --
	(285.78, 47.52);
\end{scope}
\begin{scope}
\path[clip] (246.18,  7.92) rectangle (484.44,281.16);
\definecolor{drawColor}{RGB}{0,0,0}

\node[text=drawColor,anchor=base,inner sep=0pt, outer sep=0pt, scale=  1.32] at (381.15,256.75) {\bfseries UVA-bleu transmis};

\node[text=drawColor,anchor=base,inner sep=0pt, outer sep=0pt, scale=  0.99] at (381.15, 17.42) {Temps UTC};

\node[text=drawColor,rotate= 90.00,anchor=base,inner sep=0pt, outer sep=0pt, scale=  0.99] at (260.44,144.54) {Flux photonique UVA-bleu (µmol.m$^{-2}$.s$^{-1}$)};
\end{scope}
\begin{scope}
\path[clip] (285.78, 47.52) rectangle (476.52,241.56);
\definecolor{drawColor}{RGB}{0,255,0}
\definecolor{fillColor}{RGB}{0,255,0}

\path[draw=drawColor,line width= 0.4pt,line join=round,line cap=round,fill=fillColor] (292.99, 94.64) circle (  1.49);

\path[draw=drawColor,line width= 0.4pt,line join=round,line cap=round,fill=fillColor] (293.45, 96.24) circle (  1.49);

\path[draw=drawColor,line width= 0.4pt,line join=round,line cap=round,fill=fillColor] (293.89, 98.08) circle (  1.49);

\path[draw=drawColor,line width= 0.4pt,line join=round,line cap=round,fill=fillColor] (294.35, 98.09) circle (  1.49);

\path[draw=drawColor,line width= 0.4pt,line join=round,line cap=round,fill=fillColor] (294.79, 98.05) circle (  1.49);

\path[draw=drawColor,line width= 0.4pt,line join=round,line cap=round,fill=fillColor] (295.25, 97.25) circle (  1.49);

\path[draw=drawColor,line width= 0.4pt,line join=round,line cap=round,fill=fillColor] (295.69, 96.49) circle (  1.49);

\path[draw=drawColor,line width= 0.4pt,line join=round,line cap=round,fill=fillColor] (296.17, 95.97) circle (  1.49);

\path[draw=drawColor,line width= 0.4pt,line join=round,line cap=round,fill=fillColor] (296.63, 95.51) circle (  1.49);

\path[draw=drawColor,line width= 0.4pt,line join=round,line cap=round,fill=fillColor] (297.08, 95.22) circle (  1.49);

\path[draw=drawColor,line width= 0.4pt,line join=round,line cap=round,fill=fillColor] (297.53, 95.29) circle (  1.49);

\path[draw=drawColor,line width= 0.4pt,line join=round,line cap=round,fill=fillColor] (298.00, 96.11) circle (  1.49);

\path[draw=drawColor,line width= 0.4pt,line join=round,line cap=round,fill=fillColor] (298.44, 97.42) circle (  1.49);

\path[draw=drawColor,line width= 0.4pt,line join=round,line cap=round,fill=fillColor] (298.90, 98.39) circle (  1.49);

\path[draw=drawColor,line width= 0.4pt,line join=round,line cap=round,fill=fillColor] (299.36, 98.94) circle (  1.49);

\path[draw=drawColor,line width= 0.4pt,line join=round,line cap=round,fill=fillColor] (299.82, 98.72) circle (  1.49);

\path[draw=drawColor,line width= 0.4pt,line join=round,line cap=round,fill=fillColor] (300.33, 98.36) circle (  1.49);

\path[draw=drawColor,line width= 0.4pt,line join=round,line cap=round,fill=fillColor] (300.78, 97.06) circle (  1.49);

\path[draw=drawColor,line width= 0.4pt,line join=round,line cap=round,fill=fillColor] (301.24, 98.11) circle (  1.49);

\path[draw=drawColor,line width= 0.4pt,line join=round,line cap=round,fill=fillColor] (301.68, 98.32) circle (  1.49);

\path[draw=drawColor,line width= 0.4pt,line join=round,line cap=round,fill=fillColor] (302.14, 99.36) circle (  1.49);

\path[draw=drawColor,line width= 0.4pt,line join=round,line cap=round,fill=fillColor] (302.60, 98.72) circle (  1.49);

\path[draw=drawColor,line width= 0.4pt,line join=round,line cap=round,fill=fillColor] (303.06, 97.79) circle (  1.49);

\path[draw=drawColor,line width= 0.4pt,line join=round,line cap=round,fill=fillColor] (303.53, 98.25) circle (  1.49);

\path[draw=drawColor,line width= 0.4pt,line join=round,line cap=round,fill=fillColor] (303.99, 97.97) circle (  1.49);

\path[draw=drawColor,line width= 0.4pt,line join=round,line cap=round,fill=fillColor] (304.43, 97.58) circle (  1.49);

\path[draw=drawColor,line width= 0.4pt,line join=round,line cap=round,fill=fillColor] (304.91, 96.41) circle (  1.49);

\path[draw=drawColor,line width= 0.4pt,line join=round,line cap=round,fill=fillColor] (305.37, 96.43) circle (  1.49);

\path[draw=drawColor,line width= 0.4pt,line join=round,line cap=round,fill=fillColor] (305.83, 97.22) circle (  1.49);

\path[draw=drawColor,line width= 0.4pt,line join=round,line cap=round,fill=fillColor] (306.30, 97.17) circle (  1.49);

\path[draw=drawColor,line width= 0.4pt,line join=round,line cap=round,fill=fillColor] (306.76, 97.37) circle (  1.49);

\path[draw=drawColor,line width= 0.4pt,line join=round,line cap=round,fill=fillColor] (307.22, 97.81) circle (  1.49);

\path[draw=drawColor,line width= 0.4pt,line join=round,line cap=round,fill=fillColor] (307.68, 98.12) circle (  1.49);

\path[draw=drawColor,line width= 0.4pt,line join=round,line cap=round,fill=fillColor] (308.13, 98.16) circle (  1.49);

\path[draw=drawColor,line width= 0.4pt,line join=round,line cap=round,fill=fillColor] (308.59, 98.97) circle (  1.49);

\path[draw=drawColor,line width= 0.4pt,line join=round,line cap=round,fill=fillColor] (309.05, 99.02) circle (  1.49);

\path[draw=drawColor,line width= 0.4pt,line join=round,line cap=round,fill=fillColor] (309.51,100.62) circle (  1.49);

\path[draw=drawColor,line width= 0.4pt,line join=round,line cap=round,fill=fillColor] (309.97,100.56) circle (  1.49);

\path[draw=drawColor,line width= 0.4pt,line join=round,line cap=round,fill=fillColor] (310.43,102.22) circle (  1.49);

\path[draw=drawColor,line width= 0.4pt,line join=round,line cap=round,fill=fillColor] (310.88,102.73) circle (  1.49);

\path[draw=drawColor,line width= 0.4pt,line join=round,line cap=round,fill=fillColor] (311.33,102.69) circle (  1.49);

\path[draw=drawColor,line width= 0.4pt,line join=round,line cap=round,fill=fillColor] (311.78,103.62) circle (  1.49);

\path[draw=drawColor,line width= 0.4pt,line join=round,line cap=round,fill=fillColor] (312.23,102.18) circle (  1.49);

\path[draw=drawColor,line width= 0.4pt,line join=round,line cap=round,fill=fillColor] (312.67,102.03) circle (  1.49);

\path[draw=drawColor,line width= 0.4pt,line join=round,line cap=round,fill=fillColor] (313.11,101.49) circle (  1.49);

\path[draw=drawColor,line width= 0.4pt,line join=round,line cap=round,fill=fillColor] (313.57,100.93) circle (  1.49);

\path[draw=drawColor,line width= 0.4pt,line join=round,line cap=round,fill=fillColor] (314.03, 99.92) circle (  1.49);

\path[draw=drawColor,line width= 0.4pt,line join=round,line cap=round,fill=fillColor] (314.48, 98.21) circle (  1.49);

\path[draw=drawColor,line width= 0.4pt,line join=round,line cap=round,fill=fillColor] (314.93, 96.70) circle (  1.49);

\path[draw=drawColor,line width= 0.4pt,line join=round,line cap=round,fill=fillColor] (315.39, 95.56) circle (  1.49);

\path[draw=drawColor,line width= 0.4pt,line join=round,line cap=round,fill=fillColor] (315.84, 92.98) circle (  1.49);

\path[draw=drawColor,line width= 0.4pt,line join=round,line cap=round,fill=fillColor] (316.35, 97.37) circle (  1.49);

\path[draw=drawColor,line width= 0.4pt,line join=round,line cap=round,fill=fillColor] (316.81, 88.66) circle (  1.49);

\path[draw=drawColor,line width= 0.4pt,line join=round,line cap=round,fill=fillColor] (317.25, 86.59) circle (  1.49);

\path[draw=drawColor,line width= 0.4pt,line join=round,line cap=round,fill=fillColor] (317.73, 84.89) circle (  1.49);

\path[draw=drawColor,line width= 0.4pt,line join=round,line cap=round,fill=fillColor] (318.20, 81.56) circle (  1.49);

\path[draw=drawColor,line width= 0.4pt,line join=round,line cap=round,fill=fillColor] (318.68, 82.02) circle (  1.49);

\path[draw=drawColor,line width= 0.4pt,line join=round,line cap=round,fill=fillColor] (319.17, 81.73) circle (  1.49);

\path[draw=drawColor,line width= 0.4pt,line join=round,line cap=round,fill=fillColor] (319.67, 80.33) circle (  1.49);

\path[draw=drawColor,line width= 0.4pt,line join=round,line cap=round,fill=fillColor] (320.15, 77.82) circle (  1.49);

\path[draw=drawColor,line width= 0.4pt,line join=round,line cap=round,fill=fillColor] (320.62, 76.80) circle (  1.49);

\path[draw=drawColor,line width= 0.4pt,line join=round,line cap=round,fill=fillColor] (321.11, 76.07) circle (  1.49);

\path[draw=drawColor,line width= 0.4pt,line join=round,line cap=round,fill=fillColor] (321.57, 76.32) circle (  1.49);

\path[draw=drawColor,line width= 0.4pt,line join=round,line cap=round,fill=fillColor] (322.05, 75.81) circle (  1.49);

\path[draw=drawColor,line width= 0.4pt,line join=round,line cap=round,fill=fillColor] (322.51, 73.94) circle (  1.49);

\path[draw=drawColor,line width= 0.4pt,line join=round,line cap=round,fill=fillColor] (323.00, 72.46) circle (  1.49);

\path[draw=drawColor,line width= 0.4pt,line join=round,line cap=round,fill=fillColor] (323.49, 70.35) circle (  1.49);

\path[draw=drawColor,line width= 0.4pt,line join=round,line cap=round,fill=fillColor] (323.98, 69.59) circle (  1.49);

\path[draw=drawColor,line width= 0.4pt,line join=round,line cap=round,fill=fillColor] (324.47, 69.91) circle (  1.49);

\path[draw=drawColor,line width= 0.4pt,line join=round,line cap=round,fill=fillColor] (324.96, 69.97) circle (  1.49);

\path[draw=drawColor,line width= 0.4pt,line join=round,line cap=round,fill=fillColor] (325.52, 70.84) circle (  1.49);

\path[draw=drawColor,line width= 0.4pt,line join=round,line cap=round,fill=fillColor] (326.01, 71.30) circle (  1.49);

\path[draw=drawColor,line width= 0.4pt,line join=round,line cap=round,fill=fillColor] (326.52, 71.48) circle (  1.49);

\path[draw=drawColor,line width= 0.4pt,line join=round,line cap=round,fill=fillColor] (327.02, 71.82) circle (  1.49);

\path[draw=drawColor,line width= 0.4pt,line join=round,line cap=round,fill=fillColor] (327.50, 71.93) circle (  1.49);

\path[draw=drawColor,line width= 0.4pt,line join=round,line cap=round,fill=fillColor] (327.99, 71.37) circle (  1.49);

\path[draw=drawColor,line width= 0.4pt,line join=round,line cap=round,fill=fillColor] (328.46, 72.01) circle (  1.49);

\path[draw=drawColor,line width= 0.4pt,line join=round,line cap=round,fill=fillColor] (328.96, 72.93) circle (  1.49);

\path[draw=drawColor,line width= 0.4pt,line join=round,line cap=round,fill=fillColor] (329.41, 74.34) circle (  1.49);

\path[draw=drawColor,line width= 0.4pt,line join=round,line cap=round,fill=fillColor] (329.89, 74.33) circle (  1.49);

\path[draw=drawColor,line width= 0.4pt,line join=round,line cap=round,fill=fillColor] (330.36, 74.56) circle (  1.49);

\path[draw=drawColor,line width= 0.4pt,line join=round,line cap=round,fill=fillColor] (330.84, 74.15) circle (  1.49);

\path[draw=drawColor,line width= 0.4pt,line join=round,line cap=round,fill=fillColor] (331.33, 74.91) circle (  1.49);

\path[draw=drawColor,line width= 0.4pt,line join=round,line cap=round,fill=fillColor] (331.80, 76.64) circle (  1.49);

\path[draw=drawColor,line width= 0.4pt,line join=round,line cap=round,fill=fillColor] (332.31, 78.13) circle (  1.49);

\path[draw=drawColor,line width= 0.4pt,line join=round,line cap=round,fill=fillColor] (332.79, 76.17) circle (  1.49);

\path[draw=drawColor,line width= 0.4pt,line join=round,line cap=round,fill=fillColor] (333.26, 76.19) circle (  1.49);

\path[draw=drawColor,line width= 0.4pt,line join=round,line cap=round,fill=fillColor] (333.75, 75.78) circle (  1.49);

\path[draw=drawColor,line width= 0.4pt,line join=round,line cap=round,fill=fillColor] (334.23, 75.37) circle (  1.49);

\path[draw=drawColor,line width= 0.4pt,line join=round,line cap=round,fill=fillColor] (334.70, 74.16) circle (  1.49);

\path[draw=drawColor,line width= 0.4pt,line join=round,line cap=round,fill=fillColor] (335.19, 71.99) circle (  1.49);

\path[draw=drawColor,line width= 0.4pt,line join=round,line cap=round,fill=fillColor] (335.67, 69.72) circle (  1.49);

\path[draw=drawColor,line width= 0.4pt,line join=round,line cap=round,fill=fillColor] (336.14, 68.45) circle (  1.49);

\path[draw=drawColor,line width= 0.4pt,line join=round,line cap=round,fill=fillColor] (336.63, 66.84) circle (  1.49);

\path[draw=drawColor,line width= 0.4pt,line join=round,line cap=round,fill=fillColor] (337.09, 68.66) circle (  1.49);

\path[draw=drawColor,line width= 0.4pt,line join=round,line cap=round,fill=fillColor] (337.58, 66.79) circle (  1.49);

\path[draw=drawColor,line width= 0.4pt,line join=round,line cap=round,fill=fillColor] (338.07, 67.65) circle (  1.49);

\path[draw=drawColor,line width= 0.4pt,line join=round,line cap=round,fill=fillColor] (338.55, 67.82) circle (  1.49);

\path[draw=drawColor,line width= 0.4pt,line join=round,line cap=round,fill=fillColor] (339.06, 64.71) circle (  1.49);

\path[draw=drawColor,line width= 0.4pt,line join=round,line cap=round,fill=fillColor] (339.60, 63.11) circle (  1.49);

\path[draw=drawColor,line width= 0.4pt,line join=round,line cap=round,fill=fillColor] (340.12, 62.31) circle (  1.49);

\path[draw=drawColor,line width= 0.4pt,line join=round,line cap=round,fill=fillColor] (340.63, 62.58) circle (  1.49);

\path[draw=drawColor,line width= 0.4pt,line join=round,line cap=round,fill=fillColor] (341.10, 61.75) circle (  1.49);

\path[draw=drawColor,line width= 0.4pt,line join=round,line cap=round,fill=fillColor] (341.58, 62.07) circle (  1.49);

\path[draw=drawColor,line width= 0.4pt,line join=round,line cap=round,fill=fillColor] (342.05, 62.23) circle (  1.49);

\path[draw=drawColor,line width= 0.4pt,line join=round,line cap=round,fill=fillColor] (342.54, 62.71) circle (  1.49);

\path[draw=drawColor,line width= 0.4pt,line join=round,line cap=round,fill=fillColor] (343.03, 63.01) circle (  1.49);

\path[draw=drawColor,line width= 0.4pt,line join=round,line cap=round,fill=fillColor] (343.51, 62.36) circle (  1.49);

\path[draw=drawColor,line width= 0.4pt,line join=round,line cap=round,fill=fillColor] (343.98, 61.76) circle (  1.49);

\path[draw=drawColor,line width= 0.4pt,line join=round,line cap=round,fill=fillColor] (344.59, 59.52) circle (  1.49);

\path[draw=drawColor,line width= 0.4pt,line join=round,line cap=round,fill=fillColor] (345.10, 57.87) circle (  1.49);

\path[draw=drawColor,line width= 0.4pt,line join=round,line cap=round,fill=fillColor] (345.64, 57.52) circle (  1.49);

\path[draw=drawColor,line width= 0.4pt,line join=round,line cap=round,fill=fillColor] (346.13, 57.58) circle (  1.49);

\path[draw=drawColor,line width= 0.4pt,line join=round,line cap=round,fill=fillColor] (346.81, 57.17) circle (  1.49);

\path[draw=drawColor,line width= 0.4pt,line join=round,line cap=round,fill=fillColor] (347.36, 57.04) circle (  1.49);

\path[draw=drawColor,line width= 0.4pt,line join=round,line cap=round,fill=fillColor] (347.86, 56.47) circle (  1.49);

\path[draw=drawColor,line width= 0.4pt,line join=round,line cap=round,fill=fillColor] (348.35, 57.04) circle (  1.49);

\path[draw=drawColor,line width= 0.4pt,line join=round,line cap=round,fill=fillColor] (348.86, 57.80) circle (  1.49);

\path[draw=drawColor,line width= 0.4pt,line join=round,line cap=round,fill=fillColor] (349.37, 57.02) circle (  1.49);

\path[draw=drawColor,line width= 0.4pt,line join=round,line cap=round,fill=fillColor] (349.88, 56.41) circle (  1.49);

\path[draw=drawColor,line width= 0.4pt,line join=round,line cap=round,fill=fillColor] (350.42, 56.00) circle (  1.49);

\path[draw=drawColor,line width= 0.4pt,line join=round,line cap=round,fill=fillColor] (350.92, 55.97) circle (  1.49);

\path[draw=drawColor,line width= 0.4pt,line join=round,line cap=round,fill=fillColor] (351.43, 56.30) circle (  1.49);

\path[draw=drawColor,line width= 0.4pt,line join=round,line cap=round,fill=fillColor] (351.94, 56.28) circle (  1.49);

\path[draw=drawColor,line width= 0.4pt,line join=round,line cap=round,fill=fillColor] (352.43, 56.58) circle (  1.49);

\path[draw=drawColor,line width= 0.4pt,line join=round,line cap=round,fill=fillColor] (352.95, 56.94) circle (  1.49);

\path[draw=drawColor,line width= 0.4pt,line join=round,line cap=round,fill=fillColor] (353.44, 57.49) circle (  1.49);

\path[draw=drawColor,line width= 0.4pt,line join=round,line cap=round,fill=fillColor] (353.95, 58.57) circle (  1.49);

\path[draw=drawColor,line width= 0.4pt,line join=round,line cap=round,fill=fillColor] (354.44, 58.83) circle (  1.49);

\path[draw=drawColor,line width= 0.4pt,line join=round,line cap=round,fill=fillColor] (354.95, 58.24) circle (  1.49);

\path[draw=drawColor,line width= 0.4pt,line join=round,line cap=round,fill=fillColor] (355.44, 57.83) circle (  1.49);

\path[draw=drawColor,line width= 0.4pt,line join=round,line cap=round,fill=fillColor] (356.00, 57.62) circle (  1.49);

\path[draw=drawColor,line width= 0.4pt,line join=round,line cap=round,fill=fillColor] (356.52, 56.72) circle (  1.49);

\path[draw=drawColor,line width= 0.4pt,line join=round,line cap=round,fill=fillColor] (357.01, 56.99) circle (  1.49);

\path[draw=drawColor,line width= 0.4pt,line join=round,line cap=round,fill=fillColor] (357.64, 56.60) circle (  1.49);

\path[draw=drawColor,line width= 0.4pt,line join=round,line cap=round,fill=fillColor] (358.13, 56.19) circle (  1.49);

\path[draw=drawColor,line width= 0.4pt,line join=round,line cap=round,fill=fillColor] (358.68, 55.74) circle (  1.49);

\path[draw=drawColor,line width= 0.4pt,line join=round,line cap=round,fill=fillColor] (359.37, 55.79) circle (  1.49);

\path[draw=drawColor,line width= 0.4pt,line join=round,line cap=round,fill=fillColor] (359.99, 55.93) circle (  1.49);

\path[draw=drawColor,line width= 0.4pt,line join=round,line cap=round,fill=fillColor] (360.52, 55.71) circle (  1.49);

\path[draw=drawColor,line width= 0.4pt,line join=round,line cap=round,fill=fillColor] (361.02, 55.84) circle (  1.49);

\path[draw=drawColor,line width= 0.4pt,line join=round,line cap=round,fill=fillColor] (361.76, 55.99) circle (  1.49);

\path[draw=drawColor,line width= 0.4pt,line join=round,line cap=round,fill=fillColor] (362.27, 55.91) circle (  1.49);

\path[draw=drawColor,line width= 0.4pt,line join=round,line cap=round,fill=fillColor] (362.76, 56.09) circle (  1.49);

\path[draw=drawColor,line width= 0.4pt,line join=round,line cap=round,fill=fillColor] (363.33, 56.30) circle (  1.49);

\path[draw=drawColor,line width= 0.4pt,line join=round,line cap=round,fill=fillColor] (363.82, 56.05) circle (  1.49);

\path[draw=drawColor,line width= 0.4pt,line join=round,line cap=round,fill=fillColor] (364.41, 56.20) circle (  1.49);

\path[draw=drawColor,line width= 0.4pt,line join=round,line cap=round,fill=fillColor] (364.99, 55.99) circle (  1.49);

\path[draw=drawColor,line width= 0.4pt,line join=round,line cap=round,fill=fillColor] (365.49, 55.46) circle (  1.49);

\path[draw=drawColor,line width= 0.4pt,line join=round,line cap=round,fill=fillColor] (366.00, 55.27) circle (  1.49);

\path[draw=drawColor,line width= 0.4pt,line join=round,line cap=round,fill=fillColor] (366.49, 55.97) circle (  1.49);

\path[draw=drawColor,line width= 0.4pt,line join=round,line cap=round,fill=fillColor] (367.00, 55.40) circle (  1.49);

\path[draw=drawColor,line width= 0.4pt,line join=round,line cap=round,fill=fillColor] (367.56, 55.01) circle (  1.49);

\path[draw=drawColor,line width= 0.4pt,line join=round,line cap=round,fill=fillColor] (368.13, 54.92) circle (  1.49);

\path[draw=drawColor,line width= 0.4pt,line join=round,line cap=round,fill=fillColor] (368.60, 54.84) circle (  1.49);

\path[draw=drawColor,line width= 0.4pt,line join=round,line cap=round,fill=fillColor] (369.09, 54.75) circle (  1.49);

\path[draw=drawColor,line width= 0.4pt,line join=round,line cap=round,fill=fillColor] (369.57, 54.79) circle (  1.49);

\path[draw=drawColor,line width= 0.4pt,line join=round,line cap=round,fill=fillColor] (370.04, 54.74) circle (  1.49);

\path[draw=drawColor,line width= 0.4pt,line join=round,line cap=round,fill=fillColor] (370.68, 54.73) circle (  1.49);

\path[draw=drawColor,line width= 0.4pt,line join=round,line cap=round,fill=fillColor] (371.17, 54.13) circle (  1.49);

\path[draw=drawColor,line width= 0.4pt,line join=round,line cap=round,fill=fillColor] (371.75, 54.00) circle (  1.49);

\path[draw=drawColor,line width= 0.4pt,line join=round,line cap=round,fill=fillColor] (372.27, 54.43) circle (  1.49);

\path[draw=drawColor,line width= 0.4pt,line join=round,line cap=round,fill=fillColor] (372.78, 54.24) circle (  1.49);

\path[draw=drawColor,line width= 0.4pt,line join=round,line cap=round,fill=fillColor] (373.28, 54.19) circle (  1.49);

\path[draw=drawColor,line width= 0.4pt,line join=round,line cap=round,fill=fillColor] (373.79, 54.14) circle (  1.49);

\path[draw=drawColor,line width= 0.4pt,line join=round,line cap=round,fill=fillColor] (374.35, 54.09) circle (  1.49);

\path[draw=drawColor,line width= 0.4pt,line join=round,line cap=round,fill=fillColor] (374.87, 54.12) circle (  1.49);

\path[draw=drawColor,line width= 0.4pt,line join=round,line cap=round,fill=fillColor] (375.36, 53.96) circle (  1.49);

\path[draw=drawColor,line width= 0.4pt,line join=round,line cap=round,fill=fillColor] (375.92, 53.79) circle (  1.49);

\path[draw=drawColor,line width= 0.4pt,line join=round,line cap=round,fill=fillColor] (376.43, 53.99) circle (  1.49);

\path[draw=drawColor,line width= 0.4pt,line join=round,line cap=round,fill=fillColor] (376.92, 54.36) circle (  1.49);

\path[draw=drawColor,line width= 0.4pt,line join=round,line cap=round,fill=fillColor] (377.41, 54.26) circle (  1.49);

\path[draw=drawColor,line width= 0.4pt,line join=round,line cap=round,fill=fillColor] (377.90, 54.22) circle (  1.49);

\path[draw=drawColor,line width= 0.4pt,line join=round,line cap=round,fill=fillColor] (378.39, 54.21) circle (  1.49);

\path[draw=drawColor,line width= 0.4pt,line join=round,line cap=round,fill=fillColor] (378.92, 54.34) circle (  1.49);

\path[draw=drawColor,line width= 0.4pt,line join=round,line cap=round,fill=fillColor] (379.41, 54.45) circle (  1.49);

\path[draw=drawColor,line width= 0.4pt,line join=round,line cap=round,fill=fillColor] (380.00, 54.20) circle (  1.49);

\path[draw=drawColor,line width= 0.4pt,line join=round,line cap=round,fill=fillColor] (380.54, 54.41) circle (  1.49);

\path[draw=drawColor,line width= 0.4pt,line join=round,line cap=round,fill=fillColor] (381.03, 53.97) circle (  1.49);

\path[draw=drawColor,line width= 0.4pt,line join=round,line cap=round,fill=fillColor] (381.50, 54.47) circle (  1.49);

\path[draw=drawColor,line width= 0.4pt,line join=round,line cap=round,fill=fillColor] (382.01, 54.30) circle (  1.49);

\path[draw=drawColor,line width= 0.4pt,line join=round,line cap=round,fill=fillColor] (382.53, 54.21) circle (  1.49);

\path[draw=drawColor,line width= 0.4pt,line join=round,line cap=round,fill=fillColor] (383.04, 54.33) circle (  1.49);

\path[draw=drawColor,line width= 0.4pt,line join=round,line cap=round,fill=fillColor] (383.53, 54.45) circle (  1.49);

\path[draw=drawColor,line width= 0.4pt,line join=round,line cap=round,fill=fillColor] (384.06, 54.74) circle (  1.49);

\path[draw=drawColor,line width= 0.4pt,line join=round,line cap=round,fill=fillColor] (384.53, 54.90) circle (  1.49);

\path[draw=drawColor,line width= 0.4pt,line join=round,line cap=round,fill=fillColor] (385.05, 55.68) circle (  1.49);

\path[draw=drawColor,line width= 0.4pt,line join=round,line cap=round,fill=fillColor] (385.56, 54.96) circle (  1.49);

\path[draw=drawColor,line width= 0.4pt,line join=round,line cap=round,fill=fillColor] (386.02, 54.72) circle (  1.49);

\path[draw=drawColor,line width= 0.4pt,line join=round,line cap=round,fill=fillColor] (386.54, 54.77) circle (  1.49);

\path[draw=drawColor,line width= 0.4pt,line join=round,line cap=round,fill=fillColor] (387.02, 54.59) circle (  1.49);

\path[draw=drawColor,line width= 0.4pt,line join=round,line cap=round,fill=fillColor] (387.54, 66.33) circle (  1.49);

\path[draw=drawColor,line width= 0.4pt,line join=round,line cap=round,fill=fillColor] (388.10, 55.00) circle (  1.49);

\path[draw=drawColor,line width= 0.4pt,line join=round,line cap=round,fill=fillColor] (388.61, 83.00) circle (  1.49);

\path[draw=drawColor,line width= 0.4pt,line join=round,line cap=round,fill=fillColor] (389.08, 58.63) circle (  1.49);

\path[draw=drawColor,line width= 0.4pt,line join=round,line cap=round,fill=fillColor] (389.60, 55.48) circle (  1.49);

\path[draw=drawColor,line width= 0.4pt,line join=round,line cap=round,fill=fillColor] (390.13, 60.37) circle (  1.49);

\path[draw=drawColor,line width= 0.4pt,line join=round,line cap=round,fill=fillColor] (390.64, 55.30) circle (  1.49);

\path[draw=drawColor,line width= 0.4pt,line join=round,line cap=round,fill=fillColor] (391.13, 55.07) circle (  1.49);

\path[draw=drawColor,line width= 0.4pt,line join=round,line cap=round,fill=fillColor] (391.63, 55.24) circle (  1.49);

\path[draw=drawColor,line width= 0.4pt,line join=round,line cap=round,fill=fillColor] (392.13, 55.07) circle (  1.49);

\path[draw=drawColor,line width= 0.4pt,line join=round,line cap=round,fill=fillColor] (392.67, 54.91) circle (  1.49);

\path[draw=drawColor,line width= 0.4pt,line join=round,line cap=round,fill=fillColor] (393.19, 54.80) circle (  1.49);

\path[draw=drawColor,line width= 0.4pt,line join=round,line cap=round,fill=fillColor] (393.66, 54.68) circle (  1.49);

\path[draw=drawColor,line width= 0.4pt,line join=round,line cap=round,fill=fillColor] (394.17, 54.67) circle (  1.49);

\path[draw=drawColor,line width= 0.4pt,line join=round,line cap=round,fill=fillColor] (394.68, 54.65) circle (  1.49);

\path[draw=drawColor,line width= 0.4pt,line join=round,line cap=round,fill=fillColor] (395.19, 54.58) circle (  1.49);

\path[draw=drawColor,line width= 0.4pt,line join=round,line cap=round,fill=fillColor] (395.69, 54.55) circle (  1.49);

\path[draw=drawColor,line width= 0.4pt,line join=round,line cap=round,fill=fillColor] (396.20, 54.79) circle (  1.49);

\path[draw=drawColor,line width= 0.4pt,line join=round,line cap=round,fill=fillColor] (396.69, 54.64) circle (  1.49);

\path[draw=drawColor,line width= 0.4pt,line join=round,line cap=round,fill=fillColor] (397.23, 54.57) circle (  1.49);

\path[draw=drawColor,line width= 0.4pt,line join=round,line cap=round,fill=fillColor] (397.79, 55.01) circle (  1.49);

\path[draw=drawColor,line width= 0.4pt,line join=round,line cap=round,fill=fillColor] (398.31, 54.38) circle (  1.49);

\path[draw=drawColor,line width= 0.4pt,line join=round,line cap=round,fill=fillColor] (398.79, 54.70) circle (  1.49);

\path[draw=drawColor,line width= 0.4pt,line join=round,line cap=round,fill=fillColor] (399.30, 54.37) circle (  1.49);

\path[draw=drawColor,line width= 0.4pt,line join=round,line cap=round,fill=fillColor] (399.82, 54.29) circle (  1.49);

\path[draw=drawColor,line width= 0.4pt,line join=round,line cap=round,fill=fillColor] (400.29, 54.54) circle (  1.49);

\path[draw=drawColor,line width= 0.4pt,line join=round,line cap=round,fill=fillColor] (401.01, 54.70) circle (  1.49);

\path[draw=drawColor,line width= 0.4pt,line join=round,line cap=round,fill=fillColor] (401.52, 54.61) circle (  1.49);

\path[draw=drawColor,line width= 0.4pt,line join=round,line cap=round,fill=fillColor] (402.01, 54.27) circle (  1.49);

\path[draw=drawColor,line width= 0.4pt,line join=round,line cap=round,fill=fillColor] (402.50, 54.33) circle (  1.49);

\path[draw=drawColor,line width= 0.4pt,line join=round,line cap=round,fill=fillColor] (403.03, 54.25) circle (  1.49);

\path[draw=drawColor,line width= 0.4pt,line join=round,line cap=round,fill=fillColor] (403.50, 54.66) circle (  1.49);

\path[draw=drawColor,line width= 0.4pt,line join=round,line cap=round,fill=fillColor] (404.06, 54.67) circle (  1.49);

\path[draw=drawColor,line width= 0.4pt,line join=round,line cap=round,fill=fillColor] (404.57, 54.62) circle (  1.49);

\path[draw=drawColor,line width= 0.4pt,line join=round,line cap=round,fill=fillColor] (405.09, 54.68) circle (  1.49);

\path[draw=drawColor,line width= 0.4pt,line join=round,line cap=round,fill=fillColor] (405.58, 54.61) circle (  1.49);

\path[draw=drawColor,line width= 0.4pt,line join=round,line cap=round,fill=fillColor] (406.09, 54.12) circle (  1.49);

\path[draw=drawColor,line width= 0.4pt,line join=round,line cap=round,fill=fillColor] (406.60, 54.09) circle (  1.49);

\path[draw=drawColor,line width= 0.4pt,line join=round,line cap=round,fill=fillColor] (407.23, 53.53) circle (  1.49);

\path[draw=drawColor,line width= 0.4pt,line join=round,line cap=round,fill=fillColor] (407.73, 57.25) circle (  1.49);

\path[draw=drawColor,line width= 0.4pt,line join=round,line cap=round,fill=fillColor] (408.20, 54.35) circle (  1.49);

\path[draw=drawColor,line width= 0.4pt,line join=round,line cap=round,fill=fillColor] (408.77, 54.48) circle (  1.49);

\path[draw=drawColor,line width= 0.4pt,line join=round,line cap=round,fill=fillColor] (409.28, 54.35) circle (  1.49);

\path[draw=drawColor,line width= 0.4pt,line join=round,line cap=round,fill=fillColor] (409.82, 54.48) circle (  1.49);

\path[draw=drawColor,line width= 0.4pt,line join=round,line cap=round,fill=fillColor] (410.30, 54.33) circle (  1.49);

\path[draw=drawColor,line width= 0.4pt,line join=round,line cap=round,fill=fillColor] (410.82, 54.35) circle (  1.49);

\path[draw=drawColor,line width= 0.4pt,line join=round,line cap=round,fill=fillColor] (411.33, 54.38) circle (  1.49);

\path[draw=drawColor,line width= 0.4pt,line join=round,line cap=round,fill=fillColor] (411.90,135.55) circle (  1.49);

\path[draw=drawColor,line width= 0.4pt,line join=round,line cap=round,fill=fillColor] (412.47, 54.54) circle (  1.49);

\path[draw=drawColor,line width= 0.4pt,line join=round,line cap=round,fill=fillColor] (412.98, 54.43) circle (  1.49);

\path[draw=drawColor,line width= 0.4pt,line join=round,line cap=round,fill=fillColor] (413.47, 57.10) circle (  1.49);

\path[draw=drawColor,line width= 0.4pt,line join=round,line cap=round,fill=fillColor] (414.01, 76.07) circle (  1.49);

\path[draw=drawColor,line width= 0.4pt,line join=round,line cap=round,fill=fillColor] (414.54, 56.06) circle (  1.49);

\path[draw=drawColor,line width= 0.4pt,line join=round,line cap=round,fill=fillColor] (415.01, 68.91) circle (  1.49);

\path[draw=drawColor,line width= 0.4pt,line join=round,line cap=round,fill=fillColor] (415.49, 59.68) circle (  1.49);

\path[draw=drawColor,line width= 0.4pt,line join=round,line cap=round,fill=fillColor] (415.96,106.98) circle (  1.49);

\path[draw=drawColor,line width= 0.4pt,line join=round,line cap=round,fill=fillColor] (416.45, 54.66) circle (  1.49);

\path[draw=drawColor,line width= 0.4pt,line join=round,line cap=round,fill=fillColor] (416.93, 54.59) circle (  1.49);

\path[draw=drawColor,line width= 0.4pt,line join=round,line cap=round,fill=fillColor] (417.45, 57.13) circle (  1.49);

\path[draw=drawColor,line width= 0.4pt,line join=round,line cap=round,fill=fillColor] (418.06, 55.15) circle (  1.49);

\path[draw=drawColor,line width= 0.4pt,line join=round,line cap=round,fill=fillColor] (418.63, 54.64) circle (  1.49);

\path[draw=drawColor,line width= 0.4pt,line join=round,line cap=round,fill=fillColor] (419.15, 55.11) circle (  1.49);

\path[draw=drawColor,line width= 0.4pt,line join=round,line cap=round,fill=fillColor] (419.64,215.99) circle (  1.49);

\path[draw=drawColor,line width= 0.4pt,line join=round,line cap=round,fill=fillColor] (420.12,215.60) circle (  1.49);

\path[draw=drawColor,line width= 0.4pt,line join=round,line cap=round,fill=fillColor] (420.56,192.62) circle (  1.49);

\path[draw=drawColor,line width= 0.4pt,line join=round,line cap=round,fill=fillColor] (421.03,151.35) circle (  1.49);

\path[draw=drawColor,line width= 0.4pt,line join=round,line cap=round,fill=fillColor] (421.49,144.95) circle (  1.49);

\path[draw=drawColor,line width= 0.4pt,line join=round,line cap=round,fill=fillColor] (421.97,202.80) circle (  1.49);

\path[draw=drawColor,line width= 0.4pt,line join=round,line cap=round,fill=fillColor] (422.49,202.90) circle (  1.49);

\path[draw=drawColor,line width= 0.4pt,line join=round,line cap=round,fill=fillColor] (423.06, 54.67) circle (  1.49);

\path[draw=drawColor,line width= 0.4pt,line join=round,line cap=round,fill=fillColor] (423.65, 54.42) circle (  1.49);

\path[draw=drawColor,line width= 0.4pt,line join=round,line cap=round,fill=fillColor] (424.11, 94.99) circle (  1.49);

\path[draw=drawColor,line width= 0.4pt,line join=round,line cap=round,fill=fillColor] (424.59,217.40) circle (  1.49);

\path[draw=drawColor,line width= 0.4pt,line join=round,line cap=round,fill=fillColor] (425.04, 55.30) circle (  1.49);

\path[draw=drawColor,line width= 0.4pt,line join=round,line cap=round,fill=fillColor] (425.54,215.54) circle (  1.49);

\path[draw=drawColor,line width= 0.4pt,line join=round,line cap=round,fill=fillColor] (426.09, 54.46) circle (  1.49);

\path[draw=drawColor,line width= 0.4pt,line join=round,line cap=round,fill=fillColor] (426.55,209.98) circle (  1.49);

\path[draw=drawColor,line width= 0.4pt,line join=round,line cap=round,fill=fillColor] (427.03,216.05) circle (  1.49);

\path[draw=drawColor,line width= 0.4pt,line join=round,line cap=round,fill=fillColor] (427.50,215.44) circle (  1.49);

\path[draw=drawColor,line width= 0.4pt,line join=round,line cap=round,fill=fillColor] (427.98,216.27) circle (  1.49);

\path[draw=drawColor,line width= 0.4pt,line join=round,line cap=round,fill=fillColor] (428.45,214.56) circle (  1.49);

\path[draw=drawColor,line width= 0.4pt,line join=round,line cap=round,fill=fillColor] (428.92,213.33) circle (  1.49);

\path[draw=drawColor,line width= 0.4pt,line join=round,line cap=round,fill=fillColor] (429.38,215.45) circle (  1.49);

\path[draw=drawColor,line width= 0.4pt,line join=round,line cap=round,fill=fillColor] (429.87,197.41) circle (  1.49);

\path[draw=drawColor,line width= 0.4pt,line join=round,line cap=round,fill=fillColor] (430.35,102.54) circle (  1.49);

\path[draw=drawColor,line width= 0.4pt,line join=round,line cap=round,fill=fillColor] (430.82,214.88) circle (  1.49);

\path[draw=drawColor,line width= 0.4pt,line join=round,line cap=round,fill=fillColor] (431.35,215.58) circle (  1.49);

\path[draw=drawColor,line width= 0.4pt,line join=round,line cap=round,fill=fillColor] (431.85,196.78) circle (  1.49);

\path[draw=drawColor,line width= 0.4pt,line join=round,line cap=round,fill=fillColor] (432.33,214.00) circle (  1.49);

\path[draw=drawColor,line width= 0.4pt,line join=round,line cap=round,fill=fillColor] (432.82,166.35) circle (  1.49);

\path[draw=drawColor,line width= 0.4pt,line join=round,line cap=round,fill=fillColor] (433.28,212.21) circle (  1.49);

\path[draw=drawColor,line width= 0.4pt,line join=round,line cap=round,fill=fillColor] (433.75,176.15) circle (  1.49);

\path[draw=drawColor,line width= 0.4pt,line join=round,line cap=round,fill=fillColor] (434.23,212.45) circle (  1.49);

\path[draw=drawColor,line width= 0.4pt,line join=round,line cap=round,fill=fillColor] (434.67,202.90) circle (  1.49);

\path[draw=drawColor,line width= 0.4pt,line join=round,line cap=round,fill=fillColor] (435.14,212.46) circle (  1.49);

\path[draw=drawColor,line width= 0.4pt,line join=round,line cap=round,fill=fillColor] (435.62,211.02) circle (  1.49);

\path[draw=drawColor,line width= 0.4pt,line join=round,line cap=round,fill=fillColor] (436.08,211.37) circle (  1.49);

\path[draw=drawColor,line width= 0.4pt,line join=round,line cap=round,fill=fillColor] (436.52,211.67) circle (  1.49);

\path[draw=drawColor,line width= 0.4pt,line join=round,line cap=round,fill=fillColor] (436.99,211.69) circle (  1.49);

\path[draw=drawColor,line width= 0.4pt,line join=round,line cap=round,fill=fillColor] (437.47,211.72) circle (  1.49);

\path[draw=drawColor,line width= 0.4pt,line join=round,line cap=round,fill=fillColor] (437.94,211.60) circle (  1.49);

\path[draw=drawColor,line width= 0.4pt,line join=round,line cap=round,fill=fillColor] (438.40,211.95) circle (  1.49);

\path[draw=drawColor,line width= 0.4pt,line join=round,line cap=round,fill=fillColor] (438.86,213.03) circle (  1.49);

\path[draw=drawColor,line width= 0.4pt,line join=round,line cap=round,fill=fillColor] (439.34,212.28) circle (  1.49);

\path[draw=drawColor,line width= 0.4pt,line join=round,line cap=round,fill=fillColor] (439.79,212.54) circle (  1.49);

\path[draw=drawColor,line width= 0.4pt,line join=round,line cap=round,fill=fillColor] (440.25,211.98) circle (  1.49);

\path[draw=drawColor,line width= 0.4pt,line join=round,line cap=round,fill=fillColor] (440.71,211.61) circle (  1.49);

\path[draw=drawColor,line width= 0.4pt,line join=round,line cap=round,fill=fillColor] (441.19,211.80) circle (  1.49);

\path[draw=drawColor,line width= 0.4pt,line join=round,line cap=round,fill=fillColor] (441.66,210.90) circle (  1.49);

\path[draw=drawColor,line width= 0.4pt,line join=round,line cap=round,fill=fillColor] (442.22,210.65) circle (  1.49);

\path[draw=drawColor,line width= 0.4pt,line join=round,line cap=round,fill=fillColor] (442.71,210.77) circle (  1.49);

\path[draw=drawColor,line width= 0.4pt,line join=round,line cap=round,fill=fillColor] (443.17,210.83) circle (  1.49);

\path[draw=drawColor,line width= 0.4pt,line join=round,line cap=round,fill=fillColor] (443.59,210.69) circle (  1.49);

\path[draw=drawColor,line width= 0.4pt,line join=round,line cap=round,fill=fillColor] (444.15,211.04) circle (  1.49);

\path[draw=drawColor,line width= 0.4pt,line join=round,line cap=round,fill=fillColor] (444.64,210.54) circle (  1.49);

\path[draw=drawColor,line width= 0.4pt,line join=round,line cap=round,fill=fillColor] (445.13,210.49) circle (  1.49);

\path[draw=drawColor,line width= 0.4pt,line join=round,line cap=round,fill=fillColor] (445.64,210.45) circle (  1.49);

\path[draw=drawColor,line width= 0.4pt,line join=round,line cap=round,fill=fillColor] (446.06,210.54) circle (  1.49);

\path[draw=drawColor,line width= 0.4pt,line join=round,line cap=round,fill=fillColor] (446.57,210.42) circle (  1.49);

\path[draw=drawColor,line width= 0.4pt,line join=round,line cap=round,fill=fillColor] (447.06,210.10) circle (  1.49);

\path[draw=drawColor,line width= 0.4pt,line join=round,line cap=round,fill=fillColor] (447.49,209.60) circle (  1.49);

\path[draw=drawColor,line width= 0.4pt,line join=round,line cap=round,fill=fillColor] (448.00,209.66) circle (  1.49);

\path[draw=drawColor,line width= 0.4pt,line join=round,line cap=round,fill=fillColor] (448.45,209.36) circle (  1.49);

\path[draw=drawColor,line width= 0.4pt,line join=round,line cap=round,fill=fillColor] (448.94,209.35) circle (  1.49);

\path[draw=drawColor,line width= 0.4pt,line join=round,line cap=round,fill=fillColor] (449.47,209.73) circle (  1.49);

\path[draw=drawColor,line width= 0.4pt,line join=round,line cap=round,fill=fillColor] (449.93,209.32) circle (  1.49);

\path[draw=drawColor,line width= 0.4pt,line join=round,line cap=round,fill=fillColor] (450.43,208.93) circle (  1.49);

\path[draw=drawColor,line width= 0.4pt,line join=round,line cap=round,fill=fillColor] (450.89,208.65) circle (  1.49);

\path[draw=drawColor,line width= 0.4pt,line join=round,line cap=round,fill=fillColor] (451.38,208.68) circle (  1.49);

\path[draw=drawColor,line width= 0.4pt,line join=round,line cap=round,fill=fillColor] (451.86,208.44) circle (  1.49);

\path[draw=drawColor,line width= 0.4pt,line join=round,line cap=round,fill=fillColor] (452.30,208.69) circle (  1.49);

\path[draw=drawColor,line width= 0.4pt,line join=round,line cap=round,fill=fillColor] (452.82,207.87) circle (  1.49);

\path[draw=drawColor,line width= 0.4pt,line join=round,line cap=round,fill=fillColor] (453.28,208.11) circle (  1.49);

\path[draw=drawColor,line width= 0.4pt,line join=round,line cap=round,fill=fillColor] (453.74,207.60) circle (  1.49);

\path[draw=drawColor,line width= 0.4pt,line join=round,line cap=round,fill=fillColor] (454.17,208.33) circle (  1.49);

\path[draw=drawColor,line width= 0.4pt,line join=round,line cap=round,fill=fillColor] (454.59,207.72) circle (  1.49);

\path[draw=drawColor,line width= 0.4pt,line join=round,line cap=round,fill=fillColor] (455.05,208.00) circle (  1.49);

\path[draw=drawColor,line width= 0.4pt,line join=round,line cap=round,fill=fillColor] (455.56,206.79) circle (  1.49);

\path[draw=drawColor,line width= 0.4pt,line join=round,line cap=round,fill=fillColor] (456.07,206.93) circle (  1.49);

\path[draw=drawColor,line width= 0.4pt,line join=round,line cap=round,fill=fillColor] (456.52,206.96) circle (  1.49);

\path[draw=drawColor,line width= 0.4pt,line join=round,line cap=round,fill=fillColor] (456.98,207.20) circle (  1.49);

\path[draw=drawColor,line width= 0.4pt,line join=round,line cap=round,fill=fillColor] (457.51,207.91) circle (  1.49);

\path[draw=drawColor,line width= 0.4pt,line join=round,line cap=round,fill=fillColor] (457.95,206.93) circle (  1.49);

\path[draw=drawColor,line width= 0.4pt,line join=round,line cap=round,fill=fillColor] (458.39,206.81) circle (  1.49);

\path[draw=drawColor,line width= 0.4pt,line join=round,line cap=round,fill=fillColor] (458.91,206.86) circle (  1.49);

\path[draw=drawColor,line width= 0.4pt,line join=round,line cap=round,fill=fillColor] (459.42,207.35) circle (  1.49);

\path[draw=drawColor,line width= 0.4pt,line join=round,line cap=round,fill=fillColor] (459.94,206.96) circle (  1.49);

\path[draw=drawColor,line width= 0.4pt,line join=round,line cap=round,fill=fillColor] (460.39,206.81) circle (  1.49);

\path[draw=drawColor,line width= 0.4pt,line join=round,line cap=round,fill=fillColor] (460.85,206.48) circle (  1.49);

\path[draw=drawColor,line width= 0.4pt,line join=round,line cap=round,fill=fillColor] (461.32,206.01) circle (  1.49);

\path[draw=drawColor,line width= 0.4pt,line join=round,line cap=round,fill=fillColor] (461.91,205.92) circle (  1.49);

\path[draw=drawColor,line width= 0.4pt,line join=round,line cap=round,fill=fillColor] (462.48,205.41) circle (  1.49);

\path[draw=drawColor,line width= 0.4pt,line join=round,line cap=round,fill=fillColor] (462.97,205.09) circle (  1.49);

\path[draw=drawColor,line width= 0.4pt,line join=round,line cap=round,fill=fillColor] (463.53,205.30) circle (  1.49);

\path[draw=drawColor,line width= 0.4pt,line join=round,line cap=round,fill=fillColor] (463.97,205.01) circle (  1.49);

\path[draw=drawColor,line width= 0.4pt,line join=round,line cap=round,fill=fillColor] (464.43,204.71) circle (  1.49);

\path[draw=drawColor,line width= 0.4pt,line join=round,line cap=round,fill=fillColor] (464.89,203.82) circle (  1.49);

\path[draw=drawColor,line width= 0.4pt,line join=round,line cap=round,fill=fillColor] (465.38,204.13) circle (  1.49);

\path[draw=drawColor,line width= 0.4pt,line join=round,line cap=round,fill=fillColor] (465.81,204.01) circle (  1.49);

\path[draw=drawColor,line width= 0.4pt,line join=round,line cap=round,fill=fillColor] (466.25,203.99) circle (  1.49);

\path[draw=drawColor,line width= 0.4pt,line join=round,line cap=round,fill=fillColor] (466.84,204.13) circle (  1.49);

\path[draw=drawColor,line width= 0.4pt,line join=round,line cap=round,fill=fillColor] (467.44,203.60) circle (  1.49);

\path[draw=drawColor,line width= 0.4pt,line join=round,line cap=round,fill=fillColor] (467.87,201.94) circle (  1.49);

\path[draw=drawColor,line width= 0.4pt,line join=round,line cap=round,fill=fillColor] (468.29,203.44) circle (  1.49);

\path[draw=drawColor,line width= 0.4pt,line join=round,line cap=round,fill=fillColor] (468.70,203.54) circle (  1.49);

\path[draw=drawColor,line width= 0.4pt,line join=round,line cap=round,fill=fillColor] (469.13,202.55) circle (  1.49);
\definecolor{drawColor}{RGB}{0,0,255}
\definecolor{fillColor}{RGB}{0,0,255}

\path[draw=drawColor,line width= 0.4pt,line join=round,line cap=round,fill=fillColor] (293.07, 74.29) circle (  1.49);

\path[draw=drawColor,line width= 0.4pt,line join=round,line cap=round,fill=fillColor] (293.52, 75.70) circle (  1.49);

\path[draw=drawColor,line width= 0.4pt,line join=round,line cap=round,fill=fillColor] (293.97, 76.64) circle (  1.49);

\path[draw=drawColor,line width= 0.4pt,line join=round,line cap=round,fill=fillColor] (294.42, 77.09) circle (  1.49);

\path[draw=drawColor,line width= 0.4pt,line join=round,line cap=round,fill=fillColor] (294.87, 77.73) circle (  1.49);

\path[draw=drawColor,line width= 0.4pt,line join=round,line cap=round,fill=fillColor] (295.32, 77.93) circle (  1.49);

\path[draw=drawColor,line width= 0.4pt,line join=round,line cap=round,fill=fillColor] (295.77, 76.96) circle (  1.49);

\path[draw=drawColor,line width= 0.4pt,line join=round,line cap=round,fill=fillColor] (296.25, 76.24) circle (  1.49);

\path[draw=drawColor,line width= 0.4pt,line join=round,line cap=round,fill=fillColor] (296.71, 75.69) circle (  1.49);

\path[draw=drawColor,line width= 0.4pt,line join=round,line cap=round,fill=fillColor] (297.15, 73.33) circle (  1.49);

\path[draw=drawColor,line width= 0.4pt,line join=round,line cap=round,fill=fillColor] (297.61, 75.15) circle (  1.49);

\path[draw=drawColor,line width= 0.4pt,line join=round,line cap=round,fill=fillColor] (298.08, 73.49) circle (  1.49);

\path[draw=drawColor,line width= 0.4pt,line join=round,line cap=round,fill=fillColor] (298.52, 76.94) circle (  1.49);

\path[draw=drawColor,line width= 0.4pt,line join=round,line cap=round,fill=fillColor] (298.98, 77.69) circle (  1.49);

\path[draw=drawColor,line width= 0.4pt,line join=round,line cap=round,fill=fillColor] (299.44, 77.76) circle (  1.49);

\path[draw=drawColor,line width= 0.4pt,line join=round,line cap=round,fill=fillColor] (299.90, 77.33) circle (  1.49);

\path[draw=drawColor,line width= 0.4pt,line join=round,line cap=round,fill=fillColor] (300.41, 77.69) circle (  1.49);

\path[draw=drawColor,line width= 0.4pt,line join=round,line cap=round,fill=fillColor] (300.87, 78.18) circle (  1.49);

\path[draw=drawColor,line width= 0.4pt,line join=round,line cap=round,fill=fillColor] (301.32, 78.53) circle (  1.49);

\path[draw=drawColor,line width= 0.4pt,line join=round,line cap=round,fill=fillColor] (301.77, 79.07) circle (  1.49);

\path[draw=drawColor,line width= 0.4pt,line join=round,line cap=round,fill=fillColor] (302.22, 78.42) circle (  1.49);

\path[draw=drawColor,line width= 0.4pt,line join=round,line cap=round,fill=fillColor] (302.68, 78.64) circle (  1.49);

\path[draw=drawColor,line width= 0.4pt,line join=round,line cap=round,fill=fillColor] (303.14, 77.22) circle (  1.49);

\path[draw=drawColor,line width= 0.4pt,line join=round,line cap=round,fill=fillColor] (303.62, 77.19) circle (  1.49);

\path[draw=drawColor,line width= 0.4pt,line join=round,line cap=round,fill=fillColor] (304.07, 77.22) circle (  1.49);

\path[draw=drawColor,line width= 0.4pt,line join=round,line cap=round,fill=fillColor] (304.52, 77.21) circle (  1.49);

\path[draw=drawColor,line width= 0.4pt,line join=round,line cap=round,fill=fillColor] (304.99, 77.02) circle (  1.49);

\path[draw=drawColor,line width= 0.4pt,line join=round,line cap=round,fill=fillColor] (305.45, 77.51) circle (  1.49);

\path[draw=drawColor,line width= 0.4pt,line join=round,line cap=round,fill=fillColor] (305.91, 77.44) circle (  1.49);

\path[draw=drawColor,line width= 0.4pt,line join=round,line cap=round,fill=fillColor] (306.38, 77.45) circle (  1.49);

\path[draw=drawColor,line width= 0.4pt,line join=round,line cap=round,fill=fillColor] (306.84, 77.68) circle (  1.49);

\path[draw=drawColor,line width= 0.4pt,line join=round,line cap=round,fill=fillColor] (307.30, 77.56) circle (  1.49);

\path[draw=drawColor,line width= 0.4pt,line join=round,line cap=round,fill=fillColor] (307.76, 77.98) circle (  1.49);

\path[draw=drawColor,line width= 0.4pt,line join=round,line cap=round,fill=fillColor] (308.22, 77.91) circle (  1.49);

\path[draw=drawColor,line width= 0.4pt,line join=round,line cap=round,fill=fillColor] (308.67, 77.53) circle (  1.49);

\path[draw=drawColor,line width= 0.4pt,line join=round,line cap=round,fill=fillColor] (309.13, 79.05) circle (  1.49);

\path[draw=drawColor,line width= 0.4pt,line join=round,line cap=round,fill=fillColor] (309.59, 78.98) circle (  1.49);

\path[draw=drawColor,line width= 0.4pt,line join=round,line cap=round,fill=fillColor] (310.05, 79.38) circle (  1.49);

\path[draw=drawColor,line width= 0.4pt,line join=round,line cap=round,fill=fillColor] (310.51, 81.14) circle (  1.49);

\path[draw=drawColor,line width= 0.4pt,line join=round,line cap=round,fill=fillColor] (310.95, 80.43) circle (  1.49);

\path[draw=drawColor,line width= 0.4pt,line join=round,line cap=round,fill=fillColor] (311.41, 79.07) circle (  1.49);

\path[draw=drawColor,line width= 0.4pt,line join=round,line cap=round,fill=fillColor] (311.87, 80.70) circle (  1.49);

\path[draw=drawColor,line width= 0.4pt,line join=round,line cap=round,fill=fillColor] (312.31, 80.35) circle (  1.49);

\path[draw=drawColor,line width= 0.4pt,line join=round,line cap=round,fill=fillColor] (312.75, 81.34) circle (  1.49);

\path[draw=drawColor,line width= 0.4pt,line join=round,line cap=round,fill=fillColor] (313.19, 80.88) circle (  1.49);

\path[draw=drawColor,line width= 0.4pt,line join=round,line cap=round,fill=fillColor] (313.65, 80.78) circle (  1.49);

\path[draw=drawColor,line width= 0.4pt,line join=round,line cap=round,fill=fillColor] (314.11, 79.25) circle (  1.49);

\path[draw=drawColor,line width= 0.4pt,line join=round,line cap=round,fill=fillColor] (314.57, 77.73) circle (  1.49);

\path[draw=drawColor,line width= 0.4pt,line join=round,line cap=round,fill=fillColor] (315.01, 76.34) circle (  1.49);

\path[draw=drawColor,line width= 0.4pt,line join=round,line cap=round,fill=fillColor] (315.47, 76.13) circle (  1.49);

\path[draw=drawColor,line width= 0.4pt,line join=round,line cap=round,fill=fillColor] (315.93, 74.25) circle (  1.49);

\path[draw=drawColor,line width= 0.4pt,line join=round,line cap=round,fill=fillColor] (316.43, 72.50) circle (  1.49);

\path[draw=drawColor,line width= 0.4pt,line join=round,line cap=round,fill=fillColor] (316.87, 70.87) circle (  1.49);

\path[draw=drawColor,line width= 0.4pt,line join=round,line cap=round,fill=fillColor] (317.33, 69.51) circle (  1.49);

\path[draw=drawColor,line width= 0.4pt,line join=round,line cap=round,fill=fillColor] (317.81, 67.53) circle (  1.49);

\path[draw=drawColor,line width= 0.4pt,line join=round,line cap=round,fill=fillColor] (318.28, 67.64) circle (  1.49);

\path[draw=drawColor,line width= 0.4pt,line join=round,line cap=round,fill=fillColor] (318.76, 67.90) circle (  1.49);

\path[draw=drawColor,line width= 0.4pt,line join=round,line cap=round,fill=fillColor] (319.25, 67.04) circle (  1.49);

\path[draw=drawColor,line width= 0.4pt,line join=round,line cap=round,fill=fillColor] (319.76, 66.58) circle (  1.49);

\path[draw=drawColor,line width= 0.4pt,line join=round,line cap=round,fill=fillColor] (320.23, 65.69) circle (  1.49);

\path[draw=drawColor,line width= 0.4pt,line join=round,line cap=round,fill=fillColor] (320.71, 65.35) circle (  1.49);

\path[draw=drawColor,line width= 0.4pt,line join=round,line cap=round,fill=fillColor] (321.18, 65.14) circle (  1.49);

\path[draw=drawColor,line width= 0.4pt,line join=round,line cap=round,fill=fillColor] (321.65, 64.98) circle (  1.49);

\path[draw=drawColor,line width= 0.4pt,line join=round,line cap=round,fill=fillColor] (322.13, 64.33) circle (  1.49);

\path[draw=drawColor,line width= 0.4pt,line join=round,line cap=round,fill=fillColor] (322.59, 63.39) circle (  1.49);

\path[draw=drawColor,line width= 0.4pt,line join=round,line cap=round,fill=fillColor] (323.06, 62.23) circle (  1.49);

\path[draw=drawColor,line width= 0.4pt,line join=round,line cap=round,fill=fillColor] (323.57, 61.31) circle (  1.49);

\path[draw=drawColor,line width= 0.4pt,line join=round,line cap=round,fill=fillColor] (324.06, 60.71) circle (  1.49);

\path[draw=drawColor,line width= 0.4pt,line join=round,line cap=round,fill=fillColor] (324.55, 59.47) circle (  1.49);

\path[draw=drawColor,line width= 0.4pt,line join=round,line cap=round,fill=fillColor] (325.04, 59.40) circle (  1.49);

\path[draw=drawColor,line width= 0.4pt,line join=round,line cap=round,fill=fillColor] (325.60, 58.93) circle (  1.49);

\path[draw=drawColor,line width= 0.4pt,line join=round,line cap=round,fill=fillColor] (326.09, 58.63) circle (  1.49);

\path[draw=drawColor,line width= 0.4pt,line join=round,line cap=round,fill=fillColor] (326.60, 58.95) circle (  1.49);

\path[draw=drawColor,line width= 0.4pt,line join=round,line cap=round,fill=fillColor] (327.11, 58.24) circle (  1.49);

\path[draw=drawColor,line width= 0.4pt,line join=round,line cap=round,fill=fillColor] (327.56, 58.93) circle (  1.49);

\path[draw=drawColor,line width= 0.4pt,line join=round,line cap=round,fill=fillColor] (328.07, 59.25) circle (  1.49);

\path[draw=drawColor,line width= 0.4pt,line join=round,line cap=round,fill=fillColor] (328.55, 59.48) circle (  1.49);

\path[draw=drawColor,line width= 0.4pt,line join=round,line cap=round,fill=fillColor] (329.04, 59.53) circle (  1.49);

\path[draw=drawColor,line width= 0.4pt,line join=round,line cap=round,fill=fillColor] (329.50, 59.13) circle (  1.49);

\path[draw=drawColor,line width= 0.4pt,line join=round,line cap=round,fill=fillColor] (329.97, 60.19) circle (  1.49);

\path[draw=drawColor,line width= 0.4pt,line join=round,line cap=round,fill=fillColor] (330.45, 60.36) circle (  1.49);

\path[draw=drawColor,line width= 0.4pt,line join=round,line cap=round,fill=fillColor] (330.92, 60.87) circle (  1.49);

\path[draw=drawColor,line width= 0.4pt,line join=round,line cap=round,fill=fillColor] (331.41, 62.65) circle (  1.49);

\path[draw=drawColor,line width= 0.4pt,line join=round,line cap=round,fill=fillColor] (331.89, 62.55) circle (  1.49);

\path[draw=drawColor,line width= 0.4pt,line join=round,line cap=round,fill=fillColor] (332.38, 62.71) circle (  1.49);

\path[draw=drawColor,line width= 0.4pt,line join=round,line cap=round,fill=fillColor] (332.85, 61.69) circle (  1.49);

\path[draw=drawColor,line width= 0.4pt,line join=round,line cap=round,fill=fillColor] (333.33, 60.75) circle (  1.49);

\path[draw=drawColor,line width= 0.4pt,line join=round,line cap=round,fill=fillColor] (333.83, 60.19) circle (  1.49);

\path[draw=drawColor,line width= 0.4pt,line join=round,line cap=round,fill=fillColor] (334.31, 60.26) circle (  1.49);

\path[draw=drawColor,line width= 0.4pt,line join=round,line cap=round,fill=fillColor] (334.80, 59.12) circle (  1.49);

\path[draw=drawColor,line width= 0.4pt,line join=round,line cap=round,fill=fillColor] (335.27, 58.42) circle (  1.49);

\path[draw=drawColor,line width= 0.4pt,line join=round,line cap=round,fill=fillColor] (335.75, 57.14) circle (  1.49);

\path[draw=drawColor,line width= 0.4pt,line join=round,line cap=round,fill=fillColor] (336.22, 56.83) circle (  1.49);

\path[draw=drawColor,line width= 0.4pt,line join=round,line cap=round,fill=fillColor] (336.70, 55.92) circle (  1.49);

\path[draw=drawColor,line width= 0.4pt,line join=round,line cap=round,fill=fillColor] (337.17, 58.00) circle (  1.49);

\path[draw=drawColor,line width= 0.4pt,line join=round,line cap=round,fill=fillColor] (337.66, 56.18) circle (  1.49);

\path[draw=drawColor,line width= 0.4pt,line join=round,line cap=round,fill=fillColor] (338.14, 56.32) circle (  1.49);

\path[draw=drawColor,line width= 0.4pt,line join=round,line cap=round,fill=fillColor] (338.63, 56.41) circle (  1.49);

\path[draw=drawColor,line width= 0.4pt,line join=round,line cap=round,fill=fillColor] (339.15, 54.69) circle (  1.49);

\path[draw=drawColor,line width= 0.4pt,line join=round,line cap=round,fill=fillColor] (339.69, 53.21) circle (  1.49);

\path[draw=drawColor,line width= 0.4pt,line join=round,line cap=round,fill=fillColor] (340.20, 53.09) circle (  1.49);

\path[draw=drawColor,line width= 0.4pt,line join=round,line cap=round,fill=fillColor] (340.69, 53.29) circle (  1.49);

\path[draw=drawColor,line width= 0.4pt,line join=round,line cap=round,fill=fillColor] (341.17, 53.71) circle (  1.49);

\path[draw=drawColor,line width= 0.4pt,line join=round,line cap=round,fill=fillColor] (341.66, 55.30) circle (  1.49);

\path[draw=drawColor,line width= 0.4pt,line join=round,line cap=round,fill=fillColor] (342.13, 53.29) circle (  1.49);

\path[draw=drawColor,line width= 0.4pt,line join=round,line cap=round,fill=fillColor] (342.62, 54.20) circle (  1.49);

\path[draw=drawColor,line width= 0.4pt,line join=round,line cap=round,fill=fillColor] (343.10, 53.41) circle (  1.49);

\path[draw=drawColor,line width= 0.4pt,line join=round,line cap=round,fill=fillColor] (343.59, 53.03) circle (  1.49);

\path[draw=drawColor,line width= 0.4pt,line join=round,line cap=round,fill=fillColor] (344.06, 52.35) circle (  1.49);

\path[draw=drawColor,line width= 0.4pt,line join=round,line cap=round,fill=fillColor] (344.67, 51.06) circle (  1.49);

\path[draw=drawColor,line width= 0.4pt,line join=round,line cap=round,fill=fillColor] (345.18, 50.32) circle (  1.49);

\path[draw=drawColor,line width= 0.4pt,line join=round,line cap=round,fill=fillColor] (345.72, 50.39) circle (  1.49);

\path[draw=drawColor,line width= 0.4pt,line join=round,line cap=round,fill=fillColor] (346.21, 50.40) circle (  1.49);

\path[draw=drawColor,line width= 0.4pt,line join=round,line cap=round,fill=fillColor] (346.90, 50.37) circle (  1.49);

\path[draw=drawColor,line width= 0.4pt,line join=round,line cap=round,fill=fillColor] (347.44, 50.04) circle (  1.49);

\path[draw=drawColor,line width= 0.4pt,line join=round,line cap=round,fill=fillColor] (347.94, 49.84) circle (  1.49);

\path[draw=drawColor,line width= 0.4pt,line join=round,line cap=round,fill=fillColor] (348.45, 50.02) circle (  1.49);

\path[draw=drawColor,line width= 0.4pt,line join=round,line cap=round,fill=fillColor] (348.94, 50.56) circle (  1.49);

\path[draw=drawColor,line width= 0.4pt,line join=round,line cap=round,fill=fillColor] (349.47, 50.05) circle (  1.49);

\path[draw=drawColor,line width= 0.4pt,line join=round,line cap=round,fill=fillColor] (349.96, 49.47) circle (  1.49);

\path[draw=drawColor,line width= 0.4pt,line join=round,line cap=round,fill=fillColor] (350.50, 49.17) circle (  1.49);

\path[draw=drawColor,line width= 0.4pt,line join=round,line cap=round,fill=fillColor] (351.01, 49.09) circle (  1.49);

\path[draw=drawColor,line width= 0.4pt,line join=round,line cap=round,fill=fillColor] (351.51, 49.07) circle (  1.49);

\path[draw=drawColor,line width= 0.4pt,line join=round,line cap=round,fill=fillColor] (352.02, 49.19) circle (  1.49);

\path[draw=drawColor,line width= 0.4pt,line join=round,line cap=round,fill=fillColor] (352.51, 49.18) circle (  1.49);

\path[draw=drawColor,line width= 0.4pt,line join=round,line cap=round,fill=fillColor] (353.04, 49.19) circle (  1.49);

\path[draw=drawColor,line width= 0.4pt,line join=round,line cap=round,fill=fillColor] (353.53, 49.42) circle (  1.49);

\path[draw=drawColor,line width= 0.4pt,line join=round,line cap=round,fill=fillColor] (354.03, 49.73) circle (  1.49);

\path[draw=drawColor,line width= 0.4pt,line join=round,line cap=round,fill=fillColor] (354.52, 49.97) circle (  1.49);

\path[draw=drawColor,line width= 0.4pt,line join=round,line cap=round,fill=fillColor] (355.03, 49.97) circle (  1.49);

\path[draw=drawColor,line width= 0.4pt,line join=round,line cap=round,fill=fillColor] (355.52, 50.07) circle (  1.49);

\path[draw=drawColor,line width= 0.4pt,line join=round,line cap=round,fill=fillColor] (356.08, 49.87) circle (  1.49);

\path[draw=drawColor,line width= 0.4pt,line join=round,line cap=round,fill=fillColor] (356.60, 49.39) circle (  1.49);

\path[draw=drawColor,line width= 0.4pt,line join=round,line cap=round,fill=fillColor] (357.09, 49.42) circle (  1.49);

\path[draw=drawColor,line width= 0.4pt,line join=round,line cap=round,fill=fillColor] (357.72, 49.32) circle (  1.49);

\path[draw=drawColor,line width= 0.4pt,line join=round,line cap=round,fill=fillColor] (358.21, 49.01) circle (  1.49);

\path[draw=drawColor,line width= 0.4pt,line join=round,line cap=round,fill=fillColor] (358.78, 48.96) circle (  1.49);

\path[draw=drawColor,line width= 0.4pt,line join=round,line cap=round,fill=fillColor] (359.45, 48.99) circle (  1.49);

\path[draw=drawColor,line width= 0.4pt,line join=round,line cap=round,fill=fillColor] (360.07, 48.79) circle (  1.49);

\path[draw=drawColor,line width= 0.4pt,line join=round,line cap=round,fill=fillColor] (360.60, 48.60) circle (  1.49);

\path[draw=drawColor,line width= 0.4pt,line join=round,line cap=round,fill=fillColor] (361.12, 48.40) circle (  1.49);

\path[draw=drawColor,line width= 0.4pt,line join=round,line cap=round,fill=fillColor] (361.86, 48.68) circle (  1.49);

\path[draw=drawColor,line width= 0.4pt,line join=round,line cap=round,fill=fillColor] (362.35, 48.72) circle (  1.49);

\path[draw=drawColor,line width= 0.4pt,line join=round,line cap=round,fill=fillColor] (362.86, 48.96) circle (  1.49);

\path[draw=drawColor,line width= 0.4pt,line join=round,line cap=round,fill=fillColor] (363.41, 49.18) circle (  1.49);

\path[draw=drawColor,line width= 0.4pt,line join=round,line cap=round,fill=fillColor] (363.90, 48.87) circle (  1.49);

\path[draw=drawColor,line width= 0.4pt,line join=round,line cap=round,fill=fillColor] (364.51, 49.05) circle (  1.49);

\path[draw=drawColor,line width= 0.4pt,line join=round,line cap=round,fill=fillColor] (365.07, 48.92) circle (  1.49);

\path[draw=drawColor,line width= 0.4pt,line join=round,line cap=round,fill=fillColor] (365.57, 48.80) circle (  1.49);

\path[draw=drawColor,line width= 0.4pt,line join=round,line cap=round,fill=fillColor] (366.08, 48.60) circle (  1.49);

\path[draw=drawColor,line width= 0.4pt,line join=round,line cap=round,fill=fillColor] (366.57, 48.84) circle (  1.49);

\path[draw=drawColor,line width= 0.4pt,line join=round,line cap=round,fill=fillColor] (367.08, 48.64) circle (  1.49);

\path[draw=drawColor,line width= 0.4pt,line join=round,line cap=round,fill=fillColor] (367.65, 48.37) circle (  1.49);

\path[draw=drawColor,line width= 0.4pt,line join=round,line cap=round,fill=fillColor] (368.21, 48.38) circle (  1.49);

\path[draw=drawColor,line width= 0.4pt,line join=round,line cap=round,fill=fillColor] (368.68, 48.26) circle (  1.49);

\path[draw=drawColor,line width= 0.4pt,line join=round,line cap=round,fill=fillColor] (369.18, 48.15) circle (  1.49);

\path[draw=drawColor,line width= 0.4pt,line join=round,line cap=round,fill=fillColor] (369.65, 48.21) circle (  1.49);

\path[draw=drawColor,line width= 0.4pt,line join=round,line cap=round,fill=fillColor] (370.13, 48.08) circle (  1.49);

\path[draw=drawColor,line width= 0.4pt,line join=round,line cap=round,fill=fillColor] (370.76, 48.06) circle (  1.49);

\path[draw=drawColor,line width= 0.4pt,line join=round,line cap=round,fill=fillColor] (371.27, 47.79) circle (  1.49);

\path[draw=drawColor,line width= 0.4pt,line join=round,line cap=round,fill=fillColor] (371.83, 47.64) circle (  1.49);

\path[draw=drawColor,line width= 0.4pt,line join=round,line cap=round,fill=fillColor] (372.35, 47.92) circle (  1.49);

\path[draw=drawColor,line width= 0.4pt,line join=round,line cap=round,fill=fillColor] (372.86, 47.88) circle (  1.49);

\path[draw=drawColor,line width= 0.4pt,line join=round,line cap=round,fill=fillColor] (373.38, 47.95) circle (  1.49);

\path[draw=drawColor,line width= 0.4pt,line join=round,line cap=round,fill=fillColor] (373.87, 48.03) circle (  1.49);

\path[draw=drawColor,line width= 0.4pt,line join=round,line cap=round,fill=fillColor] (374.45, 47.79) circle (  1.49);

\path[draw=drawColor,line width= 0.4pt,line join=round,line cap=round,fill=fillColor] (374.95, 47.79) circle (  1.49);

\path[draw=drawColor,line width= 0.4pt,line join=round,line cap=round,fill=fillColor] (375.46, 47.60) circle (  1.49);

\path[draw=drawColor,line width= 0.4pt,line join=round,line cap=round,fill=fillColor] (376.00, 47.70) circle (  1.49);

\path[draw=drawColor,line width= 0.4pt,line join=round,line cap=round,fill=fillColor] (376.51, 47.70) circle (  1.49);

\path[draw=drawColor,line width= 0.4pt,line join=round,line cap=round,fill=fillColor] (377.00, 48.62) circle (  1.49);

\path[draw=drawColor,line width= 0.4pt,line join=round,line cap=round,fill=fillColor] (377.49, 47.74) circle (  1.49);

\path[draw=drawColor,line width= 0.4pt,line join=round,line cap=round,fill=fillColor] (377.98, 47.90) circle (  1.49);

\path[draw=drawColor,line width= 0.4pt,line join=round,line cap=round,fill=fillColor] (378.49, 47.78) circle (  1.49);

\path[draw=drawColor,line width= 0.4pt,line join=round,line cap=round,fill=fillColor] (379.00, 47.88) circle (  1.49);

\path[draw=drawColor,line width= 0.4pt,line join=round,line cap=round,fill=fillColor] (379.49, 47.90) circle (  1.49);

\path[draw=drawColor,line width= 0.4pt,line join=round,line cap=round,fill=fillColor] (380.11, 47.92) circle (  1.49);

\path[draw=drawColor,line width= 0.4pt,line join=round,line cap=round,fill=fillColor] (380.62, 47.69) circle (  1.49);

\path[draw=drawColor,line width= 0.4pt,line join=round,line cap=round,fill=fillColor] (381.11, 47.75) circle (  1.49);

\path[draw=drawColor,line width= 0.4pt,line join=round,line cap=round,fill=fillColor] (381.60, 47.89) circle (  1.49);

\path[draw=drawColor,line width= 0.4pt,line join=round,line cap=round,fill=fillColor] (382.09, 48.02) circle (  1.49);

\path[draw=drawColor,line width= 0.4pt,line join=round,line cap=round,fill=fillColor] (382.62, 47.65) circle (  1.49);

\path[draw=drawColor,line width= 0.4pt,line join=round,line cap=round,fill=fillColor] (383.12, 47.87) circle (  1.49);

\path[draw=drawColor,line width= 0.4pt,line join=round,line cap=round,fill=fillColor] (383.61, 47.93) circle (  1.49);

\path[draw=drawColor,line width= 0.4pt,line join=round,line cap=round,fill=fillColor] (384.14, 48.05) circle (  1.49);

\path[draw=drawColor,line width= 0.4pt,line join=round,line cap=round,fill=fillColor] (384.63, 47.97) circle (  1.49);

\path[draw=drawColor,line width= 0.4pt,line join=round,line cap=round,fill=fillColor] (385.14, 48.28) circle (  1.49);

\path[draw=drawColor,line width= 0.4pt,line join=round,line cap=round,fill=fillColor] (385.64, 48.15) circle (  1.49);

\path[draw=drawColor,line width= 0.4pt,line join=round,line cap=round,fill=fillColor] (386.12, 48.08) circle (  1.49);

\path[draw=drawColor,line width= 0.4pt,line join=round,line cap=round,fill=fillColor] (386.64, 48.11) circle (  1.49);

\path[draw=drawColor,line width= 0.4pt,line join=round,line cap=round,fill=fillColor] (387.10, 48.08) circle (  1.49);

\path[draw=drawColor,line width= 0.4pt,line join=round,line cap=round,fill=fillColor] (387.62, 48.16) circle (  1.49);

\path[draw=drawColor,line width= 0.4pt,line join=round,line cap=round,fill=fillColor] (388.18, 48.27) circle (  1.49);

\path[draw=drawColor,line width= 0.4pt,line join=round,line cap=round,fill=fillColor] (388.69, 48.71) circle (  1.49);

\path[draw=drawColor,line width= 0.4pt,line join=round,line cap=round,fill=fillColor] (389.16, 48.64) circle (  1.49);

\path[draw=drawColor,line width= 0.4pt,line join=round,line cap=round,fill=fillColor] (389.70, 48.54) circle (  1.49);

\path[draw=drawColor,line width= 0.4pt,line join=round,line cap=round,fill=fillColor] (390.23, 48.43) circle (  1.49);

\path[draw=drawColor,line width= 0.4pt,line join=round,line cap=round,fill=fillColor] (390.73, 48.32) circle (  1.49);

\path[draw=drawColor,line width= 0.4pt,line join=round,line cap=round,fill=fillColor] (391.21, 48.23) circle (  1.49);

\path[draw=drawColor,line width= 0.4pt,line join=round,line cap=round,fill=fillColor] (391.72, 48.23) circle (  1.49);

\path[draw=drawColor,line width= 0.4pt,line join=round,line cap=round,fill=fillColor] (392.21, 48.26) circle (  1.49);

\path[draw=drawColor,line width= 0.4pt,line join=round,line cap=round,fill=fillColor] (392.75, 47.99) circle (  1.49);

\path[draw=drawColor,line width= 0.4pt,line join=round,line cap=round,fill=fillColor] (393.27, 48.00) circle (  1.49);

\path[draw=drawColor,line width= 0.4pt,line join=round,line cap=round,fill=fillColor] (393.76, 47.96) circle (  1.49);

\path[draw=drawColor,line width= 0.4pt,line join=round,line cap=round,fill=fillColor] (394.25, 48.02) circle (  1.49);

\path[draw=drawColor,line width= 0.4pt,line join=round,line cap=round,fill=fillColor] (394.76, 48.02) circle (  1.49);

\path[draw=drawColor,line width= 0.4pt,line join=round,line cap=round,fill=fillColor] (395.27, 48.01) circle (  1.49);

\path[draw=drawColor,line width= 0.4pt,line join=round,line cap=round,fill=fillColor] (395.78, 47.97) circle (  1.49);

\path[draw=drawColor,line width= 0.4pt,line join=round,line cap=round,fill=fillColor] (396.28, 48.03) circle (  1.49);

\path[draw=drawColor,line width= 0.4pt,line join=round,line cap=round,fill=fillColor] (396.77, 48.14) circle (  1.49);

\path[draw=drawColor,line width= 0.4pt,line join=round,line cap=round,fill=fillColor] (397.35, 48.20) circle (  1.49);

\path[draw=drawColor,line width= 0.4pt,line join=round,line cap=round,fill=fillColor] (397.87, 48.71) circle (  1.49);

\path[draw=drawColor,line width= 0.4pt,line join=round,line cap=round,fill=fillColor] (398.40, 48.26) circle (  1.49);

\path[draw=drawColor,line width= 0.4pt,line join=round,line cap=round,fill=fillColor] (398.87, 47.84) circle (  1.49);

\path[draw=drawColor,line width= 0.4pt,line join=round,line cap=round,fill=fillColor] (399.38, 48.02) circle (  1.49);

\path[draw=drawColor,line width= 0.4pt,line join=round,line cap=round,fill=fillColor] (399.90, 48.00) circle (  1.49);

\path[draw=drawColor,line width= 0.4pt,line join=round,line cap=round,fill=fillColor] (400.41, 48.29) circle (  1.49);

\path[draw=drawColor,line width= 0.4pt,line join=round,line cap=round,fill=fillColor] (401.10, 47.77) circle (  1.49);

\path[draw=drawColor,line width= 0.4pt,line join=round,line cap=round,fill=fillColor] (401.60, 47.96) circle (  1.49);

\path[draw=drawColor,line width= 0.4pt,line join=round,line cap=round,fill=fillColor] (402.09, 47.84) circle (  1.49);

\path[draw=drawColor,line width= 0.4pt,line join=round,line cap=round,fill=fillColor] (402.59, 47.16) circle (  1.49);

\path[draw=drawColor,line width= 0.4pt,line join=round,line cap=round,fill=fillColor] (403.11, 47.43) circle (  1.49);

\path[draw=drawColor,line width= 0.4pt,line join=round,line cap=round,fill=fillColor] (403.60, 47.74) circle (  1.49);

\path[draw=drawColor,line width= 0.4pt,line join=round,line cap=round,fill=fillColor] (404.16, 48.03) circle (  1.49);

\path[draw=drawColor,line width= 0.4pt,line join=round,line cap=round,fill=fillColor] (404.66, 47.99) circle (  1.49);

\path[draw=drawColor,line width= 0.4pt,line join=round,line cap=round,fill=fillColor] (405.17, 47.85) circle (  1.49);

\path[draw=drawColor,line width= 0.4pt,line join=round,line cap=round,fill=fillColor] (405.66, 47.94) circle (  1.49);

\path[draw=drawColor,line width= 0.4pt,line join=round,line cap=round,fill=fillColor] (406.17, 47.70) circle (  1.49);

\path[draw=drawColor,line width= 0.4pt,line join=round,line cap=round,fill=fillColor] (406.68, 47.61) circle (  1.49);

\path[draw=drawColor,line width= 0.4pt,line join=round,line cap=round,fill=fillColor] (407.32, 47.73) circle (  1.49);

\path[draw=drawColor,line width= 0.4pt,line join=round,line cap=round,fill=fillColor] (407.81, 48.10) circle (  1.49);

\path[draw=drawColor,line width= 0.4pt,line join=round,line cap=round,fill=fillColor] (408.28, 50.04) circle (  1.49);

\path[draw=drawColor,line width= 0.4pt,line join=round,line cap=round,fill=fillColor] (408.87, 48.15) circle (  1.49);

\path[draw=drawColor,line width= 0.4pt,line join=round,line cap=round,fill=fillColor] (409.36, 47.90) circle (  1.49);

\path[draw=drawColor,line width= 0.4pt,line join=round,line cap=round,fill=fillColor] (409.92, 53.79) circle (  1.49);

\path[draw=drawColor,line width= 0.4pt,line join=round,line cap=round,fill=fillColor] (410.38, 47.84) circle (  1.49);

\path[draw=drawColor,line width= 0.4pt,line join=round,line cap=round,fill=fillColor] (410.90, 48.09) circle (  1.49);

\path[draw=drawColor,line width= 0.4pt,line join=round,line cap=round,fill=fillColor] (411.41, 47.96) circle (  1.49);

\path[draw=drawColor,line width= 0.4pt,line join=round,line cap=round,fill=fillColor] (411.98, 47.99) circle (  1.49);

\path[draw=drawColor,line width= 0.4pt,line join=round,line cap=round,fill=fillColor] (412.57, 48.23) circle (  1.49);

\path[draw=drawColor,line width= 0.4pt,line join=round,line cap=round,fill=fillColor] (413.08, 47.85) circle (  1.49);

\path[draw=drawColor,line width= 0.4pt,line join=round,line cap=round,fill=fillColor] (413.55, 47.84) circle (  1.49);

\path[draw=drawColor,line width= 0.4pt,line join=round,line cap=round,fill=fillColor] (414.13, 47.91) circle (  1.49);

\path[draw=drawColor,line width= 0.4pt,line join=round,line cap=round,fill=fillColor] (414.62, 47.85) circle (  1.49);

\path[draw=drawColor,line width= 0.4pt,line join=round,line cap=round,fill=fillColor] (415.09, 47.69) circle (  1.49);

\path[draw=drawColor,line width= 0.4pt,line join=round,line cap=round,fill=fillColor] (415.57, 47.82) circle (  1.49);

\path[draw=drawColor,line width= 0.4pt,line join=round,line cap=round,fill=fillColor] (416.04, 47.94) circle (  1.49);

\path[draw=drawColor,line width= 0.4pt,line join=round,line cap=round,fill=fillColor] (416.53, 47.93) circle (  1.49);

\path[draw=drawColor,line width= 0.4pt,line join=round,line cap=round,fill=fillColor] (417.01, 48.12) circle (  1.49);

\path[draw=drawColor,line width= 0.4pt,line join=round,line cap=round,fill=fillColor] (417.53, 48.21) circle (  1.49);

\path[draw=drawColor,line width= 0.4pt,line join=round,line cap=round,fill=fillColor] (418.14, 48.13) circle (  1.49);

\path[draw=drawColor,line width= 0.4pt,line join=round,line cap=round,fill=fillColor] (418.71, 48.10) circle (  1.49);

\path[draw=drawColor,line width= 0.4pt,line join=round,line cap=round,fill=fillColor] (419.23, 47.88) circle (  1.49);

\path[draw=drawColor,line width= 0.4pt,line join=round,line cap=round,fill=fillColor] (419.74, 47.98) circle (  1.49);

\path[draw=drawColor,line width= 0.4pt,line join=round,line cap=round,fill=fillColor] (420.20, 48.13) circle (  1.49);

\path[draw=drawColor,line width= 0.4pt,line join=round,line cap=round,fill=fillColor] (420.64, 47.99) circle (  1.49);

\path[draw=drawColor,line width= 0.4pt,line join=round,line cap=round,fill=fillColor] (421.12, 48.05) circle (  1.49);

\path[draw=drawColor,line width= 0.4pt,line join=round,line cap=round,fill=fillColor] (421.57, 47.87) circle (  1.49);

\path[draw=drawColor,line width= 0.4pt,line join=round,line cap=round,fill=fillColor] (422.05, 47.89) circle (  1.49);

\path[draw=drawColor,line width= 0.4pt,line join=round,line cap=round,fill=fillColor] (422.59, 48.68) circle (  1.49);

\path[draw=drawColor,line width= 0.4pt,line join=round,line cap=round,fill=fillColor] (423.15, 47.96) circle (  1.49);

\path[draw=drawColor,line width= 0.4pt,line join=round,line cap=round,fill=fillColor] (423.74, 47.92) circle (  1.49);

\path[draw=drawColor,line width= 0.4pt,line join=round,line cap=round,fill=fillColor] (424.19, 47.99) circle (  1.49);

\path[draw=drawColor,line width= 0.4pt,line join=round,line cap=round,fill=fillColor] (424.67, 48.03) circle (  1.49);

\path[draw=drawColor,line width= 0.4pt,line join=round,line cap=round,fill=fillColor] (425.13, 48.21) circle (  1.49);

\path[draw=drawColor,line width= 0.4pt,line join=round,line cap=round,fill=fillColor] (425.62, 47.79) circle (  1.49);

\path[draw=drawColor,line width= 0.4pt,line join=round,line cap=round,fill=fillColor] (426.17, 47.86) circle (  1.49);

\path[draw=drawColor,line width= 0.4pt,line join=round,line cap=round,fill=fillColor] (426.65, 47.86) circle (  1.49);

\path[draw=drawColor,line width= 0.4pt,line join=round,line cap=round,fill=fillColor] (427.12, 47.94) circle (  1.49);

\path[draw=drawColor,line width= 0.4pt,line join=round,line cap=round,fill=fillColor] (427.58, 47.90) circle (  1.49);

\path[draw=drawColor,line width= 0.4pt,line join=round,line cap=round,fill=fillColor] (428.06, 47.89) circle (  1.49);

\path[draw=drawColor,line width= 0.4pt,line join=round,line cap=round,fill=fillColor] (428.53, 47.89) circle (  1.49);

\path[draw=drawColor,line width= 0.4pt,line join=round,line cap=round,fill=fillColor] (429.01, 48.00) circle (  1.49);

\path[draw=drawColor,line width= 0.4pt,line join=round,line cap=round,fill=fillColor] (429.48, 48.33) circle (  1.49);

\path[draw=drawColor,line width= 0.4pt,line join=round,line cap=round,fill=fillColor] (429.97, 48.11) circle (  1.49);

\path[draw=drawColor,line width= 0.4pt,line join=round,line cap=round,fill=fillColor] (430.43, 48.10) circle (  1.49);

\path[draw=drawColor,line width= 0.4pt,line join=round,line cap=round,fill=fillColor] (430.91, 48.29) circle (  1.49);

\path[draw=drawColor,line width= 0.4pt,line join=round,line cap=round,fill=fillColor] (431.46,177.23) circle (  1.49);

\path[draw=drawColor,line width= 0.4pt,line join=round,line cap=round,fill=fillColor] (431.94, 48.51) circle (  1.49);

\path[draw=drawColor,line width= 0.4pt,line join=round,line cap=round,fill=fillColor] (432.41, 48.57) circle (  1.49);

\path[draw=drawColor,line width= 0.4pt,line join=round,line cap=round,fill=fillColor] (432.90, 48.37) circle (  1.49);

\path[draw=drawColor,line width= 0.4pt,line join=round,line cap=round,fill=fillColor] (433.38, 48.37) circle (  1.49);

\path[draw=drawColor,line width= 0.4pt,line join=round,line cap=round,fill=fillColor] (433.84, 48.43) circle (  1.49);

\path[draw=drawColor,line width= 0.4pt,line join=round,line cap=round,fill=fillColor] (434.29,174.93) circle (  1.49);

\path[draw=drawColor,line width= 0.4pt,line join=round,line cap=round,fill=fillColor] (434.77, 48.72) circle (  1.49);

\path[draw=drawColor,line width= 0.4pt,line join=round,line cap=round,fill=fillColor] (435.23, 48.45) circle (  1.49);

\path[draw=drawColor,line width= 0.4pt,line join=round,line cap=round,fill=fillColor] (435.70, 48.44) circle (  1.49);

\path[draw=drawColor,line width= 0.4pt,line join=round,line cap=round,fill=fillColor] (436.16, 48.54) circle (  1.49);

\path[draw=drawColor,line width= 0.4pt,line join=round,line cap=round,fill=fillColor] (436.60, 68.88) circle (  1.49);

\path[draw=drawColor,line width= 0.4pt,line join=round,line cap=round,fill=fillColor] (437.09, 48.47) circle (  1.49);

\path[draw=drawColor,line width= 0.4pt,line join=round,line cap=round,fill=fillColor] (437.55, 49.49) circle (  1.49);

\path[draw=drawColor,line width= 0.4pt,line join=round,line cap=round,fill=fillColor] (438.03, 51.06) circle (  1.49);

\path[draw=drawColor,line width= 0.4pt,line join=round,line cap=round,fill=fillColor] (438.47,126.13) circle (  1.49);

\path[draw=drawColor,line width= 0.4pt,line join=round,line cap=round,fill=fillColor] (438.93,167.07) circle (  1.49);

\path[draw=drawColor,line width= 0.4pt,line join=round,line cap=round,fill=fillColor] (439.42, 96.23) circle (  1.49);

\path[draw=drawColor,line width= 0.4pt,line join=round,line cap=round,fill=fillColor] (439.88,171.55) circle (  1.49);

\path[draw=drawColor,line width= 0.4pt,line join=round,line cap=round,fill=fillColor] (440.32,174.10) circle (  1.49);

\path[draw=drawColor,line width= 0.4pt,line join=round,line cap=round,fill=fillColor] (440.81,129.39) circle (  1.49);

\path[draw=drawColor,line width= 0.4pt,line join=round,line cap=round,fill=fillColor] (441.27, 87.90) circle (  1.49);

\path[draw=drawColor,line width= 0.4pt,line join=round,line cap=round,fill=fillColor] (441.79,173.44) circle (  1.49);

\path[draw=drawColor,line width= 0.4pt,line join=round,line cap=round,fill=fillColor] (442.28,172.69) circle (  1.49);

\path[draw=drawColor,line width= 0.4pt,line join=round,line cap=round,fill=fillColor] (442.79, 48.51) circle (  1.49);

\path[draw=drawColor,line width= 0.4pt,line join=round,line cap=round,fill=fillColor] (443.23,172.75) circle (  1.49);

\path[draw=drawColor,line width= 0.4pt,line join=round,line cap=round,fill=fillColor] (443.67, 48.07) circle (  1.49);

\path[draw=drawColor,line width= 0.4pt,line join=round,line cap=round,fill=fillColor] (444.25, 48.93) circle (  1.49);

\path[draw=drawColor,line width= 0.4pt,line join=round,line cap=round,fill=fillColor] (444.72, 48.11) circle (  1.49);

\path[draw=drawColor,line width= 0.4pt,line join=round,line cap=round,fill=fillColor] (445.23, 48.01) circle (  1.49);

\path[draw=drawColor,line width= 0.4pt,line join=round,line cap=round,fill=fillColor] (445.70,172.41) circle (  1.49);

\path[draw=drawColor,line width= 0.4pt,line join=round,line cap=round,fill=fillColor] (446.18,172.29) circle (  1.49);

\path[draw=drawColor,line width= 0.4pt,line join=round,line cap=round,fill=fillColor] (446.65,172.95) circle (  1.49);

\path[draw=drawColor,line width= 0.4pt,line join=round,line cap=round,fill=fillColor] (447.13,172.33) circle (  1.49);

\path[draw=drawColor,line width= 0.4pt,line join=round,line cap=round,fill=fillColor] (447.55,172.03) circle (  1.49);

\path[draw=drawColor,line width= 0.4pt,line join=round,line cap=round,fill=fillColor] (448.08,171.68) circle (  1.49);

\path[draw=drawColor,line width= 0.4pt,line join=round,line cap=round,fill=fillColor] (448.52,171.41) circle (  1.49);

\path[draw=drawColor,line width= 0.4pt,line join=round,line cap=round,fill=fillColor] (449.01,171.73) circle (  1.49);

\path[draw=drawColor,line width= 0.4pt,line join=round,line cap=round,fill=fillColor] (449.53,171.87) circle (  1.49);

\path[draw=drawColor,line width= 0.4pt,line join=round,line cap=round,fill=fillColor] (450.01,171.61) circle (  1.49);

\path[draw=drawColor,line width= 0.4pt,line join=round,line cap=round,fill=fillColor] (450.50,171.74) circle (  1.49);

\path[draw=drawColor,line width= 0.4pt,line join=round,line cap=round,fill=fillColor] (450.96,163.51) circle (  1.49);

\path[draw=drawColor,line width= 0.4pt,line join=round,line cap=round,fill=fillColor] (451.47,170.80) circle (  1.49);

\path[draw=drawColor,line width= 0.4pt,line join=round,line cap=round,fill=fillColor] (451.92,170.61) circle (  1.49);

\path[draw=drawColor,line width= 0.4pt,line join=round,line cap=round,fill=fillColor] (452.37,170.66) circle (  1.49);

\path[draw=drawColor,line width= 0.4pt,line join=round,line cap=round,fill=fillColor] (452.91,170.61) circle (  1.49);

\path[draw=drawColor,line width= 0.4pt,line join=round,line cap=round,fill=fillColor] (453.35,171.00) circle (  1.49);

\path[draw=drawColor,line width= 0.4pt,line join=round,line cap=round,fill=fillColor] (453.81,170.43) circle (  1.49);

\path[draw=drawColor,line width= 0.4pt,line join=round,line cap=round,fill=fillColor] (454.25,172.41) circle (  1.49);

\path[draw=drawColor,line width= 0.4pt,line join=round,line cap=round,fill=fillColor] (454.66,170.45) circle (  1.49);

\path[draw=drawColor,line width= 0.4pt,line join=round,line cap=round,fill=fillColor] (455.16,171.06) circle (  1.49);

\path[draw=drawColor,line width= 0.4pt,line join=round,line cap=round,fill=fillColor] (455.66,169.60) circle (  1.49);

\path[draw=drawColor,line width= 0.4pt,line join=round,line cap=round,fill=fillColor] (456.15,166.59) circle (  1.49);

\path[draw=drawColor,line width= 0.4pt,line join=round,line cap=round,fill=fillColor] (456.61, 95.81) circle (  1.49);

\path[draw=drawColor,line width= 0.4pt,line join=round,line cap=round,fill=fillColor] (457.06, 49.10) circle (  1.49);

\path[draw=drawColor,line width= 0.4pt,line join=round,line cap=round,fill=fillColor] (457.59,171.32) circle (  1.49);

\path[draw=drawColor,line width= 0.4pt,line join=round,line cap=round,fill=fillColor] (458.03,169.70) circle (  1.49);

\path[draw=drawColor,line width= 0.4pt,line join=round,line cap=round,fill=fillColor] (458.47,169.76) circle (  1.49);

\path[draw=drawColor,line width= 0.4pt,line join=round,line cap=round,fill=fillColor] (459.00,157.81) circle (  1.49);

\path[draw=drawColor,line width= 0.4pt,line join=round,line cap=round,fill=fillColor] (459.49,170.08) circle (  1.49);

\path[draw=drawColor,line width= 0.4pt,line join=round,line cap=round,fill=fillColor] (460.03,169.50) circle (  1.49);

\path[draw=drawColor,line width= 0.4pt,line join=round,line cap=round,fill=fillColor] (460.47,169.52) circle (  1.49);

\path[draw=drawColor,line width= 0.4pt,line join=round,line cap=round,fill=fillColor] (460.93, 73.52) circle (  1.49);

\path[draw=drawColor,line width= 0.4pt,line join=round,line cap=round,fill=fillColor] (461.40,125.49) circle (  1.49);

\path[draw=drawColor,line width= 0.4pt,line join=round,line cap=round,fill=fillColor] (461.97,169.11) circle (  1.49);

\path[draw=drawColor,line width= 0.4pt,line join=round,line cap=round,fill=fillColor] (462.55,168.35) circle (  1.49);

\path[draw=drawColor,line width= 0.4pt,line join=round,line cap=round,fill=fillColor] (463.04,167.51) circle (  1.49);

\path[draw=drawColor,line width= 0.4pt,line join=round,line cap=round,fill=fillColor] (463.60,168.34) circle (  1.49);

\path[draw=drawColor,line width= 0.4pt,line join=round,line cap=round,fill=fillColor] (464.05,168.13) circle (  1.49);

\path[draw=drawColor,line width= 0.4pt,line join=round,line cap=round,fill=fillColor] (464.51,167.74) circle (  1.49);

\path[draw=drawColor,line width= 0.4pt,line join=round,line cap=round,fill=fillColor] (464.95,167.29) circle (  1.49);

\path[draw=drawColor,line width= 0.4pt,line join=round,line cap=round,fill=fillColor] (465.45,167.30) circle (  1.49);

\path[draw=drawColor,line width= 0.4pt,line join=round,line cap=round,fill=fillColor] (465.87,167.01) circle (  1.49);

\path[draw=drawColor,line width= 0.4pt,line join=round,line cap=round,fill=fillColor] (466.31,167.05) circle (  1.49);

\path[draw=drawColor,line width= 0.4pt,line join=round,line cap=round,fill=fillColor] (466.90,166.63) circle (  1.49);

\path[draw=drawColor,line width= 0.4pt,line join=round,line cap=round,fill=fillColor] (467.52,166.84) circle (  1.49);

\path[draw=drawColor,line width= 0.4pt,line join=round,line cap=round,fill=fillColor] (467.95,166.80) circle (  1.49);

\path[draw=drawColor,line width= 0.4pt,line join=round,line cap=round,fill=fillColor] (468.36,166.80) circle (  1.49);

\path[draw=drawColor,line width= 0.4pt,line join=round,line cap=round,fill=fillColor] (468.78,166.49) circle (  1.49);

\path[draw=drawColor,line width= 0.4pt,line join=round,line cap=round,fill=fillColor] (469.21,166.33) circle (  1.49);
\definecolor{drawColor}{RGB}{0,0,0}
\definecolor{fillColor}{RGB}{255,255,255}

\path[draw=drawColor,line width= 0.4pt,line join=round,line cap=round,fill=fillColor] (397.95,241.56) rectangle (476.52,182.16);
\definecolor{fillColor}{RGB}{0,0,0}

\path[draw=drawColor,line width= 0.4pt,line join=round,line cap=round,fill=fillColor] (406.86,232.65) rectangle (413.98,226.71);
\definecolor{fillColor}{RGB}{255,0,0}

\path[draw=drawColor,line width= 0.4pt,line join=round,line cap=round,fill=fillColor] (406.86,220.77) rectangle (413.98,214.83);
\definecolor{fillColor}{RGB}{0,255,0}

\path[draw=drawColor,line width= 0.4pt,line join=round,line cap=round,fill=fillColor] (406.86,208.89) rectangle (413.98,202.95);
\definecolor{fillColor}{RGB}{0,0,255}

\path[draw=drawColor,line width= 0.4pt,line join=round,line cap=round,fill=fillColor] (406.86,197.01) rectangle (413.98,191.07);

\node[text=drawColor,anchor=base west,inner sep=0pt, outer sep=0pt, scale=  0.99] at (422.89,226.27) {Incident};

\node[text=drawColor,anchor=base west,inner sep=0pt, outer sep=0pt, scale=  0.99] at (422.89,214.39) {F. excelsior};

\node[text=drawColor,anchor=base west,inner sep=0pt, outer sep=0pt, scale=  0.99] at (422.89,202.51) {A. cordata};

\node[text=drawColor,anchor=base west,inner sep=0pt, outer sep=0pt, scale=  0.99] at (422.89,190.63) {M. alba};
\end{scope}
\begin{scope}
\path[clip] (524.04, 47.52) rectangle (714.78,241.56);
\definecolor{drawColor}{RGB}{255,0,0}
\definecolor{fillColor}{RGB}{255,0,0}

\path[draw=drawColor,line width= 0.4pt,line join=round,line cap=round,fill=fillColor] (531.10,177.22) circle (  1.49);
\definecolor{drawColor}{RGB}{0,0,0}
\definecolor{fillColor}{RGB}{0,0,0}

\path[draw=drawColor,line width= 0.4pt,line join=round,line cap=round,fill=fillColor] (531.12,207.11) circle (  1.49);
\definecolor{drawColor}{RGB}{255,0,0}
\definecolor{fillColor}{RGB}{255,0,0}

\path[draw=drawColor,line width= 0.4pt,line join=round,line cap=round,fill=fillColor] (531.56,178.21) circle (  1.49);
\definecolor{drawColor}{RGB}{0,0,0}
\definecolor{fillColor}{RGB}{0,0,0}

\path[draw=drawColor,line width= 0.4pt,line join=round,line cap=round,fill=fillColor] (531.58,209.74) circle (  1.49);
\definecolor{drawColor}{RGB}{255,0,0}
\definecolor{fillColor}{RGB}{255,0,0}

\path[draw=drawColor,line width= 0.4pt,line join=round,line cap=round,fill=fillColor] (532.00,176.24) circle (  1.49);
\definecolor{drawColor}{RGB}{0,0,0}
\definecolor{fillColor}{RGB}{0,0,0}

\path[draw=drawColor,line width= 0.4pt,line join=round,line cap=round,fill=fillColor] (532.02,213.35) circle (  1.49);
\definecolor{drawColor}{RGB}{255,0,0}
\definecolor{fillColor}{RGB}{255,0,0}

\path[draw=drawColor,line width= 0.4pt,line join=round,line cap=round,fill=fillColor] (532.45,178.86) circle (  1.49);
\definecolor{drawColor}{RGB}{0,0,0}
\definecolor{fillColor}{RGB}{0,0,0}

\path[draw=drawColor,line width= 0.4pt,line join=round,line cap=round,fill=fillColor] (532.46,210.40) circle (  1.49);
\definecolor{drawColor}{RGB}{255,0,0}
\definecolor{fillColor}{RGB}{255,0,0}

\path[draw=drawColor,line width= 0.4pt,line join=round,line cap=round,fill=fillColor] (532.91,181.16) circle (  1.49);
\definecolor{drawColor}{RGB}{0,0,0}
\definecolor{fillColor}{RGB}{0,0,0}

\path[draw=drawColor,line width= 0.4pt,line join=round,line cap=round,fill=fillColor] (532.92,210.07) circle (  1.49);
\definecolor{drawColor}{RGB}{255,0,0}
\definecolor{fillColor}{RGB}{255,0,0}

\path[draw=drawColor,line width= 0.4pt,line join=round,line cap=round,fill=fillColor] (533.36,181.16) circle (  1.49);
\definecolor{drawColor}{RGB}{0,0,0}
\definecolor{fillColor}{RGB}{0,0,0}

\path[draw=drawColor,line width= 0.4pt,line join=round,line cap=round,fill=fillColor] (533.38,208.43) circle (  1.49);
\definecolor{drawColor}{RGB}{255,0,0}
\definecolor{fillColor}{RGB}{255,0,0}

\path[draw=drawColor,line width= 0.4pt,line join=round,line cap=round,fill=fillColor] (533.81,179.52) circle (  1.49);
\definecolor{drawColor}{RGB}{0,0,0}
\definecolor{fillColor}{RGB}{0,0,0}

\path[draw=drawColor,line width= 0.4pt,line join=round,line cap=round,fill=fillColor] (533.82,207.11) circle (  1.49);
\definecolor{drawColor}{RGB}{255,0,0}
\definecolor{fillColor}{RGB}{255,0,0}

\path[draw=drawColor,line width= 0.4pt,line join=round,line cap=round,fill=fillColor] (534.28,179.19) circle (  1.49);
\definecolor{drawColor}{RGB}{0,0,0}
\definecolor{fillColor}{RGB}{0,0,0}

\path[draw=drawColor,line width= 0.4pt,line join=round,line cap=round,fill=fillColor] (534.30,206.13) circle (  1.49);
\definecolor{drawColor}{RGB}{255,0,0}
\definecolor{fillColor}{RGB}{255,0,0}

\path[draw=drawColor,line width= 0.4pt,line join=round,line cap=round,fill=fillColor] (534.74,177.88) circle (  1.49);
\definecolor{drawColor}{RGB}{0,0,0}
\definecolor{fillColor}{RGB}{0,0,0}

\path[draw=drawColor,line width= 0.4pt,line join=round,line cap=round,fill=fillColor] (534.75,207.11) circle (  1.49);
\definecolor{drawColor}{RGB}{255,0,0}
\definecolor{fillColor}{RGB}{255,0,0}

\path[draw=drawColor,line width= 0.4pt,line join=round,line cap=round,fill=fillColor] (535.18,177.88) circle (  1.49);
\definecolor{drawColor}{RGB}{0,0,0}
\definecolor{fillColor}{RGB}{0,0,0}

\path[draw=drawColor,line width= 0.4pt,line join=round,line cap=round,fill=fillColor] (535.21,207.11) circle (  1.49);
\definecolor{drawColor}{RGB}{255,0,0}
\definecolor{fillColor}{RGB}{255,0,0}

\path[draw=drawColor,line width= 0.4pt,line join=round,line cap=round,fill=fillColor] (535.64,177.88) circle (  1.49);
\definecolor{drawColor}{RGB}{0,0,0}
\definecolor{fillColor}{RGB}{0,0,0}

\path[draw=drawColor,line width= 0.4pt,line join=round,line cap=round,fill=fillColor] (535.66,207.44) circle (  1.49);
\definecolor{drawColor}{RGB}{255,0,0}
\definecolor{fillColor}{RGB}{255,0,0}

\path[draw=drawColor,line width= 0.4pt,line join=round,line cap=round,fill=fillColor] (536.08,177.88) circle (  1.49);
\definecolor{drawColor}{RGB}{0,0,0}
\definecolor{fillColor}{RGB}{0,0,0}

\path[draw=drawColor,line width= 0.4pt,line join=round,line cap=round,fill=fillColor] (536.10,207.44) circle (  1.49);
\definecolor{drawColor}{RGB}{255,0,0}
\definecolor{fillColor}{RGB}{255,0,0}

\path[draw=drawColor,line width= 0.4pt,line join=round,line cap=round,fill=fillColor] (536.56,178.21) circle (  1.49);
\definecolor{drawColor}{RGB}{0,0,0}
\definecolor{fillColor}{RGB}{0,0,0}

\path[draw=drawColor,line width= 0.4pt,line join=round,line cap=round,fill=fillColor] (536.57,207.77) circle (  1.49);
\definecolor{drawColor}{RGB}{255,0,0}
\definecolor{fillColor}{RGB}{255,0,0}

\path[draw=drawColor,line width= 0.4pt,line join=round,line cap=round,fill=fillColor] (537.01,178.86) circle (  1.49);
\definecolor{drawColor}{RGB}{0,0,0}
\definecolor{fillColor}{RGB}{0,0,0}

\path[draw=drawColor,line width= 0.4pt,line join=round,line cap=round,fill=fillColor] (537.03,207.77) circle (  1.49);
\definecolor{drawColor}{RGB}{255,0,0}
\definecolor{fillColor}{RGB}{255,0,0}

\path[draw=drawColor,line width= 0.4pt,line join=round,line cap=round,fill=fillColor] (537.47,178.86) circle (  1.49);
\definecolor{drawColor}{RGB}{0,0,0}
\definecolor{fillColor}{RGB}{0,0,0}

\path[draw=drawColor,line width= 0.4pt,line join=round,line cap=round,fill=fillColor] (537.49,208.75) circle (  1.49);
\definecolor{drawColor}{RGB}{255,0,0}
\definecolor{fillColor}{RGB}{255,0,0}

\path[draw=drawColor,line width= 0.4pt,line join=round,line cap=round,fill=fillColor] (537.93,178.86) circle (  1.49);
\definecolor{drawColor}{RGB}{0,0,0}
\definecolor{fillColor}{RGB}{0,0,0}

\path[draw=drawColor,line width= 0.4pt,line join=round,line cap=round,fill=fillColor] (537.95,206.78) circle (  1.49);
\definecolor{drawColor}{RGB}{255,0,0}
\definecolor{fillColor}{RGB}{255,0,0}

\path[draw=drawColor,line width= 0.4pt,line join=round,line cap=round,fill=fillColor] (538.44,178.54) circle (  1.49);
\definecolor{drawColor}{RGB}{0,0,0}
\definecolor{fillColor}{RGB}{0,0,0}

\path[draw=drawColor,line width= 0.4pt,line join=round,line cap=round,fill=fillColor] (538.45,210.40) circle (  1.49);
\definecolor{drawColor}{RGB}{255,0,0}
\definecolor{fillColor}{RGB}{255,0,0}

\path[draw=drawColor,line width= 0.4pt,line join=round,line cap=round,fill=fillColor] (538.90,178.54) circle (  1.49);
\definecolor{drawColor}{RGB}{0,0,0}
\definecolor{fillColor}{RGB}{0,0,0}

\path[draw=drawColor,line width= 0.4pt,line join=round,line cap=round,fill=fillColor] (538.91,210.40) circle (  1.49);
\definecolor{drawColor}{RGB}{255,0,0}
\definecolor{fillColor}{RGB}{255,0,0}

\path[draw=drawColor,line width= 0.4pt,line join=round,line cap=round,fill=fillColor] (539.35,178.54) circle (  1.49);
\definecolor{drawColor}{RGB}{0,0,0}
\definecolor{fillColor}{RGB}{0,0,0}

\path[draw=drawColor,line width= 0.4pt,line join=round,line cap=round,fill=fillColor] (539.37,212.04) circle (  1.49);
\definecolor{drawColor}{RGB}{255,0,0}
\definecolor{fillColor}{RGB}{255,0,0}

\path[draw=drawColor,line width= 0.4pt,line join=round,line cap=round,fill=fillColor] (539.80,179.52) circle (  1.49);
\definecolor{drawColor}{RGB}{0,0,0}
\definecolor{fillColor}{RGB}{0,0,0}

\path[draw=drawColor,line width= 0.4pt,line join=round,line cap=round,fill=fillColor] (539.81,210.72) circle (  1.49);
\definecolor{drawColor}{RGB}{255,0,0}
\definecolor{fillColor}{RGB}{255,0,0}

\path[draw=drawColor,line width= 0.4pt,line join=round,line cap=round,fill=fillColor] (540.26,180.18) circle (  1.49);
\definecolor{drawColor}{RGB}{0,0,0}
\definecolor{fillColor}{RGB}{0,0,0}

\path[draw=drawColor,line width= 0.4pt,line join=round,line cap=round,fill=fillColor] (540.27,209.74) circle (  1.49);
\definecolor{drawColor}{RGB}{255,0,0}
\definecolor{fillColor}{RGB}{255,0,0}

\path[draw=drawColor,line width= 0.4pt,line join=round,line cap=round,fill=fillColor] (540.71,178.86) circle (  1.49);
\definecolor{drawColor}{RGB}{0,0,0}
\definecolor{fillColor}{RGB}{0,0,0}

\path[draw=drawColor,line width= 0.4pt,line join=round,line cap=round,fill=fillColor] (540.73,208.75) circle (  1.49);
\definecolor{drawColor}{RGB}{255,0,0}
\definecolor{fillColor}{RGB}{255,0,0}

\path[draw=drawColor,line width= 0.4pt,line join=round,line cap=round,fill=fillColor] (541.17,177.55) circle (  1.49);
\definecolor{drawColor}{RGB}{0,0,0}
\definecolor{fillColor}{RGB}{0,0,0}

\path[draw=drawColor,line width= 0.4pt,line join=round,line cap=round,fill=fillColor] (541.19,209.41) circle (  1.49);
\definecolor{drawColor}{RGB}{255,0,0}
\definecolor{fillColor}{RGB}{255,0,0}

\path[draw=drawColor,line width= 0.4pt,line join=round,line cap=round,fill=fillColor] (541.65,177.55) circle (  1.49);
\definecolor{drawColor}{RGB}{0,0,0}
\definecolor{fillColor}{RGB}{0,0,0}

\path[draw=drawColor,line width= 0.4pt,line join=round,line cap=round,fill=fillColor] (541.66,210.40) circle (  1.49);
\definecolor{drawColor}{RGB}{255,0,0}
\definecolor{fillColor}{RGB}{255,0,0}

\path[draw=drawColor,line width= 0.4pt,line join=round,line cap=round,fill=fillColor] (542.10,178.21) circle (  1.49);
\definecolor{drawColor}{RGB}{0,0,0}
\definecolor{fillColor}{RGB}{0,0,0}

\path[draw=drawColor,line width= 0.4pt,line join=round,line cap=round,fill=fillColor] (542.12,210.07) circle (  1.49);
\definecolor{drawColor}{RGB}{255,0,0}
\definecolor{fillColor}{RGB}{255,0,0}

\path[draw=drawColor,line width= 0.4pt,line join=round,line cap=round,fill=fillColor] (542.55,178.86) circle (  1.49);
\definecolor{drawColor}{RGB}{0,0,0}
\definecolor{fillColor}{RGB}{0,0,0}

\path[draw=drawColor,line width= 0.4pt,line join=round,line cap=round,fill=fillColor] (542.56,209.74) circle (  1.49);
\definecolor{drawColor}{RGB}{255,0,0}
\definecolor{fillColor}{RGB}{255,0,0}

\path[draw=drawColor,line width= 0.4pt,line join=round,line cap=round,fill=fillColor] (543.01,178.54) circle (  1.49);
\definecolor{drawColor}{RGB}{0,0,0}
\definecolor{fillColor}{RGB}{0,0,0}

\path[draw=drawColor,line width= 0.4pt,line join=round,line cap=round,fill=fillColor] (543.04,210.07) circle (  1.49);
\definecolor{drawColor}{RGB}{255,0,0}
\definecolor{fillColor}{RGB}{255,0,0}

\path[draw=drawColor,line width= 0.4pt,line join=round,line cap=round,fill=fillColor] (543.46,178.54) circle (  1.49);
\definecolor{drawColor}{RGB}{0,0,0}
\definecolor{fillColor}{RGB}{0,0,0}

\path[draw=drawColor,line width= 0.4pt,line join=round,line cap=round,fill=fillColor] (543.48,208.43) circle (  1.49);
\definecolor{drawColor}{RGB}{255,0,0}
\definecolor{fillColor}{RGB}{255,0,0}

\path[draw=drawColor,line width= 0.4pt,line join=round,line cap=round,fill=fillColor] (543.92,179.52) circle (  1.49);
\definecolor{drawColor}{RGB}{0,0,0}
\definecolor{fillColor}{RGB}{0,0,0}

\path[draw=drawColor,line width= 0.4pt,line join=round,line cap=round,fill=fillColor] (543.94,207.77) circle (  1.49);
\definecolor{drawColor}{RGB}{255,0,0}
\definecolor{fillColor}{RGB}{255,0,0}

\path[draw=drawColor,line width= 0.4pt,line join=round,line cap=round,fill=fillColor] (544.41,179.19) circle (  1.49);
\definecolor{drawColor}{RGB}{0,0,0}
\definecolor{fillColor}{RGB}{0,0,0}

\path[draw=drawColor,line width= 0.4pt,line join=round,line cap=round,fill=fillColor] (544.43,207.11) circle (  1.49);
\definecolor{drawColor}{RGB}{255,0,0}
\definecolor{fillColor}{RGB}{255,0,0}

\path[draw=drawColor,line width= 0.4pt,line join=round,line cap=round,fill=fillColor] (544.87,179.19) circle (  1.49);
\definecolor{drawColor}{RGB}{0,0,0}
\definecolor{fillColor}{RGB}{0,0,0}

\path[draw=drawColor,line width= 0.4pt,line join=round,line cap=round,fill=fillColor] (544.89,207.44) circle (  1.49);
\definecolor{drawColor}{RGB}{255,0,0}
\definecolor{fillColor}{RGB}{255,0,0}

\path[draw=drawColor,line width= 0.4pt,line join=round,line cap=round,fill=fillColor] (545.33,178.54) circle (  1.49);
\definecolor{drawColor}{RGB}{0,0,0}
\definecolor{fillColor}{RGB}{0,0,0}

\path[draw=drawColor,line width= 0.4pt,line join=round,line cap=round,fill=fillColor] (545.35,206.45) circle (  1.49);
\definecolor{drawColor}{RGB}{255,0,0}
\definecolor{fillColor}{RGB}{255,0,0}

\path[draw=drawColor,line width= 0.4pt,line join=round,line cap=round,fill=fillColor] (545.79,177.55) circle (  1.49);
\definecolor{drawColor}{RGB}{0,0,0}
\definecolor{fillColor}{RGB}{0,0,0}

\path[draw=drawColor,line width= 0.4pt,line join=round,line cap=round,fill=fillColor] (545.80,204.81) circle (  1.49);
\definecolor{drawColor}{RGB}{255,0,0}
\definecolor{fillColor}{RGB}{255,0,0}

\path[draw=drawColor,line width= 0.4pt,line join=round,line cap=round,fill=fillColor] (546.25,177.88) circle (  1.49);
\definecolor{drawColor}{RGB}{0,0,0}
\definecolor{fillColor}{RGB}{0,0,0}

\path[draw=drawColor,line width= 0.4pt,line join=round,line cap=round,fill=fillColor] (546.26,207.11) circle (  1.49);
\definecolor{drawColor}{RGB}{255,0,0}
\definecolor{fillColor}{RGB}{255,0,0}

\path[draw=drawColor,line width= 0.4pt,line join=round,line cap=round,fill=fillColor] (546.70,176.56) circle (  1.49);
\definecolor{drawColor}{RGB}{0,0,0}
\definecolor{fillColor}{RGB}{0,0,0}

\path[draw=drawColor,line width= 0.4pt,line join=round,line cap=round,fill=fillColor] (546.72,208.43) circle (  1.49);
\definecolor{drawColor}{RGB}{255,0,0}
\definecolor{fillColor}{RGB}{255,0,0}

\path[draw=drawColor,line width= 0.4pt,line join=round,line cap=round,fill=fillColor] (547.16,175.91) circle (  1.49);
\definecolor{drawColor}{RGB}{0,0,0}
\definecolor{fillColor}{RGB}{0,0,0}

\path[draw=drawColor,line width= 0.4pt,line join=round,line cap=round,fill=fillColor] (547.18,209.41) circle (  1.49);
\definecolor{drawColor}{RGB}{255,0,0}
\definecolor{fillColor}{RGB}{255,0,0}

\path[draw=drawColor,line width= 0.4pt,line join=round,line cap=round,fill=fillColor] (547.62,175.91) circle (  1.49);
\definecolor{drawColor}{RGB}{0,0,0}
\definecolor{fillColor}{RGB}{0,0,0}

\path[draw=drawColor,line width= 0.4pt,line join=round,line cap=round,fill=fillColor] (547.64,211.05) circle (  1.49);
\definecolor{drawColor}{RGB}{255,0,0}
\definecolor{fillColor}{RGB}{255,0,0}

\path[draw=drawColor,line width= 0.4pt,line join=round,line cap=round,fill=fillColor] (548.08,175.58) circle (  1.49);
\definecolor{drawColor}{RGB}{0,0,0}
\definecolor{fillColor}{RGB}{0,0,0}

\path[draw=drawColor,line width= 0.4pt,line join=round,line cap=round,fill=fillColor] (548.10,210.07) circle (  1.49);
\definecolor{drawColor}{RGB}{255,0,0}
\definecolor{fillColor}{RGB}{255,0,0}

\path[draw=drawColor,line width= 0.4pt,line join=round,line cap=round,fill=fillColor] (548.54,178.21) circle (  1.49);
\definecolor{drawColor}{RGB}{0,0,0}
\definecolor{fillColor}{RGB}{0,0,0}

\path[draw=drawColor,line width= 0.4pt,line join=round,line cap=round,fill=fillColor] (548.55,210.72) circle (  1.49);
\definecolor{drawColor}{RGB}{255,0,0}
\definecolor{fillColor}{RGB}{255,0,0}

\path[draw=drawColor,line width= 0.4pt,line join=round,line cap=round,fill=fillColor] (548.98,176.89) circle (  1.49);
\definecolor{drawColor}{RGB}{0,0,0}
\definecolor{fillColor}{RGB}{0,0,0}

\path[draw=drawColor,line width= 0.4pt,line join=round,line cap=round,fill=fillColor] (549.01,210.72) circle (  1.49);
\definecolor{drawColor}{RGB}{255,0,0}
\definecolor{fillColor}{RGB}{255,0,0}

\path[draw=drawColor,line width= 0.4pt,line join=round,line cap=round,fill=fillColor] (549.44,177.88) circle (  1.49);
\definecolor{drawColor}{RGB}{0,0,0}
\definecolor{fillColor}{RGB}{0,0,0}

\path[draw=drawColor,line width= 0.4pt,line join=round,line cap=round,fill=fillColor] (549.45,211.05) circle (  1.49);
\definecolor{drawColor}{RGB}{255,0,0}
\definecolor{fillColor}{RGB}{255,0,0}

\path[draw=drawColor,line width= 0.4pt,line join=round,line cap=round,fill=fillColor] (549.88,176.56) circle (  1.49);
\definecolor{drawColor}{RGB}{0,0,0}
\definecolor{fillColor}{RGB}{0,0,0}

\path[draw=drawColor,line width= 0.4pt,line join=round,line cap=round,fill=fillColor] (549.90,213.02) circle (  1.49);
\definecolor{drawColor}{RGB}{255,0,0}
\definecolor{fillColor}{RGB}{255,0,0}

\path[draw=drawColor,line width= 0.4pt,line join=round,line cap=round,fill=fillColor] (550.34,171.64) circle (  1.49);
\definecolor{drawColor}{RGB}{0,0,0}
\definecolor{fillColor}{RGB}{0,0,0}

\path[draw=drawColor,line width= 0.4pt,line join=round,line cap=round,fill=fillColor] (550.36,214.67) circle (  1.49);
\definecolor{drawColor}{RGB}{255,0,0}
\definecolor{fillColor}{RGB}{255,0,0}

\path[draw=drawColor,line width= 0.4pt,line join=round,line cap=round,fill=fillColor] (550.78,174.92) circle (  1.49);
\definecolor{drawColor}{RGB}{0,0,0}
\definecolor{fillColor}{RGB}{0,0,0}

\path[draw=drawColor,line width= 0.4pt,line join=round,line cap=round,fill=fillColor] (550.80,216.64) circle (  1.49);
\definecolor{drawColor}{RGB}{255,0,0}
\definecolor{fillColor}{RGB}{255,0,0}

\path[draw=drawColor,line width= 0.4pt,line join=round,line cap=round,fill=fillColor] (551.22,176.89) circle (  1.49);
\definecolor{drawColor}{RGB}{0,0,0}
\definecolor{fillColor}{RGB}{0,0,0}

\path[draw=drawColor,line width= 0.4pt,line join=round,line cap=round,fill=fillColor] (551.24,213.68) circle (  1.49);
\definecolor{drawColor}{RGB}{255,0,0}
\definecolor{fillColor}{RGB}{255,0,0}

\path[draw=drawColor,line width= 0.4pt,line join=round,line cap=round,fill=fillColor] (551.68,177.22) circle (  1.49);
\definecolor{drawColor}{RGB}{0,0,0}
\definecolor{fillColor}{RGB}{0,0,0}

\path[draw=drawColor,line width= 0.4pt,line join=round,line cap=round,fill=fillColor] (551.70,213.68) circle (  1.49);
\definecolor{drawColor}{RGB}{255,0,0}
\definecolor{fillColor}{RGB}{255,0,0}

\path[draw=drawColor,line width= 0.4pt,line join=round,line cap=round,fill=fillColor] (552.12,177.22) circle (  1.49);
\definecolor{drawColor}{RGB}{0,0,0}
\definecolor{fillColor}{RGB}{0,0,0}

\path[draw=drawColor,line width= 0.4pt,line join=round,line cap=round,fill=fillColor] (552.14,213.02) circle (  1.49);
\definecolor{drawColor}{RGB}{255,0,0}
\definecolor{fillColor}{RGB}{255,0,0}

\path[draw=drawColor,line width= 0.4pt,line join=round,line cap=round,fill=fillColor] (552.58,174.27) circle (  1.49);
\definecolor{drawColor}{RGB}{0,0,0}
\definecolor{fillColor}{RGB}{0,0,0}

\path[draw=drawColor,line width= 0.4pt,line join=round,line cap=round,fill=fillColor] (552.60,216.97) circle (  1.49);
\definecolor{drawColor}{RGB}{255,0,0}
\definecolor{fillColor}{RGB}{255,0,0}

\path[draw=drawColor,line width= 0.4pt,line join=round,line cap=round,fill=fillColor] (553.04,171.64) circle (  1.49);
\definecolor{drawColor}{RGB}{0,0,0}
\definecolor{fillColor}{RGB}{0,0,0}

\path[draw=drawColor,line width= 0.4pt,line join=round,line cap=round,fill=fillColor] (553.06,215.32) circle (  1.49);
\definecolor{drawColor}{RGB}{255,0,0}
\definecolor{fillColor}{RGB}{255,0,0}

\path[draw=drawColor,line width= 0.4pt,line join=round,line cap=round,fill=fillColor] (553.48,171.31) circle (  1.49);
\definecolor{drawColor}{RGB}{0,0,0}
\definecolor{fillColor}{RGB}{0,0,0}

\path[draw=drawColor,line width= 0.4pt,line join=round,line cap=round,fill=fillColor] (553.50,218.61) circle (  1.49);
\definecolor{drawColor}{RGB}{255,0,0}
\definecolor{fillColor}{RGB}{255,0,0}

\path[draw=drawColor,line width= 0.4pt,line join=round,line cap=round,fill=fillColor] (553.96,154.56) circle (  1.49);
\definecolor{drawColor}{RGB}{0,0,0}
\definecolor{fillColor}{RGB}{0,0,0}

\path[draw=drawColor,line width= 0.4pt,line join=round,line cap=round,fill=fillColor] (553.97,215.98) circle (  1.49);
\definecolor{drawColor}{RGB}{255,0,0}
\definecolor{fillColor}{RGB}{255,0,0}

\path[draw=drawColor,line width= 0.4pt,line join=round,line cap=round,fill=fillColor] (554.46,153.57) circle (  1.49);
\definecolor{drawColor}{RGB}{0,0,0}
\definecolor{fillColor}{RGB}{0,0,0}

\path[draw=drawColor,line width= 0.4pt,line join=round,line cap=round,fill=fillColor] (554.50,215.65) circle (  1.49);
\definecolor{drawColor}{RGB}{255,0,0}
\definecolor{fillColor}{RGB}{255,0,0}

\path[draw=drawColor,line width= 0.4pt,line join=round,line cap=round,fill=fillColor] (554.92,160.47) circle (  1.49);
\definecolor{drawColor}{RGB}{0,0,0}
\definecolor{fillColor}{RGB}{0,0,0}

\path[draw=drawColor,line width= 0.4pt,line join=round,line cap=round,fill=fillColor] (554.94,216.97) circle (  1.49);
\definecolor{drawColor}{RGB}{255,0,0}
\definecolor{fillColor}{RGB}{255,0,0}

\path[draw=drawColor,line width= 0.4pt,line join=round,line cap=round,fill=fillColor] (555.36,167.04) circle (  1.49);
\definecolor{drawColor}{RGB}{0,0,0}
\definecolor{fillColor}{RGB}{0,0,0}

\path[draw=drawColor,line width= 0.4pt,line join=round,line cap=round,fill=fillColor] (555.38,219.26) circle (  1.49);
\definecolor{drawColor}{RGB}{255,0,0}
\definecolor{fillColor}{RGB}{255,0,0}

\path[draw=drawColor,line width= 0.4pt,line join=round,line cap=round,fill=fillColor] (555.82,172.62) circle (  1.49);
\definecolor{drawColor}{RGB}{0,0,0}
\definecolor{fillColor}{RGB}{0,0,0}

\path[draw=drawColor,line width= 0.4pt,line join=round,line cap=round,fill=fillColor] (555.84,218.61) circle (  1.49);
\definecolor{drawColor}{RGB}{255,0,0}
\definecolor{fillColor}{RGB}{255,0,0}

\path[draw=drawColor,line width= 0.4pt,line join=round,line cap=round,fill=fillColor] (556.30,180.83) circle (  1.49);
\definecolor{drawColor}{RGB}{0,0,0}
\definecolor{fillColor}{RGB}{0,0,0}

\path[draw=drawColor,line width= 0.4pt,line join=round,line cap=round,fill=fillColor] (556.31,214.34) circle (  1.49);
\definecolor{drawColor}{RGB}{255,0,0}
\definecolor{fillColor}{RGB}{255,0,0}

\path[draw=drawColor,line width= 0.4pt,line join=round,line cap=round,fill=fillColor] (556.77,182.48) circle (  1.49);
\definecolor{drawColor}{RGB}{0,0,0}
\definecolor{fillColor}{RGB}{0,0,0}

\path[draw=drawColor,line width= 0.4pt,line join=round,line cap=round,fill=fillColor] (556.79,214.67) circle (  1.49);
\definecolor{drawColor}{RGB}{255,0,0}
\definecolor{fillColor}{RGB}{255,0,0}

\path[draw=drawColor,line width= 0.4pt,line join=round,line cap=round,fill=fillColor] (557.28,181.49) circle (  1.49);
\definecolor{drawColor}{RGB}{0,0,0}
\definecolor{fillColor}{RGB}{0,0,0}

\path[draw=drawColor,line width= 0.4pt,line join=round,line cap=round,fill=fillColor] (557.30,214.67) circle (  1.49);
\definecolor{drawColor}{RGB}{255,0,0}
\definecolor{fillColor}{RGB}{255,0,0}

\path[draw=drawColor,line width= 0.4pt,line join=round,line cap=round,fill=fillColor] (557.77,177.88) circle (  1.49);
\definecolor{drawColor}{RGB}{0,0,0}
\definecolor{fillColor}{RGB}{0,0,0}

\path[draw=drawColor,line width= 0.4pt,line join=round,line cap=round,fill=fillColor] (557.79,218.28) circle (  1.49);
\definecolor{drawColor}{RGB}{255,0,0}
\definecolor{fillColor}{RGB}{255,0,0}

\path[draw=drawColor,line width= 0.4pt,line join=round,line cap=round,fill=fillColor] (558.25,177.22) circle (  1.49);
\definecolor{drawColor}{RGB}{0,0,0}
\definecolor{fillColor}{RGB}{0,0,0}

\path[draw=drawColor,line width= 0.4pt,line join=round,line cap=round,fill=fillColor] (558.26,218.94) circle (  1.49);
\definecolor{drawColor}{RGB}{255,0,0}
\definecolor{fillColor}{RGB}{255,0,0}

\path[draw=drawColor,line width= 0.4pt,line join=round,line cap=round,fill=fillColor] (558.72,171.97) circle (  1.49);
\definecolor{drawColor}{RGB}{0,0,0}
\definecolor{fillColor}{RGB}{0,0,0}

\path[draw=drawColor,line width= 0.4pt,line join=round,line cap=round,fill=fillColor] (558.74,221.24) circle (  1.49);
\definecolor{drawColor}{RGB}{255,0,0}
\definecolor{fillColor}{RGB}{255,0,0}

\path[draw=drawColor,line width= 0.4pt,line join=round,line cap=round,fill=fillColor] (559.21,150.29) circle (  1.49);
\definecolor{drawColor}{RGB}{0,0,0}
\definecolor{fillColor}{RGB}{0,0,0}

\path[draw=drawColor,line width= 0.4pt,line join=round,line cap=round,fill=fillColor] (559.23,216.97) circle (  1.49);
\definecolor{drawColor}{RGB}{255,0,0}
\definecolor{fillColor}{RGB}{255,0,0}

\path[draw=drawColor,line width= 0.4pt,line join=round,line cap=round,fill=fillColor] (559.69,139.45) circle (  1.49);
\definecolor{drawColor}{RGB}{0,0,0}
\definecolor{fillColor}{RGB}{0,0,0}

\path[draw=drawColor,line width= 0.4pt,line join=round,line cap=round,fill=fillColor] (559.70,217.29) circle (  1.49);
\definecolor{drawColor}{RGB}{255,0,0}
\definecolor{fillColor}{RGB}{255,0,0}

\path[draw=drawColor,line width= 0.4pt,line join=round,line cap=round,fill=fillColor] (560.14,142.73) circle (  1.49);
\definecolor{drawColor}{RGB}{0,0,0}
\definecolor{fillColor}{RGB}{0,0,0}

\path[draw=drawColor,line width= 0.4pt,line join=round,line cap=round,fill=fillColor] (560.16,218.61) circle (  1.49);
\definecolor{drawColor}{RGB}{255,0,0}
\definecolor{fillColor}{RGB}{255,0,0}

\path[draw=drawColor,line width= 0.4pt,line join=round,line cap=round,fill=fillColor] (560.62,138.79) circle (  1.49);
\definecolor{drawColor}{RGB}{0,0,0}
\definecolor{fillColor}{RGB}{0,0,0}

\path[draw=drawColor,line width= 0.4pt,line join=round,line cap=round,fill=fillColor] (560.64,217.62) circle (  1.49);
\definecolor{drawColor}{RGB}{255,0,0}
\definecolor{fillColor}{RGB}{255,0,0}

\path[draw=drawColor,line width= 0.4pt,line join=round,line cap=round,fill=fillColor] (561.09,142.08) circle (  1.49);
\definecolor{drawColor}{RGB}{0,0,0}
\definecolor{fillColor}{RGB}{0,0,0}

\path[draw=drawColor,line width= 0.4pt,line join=round,line cap=round,fill=fillColor] (561.11,219.26) circle (  1.49);
\definecolor{drawColor}{RGB}{255,0,0}
\definecolor{fillColor}{RGB}{255,0,0}

\path[draw=drawColor,line width= 0.4pt,line join=round,line cap=round,fill=fillColor] (561.57,136.16) circle (  1.49);
\definecolor{drawColor}{RGB}{0,0,0}
\definecolor{fillColor}{RGB}{0,0,0}

\path[draw=drawColor,line width= 0.4pt,line join=round,line cap=round,fill=fillColor] (561.58,219.26) circle (  1.49);
\definecolor{drawColor}{RGB}{255,0,0}
\definecolor{fillColor}{RGB}{255,0,0}

\path[draw=drawColor,line width= 0.4pt,line join=round,line cap=round,fill=fillColor] (562.09,137.48) circle (  1.49);
\definecolor{drawColor}{RGB}{0,0,0}
\definecolor{fillColor}{RGB}{0,0,0}

\path[draw=drawColor,line width= 0.4pt,line join=round,line cap=round,fill=fillColor] (562.11,219.59) circle (  1.49);
\definecolor{drawColor}{RGB}{255,0,0}
\definecolor{fillColor}{RGB}{255,0,0}

\path[draw=drawColor,line width= 0.4pt,line join=round,line cap=round,fill=fillColor] (562.58,142.73) circle (  1.49);
\definecolor{drawColor}{RGB}{0,0,0}
\definecolor{fillColor}{RGB}{0,0,0}

\path[draw=drawColor,line width= 0.4pt,line join=round,line cap=round,fill=fillColor] (562.60,219.92) circle (  1.49);
\definecolor{drawColor}{RGB}{255,0,0}
\definecolor{fillColor}{RGB}{255,0,0}

\path[draw=drawColor,line width= 0.4pt,line join=round,line cap=round,fill=fillColor] (563.06,129.92) circle (  1.49);
\definecolor{drawColor}{RGB}{0,0,0}
\definecolor{fillColor}{RGB}{0,0,0}

\path[draw=drawColor,line width= 0.4pt,line join=round,line cap=round,fill=fillColor] (563.07,220.58) circle (  1.49);
\definecolor{drawColor}{RGB}{255,0,0}
\definecolor{fillColor}{RGB}{255,0,0}

\path[draw=drawColor,line width= 0.4pt,line join=round,line cap=round,fill=fillColor] (563.61,127.62) circle (  1.49);
\definecolor{drawColor}{RGB}{0,0,0}
\definecolor{fillColor}{RGB}{0,0,0}

\path[draw=drawColor,line width= 0.4pt,line join=round,line cap=round,fill=fillColor] (563.63,218.61) circle (  1.49);
\definecolor{drawColor}{RGB}{255,0,0}
\definecolor{fillColor}{RGB}{255,0,0}

\path[draw=drawColor,line width= 0.4pt,line join=round,line cap=round,fill=fillColor] (564.11,128.94) circle (  1.49);
\definecolor{drawColor}{RGB}{0,0,0}
\definecolor{fillColor}{RGB}{0,0,0}

\path[draw=drawColor,line width= 0.4pt,line join=round,line cap=round,fill=fillColor] (564.12,219.92) circle (  1.49);
\definecolor{drawColor}{RGB}{255,0,0}
\definecolor{fillColor}{RGB}{255,0,0}

\path[draw=drawColor,line width= 0.4pt,line join=round,line cap=round,fill=fillColor] (564.60,131.24) circle (  1.49);
\definecolor{drawColor}{RGB}{0,0,0}
\definecolor{fillColor}{RGB}{0,0,0}

\path[draw=drawColor,line width= 0.4pt,line join=round,line cap=round,fill=fillColor] (564.61,218.28) circle (  1.49);
\definecolor{drawColor}{RGB}{255,0,0}
\definecolor{fillColor}{RGB}{255,0,0}

\path[draw=drawColor,line width= 0.4pt,line join=round,line cap=round,fill=fillColor] (565.12,127.95) circle (  1.49);
\definecolor{drawColor}{RGB}{0,0,0}
\definecolor{fillColor}{RGB}{0,0,0}

\path[draw=drawColor,line width= 0.4pt,line join=round,line cap=round,fill=fillColor] (565.14,218.61) circle (  1.49);
\definecolor{drawColor}{RGB}{255,0,0}
\definecolor{fillColor}{RGB}{255,0,0}

\path[draw=drawColor,line width= 0.4pt,line join=round,line cap=round,fill=fillColor] (565.60,160.47) circle (  1.49);
\definecolor{drawColor}{RGB}{0,0,0}
\definecolor{fillColor}{RGB}{0,0,0}

\path[draw=drawColor,line width= 0.4pt,line join=round,line cap=round,fill=fillColor] (565.61,224.19) circle (  1.49);
\definecolor{drawColor}{RGB}{255,0,0}
\definecolor{fillColor}{RGB}{255,0,0}

\path[draw=drawColor,line width= 0.4pt,line join=round,line cap=round,fill=fillColor] (566.07,149.63) circle (  1.49);
\definecolor{drawColor}{RGB}{0,0,0}
\definecolor{fillColor}{RGB}{0,0,0}

\path[draw=drawColor,line width= 0.4pt,line join=round,line cap=round,fill=fillColor] (566.10,221.24) circle (  1.49);
\definecolor{drawColor}{RGB}{255,0,0}
\definecolor{fillColor}{RGB}{255,0,0}

\path[draw=drawColor,line width= 0.4pt,line join=round,line cap=round,fill=fillColor] (566.56,142.73) circle (  1.49);
\definecolor{drawColor}{RGB}{0,0,0}
\definecolor{fillColor}{RGB}{0,0,0}

\path[draw=drawColor,line width= 0.4pt,line join=round,line cap=round,fill=fillColor] (566.59,220.58) circle (  1.49);
\definecolor{drawColor}{RGB}{255,0,0}
\definecolor{fillColor}{RGB}{255,0,0}

\path[draw=drawColor,line width= 0.4pt,line join=round,line cap=round,fill=fillColor] (567.05,152.92) circle (  1.49);
\definecolor{drawColor}{RGB}{0,0,0}
\definecolor{fillColor}{RGB}{0,0,0}

\path[draw=drawColor,line width= 0.4pt,line join=round,line cap=round,fill=fillColor] (567.07,220.91) circle (  1.49);
\definecolor{drawColor}{RGB}{255,0,0}
\definecolor{fillColor}{RGB}{255,0,0}

\path[draw=drawColor,line width= 0.4pt,line join=round,line cap=round,fill=fillColor] (567.53,138.14) circle (  1.49);
\definecolor{drawColor}{RGB}{0,0,0}
\definecolor{fillColor}{RGB}{0,0,0}

\path[draw=drawColor,line width= 0.4pt,line join=round,line cap=round,fill=fillColor] (567.54,217.62) circle (  1.49);
\definecolor{drawColor}{RGB}{255,0,0}
\definecolor{fillColor}{RGB}{255,0,0}

\path[draw=drawColor,line width= 0.4pt,line join=round,line cap=round,fill=fillColor] (568.00,141.42) circle (  1.49);
\definecolor{drawColor}{RGB}{0,0,0}
\definecolor{fillColor}{RGB}{0,0,0}

\path[draw=drawColor,line width= 0.4pt,line join=round,line cap=round,fill=fillColor] (568.02,218.28) circle (  1.49);
\definecolor{drawColor}{RGB}{255,0,0}
\definecolor{fillColor}{RGB}{255,0,0}

\path[draw=drawColor,line width= 0.4pt,line join=round,line cap=round,fill=fillColor] (568.46,146.67) circle (  1.49);
\definecolor{drawColor}{RGB}{0,0,0}
\definecolor{fillColor}{RGB}{0,0,0}

\path[draw=drawColor,line width= 0.4pt,line join=round,line cap=round,fill=fillColor] (568.48,219.92) circle (  1.49);
\definecolor{drawColor}{RGB}{255,0,0}
\definecolor{fillColor}{RGB}{255,0,0}

\path[draw=drawColor,line width= 0.4pt,line join=round,line cap=round,fill=fillColor] (568.93,145.03) circle (  1.49);
\definecolor{drawColor}{RGB}{0,0,0}
\definecolor{fillColor}{RGB}{0,0,0}

\path[draw=drawColor,line width= 0.4pt,line join=round,line cap=round,fill=fillColor] (568.95,217.95) circle (  1.49);
\definecolor{drawColor}{RGB}{255,0,0}
\definecolor{fillColor}{RGB}{255,0,0}

\path[draw=drawColor,line width= 0.4pt,line join=round,line cap=round,fill=fillColor] (569.43,172.95) circle (  1.49);
\definecolor{drawColor}{RGB}{0,0,0}
\definecolor{fillColor}{RGB}{0,0,0}

\path[draw=drawColor,line width= 0.4pt,line join=round,line cap=round,fill=fillColor] (569.44,226.16) circle (  1.49);
\definecolor{drawColor}{RGB}{255,0,0}
\definecolor{fillColor}{RGB}{255,0,0}

\path[draw=drawColor,line width= 0.4pt,line join=round,line cap=round,fill=fillColor] (569.90,149.96) circle (  1.49);
\definecolor{drawColor}{RGB}{0,0,0}
\definecolor{fillColor}{RGB}{0,0,0}

\path[draw=drawColor,line width= 0.4pt,line join=round,line cap=round,fill=fillColor] (569.92,218.94) circle (  1.49);
\definecolor{drawColor}{RGB}{255,0,0}
\definecolor{fillColor}{RGB}{255,0,0}

\path[draw=drawColor,line width= 0.4pt,line join=round,line cap=round,fill=fillColor] (570.38,149.30) circle (  1.49);
\definecolor{drawColor}{RGB}{0,0,0}
\definecolor{fillColor}{RGB}{0,0,0}

\path[draw=drawColor,line width= 0.4pt,line join=round,line cap=round,fill=fillColor] (570.39,217.95) circle (  1.49);
\definecolor{drawColor}{RGB}{255,0,0}
\definecolor{fillColor}{RGB}{255,0,0}

\path[draw=drawColor,line width= 0.4pt,line join=round,line cap=round,fill=fillColor] (570.88,148.97) circle (  1.49);
\definecolor{drawColor}{RGB}{0,0,0}
\definecolor{fillColor}{RGB}{0,0,0}

\path[draw=drawColor,line width= 0.4pt,line join=round,line cap=round,fill=fillColor] (570.90,217.95) circle (  1.49);
\definecolor{drawColor}{RGB}{255,0,0}
\definecolor{fillColor}{RGB}{255,0,0}

\path[draw=drawColor,line width= 0.4pt,line join=round,line cap=round,fill=fillColor] (571.36,145.69) circle (  1.49);
\definecolor{drawColor}{RGB}{0,0,0}
\definecolor{fillColor}{RGB}{0,0,0}

\path[draw=drawColor,line width= 0.4pt,line join=round,line cap=round,fill=fillColor] (571.37,218.61) circle (  1.49);
\definecolor{drawColor}{RGB}{255,0,0}
\definecolor{fillColor}{RGB}{255,0,0}

\path[draw=drawColor,line width= 0.4pt,line join=round,line cap=round,fill=fillColor] (571.85,143.06) circle (  1.49);
\definecolor{drawColor}{RGB}{0,0,0}
\definecolor{fillColor}{RGB}{0,0,0}

\path[draw=drawColor,line width= 0.4pt,line join=round,line cap=round,fill=fillColor] (571.88,218.61) circle (  1.49);
\definecolor{drawColor}{RGB}{255,0,0}
\definecolor{fillColor}{RGB}{255,0,0}

\path[draw=drawColor,line width= 0.4pt,line join=round,line cap=round,fill=fillColor] (572.32,145.36) circle (  1.49);
\definecolor{drawColor}{RGB}{0,0,0}
\definecolor{fillColor}{RGB}{0,0,0}

\path[draw=drawColor,line width= 0.4pt,line join=round,line cap=round,fill=fillColor] (572.34,218.94) circle (  1.49);
\definecolor{drawColor}{RGB}{255,0,0}
\definecolor{fillColor}{RGB}{255,0,0}

\path[draw=drawColor,line width= 0.4pt,line join=round,line cap=round,fill=fillColor] (572.81,154.23) circle (  1.49);
\definecolor{drawColor}{RGB}{0,0,0}
\definecolor{fillColor}{RGB}{0,0,0}

\path[draw=drawColor,line width= 0.4pt,line join=round,line cap=round,fill=fillColor] (572.83,219.59) circle (  1.49);
\definecolor{drawColor}{RGB}{255,0,0}
\definecolor{fillColor}{RGB}{255,0,0}

\path[draw=drawColor,line width= 0.4pt,line join=round,line cap=round,fill=fillColor] (573.29,142.41) circle (  1.49);
\definecolor{drawColor}{RGB}{0,0,0}
\definecolor{fillColor}{RGB}{0,0,0}

\path[draw=drawColor,line width= 0.4pt,line join=round,line cap=round,fill=fillColor] (573.31,218.94) circle (  1.49);
\definecolor{drawColor}{RGB}{255,0,0}
\definecolor{fillColor}{RGB}{255,0,0}

\path[draw=drawColor,line width= 0.4pt,line join=round,line cap=round,fill=fillColor] (573.78,141.75) circle (  1.49);
\definecolor{drawColor}{RGB}{0,0,0}
\definecolor{fillColor}{RGB}{0,0,0}

\path[draw=drawColor,line width= 0.4pt,line join=round,line cap=round,fill=fillColor] (573.80,219.59) circle (  1.49);
\definecolor{drawColor}{RGB}{255,0,0}
\definecolor{fillColor}{RGB}{255,0,0}

\path[draw=drawColor,line width= 0.4pt,line join=round,line cap=round,fill=fillColor] (574.24,142.73) circle (  1.49);
\definecolor{drawColor}{RGB}{0,0,0}
\definecolor{fillColor}{RGB}{0,0,0}

\path[draw=drawColor,line width= 0.4pt,line join=round,line cap=round,fill=fillColor] (574.25,219.59) circle (  1.49);
\definecolor{drawColor}{RGB}{255,0,0}
\definecolor{fillColor}{RGB}{255,0,0}

\path[draw=drawColor,line width= 0.4pt,line join=round,line cap=round,fill=fillColor] (574.73,136.49) circle (  1.49);
\definecolor{drawColor}{RGB}{0,0,0}
\definecolor{fillColor}{RGB}{0,0,0}

\path[draw=drawColor,line width= 0.4pt,line join=round,line cap=round,fill=fillColor] (574.75,221.89) circle (  1.49);
\definecolor{drawColor}{RGB}{255,0,0}
\definecolor{fillColor}{RGB}{255,0,0}

\path[draw=drawColor,line width= 0.4pt,line join=round,line cap=round,fill=fillColor] (575.20,133.54) circle (  1.49);
\definecolor{drawColor}{RGB}{0,0,0}
\definecolor{fillColor}{RGB}{0,0,0}

\path[draw=drawColor,line width= 0.4pt,line join=round,line cap=round,fill=fillColor] (575.22,220.91) circle (  1.49);
\definecolor{drawColor}{RGB}{255,0,0}
\definecolor{fillColor}{RGB}{255,0,0}

\path[draw=drawColor,line width= 0.4pt,line join=round,line cap=round,fill=fillColor] (575.68,153.24) circle (  1.49);
\definecolor{drawColor}{RGB}{0,0,0}
\definecolor{fillColor}{RGB}{0,0,0}

\path[draw=drawColor,line width= 0.4pt,line join=round,line cap=round,fill=fillColor] (575.70,225.83) circle (  1.49);
\definecolor{drawColor}{RGB}{255,0,0}
\definecolor{fillColor}{RGB}{255,0,0}

\path[draw=drawColor,line width= 0.4pt,line join=round,line cap=round,fill=fillColor] (576.17,155.54) circle (  1.49);
\definecolor{drawColor}{RGB}{0,0,0}
\definecolor{fillColor}{RGB}{0,0,0}

\path[draw=drawColor,line width= 0.4pt,line join=round,line cap=round,fill=fillColor] (576.19,223.86) circle (  1.49);
\definecolor{drawColor}{RGB}{255,0,0}
\definecolor{fillColor}{RGB}{255,0,0}

\path[draw=drawColor,line width= 0.4pt,line join=round,line cap=round,fill=fillColor] (576.64,139.78) circle (  1.49);
\definecolor{drawColor}{RGB}{0,0,0}
\definecolor{fillColor}{RGB}{0,0,0}

\path[draw=drawColor,line width= 0.4pt,line join=round,line cap=round,fill=fillColor] (576.66,220.58) circle (  1.49);
\definecolor{drawColor}{RGB}{255,0,0}
\definecolor{fillColor}{RGB}{255,0,0}

\path[draw=drawColor,line width= 0.4pt,line join=round,line cap=round,fill=fillColor] (577.12,144.38) circle (  1.49);
\definecolor{drawColor}{RGB}{0,0,0}
\definecolor{fillColor}{RGB}{0,0,0}

\path[draw=drawColor,line width= 0.4pt,line join=round,line cap=round,fill=fillColor] (577.14,225.18) circle (  1.49);
\definecolor{drawColor}{RGB}{255,0,0}
\definecolor{fillColor}{RGB}{255,0,0}

\path[draw=drawColor,line width= 0.4pt,line join=round,line cap=round,fill=fillColor] (577.68,187.08) circle (  1.49);
\definecolor{drawColor}{RGB}{0,0,0}
\definecolor{fillColor}{RGB}{0,0,0}

\path[draw=drawColor,line width= 0.4pt,line join=round,line cap=round,fill=fillColor] (577.69,234.37) circle (  1.49);
\definecolor{drawColor}{RGB}{255,0,0}
\definecolor{fillColor}{RGB}{255,0,0}

\path[draw=drawColor,line width= 0.4pt,line join=round,line cap=round,fill=fillColor] (578.22,172.62) circle (  1.49);
\definecolor{drawColor}{RGB}{0,0,0}
\definecolor{fillColor}{RGB}{0,0,0}

\path[draw=drawColor,line width= 0.4pt,line join=round,line cap=round,fill=fillColor] (578.23,233.72) circle (  1.49);
\definecolor{drawColor}{RGB}{255,0,0}
\definecolor{fillColor}{RGB}{255,0,0}

\path[draw=drawColor,line width= 0.4pt,line join=round,line cap=round,fill=fillColor] (578.71,118.76) circle (  1.49);
\definecolor{drawColor}{RGB}{0,0,0}
\definecolor{fillColor}{RGB}{0,0,0}

\path[draw=drawColor,line width= 0.4pt,line join=round,line cap=round,fill=fillColor] (578.72,220.25) circle (  1.49);
\definecolor{drawColor}{RGB}{255,0,0}
\definecolor{fillColor}{RGB}{255,0,0}

\path[draw=drawColor,line width= 0.4pt,line join=round,line cap=round,fill=fillColor] (579.20,147.99) circle (  1.49);
\definecolor{drawColor}{RGB}{0,0,0}
\definecolor{fillColor}{RGB}{0,0,0}

\path[draw=drawColor,line width= 0.4pt,line join=round,line cap=round,fill=fillColor] (579.21,219.26) circle (  1.49);
\definecolor{drawColor}{RGB}{255,0,0}
\definecolor{fillColor}{RGB}{255,0,0}

\path[draw=drawColor,line width= 0.4pt,line join=round,line cap=round,fill=fillColor] (579.67,128.94) circle (  1.49);
\definecolor{drawColor}{RGB}{0,0,0}
\definecolor{fillColor}{RGB}{0,0,0}

\path[draw=drawColor,line width= 0.4pt,line join=round,line cap=round,fill=fillColor] (579.69,220.25) circle (  1.49);
\definecolor{drawColor}{RGB}{255,0,0}
\definecolor{fillColor}{RGB}{255,0,0}

\path[draw=drawColor,line width= 0.4pt,line join=round,line cap=round,fill=fillColor] (580.15,118.76) circle (  1.49);
\definecolor{drawColor}{RGB}{0,0,0}
\definecolor{fillColor}{RGB}{0,0,0}

\path[draw=drawColor,line width= 0.4pt,line join=round,line cap=round,fill=fillColor] (580.16,220.91) circle (  1.49);
\definecolor{drawColor}{RGB}{255,0,0}
\definecolor{fillColor}{RGB}{255,0,0}

\path[draw=drawColor,line width= 0.4pt,line join=round,line cap=round,fill=fillColor] (580.64,122.04) circle (  1.49);
\definecolor{drawColor}{RGB}{0,0,0}
\definecolor{fillColor}{RGB}{0,0,0}

\path[draw=drawColor,line width= 0.4pt,line join=round,line cap=round,fill=fillColor] (580.66,221.24) circle (  1.49);
\definecolor{drawColor}{RGB}{255,0,0}
\definecolor{fillColor}{RGB}{255,0,0}

\path[draw=drawColor,line width= 0.4pt,line join=round,line cap=round,fill=fillColor] (581.13,122.04) circle (  1.49);
\definecolor{drawColor}{RGB}{0,0,0}
\definecolor{fillColor}{RGB}{0,0,0}

\path[draw=drawColor,line width= 0.4pt,line join=round,line cap=round,fill=fillColor] (581.15,221.56) circle (  1.49);
\definecolor{drawColor}{RGB}{255,0,0}
\definecolor{fillColor}{RGB}{255,0,0}

\path[draw=drawColor,line width= 0.4pt,line join=round,line cap=round,fill=fillColor] (581.60,153.90) circle (  1.49);
\definecolor{drawColor}{RGB}{0,0,0}
\definecolor{fillColor}{RGB}{0,0,0}

\path[draw=drawColor,line width= 0.4pt,line join=round,line cap=round,fill=fillColor] (581.62,220.91) circle (  1.49);
\definecolor{drawColor}{RGB}{255,0,0}
\definecolor{fillColor}{RGB}{255,0,0}

\path[draw=drawColor,line width= 0.4pt,line join=round,line cap=round,fill=fillColor] (582.08,140.43) circle (  1.49);
\definecolor{drawColor}{RGB}{0,0,0}
\definecolor{fillColor}{RGB}{0,0,0}

\path[draw=drawColor,line width= 0.4pt,line join=round,line cap=round,fill=fillColor] (582.10,221.24) circle (  1.49);
\definecolor{drawColor}{RGB}{255,0,0}
\definecolor{fillColor}{RGB}{255,0,0}

\path[draw=drawColor,line width= 0.4pt,line join=round,line cap=round,fill=fillColor] (582.67,205.14) circle (  1.49);
\definecolor{drawColor}{RGB}{0,0,0}
\definecolor{fillColor}{RGB}{0,0,0}

\path[draw=drawColor,line width= 0.4pt,line join=round,line cap=round,fill=fillColor] (582.68,228.13) circle (  1.49);
\definecolor{drawColor}{RGB}{255,0,0}
\definecolor{fillColor}{RGB}{255,0,0}

\path[draw=drawColor,line width= 0.4pt,line join=round,line cap=round,fill=fillColor] (583.18, 91.17) circle (  1.49);
\definecolor{drawColor}{RGB}{0,0,0}
\definecolor{fillColor}{RGB}{0,0,0}

\path[draw=drawColor,line width= 0.4pt,line join=round,line cap=round,fill=fillColor] (583.19,221.56) circle (  1.49);
\definecolor{drawColor}{RGB}{255,0,0}
\definecolor{fillColor}{RGB}{255,0,0}

\path[draw=drawColor,line width= 0.4pt,line join=round,line cap=round,fill=fillColor] (583.68,129.60) circle (  1.49);
\definecolor{drawColor}{RGB}{0,0,0}
\definecolor{fillColor}{RGB}{0,0,0}

\path[draw=drawColor,line width= 0.4pt,line join=round,line cap=round,fill=fillColor] (583.70,223.53) circle (  1.49);
\definecolor{drawColor}{RGB}{255,0,0}
\definecolor{fillColor}{RGB}{255,0,0}

\path[draw=drawColor,line width= 0.4pt,line join=round,line cap=round,fill=fillColor] (584.22,161.46) circle (  1.49);
\definecolor{drawColor}{RGB}{0,0,0}
\definecolor{fillColor}{RGB}{0,0,0}

\path[draw=drawColor,line width= 0.4pt,line join=round,line cap=round,fill=fillColor] (584.24,222.88) circle (  1.49);
\definecolor{drawColor}{RGB}{255,0,0}
\definecolor{fillColor}{RGB}{255,0,0}

\path[draw=drawColor,line width= 0.4pt,line join=round,line cap=round,fill=fillColor] (584.91, 96.42) circle (  1.49);
\definecolor{drawColor}{RGB}{0,0,0}
\definecolor{fillColor}{RGB}{0,0,0}

\path[draw=drawColor,line width= 0.4pt,line join=round,line cap=round,fill=fillColor] (584.93,218.61) circle (  1.49);
\definecolor{drawColor}{RGB}{255,0,0}
\definecolor{fillColor}{RGB}{255,0,0}

\path[draw=drawColor,line width= 0.4pt,line join=round,line cap=round,fill=fillColor] (585.45, 96.75) circle (  1.49);
\definecolor{drawColor}{RGB}{0,0,0}
\definecolor{fillColor}{RGB}{0,0,0}

\path[draw=drawColor,line width= 0.4pt,line join=round,line cap=round,fill=fillColor] (585.47,221.56) circle (  1.49);
\definecolor{drawColor}{RGB}{255,0,0}
\definecolor{fillColor}{RGB}{255,0,0}

\path[draw=drawColor,line width= 0.4pt,line join=round,line cap=round,fill=fillColor] (585.94, 96.09) circle (  1.49);
\definecolor{drawColor}{RGB}{0,0,0}
\definecolor{fillColor}{RGB}{0,0,0}

\path[draw=drawColor,line width= 0.4pt,line join=round,line cap=round,fill=fillColor] (585.96,220.91) circle (  1.49);
\definecolor{drawColor}{RGB}{255,0,0}
\definecolor{fillColor}{RGB}{255,0,0}

\path[draw=drawColor,line width= 0.4pt,line join=round,line cap=round,fill=fillColor] (586.45, 92.48) circle (  1.49);
\definecolor{drawColor}{RGB}{0,0,0}
\definecolor{fillColor}{RGB}{0,0,0}

\path[draw=drawColor,line width= 0.4pt,line join=round,line cap=round,fill=fillColor] (586.47,221.56) circle (  1.49);
\definecolor{drawColor}{RGB}{255,0,0}
\definecolor{fillColor}{RGB}{255,0,0}

\path[draw=drawColor,line width= 0.4pt,line join=round,line cap=round,fill=fillColor] (586.97, 93.14) circle (  1.49);
\definecolor{drawColor}{RGB}{0,0,0}
\definecolor{fillColor}{RGB}{0,0,0}

\path[draw=drawColor,line width= 0.4pt,line join=round,line cap=round,fill=fillColor] (586.99,220.91) circle (  1.49);
\definecolor{drawColor}{RGB}{255,0,0}
\definecolor{fillColor}{RGB}{255,0,0}

\path[draw=drawColor,line width= 0.4pt,line join=round,line cap=round,fill=fillColor] (587.46, 80.65) circle (  1.49);
\definecolor{drawColor}{RGB}{0,0,0}
\definecolor{fillColor}{RGB}{0,0,0}

\path[draw=drawColor,line width= 0.4pt,line join=round,line cap=round,fill=fillColor] (587.48,219.92) circle (  1.49);
\definecolor{drawColor}{RGB}{255,0,0}
\definecolor{fillColor}{RGB}{255,0,0}

\path[draw=drawColor,line width= 0.4pt,line join=round,line cap=round,fill=fillColor] (587.97, 71.46) circle (  1.49);
\definecolor{drawColor}{RGB}{0,0,0}
\definecolor{fillColor}{RGB}{0,0,0}

\path[draw=drawColor,line width= 0.4pt,line join=round,line cap=round,fill=fillColor] (587.99,221.24) circle (  1.49);
\definecolor{drawColor}{RGB}{255,0,0}
\definecolor{fillColor}{RGB}{255,0,0}

\path[draw=drawColor,line width= 0.4pt,line join=round,line cap=round,fill=fillColor] (588.51, 99.38) circle (  1.49);
\definecolor{drawColor}{RGB}{0,0,0}
\definecolor{fillColor}{RGB}{0,0,0}

\path[draw=drawColor,line width= 0.4pt,line join=round,line cap=round,fill=fillColor] (588.53,221.89) circle (  1.49);
\definecolor{drawColor}{RGB}{255,0,0}
\definecolor{fillColor}{RGB}{255,0,0}

\path[draw=drawColor,line width= 0.4pt,line join=round,line cap=round,fill=fillColor] (589.00,111.20) circle (  1.49);
\definecolor{drawColor}{RGB}{0,0,0}
\definecolor{fillColor}{RGB}{0,0,0}

\path[draw=drawColor,line width= 0.4pt,line join=round,line cap=round,fill=fillColor] (589.04,219.92) circle (  1.49);
\definecolor{drawColor}{RGB}{255,0,0}
\definecolor{fillColor}{RGB}{255,0,0}

\path[draw=drawColor,line width= 0.4pt,line join=round,line cap=round,fill=fillColor] (589.51,128.94) circle (  1.49);
\definecolor{drawColor}{RGB}{0,0,0}
\definecolor{fillColor}{RGB}{0,0,0}

\path[draw=drawColor,line width= 0.4pt,line join=round,line cap=round,fill=fillColor] (589.53,220.91) circle (  1.49);
\definecolor{drawColor}{RGB}{255,0,0}
\definecolor{fillColor}{RGB}{255,0,0}

\path[draw=drawColor,line width= 0.4pt,line join=round,line cap=round,fill=fillColor] (590.02,178.86) circle (  1.49);
\definecolor{drawColor}{RGB}{0,0,0}
\definecolor{fillColor}{RGB}{0,0,0}

\path[draw=drawColor,line width= 0.4pt,line join=round,line cap=round,fill=fillColor] (590.03,221.24) circle (  1.49);
\definecolor{drawColor}{RGB}{255,0,0}
\definecolor{fillColor}{RGB}{255,0,0}

\path[draw=drawColor,line width= 0.4pt,line join=round,line cap=round,fill=fillColor] (590.53, 77.37) circle (  1.49);
\definecolor{drawColor}{RGB}{0,0,0}
\definecolor{fillColor}{RGB}{0,0,0}

\path[draw=drawColor,line width= 0.4pt,line join=round,line cap=round,fill=fillColor] (590.54,221.24) circle (  1.49);
\definecolor{drawColor}{RGB}{255,0,0}
\definecolor{fillColor}{RGB}{255,0,0}

\path[draw=drawColor,line width= 0.4pt,line join=round,line cap=round,fill=fillColor] (591.03,101.68) circle (  1.49);
\definecolor{drawColor}{RGB}{0,0,0}
\definecolor{fillColor}{RGB}{0,0,0}

\path[draw=drawColor,line width= 0.4pt,line join=round,line cap=round,fill=fillColor] (591.05,222.22) circle (  1.49);
\definecolor{drawColor}{RGB}{255,0,0}
\definecolor{fillColor}{RGB}{255,0,0}

\path[draw=drawColor,line width= 0.4pt,line join=round,line cap=round,fill=fillColor] (591.54,105.95) circle (  1.49);
\definecolor{drawColor}{RGB}{0,0,0}
\definecolor{fillColor}{RGB}{0,0,0}

\path[draw=drawColor,line width= 0.4pt,line join=round,line cap=round,fill=fillColor] (591.56,221.56) circle (  1.49);
\definecolor{drawColor}{RGB}{255,0,0}
\definecolor{fillColor}{RGB}{255,0,0}

\path[draw=drawColor,line width= 0.4pt,line join=round,line cap=round,fill=fillColor] (592.05,149.96) circle (  1.49);
\definecolor{drawColor}{RGB}{0,0,0}
\definecolor{fillColor}{RGB}{0,0,0}

\path[draw=drawColor,line width= 0.4pt,line join=round,line cap=round,fill=fillColor] (592.06,221.89) circle (  1.49);
\definecolor{drawColor}{RGB}{255,0,0}
\definecolor{fillColor}{RGB}{255,0,0}

\path[draw=drawColor,line width= 0.4pt,line join=round,line cap=round,fill=fillColor] (592.54,143.06) circle (  1.49);
\definecolor{drawColor}{RGB}{0,0,0}
\definecolor{fillColor}{RGB}{0,0,0}

\path[draw=drawColor,line width= 0.4pt,line join=round,line cap=round,fill=fillColor] (592.56,221.89) circle (  1.49);
\definecolor{drawColor}{RGB}{255,0,0}
\definecolor{fillColor}{RGB}{255,0,0}

\path[draw=drawColor,line width= 0.4pt,line join=round,line cap=round,fill=fillColor] (593.03, 99.05) circle (  1.49);
\definecolor{drawColor}{RGB}{0,0,0}
\definecolor{fillColor}{RGB}{0,0,0}

\path[draw=drawColor,line width= 0.4pt,line join=round,line cap=round,fill=fillColor] (593.05,218.94) circle (  1.49);
\definecolor{drawColor}{RGB}{255,0,0}
\definecolor{fillColor}{RGB}{255,0,0}

\path[draw=drawColor,line width= 0.4pt,line join=round,line cap=round,fill=fillColor] (593.54,183.46) circle (  1.49);
\definecolor{drawColor}{RGB}{0,0,0}
\definecolor{fillColor}{RGB}{0,0,0}

\path[draw=drawColor,line width= 0.4pt,line join=round,line cap=round,fill=fillColor] (593.55,221.89) circle (  1.49);
\definecolor{drawColor}{RGB}{255,0,0}
\definecolor{fillColor}{RGB}{255,0,0}

\path[draw=drawColor,line width= 0.4pt,line join=round,line cap=round,fill=fillColor] (594.08, 93.46) circle (  1.49);
\definecolor{drawColor}{RGB}{0,0,0}
\definecolor{fillColor}{RGB}{0,0,0}

\path[draw=drawColor,line width= 0.4pt,line join=round,line cap=round,fill=fillColor] (594.11,221.24) circle (  1.49);
\definecolor{drawColor}{RGB}{255,0,0}
\definecolor{fillColor}{RGB}{255,0,0}

\path[draw=drawColor,line width= 0.4pt,line join=round,line cap=round,fill=fillColor] (594.59, 85.91) circle (  1.49);
\definecolor{drawColor}{RGB}{0,0,0}
\definecolor{fillColor}{RGB}{0,0,0}

\path[draw=drawColor,line width= 0.4pt,line join=round,line cap=round,fill=fillColor] (594.60,221.89) circle (  1.49);
\definecolor{drawColor}{RGB}{255,0,0}
\definecolor{fillColor}{RGB}{255,0,0}

\path[draw=drawColor,line width= 0.4pt,line join=round,line cap=round,fill=fillColor] (595.11,192.66) circle (  1.49);
\definecolor{drawColor}{RGB}{0,0,0}
\definecolor{fillColor}{RGB}{0,0,0}

\path[draw=drawColor,line width= 0.4pt,line join=round,line cap=round,fill=fillColor] (595.13,222.55) circle (  1.49);
\definecolor{drawColor}{RGB}{255,0,0}
\definecolor{fillColor}{RGB}{255,0,0}

\path[draw=drawColor,line width= 0.4pt,line join=round,line cap=round,fill=fillColor] (595.72,179.19) circle (  1.49);
\definecolor{drawColor}{RGB}{0,0,0}
\definecolor{fillColor}{RGB}{0,0,0}

\path[draw=drawColor,line width= 0.4pt,line join=round,line cap=round,fill=fillColor] (595.73,222.22) circle (  1.49);
\definecolor{drawColor}{RGB}{255,0,0}
\definecolor{fillColor}{RGB}{255,0,0}

\path[draw=drawColor,line width= 0.4pt,line join=round,line cap=round,fill=fillColor] (596.22,201.20) circle (  1.49);
\definecolor{drawColor}{RGB}{0,0,0}
\definecolor{fillColor}{RGB}{0,0,0}

\path[draw=drawColor,line width= 0.4pt,line join=round,line cap=round,fill=fillColor] (596.24,221.89) circle (  1.49);
\definecolor{drawColor}{RGB}{255,0,0}
\definecolor{fillColor}{RGB}{255,0,0}

\path[draw=drawColor,line width= 0.4pt,line join=round,line cap=round,fill=fillColor] (596.71,147.00) circle (  1.49);
\definecolor{drawColor}{RGB}{0,0,0}
\definecolor{fillColor}{RGB}{0,0,0}

\path[draw=drawColor,line width= 0.4pt,line join=round,line cap=round,fill=fillColor] (596.75,221.56) circle (  1.49);
\definecolor{drawColor}{RGB}{255,0,0}
\definecolor{fillColor}{RGB}{255,0,0}

\path[draw=drawColor,line width= 0.4pt,line join=round,line cap=round,fill=fillColor] (597.37, 54.71) circle (  1.49);
\definecolor{drawColor}{RGB}{0,0,0}
\definecolor{fillColor}{RGB}{0,0,0}

\path[draw=drawColor,line width= 0.4pt,line join=round,line cap=round,fill=fillColor] (597.38,219.59) circle (  1.49);
\definecolor{drawColor}{RGB}{255,0,0}
\definecolor{fillColor}{RGB}{255,0,0}

\path[draw=drawColor,line width= 0.4pt,line join=round,line cap=round,fill=fillColor] (598.09,174.27) circle (  1.49);
\definecolor{drawColor}{RGB}{0,0,0}
\definecolor{fillColor}{RGB}{0,0,0}

\path[draw=drawColor,line width= 0.4pt,line join=round,line cap=round,fill=fillColor] (598.11,220.58) circle (  1.49);
\definecolor{drawColor}{RGB}{255,0,0}
\definecolor{fillColor}{RGB}{255,0,0}

\path[draw=drawColor,line width= 0.4pt,line join=round,line cap=round,fill=fillColor] (598.61,147.99) circle (  1.49);
\definecolor{drawColor}{RGB}{0,0,0}
\definecolor{fillColor}{RGB}{0,0,0}

\path[draw=drawColor,line width= 0.4pt,line join=round,line cap=round,fill=fillColor] (598.63,219.59) circle (  1.49);
\definecolor{drawColor}{RGB}{255,0,0}
\definecolor{fillColor}{RGB}{255,0,0}

\path[draw=drawColor,line width= 0.4pt,line join=round,line cap=round,fill=fillColor] (599.12,188.39) circle (  1.49);
\definecolor{drawColor}{RGB}{0,0,0}
\definecolor{fillColor}{RGB}{0,0,0}

\path[draw=drawColor,line width= 0.4pt,line join=round,line cap=round,fill=fillColor] (599.14,219.26) circle (  1.49);
\definecolor{drawColor}{RGB}{255,0,0}
\definecolor{fillColor}{RGB}{255,0,0}

\path[draw=drawColor,line width= 0.4pt,line join=round,line cap=round,fill=fillColor] (599.77,193.64) circle (  1.49);
\definecolor{drawColor}{RGB}{0,0,0}
\definecolor{fillColor}{RGB}{0,0,0}

\path[draw=drawColor,line width= 0.4pt,line join=round,line cap=round,fill=fillColor] (599.79,220.58) circle (  1.49);
\definecolor{drawColor}{RGB}{255,0,0}
\definecolor{fillColor}{RGB}{255,0,0}

\path[draw=drawColor,line width= 0.4pt,line join=round,line cap=round,fill=fillColor] (600.36,196.60) circle (  1.49);
\definecolor{drawColor}{RGB}{0,0,0}
\definecolor{fillColor}{RGB}{0,0,0}

\path[draw=drawColor,line width= 0.4pt,line join=round,line cap=round,fill=fillColor] (600.38,222.88) circle (  1.49);
\definecolor{drawColor}{RGB}{255,0,0}
\definecolor{fillColor}{RGB}{255,0,0}

\path[draw=drawColor,line width= 0.4pt,line join=round,line cap=round,fill=fillColor] (600.86, 99.05) circle (  1.49);
\definecolor{drawColor}{RGB}{0,0,0}
\definecolor{fillColor}{RGB}{0,0,0}

\path[draw=drawColor,line width= 0.4pt,line join=round,line cap=round,fill=fillColor] (600.87,222.88) circle (  1.49);
\definecolor{drawColor}{RGB}{255,0,0}
\definecolor{fillColor}{RGB}{255,0,0}

\path[draw=drawColor,line width= 0.4pt,line join=round,line cap=round,fill=fillColor] (601.40,206.78) circle (  1.49);
\definecolor{drawColor}{RGB}{0,0,0}
\definecolor{fillColor}{RGB}{0,0,0}

\path[draw=drawColor,line width= 0.4pt,line join=round,line cap=round,fill=fillColor] (601.41,223.21) circle (  1.49);
\definecolor{drawColor}{RGB}{255,0,0}
\definecolor{fillColor}{RGB}{255,0,0}

\path[draw=drawColor,line width= 0.4pt,line join=round,line cap=round,fill=fillColor] (601.92,207.11) circle (  1.49);
\definecolor{drawColor}{RGB}{0,0,0}
\definecolor{fillColor}{RGB}{0,0,0}

\path[draw=drawColor,line width= 0.4pt,line join=round,line cap=round,fill=fillColor] (601.94,222.22) circle (  1.49);
\definecolor{drawColor}{RGB}{255,0,0}
\definecolor{fillColor}{RGB}{255,0,0}

\path[draw=drawColor,line width= 0.4pt,line join=round,line cap=round,fill=fillColor] (602.51,208.10) circle (  1.49);
\definecolor{drawColor}{RGB}{0,0,0}
\definecolor{fillColor}{RGB}{0,0,0}

\path[draw=drawColor,line width= 0.4pt,line join=round,line cap=round,fill=fillColor] (602.52,221.89) circle (  1.49);
\definecolor{drawColor}{RGB}{255,0,0}
\definecolor{fillColor}{RGB}{255,0,0}

\path[draw=drawColor,line width= 0.4pt,line join=round,line cap=round,fill=fillColor] (603.08,207.77) circle (  1.49);
\definecolor{drawColor}{RGB}{0,0,0}
\definecolor{fillColor}{RGB}{0,0,0}

\path[draw=drawColor,line width= 0.4pt,line join=round,line cap=round,fill=fillColor] (603.10,221.56) circle (  1.49);
\definecolor{drawColor}{RGB}{255,0,0}
\definecolor{fillColor}{RGB}{255,0,0}

\path[draw=drawColor,line width= 0.4pt,line join=round,line cap=round,fill=fillColor] (603.57,207.11) circle (  1.49);
\definecolor{drawColor}{RGB}{0,0,0}
\definecolor{fillColor}{RGB}{0,0,0}

\path[draw=drawColor,line width= 0.4pt,line join=round,line cap=round,fill=fillColor] (603.59,221.56) circle (  1.49);
\definecolor{drawColor}{RGB}{255,0,0}
\definecolor{fillColor}{RGB}{255,0,0}

\path[draw=drawColor,line width= 0.4pt,line join=round,line cap=round,fill=fillColor] (604.08,207.77) circle (  1.49);
\definecolor{drawColor}{RGB}{0,0,0}
\definecolor{fillColor}{RGB}{0,0,0}

\path[draw=drawColor,line width= 0.4pt,line join=round,line cap=round,fill=fillColor] (604.10,219.92) circle (  1.49);
\definecolor{drawColor}{RGB}{255,0,0}
\definecolor{fillColor}{RGB}{255,0,0}

\path[draw=drawColor,line width= 0.4pt,line join=round,line cap=round,fill=fillColor] (604.59,206.45) circle (  1.49);
\definecolor{drawColor}{RGB}{0,0,0}
\definecolor{fillColor}{RGB}{0,0,0}

\path[draw=drawColor,line width= 0.4pt,line join=round,line cap=round,fill=fillColor] (604.60,221.89) circle (  1.49);
\definecolor{drawColor}{RGB}{255,0,0}
\definecolor{fillColor}{RGB}{255,0,0}

\path[draw=drawColor,line width= 0.4pt,line join=round,line cap=round,fill=fillColor] (605.08,207.77) circle (  1.49);
\definecolor{drawColor}{RGB}{0,0,0}
\definecolor{fillColor}{RGB}{0,0,0}

\path[draw=drawColor,line width= 0.4pt,line join=round,line cap=round,fill=fillColor] (605.09,221.56) circle (  1.49);
\definecolor{drawColor}{RGB}{255,0,0}
\definecolor{fillColor}{RGB}{255,0,0}

\path[draw=drawColor,line width= 0.4pt,line join=round,line cap=round,fill=fillColor] (605.64,207.11) circle (  1.49);
\definecolor{drawColor}{RGB}{0,0,0}
\definecolor{fillColor}{RGB}{0,0,0}

\path[draw=drawColor,line width= 0.4pt,line join=round,line cap=round,fill=fillColor] (605.65,219.59) circle (  1.49);
\definecolor{drawColor}{RGB}{255,0,0}
\definecolor{fillColor}{RGB}{255,0,0}

\path[draw=drawColor,line width= 0.4pt,line join=round,line cap=round,fill=fillColor] (606.19,207.11) circle (  1.49);
\definecolor{drawColor}{RGB}{0,0,0}
\definecolor{fillColor}{RGB}{0,0,0}

\path[draw=drawColor,line width= 0.4pt,line join=round,line cap=round,fill=fillColor] (606.21,221.56) circle (  1.49);
\definecolor{drawColor}{RGB}{255,0,0}
\definecolor{fillColor}{RGB}{255,0,0}

\path[draw=drawColor,line width= 0.4pt,line join=round,line cap=round,fill=fillColor] (606.70,207.77) circle (  1.49);
\definecolor{drawColor}{RGB}{0,0,0}
\definecolor{fillColor}{RGB}{0,0,0}

\path[draw=drawColor,line width= 0.4pt,line join=round,line cap=round,fill=fillColor] (606.72,221.24) circle (  1.49);
\definecolor{drawColor}{RGB}{255,0,0}
\definecolor{fillColor}{RGB}{255,0,0}

\path[draw=drawColor,line width= 0.4pt,line join=round,line cap=round,fill=fillColor] (607.17,206.78) circle (  1.49);
\definecolor{drawColor}{RGB}{0,0,0}
\definecolor{fillColor}{RGB}{0,0,0}

\path[draw=drawColor,line width= 0.4pt,line join=round,line cap=round,fill=fillColor] (607.19,218.94) circle (  1.49);
\definecolor{drawColor}{RGB}{255,0,0}
\definecolor{fillColor}{RGB}{255,0,0}

\path[draw=drawColor,line width= 0.4pt,line join=round,line cap=round,fill=fillColor] (607.66,206.13) circle (  1.49);
\definecolor{drawColor}{RGB}{0,0,0}
\definecolor{fillColor}{RGB}{0,0,0}

\path[draw=drawColor,line width= 0.4pt,line join=round,line cap=round,fill=fillColor] (607.68,221.24) circle (  1.49);
\definecolor{drawColor}{RGB}{255,0,0}
\definecolor{fillColor}{RGB}{255,0,0}

\path[draw=drawColor,line width= 0.4pt,line join=round,line cap=round,fill=fillColor] (608.14,206.78) circle (  1.49);
\definecolor{drawColor}{RGB}{0,0,0}
\definecolor{fillColor}{RGB}{0,0,0}

\path[draw=drawColor,line width= 0.4pt,line join=round,line cap=round,fill=fillColor] (608.16,220.91) circle (  1.49);
\definecolor{drawColor}{RGB}{255,0,0}
\definecolor{fillColor}{RGB}{255,0,0}

\path[draw=drawColor,line width= 0.4pt,line join=round,line cap=round,fill=fillColor] (608.71,193.97) circle (  1.49);
\definecolor{drawColor}{RGB}{0,0,0}
\definecolor{fillColor}{RGB}{0,0,0}

\path[draw=drawColor,line width= 0.4pt,line join=round,line cap=round,fill=fillColor] (608.73,219.92) circle (  1.49);
\definecolor{drawColor}{RGB}{255,0,0}
\definecolor{fillColor}{RGB}{255,0,0}

\path[draw=drawColor,line width= 0.4pt,line join=round,line cap=round,fill=fillColor] (609.27,201.53) circle (  1.49);
\definecolor{drawColor}{RGB}{0,0,0}
\definecolor{fillColor}{RGB}{0,0,0}

\path[draw=drawColor,line width= 0.4pt,line join=round,line cap=round,fill=fillColor] (609.29,221.56) circle (  1.49);
\definecolor{drawColor}{RGB}{255,0,0}
\definecolor{fillColor}{RGB}{255,0,0}

\path[draw=drawColor,line width= 0.4pt,line join=round,line cap=round,fill=fillColor] (609.79,202.51) circle (  1.49);
\definecolor{drawColor}{RGB}{0,0,0}
\definecolor{fillColor}{RGB}{0,0,0}

\path[draw=drawColor,line width= 0.4pt,line join=round,line cap=round,fill=fillColor] (609.81,220.91) circle (  1.49);
\definecolor{drawColor}{RGB}{255,0,0}
\definecolor{fillColor}{RGB}{255,0,0}

\path[draw=drawColor,line width= 0.4pt,line join=round,line cap=round,fill=fillColor] (610.32,204.81) circle (  1.49);
\definecolor{drawColor}{RGB}{0,0,0}
\definecolor{fillColor}{RGB}{0,0,0}

\path[draw=drawColor,line width= 0.4pt,line join=round,line cap=round,fill=fillColor] (610.33,220.25) circle (  1.49);
\definecolor{drawColor}{RGB}{255,0,0}
\definecolor{fillColor}{RGB}{255,0,0}

\path[draw=drawColor,line width= 0.4pt,line join=round,line cap=round,fill=fillColor] (610.86,205.80) circle (  1.49);
\definecolor{drawColor}{RGB}{0,0,0}
\definecolor{fillColor}{RGB}{0,0,0}

\path[draw=drawColor,line width= 0.4pt,line join=round,line cap=round,fill=fillColor] (610.87,219.92) circle (  1.49);
\definecolor{drawColor}{RGB}{255,0,0}
\definecolor{fillColor}{RGB}{255,0,0}

\path[draw=drawColor,line width= 0.4pt,line join=round,line cap=round,fill=fillColor] (611.36,204.48) circle (  1.49);
\definecolor{drawColor}{RGB}{0,0,0}
\definecolor{fillColor}{RGB}{0,0,0}

\path[draw=drawColor,line width= 0.4pt,line join=round,line cap=round,fill=fillColor] (611.38,220.25) circle (  1.49);
\definecolor{drawColor}{RGB}{255,0,0}
\definecolor{fillColor}{RGB}{255,0,0}

\path[draw=drawColor,line width= 0.4pt,line join=round,line cap=round,fill=fillColor] (611.87,206.13) circle (  1.49);
\definecolor{drawColor}{RGB}{0,0,0}
\definecolor{fillColor}{RGB}{0,0,0}

\path[draw=drawColor,line width= 0.4pt,line join=round,line cap=round,fill=fillColor] (611.89,220.25) circle (  1.49);
\definecolor{drawColor}{RGB}{255,0,0}
\definecolor{fillColor}{RGB}{255,0,0}

\path[draw=drawColor,line width= 0.4pt,line join=round,line cap=round,fill=fillColor] (612.43,199.23) circle (  1.49);
\definecolor{drawColor}{RGB}{0,0,0}
\definecolor{fillColor}{RGB}{0,0,0}

\path[draw=drawColor,line width= 0.4pt,line join=round,line cap=round,fill=fillColor] (612.44,218.94) circle (  1.49);
\definecolor{drawColor}{RGB}{255,0,0}
\definecolor{fillColor}{RGB}{255,0,0}

\path[draw=drawColor,line width= 0.4pt,line join=round,line cap=round,fill=fillColor] (612.95,193.97) circle (  1.49);
\definecolor{drawColor}{RGB}{0,0,0}
\definecolor{fillColor}{RGB}{0,0,0}

\path[draw=drawColor,line width= 0.4pt,line join=round,line cap=round,fill=fillColor] (612.97,219.26) circle (  1.49);
\definecolor{drawColor}{RGB}{255,0,0}
\definecolor{fillColor}{RGB}{255,0,0}

\path[draw=drawColor,line width= 0.4pt,line join=round,line cap=round,fill=fillColor] (613.44,204.81) circle (  1.49);
\definecolor{drawColor}{RGB}{0,0,0}
\definecolor{fillColor}{RGB}{0,0,0}

\path[draw=drawColor,line width= 0.4pt,line join=round,line cap=round,fill=fillColor] (613.46,219.26) circle (  1.49);
\definecolor{drawColor}{RGB}{255,0,0}
\definecolor{fillColor}{RGB}{255,0,0}

\path[draw=drawColor,line width= 0.4pt,line join=round,line cap=round,fill=fillColor] (614.00,205.47) circle (  1.49);
\definecolor{drawColor}{RGB}{0,0,0}
\definecolor{fillColor}{RGB}{0,0,0}

\path[draw=drawColor,line width= 0.4pt,line join=round,line cap=round,fill=fillColor] (614.03,210.40) circle (  1.49);
\definecolor{drawColor}{RGB}{255,0,0}
\definecolor{fillColor}{RGB}{255,0,0}

\path[draw=drawColor,line width= 0.4pt,line join=round,line cap=round,fill=fillColor] (614.54,205.47) circle (  1.49);
\definecolor{drawColor}{RGB}{0,0,0}
\definecolor{fillColor}{RGB}{0,0,0}

\path[draw=drawColor,line width= 0.4pt,line join=round,line cap=round,fill=fillColor] (614.56,219.92) circle (  1.49);
\definecolor{drawColor}{RGB}{255,0,0}
\definecolor{fillColor}{RGB}{255,0,0}

\path[draw=drawColor,line width= 0.4pt,line join=round,line cap=round,fill=fillColor] (615.01,204.81) circle (  1.49);
\definecolor{drawColor}{RGB}{0,0,0}
\definecolor{fillColor}{RGB}{0,0,0}

\path[draw=drawColor,line width= 0.4pt,line join=round,line cap=round,fill=fillColor] (615.03,219.59) circle (  1.49);
\definecolor{drawColor}{RGB}{255,0,0}
\definecolor{fillColor}{RGB}{255,0,0}

\path[draw=drawColor,line width= 0.4pt,line join=round,line cap=round,fill=fillColor] (615.51,142.73) circle (  1.49);
\definecolor{drawColor}{RGB}{0,0,0}
\definecolor{fillColor}{RGB}{0,0,0}

\path[draw=drawColor,line width= 0.4pt,line join=round,line cap=round,fill=fillColor] (615.52,220.25) circle (  1.49);
\definecolor{drawColor}{RGB}{255,0,0}
\definecolor{fillColor}{RGB}{255,0,0}

\path[draw=drawColor,line width= 0.4pt,line join=round,line cap=round,fill=fillColor] (616.01,206.13) circle (  1.49);
\definecolor{drawColor}{RGB}{0,0,0}
\definecolor{fillColor}{RGB}{0,0,0}

\path[draw=drawColor,line width= 0.4pt,line join=round,line cap=round,fill=fillColor] (616.03,220.58) circle (  1.49);
\definecolor{drawColor}{RGB}{255,0,0}
\definecolor{fillColor}{RGB}{255,0,0}

\path[draw=drawColor,line width= 0.4pt,line join=round,line cap=round,fill=fillColor] (616.49,206.45) circle (  1.49);
\definecolor{drawColor}{RGB}{0,0,0}
\definecolor{fillColor}{RGB}{0,0,0}

\path[draw=drawColor,line width= 0.4pt,line join=round,line cap=round,fill=fillColor] (616.50,219.26) circle (  1.49);
\definecolor{drawColor}{RGB}{255,0,0}
\definecolor{fillColor}{RGB}{255,0,0}

\path[draw=drawColor,line width= 0.4pt,line join=round,line cap=round,fill=fillColor] (617.01,205.80) circle (  1.49);
\definecolor{drawColor}{RGB}{0,0,0}
\definecolor{fillColor}{RGB}{0,0,0}

\path[draw=drawColor,line width= 0.4pt,line join=round,line cap=round,fill=fillColor] (617.03,221.24) circle (  1.49);
\definecolor{drawColor}{RGB}{255,0,0}
\definecolor{fillColor}{RGB}{255,0,0}

\path[draw=drawColor,line width= 0.4pt,line join=round,line cap=round,fill=fillColor] (617.50,206.13) circle (  1.49);
\definecolor{drawColor}{RGB}{0,0,0}
\definecolor{fillColor}{RGB}{0,0,0}

\path[draw=drawColor,line width= 0.4pt,line join=round,line cap=round,fill=fillColor] (617.52,220.91) circle (  1.49);
\definecolor{drawColor}{RGB}{255,0,0}
\definecolor{fillColor}{RGB}{255,0,0}

\path[draw=drawColor,line width= 0.4pt,line join=round,line cap=round,fill=fillColor] (618.03,206.45) circle (  1.49);
\definecolor{drawColor}{RGB}{0,0,0}
\definecolor{fillColor}{RGB}{0,0,0}

\path[draw=drawColor,line width= 0.4pt,line join=round,line cap=round,fill=fillColor] (618.04,221.24) circle (  1.49);
\definecolor{drawColor}{RGB}{255,0,0}
\definecolor{fillColor}{RGB}{255,0,0}

\path[draw=drawColor,line width= 0.4pt,line join=round,line cap=round,fill=fillColor] (618.63,205.47) circle (  1.49);
\definecolor{drawColor}{RGB}{0,0,0}
\definecolor{fillColor}{RGB}{0,0,0}

\path[draw=drawColor,line width= 0.4pt,line join=round,line cap=round,fill=fillColor] (618.65,220.91) circle (  1.49);
\definecolor{drawColor}{RGB}{255,0,0}
\definecolor{fillColor}{RGB}{255,0,0}

\path[draw=drawColor,line width= 0.4pt,line join=round,line cap=round,fill=fillColor] (619.12,206.78) circle (  1.49);
\definecolor{drawColor}{RGB}{0,0,0}
\definecolor{fillColor}{RGB}{0,0,0}

\path[draw=drawColor,line width= 0.4pt,line join=round,line cap=round,fill=fillColor] (619.14,221.24) circle (  1.49);
\definecolor{drawColor}{RGB}{255,0,0}
\definecolor{fillColor}{RGB}{255,0,0}

\path[draw=drawColor,line width= 0.4pt,line join=round,line cap=round,fill=fillColor] (619.61,205.80) circle (  1.49);
\definecolor{drawColor}{RGB}{0,0,0}
\definecolor{fillColor}{RGB}{0,0,0}

\path[draw=drawColor,line width= 0.4pt,line join=round,line cap=round,fill=fillColor] (619.63,221.56) circle (  1.49);
\definecolor{drawColor}{RGB}{255,0,0}
\definecolor{fillColor}{RGB}{255,0,0}

\path[draw=drawColor,line width= 0.4pt,line join=round,line cap=round,fill=fillColor] (620.09,206.13) circle (  1.49);
\definecolor{drawColor}{RGB}{0,0,0}
\definecolor{fillColor}{RGB}{0,0,0}

\path[draw=drawColor,line width= 0.4pt,line join=round,line cap=round,fill=fillColor] (620.11,221.56) circle (  1.49);
\definecolor{drawColor}{RGB}{255,0,0}
\definecolor{fillColor}{RGB}{255,0,0}

\path[draw=drawColor,line width= 0.4pt,line join=round,line cap=round,fill=fillColor] (620.61,183.79) circle (  1.49);
\definecolor{drawColor}{RGB}{0,0,0}
\definecolor{fillColor}{RGB}{0,0,0}

\path[draw=drawColor,line width= 0.4pt,line join=round,line cap=round,fill=fillColor] (620.63,221.89) circle (  1.49);
\definecolor{drawColor}{RGB}{255,0,0}
\definecolor{fillColor}{RGB}{255,0,0}

\path[draw=drawColor,line width= 0.4pt,line join=round,line cap=round,fill=fillColor] (621.12,206.45) circle (  1.49);
\definecolor{drawColor}{RGB}{0,0,0}
\definecolor{fillColor}{RGB}{0,0,0}

\path[draw=drawColor,line width= 0.4pt,line join=round,line cap=round,fill=fillColor] (621.14,220.58) circle (  1.49);
\definecolor{drawColor}{RGB}{255,0,0}
\definecolor{fillColor}{RGB}{255,0,0}

\path[draw=drawColor,line width= 0.4pt,line join=round,line cap=round,fill=fillColor] (621.63,206.78) circle (  1.49);
\definecolor{drawColor}{RGB}{0,0,0}
\definecolor{fillColor}{RGB}{0,0,0}

\path[draw=drawColor,line width= 0.4pt,line join=round,line cap=round,fill=fillColor] (621.64,220.91) circle (  1.49);
\definecolor{drawColor}{RGB}{255,0,0}
\definecolor{fillColor}{RGB}{255,0,0}

\path[draw=drawColor,line width= 0.4pt,line join=round,line cap=round,fill=fillColor] (622.12,206.78) circle (  1.49);
\definecolor{drawColor}{RGB}{0,0,0}
\definecolor{fillColor}{RGB}{0,0,0}

\path[draw=drawColor,line width= 0.4pt,line join=round,line cap=round,fill=fillColor] (622.14,219.92) circle (  1.49);
\definecolor{drawColor}{RGB}{255,0,0}
\definecolor{fillColor}{RGB}{255,0,0}

\path[draw=drawColor,line width= 0.4pt,line join=round,line cap=round,fill=fillColor] (622.63,203.83) circle (  1.49);
\definecolor{drawColor}{RGB}{0,0,0}
\definecolor{fillColor}{RGB}{0,0,0}

\path[draw=drawColor,line width= 0.4pt,line join=round,line cap=round,fill=fillColor] (622.64,216.64) circle (  1.49);
\definecolor{drawColor}{RGB}{255,0,0}
\definecolor{fillColor}{RGB}{255,0,0}

\path[draw=drawColor,line width= 0.4pt,line join=round,line cap=round,fill=fillColor] (623.12,205.80) circle (  1.49);
\definecolor{drawColor}{RGB}{0,0,0}
\definecolor{fillColor}{RGB}{0,0,0}

\path[draw=drawColor,line width= 0.4pt,line join=round,line cap=round,fill=fillColor] (623.13,221.24) circle (  1.49);
\definecolor{drawColor}{RGB}{255,0,0}
\definecolor{fillColor}{RGB}{255,0,0}

\path[draw=drawColor,line width= 0.4pt,line join=round,line cap=round,fill=fillColor] (623.63,206.45) circle (  1.49);
\definecolor{drawColor}{RGB}{0,0,0}
\definecolor{fillColor}{RGB}{0,0,0}

\path[draw=drawColor,line width= 0.4pt,line join=round,line cap=round,fill=fillColor] (623.64,220.91) circle (  1.49);
\definecolor{drawColor}{RGB}{255,0,0}
\definecolor{fillColor}{RGB}{255,0,0}

\path[draw=drawColor,line width= 0.4pt,line join=round,line cap=round,fill=fillColor] (624.13,207.11) circle (  1.49);
\definecolor{drawColor}{RGB}{0,0,0}
\definecolor{fillColor}{RGB}{0,0,0}

\path[draw=drawColor,line width= 0.4pt,line join=round,line cap=round,fill=fillColor] (624.15,221.24) circle (  1.49);
\definecolor{drawColor}{RGB}{255,0,0}
\definecolor{fillColor}{RGB}{255,0,0}

\path[draw=drawColor,line width= 0.4pt,line join=round,line cap=round,fill=fillColor] (624.61,206.78) circle (  1.49);
\definecolor{drawColor}{RGB}{0,0,0}
\definecolor{fillColor}{RGB}{0,0,0}

\path[draw=drawColor,line width= 0.4pt,line join=round,line cap=round,fill=fillColor] (624.62,221.24) circle (  1.49);
\definecolor{drawColor}{RGB}{255,0,0}
\definecolor{fillColor}{RGB}{255,0,0}

\path[draw=drawColor,line width= 0.4pt,line join=round,line cap=round,fill=fillColor] (625.13,205.47) circle (  1.49);
\definecolor{drawColor}{RGB}{0,0,0}
\definecolor{fillColor}{RGB}{0,0,0}

\path[draw=drawColor,line width= 0.4pt,line join=round,line cap=round,fill=fillColor] (625.15,222.22) circle (  1.49);
\definecolor{drawColor}{RGB}{255,0,0}
\definecolor{fillColor}{RGB}{255,0,0}

\path[draw=drawColor,line width= 0.4pt,line join=round,line cap=round,fill=fillColor] (625.62,206.13) circle (  1.49);
\definecolor{drawColor}{RGB}{0,0,0}
\definecolor{fillColor}{RGB}{0,0,0}

\path[draw=drawColor,line width= 0.4pt,line join=round,line cap=round,fill=fillColor] (625.64,222.88) circle (  1.49);
\definecolor{drawColor}{RGB}{255,0,0}
\definecolor{fillColor}{RGB}{255,0,0}

\path[draw=drawColor,line width= 0.4pt,line join=round,line cap=round,fill=fillColor] (626.21,205.80) circle (  1.49);
\definecolor{drawColor}{RGB}{0,0,0}
\definecolor{fillColor}{RGB}{0,0,0}

\path[draw=drawColor,line width= 0.4pt,line join=round,line cap=round,fill=fillColor] (626.23,221.24) circle (  1.49);
\definecolor{drawColor}{RGB}{255,0,0}
\definecolor{fillColor}{RGB}{255,0,0}

\path[draw=drawColor,line width= 0.4pt,line join=round,line cap=round,fill=fillColor] (626.74,205.47) circle (  1.49);
\definecolor{drawColor}{RGB}{0,0,0}
\definecolor{fillColor}{RGB}{0,0,0}

\path[draw=drawColor,line width= 0.4pt,line join=round,line cap=round,fill=fillColor] (626.75,220.58) circle (  1.49);
\definecolor{drawColor}{RGB}{255,0,0}
\definecolor{fillColor}{RGB}{255,0,0}

\path[draw=drawColor,line width= 0.4pt,line join=round,line cap=round,fill=fillColor] (627.19,204.81) circle (  1.49);
\definecolor{drawColor}{RGB}{0,0,0}
\definecolor{fillColor}{RGB}{0,0,0}

\path[draw=drawColor,line width= 0.4pt,line join=round,line cap=round,fill=fillColor] (627.21,219.92) circle (  1.49);
\definecolor{drawColor}{RGB}{255,0,0}
\definecolor{fillColor}{RGB}{255,0,0}

\path[draw=drawColor,line width= 0.4pt,line join=round,line cap=round,fill=fillColor] (627.67,206.13) circle (  1.49);
\definecolor{drawColor}{RGB}{0,0,0}
\definecolor{fillColor}{RGB}{0,0,0}

\path[draw=drawColor,line width= 0.4pt,line join=round,line cap=round,fill=fillColor] (627.68,220.25) circle (  1.49);
\definecolor{drawColor}{RGB}{255,0,0}
\definecolor{fillColor}{RGB}{255,0,0}

\path[draw=drawColor,line width= 0.4pt,line join=round,line cap=round,fill=fillColor] (628.21,205.80) circle (  1.49);
\definecolor{drawColor}{RGB}{0,0,0}
\definecolor{fillColor}{RGB}{0,0,0}

\path[draw=drawColor,line width= 0.4pt,line join=round,line cap=round,fill=fillColor] (628.23,221.24) circle (  1.49);
\definecolor{drawColor}{RGB}{255,0,0}
\definecolor{fillColor}{RGB}{255,0,0}

\path[draw=drawColor,line width= 0.4pt,line join=round,line cap=round,fill=fillColor] (628.73,206.78) circle (  1.49);
\definecolor{drawColor}{RGB}{0,0,0}
\definecolor{fillColor}{RGB}{0,0,0}

\path[draw=drawColor,line width= 0.4pt,line join=round,line cap=round,fill=fillColor] (628.75,219.59) circle (  1.49);
\definecolor{drawColor}{RGB}{255,0,0}
\definecolor{fillColor}{RGB}{255,0,0}

\path[draw=drawColor,line width= 0.4pt,line join=round,line cap=round,fill=fillColor] (629.22,206.78) circle (  1.49);
\definecolor{drawColor}{RGB}{0,0,0}
\definecolor{fillColor}{RGB}{0,0,0}

\path[draw=drawColor,line width= 0.4pt,line join=round,line cap=round,fill=fillColor] (629.26,221.89) circle (  1.49);
\definecolor{drawColor}{RGB}{255,0,0}
\definecolor{fillColor}{RGB}{255,0,0}

\path[draw=drawColor,line width= 0.4pt,line join=round,line cap=round,fill=fillColor] (629.71,208.10) circle (  1.49);
\definecolor{drawColor}{RGB}{0,0,0}
\definecolor{fillColor}{RGB}{0,0,0}

\path[draw=drawColor,line width= 0.4pt,line join=round,line cap=round,fill=fillColor] (629.73,221.24) circle (  1.49);
\definecolor{drawColor}{RGB}{255,0,0}
\definecolor{fillColor}{RGB}{255,0,0}

\path[draw=drawColor,line width= 0.4pt,line join=round,line cap=round,fill=fillColor] (630.22,207.11) circle (  1.49);
\definecolor{drawColor}{RGB}{0,0,0}
\definecolor{fillColor}{RGB}{0,0,0}

\path[draw=drawColor,line width= 0.4pt,line join=round,line cap=round,fill=fillColor] (630.24,221.89) circle (  1.49);
\definecolor{drawColor}{RGB}{255,0,0}
\definecolor{fillColor}{RGB}{255,0,0}

\path[draw=drawColor,line width= 0.4pt,line join=round,line cap=round,fill=fillColor] (630.73,207.11) circle (  1.49);
\definecolor{drawColor}{RGB}{0,0,0}
\definecolor{fillColor}{RGB}{0,0,0}

\path[draw=drawColor,line width= 0.4pt,line join=round,line cap=round,fill=fillColor] (630.75,221.89) circle (  1.49);
\definecolor{drawColor}{RGB}{255,0,0}
\definecolor{fillColor}{RGB}{255,0,0}

\path[draw=drawColor,line width= 0.4pt,line join=round,line cap=round,fill=fillColor] (631.25,207.11) circle (  1.49);
\definecolor{drawColor}{RGB}{0,0,0}
\definecolor{fillColor}{RGB}{0,0,0}

\path[draw=drawColor,line width= 0.4pt,line join=round,line cap=round,fill=fillColor] (631.27,221.24) circle (  1.49);
\definecolor{drawColor}{RGB}{255,0,0}
\definecolor{fillColor}{RGB}{255,0,0}

\path[draw=drawColor,line width= 0.4pt,line join=round,line cap=round,fill=fillColor] (631.76,207.77) circle (  1.49);
\definecolor{drawColor}{RGB}{0,0,0}
\definecolor{fillColor}{RGB}{0,0,0}

\path[draw=drawColor,line width= 0.4pt,line join=round,line cap=round,fill=fillColor] (631.78,222.22) circle (  1.49);
\definecolor{drawColor}{RGB}{255,0,0}
\definecolor{fillColor}{RGB}{255,0,0}

\path[draw=drawColor,line width= 0.4pt,line join=round,line cap=round,fill=fillColor] (632.27,207.11) circle (  1.49);
\definecolor{drawColor}{RGB}{0,0,0}
\definecolor{fillColor}{RGB}{0,0,0}

\path[draw=drawColor,line width= 0.4pt,line join=round,line cap=round,fill=fillColor] (632.28,222.55) circle (  1.49);
\definecolor{drawColor}{RGB}{255,0,0}
\definecolor{fillColor}{RGB}{255,0,0}

\path[draw=drawColor,line width= 0.4pt,line join=round,line cap=round,fill=fillColor] (632.76,207.44) circle (  1.49);
\definecolor{drawColor}{RGB}{0,0,0}
\definecolor{fillColor}{RGB}{0,0,0}

\path[draw=drawColor,line width= 0.4pt,line join=round,line cap=round,fill=fillColor] (632.78,222.88) circle (  1.49);
\definecolor{drawColor}{RGB}{255,0,0}
\definecolor{fillColor}{RGB}{255,0,0}

\path[draw=drawColor,line width= 0.4pt,line join=round,line cap=round,fill=fillColor] (633.27,207.77) circle (  1.49);
\definecolor{drawColor}{RGB}{0,0,0}
\definecolor{fillColor}{RGB}{0,0,0}

\path[draw=drawColor,line width= 0.4pt,line join=round,line cap=round,fill=fillColor] (633.28,221.89) circle (  1.49);
\definecolor{drawColor}{RGB}{255,0,0}
\definecolor{fillColor}{RGB}{255,0,0}

\path[draw=drawColor,line width= 0.4pt,line join=round,line cap=round,fill=fillColor] (633.77,207.44) circle (  1.49);
\definecolor{drawColor}{RGB}{0,0,0}
\definecolor{fillColor}{RGB}{0,0,0}

\path[draw=drawColor,line width= 0.4pt,line join=round,line cap=round,fill=fillColor] (633.79,221.89) circle (  1.49);
\definecolor{drawColor}{RGB}{255,0,0}
\definecolor{fillColor}{RGB}{255,0,0}

\path[draw=drawColor,line width= 0.4pt,line join=round,line cap=round,fill=fillColor] (634.28,207.11) circle (  1.49);
\definecolor{drawColor}{RGB}{0,0,0}
\definecolor{fillColor}{RGB}{0,0,0}

\path[draw=drawColor,line width= 0.4pt,line join=round,line cap=round,fill=fillColor] (634.30,220.91) circle (  1.49);
\definecolor{drawColor}{RGB}{255,0,0}
\definecolor{fillColor}{RGB}{255,0,0}

\path[draw=drawColor,line width= 0.4pt,line join=round,line cap=round,fill=fillColor] (634.77,207.11) circle (  1.49);
\definecolor{drawColor}{RGB}{0,0,0}
\definecolor{fillColor}{RGB}{0,0,0}

\path[draw=drawColor,line width= 0.4pt,line join=round,line cap=round,fill=fillColor] (634.81,221.24) circle (  1.49);
\definecolor{drawColor}{RGB}{255,0,0}
\definecolor{fillColor}{RGB}{255,0,0}

\path[draw=drawColor,line width= 0.4pt,line join=round,line cap=round,fill=fillColor] (635.30,207.11) circle (  1.49);
\definecolor{drawColor}{RGB}{0,0,0}
\definecolor{fillColor}{RGB}{0,0,0}

\path[draw=drawColor,line width= 0.4pt,line join=round,line cap=round,fill=fillColor] (635.33,217.95) circle (  1.49);
\definecolor{drawColor}{RGB}{255,0,0}
\definecolor{fillColor}{RGB}{255,0,0}

\path[draw=drawColor,line width= 0.4pt,line join=round,line cap=round,fill=fillColor] (635.85,206.45) circle (  1.49);
\definecolor{drawColor}{RGB}{0,0,0}
\definecolor{fillColor}{RGB}{0,0,0}

\path[draw=drawColor,line width= 0.4pt,line join=round,line cap=round,fill=fillColor] (635.87,221.24) circle (  1.49);
\definecolor{drawColor}{RGB}{255,0,0}
\definecolor{fillColor}{RGB}{255,0,0}

\path[draw=drawColor,line width= 0.4pt,line join=round,line cap=round,fill=fillColor] (636.38,205.80) circle (  1.49);
\definecolor{drawColor}{RGB}{0,0,0}
\definecolor{fillColor}{RGB}{0,0,0}

\path[draw=drawColor,line width= 0.4pt,line join=round,line cap=round,fill=fillColor] (636.39,220.91) circle (  1.49);
\definecolor{drawColor}{RGB}{255,0,0}
\definecolor{fillColor}{RGB}{255,0,0}

\path[draw=drawColor,line width= 0.4pt,line join=round,line cap=round,fill=fillColor] (636.90,203.83) circle (  1.49);
\definecolor{drawColor}{RGB}{0,0,0}
\definecolor{fillColor}{RGB}{0,0,0}

\path[draw=drawColor,line width= 0.4pt,line join=round,line cap=round,fill=fillColor] (636.92,220.58) circle (  1.49);
\definecolor{drawColor}{RGB}{255,0,0}
\definecolor{fillColor}{RGB}{255,0,0}

\path[draw=drawColor,line width= 0.4pt,line join=round,line cap=round,fill=fillColor] (637.38,205.80) circle (  1.49);
\definecolor{drawColor}{RGB}{0,0,0}
\definecolor{fillColor}{RGB}{0,0,0}

\path[draw=drawColor,line width= 0.4pt,line join=round,line cap=round,fill=fillColor] (637.39,220.91) circle (  1.49);
\definecolor{drawColor}{RGB}{255,0,0}
\definecolor{fillColor}{RGB}{255,0,0}

\path[draw=drawColor,line width= 0.4pt,line join=round,line cap=round,fill=fillColor] (637.92,205.14) circle (  1.49);
\definecolor{drawColor}{RGB}{0,0,0}
\definecolor{fillColor}{RGB}{0,0,0}

\path[draw=drawColor,line width= 0.4pt,line join=round,line cap=round,fill=fillColor] (637.93,220.58) circle (  1.49);
\definecolor{drawColor}{RGB}{255,0,0}
\definecolor{fillColor}{RGB}{255,0,0}

\path[draw=drawColor,line width= 0.4pt,line join=round,line cap=round,fill=fillColor] (638.41,205.14) circle (  1.49);
\definecolor{drawColor}{RGB}{0,0,0}
\definecolor{fillColor}{RGB}{0,0,0}

\path[draw=drawColor,line width= 0.4pt,line join=round,line cap=round,fill=fillColor] (638.42,220.91) circle (  1.49);
\definecolor{drawColor}{RGB}{255,0,0}
\definecolor{fillColor}{RGB}{255,0,0}

\path[draw=drawColor,line width= 0.4pt,line join=round,line cap=round,fill=fillColor] (638.93,205.14) circle (  1.49);
\definecolor{drawColor}{RGB}{0,0,0}
\definecolor{fillColor}{RGB}{0,0,0}

\path[draw=drawColor,line width= 0.4pt,line join=round,line cap=round,fill=fillColor] (638.96,221.24) circle (  1.49);
\definecolor{drawColor}{RGB}{255,0,0}
\definecolor{fillColor}{RGB}{255,0,0}

\path[draw=drawColor,line width= 0.4pt,line join=round,line cap=round,fill=fillColor] (639.63,205.80) circle (  1.49);
\definecolor{drawColor}{RGB}{0,0,0}
\definecolor{fillColor}{RGB}{0,0,0}

\path[draw=drawColor,line width= 0.4pt,line join=round,line cap=round,fill=fillColor] (639.65,219.59) circle (  1.49);
\definecolor{drawColor}{RGB}{255,0,0}
\definecolor{fillColor}{RGB}{255,0,0}

\path[draw=drawColor,line width= 0.4pt,line join=round,line cap=round,fill=fillColor] (640.11,206.13) circle (  1.49);
\definecolor{drawColor}{RGB}{0,0,0}
\definecolor{fillColor}{RGB}{0,0,0}

\path[draw=drawColor,line width= 0.4pt,line join=round,line cap=round,fill=fillColor] (640.13,222.22) circle (  1.49);
\definecolor{drawColor}{RGB}{255,0,0}
\definecolor{fillColor}{RGB}{255,0,0}

\path[draw=drawColor,line width= 0.4pt,line join=round,line cap=round,fill=fillColor] (640.62,205.80) circle (  1.49);
\definecolor{drawColor}{RGB}{0,0,0}
\definecolor{fillColor}{RGB}{0,0,0}

\path[draw=drawColor,line width= 0.4pt,line join=round,line cap=round,fill=fillColor] (640.63,222.22) circle (  1.49);
\definecolor{drawColor}{RGB}{255,0,0}
\definecolor{fillColor}{RGB}{255,0,0}

\path[draw=drawColor,line width= 0.4pt,line join=round,line cap=round,fill=fillColor] (641.09,207.11) circle (  1.49);
\definecolor{drawColor}{RGB}{0,0,0}
\definecolor{fillColor}{RGB}{0,0,0}

\path[draw=drawColor,line width= 0.4pt,line join=round,line cap=round,fill=fillColor] (641.11,223.21) circle (  1.49);
\definecolor{drawColor}{RGB}{255,0,0}
\definecolor{fillColor}{RGB}{255,0,0}

\path[draw=drawColor,line width= 0.4pt,line join=round,line cap=round,fill=fillColor] (641.62,207.11) circle (  1.49);
\definecolor{drawColor}{RGB}{0,0,0}
\definecolor{fillColor}{RGB}{0,0,0}

\path[draw=drawColor,line width= 0.4pt,line join=round,line cap=round,fill=fillColor] (641.63,221.56) circle (  1.49);
\definecolor{drawColor}{RGB}{255,0,0}
\definecolor{fillColor}{RGB}{255,0,0}

\path[draw=drawColor,line width= 0.4pt,line join=round,line cap=round,fill=fillColor] (642.11,205.47) circle (  1.49);
\definecolor{drawColor}{RGB}{0,0,0}
\definecolor{fillColor}{RGB}{0,0,0}

\path[draw=drawColor,line width= 0.4pt,line join=round,line cap=round,fill=fillColor] (642.12,220.91) circle (  1.49);
\definecolor{drawColor}{RGB}{255,0,0}
\definecolor{fillColor}{RGB}{255,0,0}

\path[draw=drawColor,line width= 0.4pt,line join=round,line cap=round,fill=fillColor] (642.66,206.13) circle (  1.49);
\definecolor{drawColor}{RGB}{0,0,0}
\definecolor{fillColor}{RGB}{0,0,0}

\path[draw=drawColor,line width= 0.4pt,line join=round,line cap=round,fill=fillColor] (642.68,222.22) circle (  1.49);
\definecolor{drawColor}{RGB}{255,0,0}
\definecolor{fillColor}{RGB}{255,0,0}

\path[draw=drawColor,line width= 0.4pt,line join=round,line cap=round,fill=fillColor] (643.17,206.13) circle (  1.49);
\definecolor{drawColor}{RGB}{0,0,0}
\definecolor{fillColor}{RGB}{0,0,0}

\path[draw=drawColor,line width= 0.4pt,line join=round,line cap=round,fill=fillColor] (643.19,222.55) circle (  1.49);
\definecolor{drawColor}{RGB}{255,0,0}
\definecolor{fillColor}{RGB}{255,0,0}

\path[draw=drawColor,line width= 0.4pt,line join=round,line cap=round,fill=fillColor] (643.68,207.11) circle (  1.49);
\definecolor{drawColor}{RGB}{0,0,0}
\definecolor{fillColor}{RGB}{0,0,0}

\path[draw=drawColor,line width= 0.4pt,line join=round,line cap=round,fill=fillColor] (643.69,222.55) circle (  1.49);
\definecolor{drawColor}{RGB}{255,0,0}
\definecolor{fillColor}{RGB}{255,0,0}

\path[draw=drawColor,line width= 0.4pt,line join=round,line cap=round,fill=fillColor] (644.19,206.45) circle (  1.49);
\definecolor{drawColor}{RGB}{0,0,0}
\definecolor{fillColor}{RGB}{0,0,0}

\path[draw=drawColor,line width= 0.4pt,line join=round,line cap=round,fill=fillColor] (644.20,222.22) circle (  1.49);
\definecolor{drawColor}{RGB}{255,0,0}
\definecolor{fillColor}{RGB}{255,0,0}

\path[draw=drawColor,line width= 0.4pt,line join=round,line cap=round,fill=fillColor] (644.68,206.13) circle (  1.49);
\definecolor{drawColor}{RGB}{0,0,0}
\definecolor{fillColor}{RGB}{0,0,0}

\path[draw=drawColor,line width= 0.4pt,line join=round,line cap=round,fill=fillColor] (644.69,220.58) circle (  1.49);
\definecolor{drawColor}{RGB}{255,0,0}
\definecolor{fillColor}{RGB}{255,0,0}

\path[draw=drawColor,line width= 0.4pt,line join=round,line cap=round,fill=fillColor] (645.33,205.47) circle (  1.49);
\definecolor{drawColor}{RGB}{0,0,0}
\definecolor{fillColor}{RGB}{0,0,0}

\path[draw=drawColor,line width= 0.4pt,line join=round,line cap=round,fill=fillColor] (645.35,219.26) circle (  1.49);
\definecolor{drawColor}{RGB}{255,0,0}
\definecolor{fillColor}{RGB}{255,0,0}

\path[draw=drawColor,line width= 0.4pt,line join=round,line cap=round,fill=fillColor] (645.82,205.14) circle (  1.49);
\definecolor{drawColor}{RGB}{0,0,0}
\definecolor{fillColor}{RGB}{0,0,0}

\path[draw=drawColor,line width= 0.4pt,line join=round,line cap=round,fill=fillColor] (645.84,221.89) circle (  1.49);
\definecolor{drawColor}{RGB}{255,0,0}
\definecolor{fillColor}{RGB}{255,0,0}

\path[draw=drawColor,line width= 0.4pt,line join=round,line cap=round,fill=fillColor] (646.31,204.81) circle (  1.49);
\definecolor{drawColor}{RGB}{0,0,0}
\definecolor{fillColor}{RGB}{0,0,0}

\path[draw=drawColor,line width= 0.4pt,line join=round,line cap=round,fill=fillColor] (646.33,221.89) circle (  1.49);
\definecolor{drawColor}{RGB}{255,0,0}
\definecolor{fillColor}{RGB}{255,0,0}

\path[draw=drawColor,line width= 0.4pt,line join=round,line cap=round,fill=fillColor] (646.87,205.14) circle (  1.49);
\definecolor{drawColor}{RGB}{0,0,0}
\definecolor{fillColor}{RGB}{0,0,0}

\path[draw=drawColor,line width= 0.4pt,line join=round,line cap=round,fill=fillColor] (646.89,221.24) circle (  1.49);
\definecolor{drawColor}{RGB}{255,0,0}
\definecolor{fillColor}{RGB}{255,0,0}

\path[draw=drawColor,line width= 0.4pt,line join=round,line cap=round,fill=fillColor] (647.38,205.47) circle (  1.49);
\definecolor{drawColor}{RGB}{0,0,0}
\definecolor{fillColor}{RGB}{0,0,0}

\path[draw=drawColor,line width= 0.4pt,line join=round,line cap=round,fill=fillColor] (647.39,220.91) circle (  1.49);
\definecolor{drawColor}{RGB}{255,0,0}
\definecolor{fillColor}{RGB}{255,0,0}

\path[draw=drawColor,line width= 0.4pt,line join=round,line cap=round,fill=fillColor] (647.92,204.48) circle (  1.49);
\definecolor{drawColor}{RGB}{0,0,0}
\definecolor{fillColor}{RGB}{0,0,0}

\path[draw=drawColor,line width= 0.4pt,line join=round,line cap=round,fill=fillColor] (647.93,220.58) circle (  1.49);
\definecolor{drawColor}{RGB}{255,0,0}
\definecolor{fillColor}{RGB}{255,0,0}

\path[draw=drawColor,line width= 0.4pt,line join=round,line cap=round,fill=fillColor] (648.41,203.17) circle (  1.49);
\definecolor{drawColor}{RGB}{0,0,0}
\definecolor{fillColor}{RGB}{0,0,0}

\path[draw=drawColor,line width= 0.4pt,line join=round,line cap=round,fill=fillColor] (648.43,221.56) circle (  1.49);
\definecolor{drawColor}{RGB}{255,0,0}
\definecolor{fillColor}{RGB}{255,0,0}

\path[draw=drawColor,line width= 0.4pt,line join=round,line cap=round,fill=fillColor] (648.88,205.80) circle (  1.49);
\definecolor{drawColor}{RGB}{0,0,0}
\definecolor{fillColor}{RGB}{0,0,0}

\path[draw=drawColor,line width= 0.4pt,line join=round,line cap=round,fill=fillColor] (648.90,221.24) circle (  1.49);
\definecolor{drawColor}{RGB}{255,0,0}
\definecolor{fillColor}{RGB}{255,0,0}

\path[draw=drawColor,line width= 0.4pt,line join=round,line cap=round,fill=fillColor] (649.41,206.13) circle (  1.49);
\definecolor{drawColor}{RGB}{0,0,0}
\definecolor{fillColor}{RGB}{0,0,0}

\path[draw=drawColor,line width= 0.4pt,line join=round,line cap=round,fill=fillColor] (649.42,221.89) circle (  1.49);
\definecolor{drawColor}{RGB}{255,0,0}
\definecolor{fillColor}{RGB}{255,0,0}

\path[draw=drawColor,line width= 0.4pt,line join=round,line cap=round,fill=fillColor] (649.91,205.80) circle (  1.49);
\definecolor{drawColor}{RGB}{0,0,0}
\definecolor{fillColor}{RGB}{0,0,0}

\path[draw=drawColor,line width= 0.4pt,line join=round,line cap=round,fill=fillColor] (649.93,221.56) circle (  1.49);
\definecolor{drawColor}{RGB}{255,0,0}
\definecolor{fillColor}{RGB}{255,0,0}

\path[draw=drawColor,line width= 0.4pt,line join=round,line cap=round,fill=fillColor] (650.49,206.45) circle (  1.49);
\definecolor{drawColor}{RGB}{0,0,0}
\definecolor{fillColor}{RGB}{0,0,0}

\path[draw=drawColor,line width= 0.4pt,line join=round,line cap=round,fill=fillColor] (650.50,221.56) circle (  1.49);
\definecolor{drawColor}{RGB}{255,0,0}
\definecolor{fillColor}{RGB}{255,0,0}

\path[draw=drawColor,line width= 0.4pt,line join=round,line cap=round,fill=fillColor] (651.08,206.45) circle (  1.49);
\definecolor{drawColor}{RGB}{0,0,0}
\definecolor{fillColor}{RGB}{0,0,0}

\path[draw=drawColor,line width= 0.4pt,line join=round,line cap=round,fill=fillColor] (651.09,222.22) circle (  1.49);
\definecolor{drawColor}{RGB}{255,0,0}
\definecolor{fillColor}{RGB}{255,0,0}

\path[draw=drawColor,line width= 0.4pt,line join=round,line cap=round,fill=fillColor] (651.58,206.78) circle (  1.49);
\definecolor{drawColor}{RGB}{0,0,0}
\definecolor{fillColor}{RGB}{0,0,0}

\path[draw=drawColor,line width= 0.4pt,line join=round,line cap=round,fill=fillColor] (651.60,222.55) circle (  1.49);
\definecolor{drawColor}{RGB}{255,0,0}
\definecolor{fillColor}{RGB}{255,0,0}

\path[draw=drawColor,line width= 0.4pt,line join=round,line cap=round,fill=fillColor] (652.06,206.78) circle (  1.49);
\definecolor{drawColor}{RGB}{0,0,0}
\definecolor{fillColor}{RGB}{0,0,0}

\path[draw=drawColor,line width= 0.4pt,line join=round,line cap=round,fill=fillColor] (652.08,223.21) circle (  1.49);
\definecolor{drawColor}{RGB}{255,0,0}
\definecolor{fillColor}{RGB}{255,0,0}

\path[draw=drawColor,line width= 0.4pt,line join=round,line cap=round,fill=fillColor] (652.62,207.11) circle (  1.49);
\definecolor{drawColor}{RGB}{0,0,0}
\definecolor{fillColor}{RGB}{0,0,0}

\path[draw=drawColor,line width= 0.4pt,line join=round,line cap=round,fill=fillColor] (652.63,222.55) circle (  1.49);
\definecolor{drawColor}{RGB}{255,0,0}
\definecolor{fillColor}{RGB}{255,0,0}

\path[draw=drawColor,line width= 0.4pt,line join=round,line cap=round,fill=fillColor] (653.12,206.78) circle (  1.49);
\definecolor{drawColor}{RGB}{0,0,0}
\definecolor{fillColor}{RGB}{0,0,0}

\path[draw=drawColor,line width= 0.4pt,line join=round,line cap=round,fill=fillColor] (653.14,222.55) circle (  1.49);
\definecolor{drawColor}{RGB}{255,0,0}
\definecolor{fillColor}{RGB}{255,0,0}

\path[draw=drawColor,line width= 0.4pt,line join=round,line cap=round,fill=fillColor] (653.60,205.80) circle (  1.49);
\definecolor{drawColor}{RGB}{0,0,0}
\definecolor{fillColor}{RGB}{0,0,0}

\path[draw=drawColor,line width= 0.4pt,line join=round,line cap=round,fill=fillColor] (653.61,222.22) circle (  1.49);
\definecolor{drawColor}{RGB}{255,0,0}
\definecolor{fillColor}{RGB}{255,0,0}

\path[draw=drawColor,line width= 0.4pt,line join=round,line cap=round,fill=fillColor] (654.07,205.47) circle (  1.49);
\definecolor{drawColor}{RGB}{0,0,0}
\definecolor{fillColor}{RGB}{0,0,0}

\path[draw=drawColor,line width= 0.4pt,line join=round,line cap=round,fill=fillColor] (654.09,222.55) circle (  1.49);
\definecolor{drawColor}{RGB}{255,0,0}
\definecolor{fillColor}{RGB}{255,0,0}

\path[draw=drawColor,line width= 0.4pt,line join=round,line cap=round,fill=fillColor] (654.55,205.47) circle (  1.49);
\definecolor{drawColor}{RGB}{0,0,0}
\definecolor{fillColor}{RGB}{0,0,0}

\path[draw=drawColor,line width= 0.4pt,line join=round,line cap=round,fill=fillColor] (654.56,220.91) circle (  1.49);
\definecolor{drawColor}{RGB}{255,0,0}
\definecolor{fillColor}{RGB}{255,0,0}

\path[draw=drawColor,line width= 0.4pt,line join=round,line cap=round,fill=fillColor] (655.02,205.47) circle (  1.49);
\definecolor{drawColor}{RGB}{0,0,0}
\definecolor{fillColor}{RGB}{0,0,0}

\path[draw=drawColor,line width= 0.4pt,line join=round,line cap=round,fill=fillColor] (655.04,221.24) circle (  1.49);
\definecolor{drawColor}{RGB}{255,0,0}
\definecolor{fillColor}{RGB}{255,0,0}

\path[draw=drawColor,line width= 0.4pt,line join=round,line cap=round,fill=fillColor] (655.55,204.81) circle (  1.49);
\definecolor{drawColor}{RGB}{0,0,0}
\definecolor{fillColor}{RGB}{0,0,0}

\path[draw=drawColor,line width= 0.4pt,line join=round,line cap=round,fill=fillColor] (655.56,220.58) circle (  1.49);
\definecolor{drawColor}{RGB}{255,0,0}
\definecolor{fillColor}{RGB}{255,0,0}

\path[draw=drawColor,line width= 0.4pt,line join=round,line cap=round,fill=fillColor] (656.02,205.14) circle (  1.49);
\definecolor{drawColor}{RGB}{0,0,0}
\definecolor{fillColor}{RGB}{0,0,0}

\path[draw=drawColor,line width= 0.4pt,line join=round,line cap=round,fill=fillColor] (656.04,221.56) circle (  1.49);
\definecolor{drawColor}{RGB}{255,0,0}
\definecolor{fillColor}{RGB}{255,0,0}

\path[draw=drawColor,line width= 0.4pt,line join=round,line cap=round,fill=fillColor] (656.64,206.13) circle (  1.49);
\definecolor{drawColor}{RGB}{0,0,0}
\definecolor{fillColor}{RGB}{0,0,0}

\path[draw=drawColor,line width= 0.4pt,line join=round,line cap=round,fill=fillColor] (656.66,220.58) circle (  1.49);
\definecolor{drawColor}{RGB}{255,0,0}
\definecolor{fillColor}{RGB}{255,0,0}

\path[draw=drawColor,line width= 0.4pt,line join=round,line cap=round,fill=fillColor] (657.22,206.45) circle (  1.49);
\definecolor{drawColor}{RGB}{0,0,0}
\definecolor{fillColor}{RGB}{0,0,0}

\path[draw=drawColor,line width= 0.4pt,line join=round,line cap=round,fill=fillColor] (657.23,223.53) circle (  1.49);
\definecolor{drawColor}{RGB}{255,0,0}
\definecolor{fillColor}{RGB}{255,0,0}

\path[draw=drawColor,line width= 0.4pt,line join=round,line cap=round,fill=fillColor] (657.77,207.11) circle (  1.49);
\definecolor{drawColor}{RGB}{0,0,0}
\definecolor{fillColor}{RGB}{0,0,0}

\path[draw=drawColor,line width= 0.4pt,line join=round,line cap=round,fill=fillColor] (657.79,223.53) circle (  1.49);
\definecolor{drawColor}{RGB}{255,0,0}
\definecolor{fillColor}{RGB}{255,0,0}

\path[draw=drawColor,line width= 0.4pt,line join=round,line cap=round,fill=fillColor] (658.23,206.78) circle (  1.49);
\definecolor{drawColor}{RGB}{0,0,0}
\definecolor{fillColor}{RGB}{0,0,0}

\path[draw=drawColor,line width= 0.4pt,line join=round,line cap=round,fill=fillColor] (658.25,222.55) circle (  1.49);
\definecolor{drawColor}{RGB}{255,0,0}
\definecolor{fillColor}{RGB}{255,0,0}

\path[draw=drawColor,line width= 0.4pt,line join=round,line cap=round,fill=fillColor] (658.69,205.47) circle (  1.49);
\definecolor{drawColor}{RGB}{0,0,0}
\definecolor{fillColor}{RGB}{0,0,0}

\path[draw=drawColor,line width= 0.4pt,line join=round,line cap=round,fill=fillColor] (658.71,222.22) circle (  1.49);
\definecolor{drawColor}{RGB}{255,0,0}
\definecolor{fillColor}{RGB}{255,0,0}

\path[draw=drawColor,line width= 0.4pt,line join=round,line cap=round,fill=fillColor] (659.15,205.14) circle (  1.49);
\definecolor{drawColor}{RGB}{0,0,0}
\definecolor{fillColor}{RGB}{0,0,0}

\path[draw=drawColor,line width= 0.4pt,line join=round,line cap=round,fill=fillColor] (659.16,221.56) circle (  1.49);
\definecolor{drawColor}{RGB}{255,0,0}
\definecolor{fillColor}{RGB}{255,0,0}

\path[draw=drawColor,line width= 0.4pt,line join=round,line cap=round,fill=fillColor] (659.62,205.80) circle (  1.49);
\definecolor{drawColor}{RGB}{0,0,0}
\definecolor{fillColor}{RGB}{0,0,0}

\path[draw=drawColor,line width= 0.4pt,line join=round,line cap=round,fill=fillColor] (659.64,222.22) circle (  1.49);
\definecolor{drawColor}{RGB}{255,0,0}
\definecolor{fillColor}{RGB}{255,0,0}

\path[draw=drawColor,line width= 0.4pt,line join=round,line cap=round,fill=fillColor] (660.08,205.14) circle (  1.49);
\definecolor{drawColor}{RGB}{0,0,0}
\definecolor{fillColor}{RGB}{0,0,0}

\path[draw=drawColor,line width= 0.4pt,line join=round,line cap=round,fill=fillColor] (660.10,221.24) circle (  1.49);
\definecolor{drawColor}{RGB}{255,0,0}
\definecolor{fillColor}{RGB}{255,0,0}

\path[draw=drawColor,line width= 0.4pt,line join=round,line cap=round,fill=fillColor] (660.55,205.14) circle (  1.49);
\definecolor{drawColor}{RGB}{0,0,0}
\definecolor{fillColor}{RGB}{0,0,0}

\path[draw=drawColor,line width= 0.4pt,line join=round,line cap=round,fill=fillColor] (660.57,222.22) circle (  1.49);
\definecolor{drawColor}{RGB}{255,0,0}
\definecolor{fillColor}{RGB}{255,0,0}

\path[draw=drawColor,line width= 0.4pt,line join=round,line cap=round,fill=fillColor] (661.16,204.81) circle (  1.49);
\definecolor{drawColor}{RGB}{0,0,0}
\definecolor{fillColor}{RGB}{0,0,0}

\path[draw=drawColor,line width= 0.4pt,line join=round,line cap=round,fill=fillColor] (661.18,222.22) circle (  1.49);
\definecolor{drawColor}{RGB}{255,0,0}
\definecolor{fillColor}{RGB}{255,0,0}

\path[draw=drawColor,line width= 0.4pt,line join=round,line cap=round,fill=fillColor] (661.68,204.81) circle (  1.49);
\definecolor{drawColor}{RGB}{0,0,0}
\definecolor{fillColor}{RGB}{0,0,0}

\path[draw=drawColor,line width= 0.4pt,line join=round,line cap=round,fill=fillColor] (661.70,221.89) circle (  1.49);
\definecolor{drawColor}{RGB}{255,0,0}
\definecolor{fillColor}{RGB}{255,0,0}

\path[draw=drawColor,line width= 0.4pt,line join=round,line cap=round,fill=fillColor] (662.24,204.16) circle (  1.49);
\definecolor{drawColor}{RGB}{0,0,0}
\definecolor{fillColor}{RGB}{0,0,0}

\path[draw=drawColor,line width= 0.4pt,line join=round,line cap=round,fill=fillColor] (662.26,220.91) circle (  1.49);
\definecolor{drawColor}{RGB}{255,0,0}
\definecolor{fillColor}{RGB}{255,0,0}

\path[draw=drawColor,line width= 0.4pt,line join=round,line cap=round,fill=fillColor] (662.70,204.16) circle (  1.49);
\definecolor{drawColor}{RGB}{0,0,0}
\definecolor{fillColor}{RGB}{0,0,0}

\path[draw=drawColor,line width= 0.4pt,line join=round,line cap=round,fill=fillColor] (662.73,220.91) circle (  1.49);
\definecolor{drawColor}{RGB}{255,0,0}
\definecolor{fillColor}{RGB}{255,0,0}

\path[draw=drawColor,line width= 0.4pt,line join=round,line cap=round,fill=fillColor] (663.14,204.81) circle (  1.49);
\definecolor{drawColor}{RGB}{0,0,0}
\definecolor{fillColor}{RGB}{0,0,0}

\path[draw=drawColor,line width= 0.4pt,line join=round,line cap=round,fill=fillColor] (663.17,221.89) circle (  1.49);
\definecolor{drawColor}{RGB}{255,0,0}
\definecolor{fillColor}{RGB}{255,0,0}

\path[draw=drawColor,line width= 0.4pt,line join=round,line cap=round,fill=fillColor] (663.63,205.47) circle (  1.49);
\definecolor{drawColor}{RGB}{0,0,0}
\definecolor{fillColor}{RGB}{0,0,0}

\path[draw=drawColor,line width= 0.4pt,line join=round,line cap=round,fill=fillColor] (663.65,221.89) circle (  1.49);
\definecolor{drawColor}{RGB}{255,0,0}
\definecolor{fillColor}{RGB}{255,0,0}

\path[draw=drawColor,line width= 0.4pt,line join=round,line cap=round,fill=fillColor] (664.19,206.13) circle (  1.49);
\definecolor{drawColor}{RGB}{0,0,0}
\definecolor{fillColor}{RGB}{0,0,0}

\path[draw=drawColor,line width= 0.4pt,line join=round,line cap=round,fill=fillColor] (664.21,222.55) circle (  1.49);
\definecolor{drawColor}{RGB}{255,0,0}
\definecolor{fillColor}{RGB}{255,0,0}

\path[draw=drawColor,line width= 0.4pt,line join=round,line cap=round,fill=fillColor] (664.68,205.14) circle (  1.49);
\definecolor{drawColor}{RGB}{0,0,0}
\definecolor{fillColor}{RGB}{0,0,0}

\path[draw=drawColor,line width= 0.4pt,line join=round,line cap=round,fill=fillColor] (664.70,220.91) circle (  1.49);
\definecolor{drawColor}{RGB}{255,0,0}
\definecolor{fillColor}{RGB}{255,0,0}

\path[draw=drawColor,line width= 0.4pt,line join=round,line cap=round,fill=fillColor] (665.15,205.47) circle (  1.49);
\definecolor{drawColor}{RGB}{0,0,0}
\definecolor{fillColor}{RGB}{0,0,0}

\path[draw=drawColor,line width= 0.4pt,line join=round,line cap=round,fill=fillColor] (665.17,218.61) circle (  1.49);
\definecolor{drawColor}{RGB}{255,0,0}
\definecolor{fillColor}{RGB}{255,0,0}

\path[draw=drawColor,line width= 0.4pt,line join=round,line cap=round,fill=fillColor] (665.63,205.80) circle (  1.49);
\definecolor{drawColor}{RGB}{0,0,0}
\definecolor{fillColor}{RGB}{0,0,0}

\path[draw=drawColor,line width= 0.4pt,line join=round,line cap=round,fill=fillColor] (665.65,222.22) circle (  1.49);
\definecolor{drawColor}{RGB}{255,0,0}
\definecolor{fillColor}{RGB}{255,0,0}

\path[draw=drawColor,line width= 0.4pt,line join=round,line cap=round,fill=fillColor] (666.09,205.14) circle (  1.49);
\definecolor{drawColor}{RGB}{0,0,0}
\definecolor{fillColor}{RGB}{0,0,0}

\path[draw=drawColor,line width= 0.4pt,line join=round,line cap=round,fill=fillColor] (666.12,221.89) circle (  1.49);
\definecolor{drawColor}{RGB}{255,0,0}
\definecolor{fillColor}{RGB}{255,0,0}

\path[draw=drawColor,line width= 0.4pt,line join=round,line cap=round,fill=fillColor] (666.56,205.80) circle (  1.49);
\definecolor{drawColor}{RGB}{0,0,0}
\definecolor{fillColor}{RGB}{0,0,0}

\path[draw=drawColor,line width= 0.4pt,line join=round,line cap=round,fill=fillColor] (666.58,222.22) circle (  1.49);
\definecolor{drawColor}{RGB}{255,0,0}
\definecolor{fillColor}{RGB}{255,0,0}

\path[draw=drawColor,line width= 0.4pt,line join=round,line cap=round,fill=fillColor] (667.04,205.47) circle (  1.49);
\definecolor{drawColor}{RGB}{0,0,0}
\definecolor{fillColor}{RGB}{0,0,0}

\path[draw=drawColor,line width= 0.4pt,line join=round,line cap=round,fill=fillColor] (667.05,222.22) circle (  1.49);
\definecolor{drawColor}{RGB}{255,0,0}
\definecolor{fillColor}{RGB}{255,0,0}

\path[draw=drawColor,line width= 0.4pt,line join=round,line cap=round,fill=fillColor] (667.51,206.45) circle (  1.49);
\definecolor{drawColor}{RGB}{0,0,0}
\definecolor{fillColor}{RGB}{0,0,0}

\path[draw=drawColor,line width= 0.4pt,line join=round,line cap=round,fill=fillColor] (667.53,221.24) circle (  1.49);
\definecolor{drawColor}{RGB}{255,0,0}
\definecolor{fillColor}{RGB}{255,0,0}

\path[draw=drawColor,line width= 0.4pt,line join=round,line cap=round,fill=fillColor] (668.00,205.47) circle (  1.49);
\definecolor{drawColor}{RGB}{0,0,0}
\definecolor{fillColor}{RGB}{0,0,0}

\path[draw=drawColor,line width= 0.4pt,line join=round,line cap=round,fill=fillColor] (668.02,219.92) circle (  1.49);
\definecolor{drawColor}{RGB}{255,0,0}
\definecolor{fillColor}{RGB}{255,0,0}

\path[draw=drawColor,line width= 0.4pt,line join=round,line cap=round,fill=fillColor] (668.46,205.80) circle (  1.49);
\definecolor{drawColor}{RGB}{0,0,0}
\definecolor{fillColor}{RGB}{0,0,0}

\path[draw=drawColor,line width= 0.4pt,line join=round,line cap=round,fill=fillColor] (668.49,220.91) circle (  1.49);
\definecolor{drawColor}{RGB}{255,0,0}
\definecolor{fillColor}{RGB}{255,0,0}

\path[draw=drawColor,line width= 0.4pt,line join=round,line cap=round,fill=fillColor] (668.94,204.81) circle (  1.49);
\definecolor{drawColor}{RGB}{0,0,0}
\definecolor{fillColor}{RGB}{0,0,0}

\path[draw=drawColor,line width= 0.4pt,line join=round,line cap=round,fill=fillColor] (668.95,221.56) circle (  1.49);
\definecolor{drawColor}{RGB}{255,0,0}
\definecolor{fillColor}{RGB}{255,0,0}

\path[draw=drawColor,line width= 0.4pt,line join=round,line cap=round,fill=fillColor] (669.46,205.14) circle (  1.49);
\definecolor{drawColor}{RGB}{0,0,0}
\definecolor{fillColor}{RGB}{0,0,0}

\path[draw=drawColor,line width= 0.4pt,line join=round,line cap=round,fill=fillColor] (669.48,219.59) circle (  1.49);
\definecolor{drawColor}{RGB}{255,0,0}
\definecolor{fillColor}{RGB}{255,0,0}

\path[draw=drawColor,line width= 0.4pt,line join=round,line cap=round,fill=fillColor] (669.97,205.14) circle (  1.49);
\definecolor{drawColor}{RGB}{0,0,0}
\definecolor{fillColor}{RGB}{0,0,0}

\path[draw=drawColor,line width= 0.4pt,line join=round,line cap=round,fill=fillColor] (669.98,219.59) circle (  1.49);
\definecolor{drawColor}{RGB}{255,0,0}
\definecolor{fillColor}{RGB}{255,0,0}

\path[draw=drawColor,line width= 0.4pt,line join=round,line cap=round,fill=fillColor] (670.44,205.80) circle (  1.49);
\definecolor{drawColor}{RGB}{0,0,0}
\definecolor{fillColor}{RGB}{0,0,0}

\path[draw=drawColor,line width= 0.4pt,line join=round,line cap=round,fill=fillColor] (670.46,222.55) circle (  1.49);
\definecolor{drawColor}{RGB}{255,0,0}
\definecolor{fillColor}{RGB}{255,0,0}

\path[draw=drawColor,line width= 0.4pt,line join=round,line cap=round,fill=fillColor] (670.93,207.11) circle (  1.49);
\definecolor{drawColor}{RGB}{0,0,0}
\definecolor{fillColor}{RGB}{0,0,0}

\path[draw=drawColor,line width= 0.4pt,line join=round,line cap=round,fill=fillColor] (670.95,224.52) circle (  1.49);
\definecolor{drawColor}{RGB}{255,0,0}
\definecolor{fillColor}{RGB}{255,0,0}

\path[draw=drawColor,line width= 0.4pt,line join=round,line cap=round,fill=fillColor] (671.41,208.10) circle (  1.49);
\definecolor{drawColor}{RGB}{0,0,0}
\definecolor{fillColor}{RGB}{0,0,0}

\path[draw=drawColor,line width= 0.4pt,line join=round,line cap=round,fill=fillColor] (671.42,224.85) circle (  1.49);
\definecolor{drawColor}{RGB}{255,0,0}
\definecolor{fillColor}{RGB}{255,0,0}

\path[draw=drawColor,line width= 0.4pt,line join=round,line cap=round,fill=fillColor] (671.88,207.44) circle (  1.49);
\definecolor{drawColor}{RGB}{0,0,0}
\definecolor{fillColor}{RGB}{0,0,0}

\path[draw=drawColor,line width= 0.4pt,line join=round,line cap=round,fill=fillColor] (671.90,224.19) circle (  1.49);
\definecolor{drawColor}{RGB}{255,0,0}
\definecolor{fillColor}{RGB}{255,0,0}

\path[draw=drawColor,line width= 0.4pt,line join=round,line cap=round,fill=fillColor] (672.34,207.44) circle (  1.49);
\definecolor{drawColor}{RGB}{0,0,0}
\definecolor{fillColor}{RGB}{0,0,0}

\path[draw=drawColor,line width= 0.4pt,line join=round,line cap=round,fill=fillColor] (672.36,224.52) circle (  1.49);
\definecolor{drawColor}{RGB}{255,0,0}
\definecolor{fillColor}{RGB}{255,0,0}

\path[draw=drawColor,line width= 0.4pt,line join=round,line cap=round,fill=fillColor] (672.78,207.11) circle (  1.49);
\definecolor{drawColor}{RGB}{0,0,0}
\definecolor{fillColor}{RGB}{0,0,0}

\path[draw=drawColor,line width= 0.4pt,line join=round,line cap=round,fill=fillColor] (672.80,224.52) circle (  1.49);
\definecolor{drawColor}{RGB}{255,0,0}
\definecolor{fillColor}{RGB}{255,0,0}

\path[draw=drawColor,line width= 0.4pt,line join=round,line cap=round,fill=fillColor] (673.27,207.77) circle (  1.49);
\definecolor{drawColor}{RGB}{0,0,0}
\definecolor{fillColor}{RGB}{0,0,0}

\path[draw=drawColor,line width= 0.4pt,line join=round,line cap=round,fill=fillColor] (673.29,223.86) circle (  1.49);
\definecolor{drawColor}{RGB}{255,0,0}
\definecolor{fillColor}{RGB}{255,0,0}

\path[draw=drawColor,line width= 0.4pt,line join=round,line cap=round,fill=fillColor] (673.73,208.10) circle (  1.49);
\definecolor{drawColor}{RGB}{0,0,0}
\definecolor{fillColor}{RGB}{0,0,0}

\path[draw=drawColor,line width= 0.4pt,line join=round,line cap=round,fill=fillColor] (673.75,224.52) circle (  1.49);
\definecolor{drawColor}{RGB}{255,0,0}
\definecolor{fillColor}{RGB}{255,0,0}

\path[draw=drawColor,line width= 0.4pt,line join=round,line cap=round,fill=fillColor] (674.19,207.77) circle (  1.49);
\definecolor{drawColor}{RGB}{0,0,0}
\definecolor{fillColor}{RGB}{0,0,0}

\path[draw=drawColor,line width= 0.4pt,line join=round,line cap=round,fill=fillColor] (674.21,226.16) circle (  1.49);
\definecolor{drawColor}{RGB}{255,0,0}
\definecolor{fillColor}{RGB}{255,0,0}

\path[draw=drawColor,line width= 0.4pt,line join=round,line cap=round,fill=fillColor] (674.65,207.77) circle (  1.49);
\definecolor{drawColor}{RGB}{0,0,0}
\definecolor{fillColor}{RGB}{0,0,0}

\path[draw=drawColor,line width= 0.4pt,line join=round,line cap=round,fill=fillColor] (674.67,224.85) circle (  1.49);
\definecolor{drawColor}{RGB}{255,0,0}
\definecolor{fillColor}{RGB}{255,0,0}

\path[draw=drawColor,line width= 0.4pt,line join=round,line cap=round,fill=fillColor] (675.11,207.77) circle (  1.49);
\definecolor{drawColor}{RGB}{0,0,0}
\definecolor{fillColor}{RGB}{0,0,0}

\path[draw=drawColor,line width= 0.4pt,line join=round,line cap=round,fill=fillColor] (675.12,225.50) circle (  1.49);
\definecolor{drawColor}{RGB}{255,0,0}
\definecolor{fillColor}{RGB}{255,0,0}

\path[draw=drawColor,line width= 0.4pt,line join=round,line cap=round,fill=fillColor] (675.58,206.78) circle (  1.49);
\definecolor{drawColor}{RGB}{0,0,0}
\definecolor{fillColor}{RGB}{0,0,0}

\path[draw=drawColor,line width= 0.4pt,line join=round,line cap=round,fill=fillColor] (675.60,223.86) circle (  1.49);
\definecolor{drawColor}{RGB}{255,0,0}
\definecolor{fillColor}{RGB}{255,0,0}

\path[draw=drawColor,line width= 0.4pt,line join=round,line cap=round,fill=fillColor] (676.06,206.78) circle (  1.49);
\definecolor{drawColor}{RGB}{0,0,0}
\definecolor{fillColor}{RGB}{0,0,0}

\path[draw=drawColor,line width= 0.4pt,line join=round,line cap=round,fill=fillColor] (676.07,224.52) circle (  1.49);
\definecolor{drawColor}{RGB}{255,0,0}
\definecolor{fillColor}{RGB}{255,0,0}

\path[draw=drawColor,line width= 0.4pt,line join=round,line cap=round,fill=fillColor] (676.53,206.13) circle (  1.49);
\definecolor{drawColor}{RGB}{0,0,0}
\definecolor{fillColor}{RGB}{0,0,0}

\path[draw=drawColor,line width= 0.4pt,line join=round,line cap=round,fill=fillColor] (676.55,223.86) circle (  1.49);
\definecolor{drawColor}{RGB}{255,0,0}
\definecolor{fillColor}{RGB}{255,0,0}

\path[draw=drawColor,line width= 0.4pt,line join=round,line cap=round,fill=fillColor] (676.99,205.47) circle (  1.49);
\definecolor{drawColor}{RGB}{0,0,0}
\definecolor{fillColor}{RGB}{0,0,0}

\path[draw=drawColor,line width= 0.4pt,line join=round,line cap=round,fill=fillColor] (677.01,222.88) circle (  1.49);
\definecolor{drawColor}{RGB}{255,0,0}
\definecolor{fillColor}{RGB}{255,0,0}

\path[draw=drawColor,line width= 0.4pt,line join=round,line cap=round,fill=fillColor] (677.46,205.47) circle (  1.49);
\definecolor{drawColor}{RGB}{0,0,0}
\definecolor{fillColor}{RGB}{0,0,0}

\path[draw=drawColor,line width= 0.4pt,line join=round,line cap=round,fill=fillColor] (677.48,222.88) circle (  1.49);
\definecolor{drawColor}{RGB}{255,0,0}
\definecolor{fillColor}{RGB}{255,0,0}

\path[draw=drawColor,line width= 0.4pt,line join=round,line cap=round,fill=fillColor] (677.92,205.14) circle (  1.49);
\definecolor{drawColor}{RGB}{0,0,0}
\definecolor{fillColor}{RGB}{0,0,0}

\path[draw=drawColor,line width= 0.4pt,line join=round,line cap=round,fill=fillColor] (677.94,223.53) circle (  1.49);
\definecolor{drawColor}{RGB}{255,0,0}
\definecolor{fillColor}{RGB}{255,0,0}

\path[draw=drawColor,line width= 0.4pt,line join=round,line cap=round,fill=fillColor] (678.36,205.47) circle (  1.49);
\definecolor{drawColor}{RGB}{0,0,0}
\definecolor{fillColor}{RGB}{0,0,0}

\path[draw=drawColor,line width= 0.4pt,line join=round,line cap=round,fill=fillColor] (678.38,221.24) circle (  1.49);
\definecolor{drawColor}{RGB}{255,0,0}
\definecolor{fillColor}{RGB}{255,0,0}

\path[draw=drawColor,line width= 0.4pt,line join=round,line cap=round,fill=fillColor] (678.84,205.80) circle (  1.49);
\definecolor{drawColor}{RGB}{0,0,0}
\definecolor{fillColor}{RGB}{0,0,0}

\path[draw=drawColor,line width= 0.4pt,line join=round,line cap=round,fill=fillColor] (678.86,223.86) circle (  1.49);
\definecolor{drawColor}{RGB}{255,0,0}
\definecolor{fillColor}{RGB}{255,0,0}

\path[draw=drawColor,line width= 0.4pt,line join=round,line cap=round,fill=fillColor] (679.31,205.80) circle (  1.49);
\definecolor{drawColor}{RGB}{0,0,0}
\definecolor{fillColor}{RGB}{0,0,0}

\path[draw=drawColor,line width= 0.4pt,line join=round,line cap=round,fill=fillColor] (679.33,223.53) circle (  1.49);
\definecolor{drawColor}{RGB}{255,0,0}
\definecolor{fillColor}{RGB}{255,0,0}

\path[draw=drawColor,line width= 0.4pt,line join=round,line cap=round,fill=fillColor] (679.79,205.80) circle (  1.49);
\definecolor{drawColor}{RGB}{0,0,0}
\definecolor{fillColor}{RGB}{0,0,0}

\path[draw=drawColor,line width= 0.4pt,line join=round,line cap=round,fill=fillColor] (679.81,223.53) circle (  1.49);
\definecolor{drawColor}{RGB}{255,0,0}
\definecolor{fillColor}{RGB}{255,0,0}

\path[draw=drawColor,line width= 0.4pt,line join=round,line cap=round,fill=fillColor] (680.33,206.13) circle (  1.49);
\definecolor{drawColor}{RGB}{0,0,0}
\definecolor{fillColor}{RGB}{0,0,0}

\path[draw=drawColor,line width= 0.4pt,line join=round,line cap=round,fill=fillColor] (680.35,224.19) circle (  1.49);
\definecolor{drawColor}{RGB}{255,0,0}
\definecolor{fillColor}{RGB}{255,0,0}

\path[draw=drawColor,line width= 0.4pt,line join=round,line cap=round,fill=fillColor] (680.82,205.80) circle (  1.49);
\definecolor{drawColor}{RGB}{0,0,0}
\definecolor{fillColor}{RGB}{0,0,0}

\path[draw=drawColor,line width= 0.4pt,line join=round,line cap=round,fill=fillColor] (680.84,224.19) circle (  1.49);
\definecolor{drawColor}{RGB}{255,0,0}
\definecolor{fillColor}{RGB}{255,0,0}

\path[draw=drawColor,line width= 0.4pt,line join=round,line cap=round,fill=fillColor] (681.28,205.80) circle (  1.49);
\definecolor{drawColor}{RGB}{0,0,0}
\definecolor{fillColor}{RGB}{0,0,0}

\path[draw=drawColor,line width= 0.4pt,line join=round,line cap=round,fill=fillColor] (681.30,221.24) circle (  1.49);
\definecolor{drawColor}{RGB}{255,0,0}
\definecolor{fillColor}{RGB}{255,0,0}

\path[draw=drawColor,line width= 0.4pt,line join=round,line cap=round,fill=fillColor] (681.70,205.80) circle (  1.49);
\definecolor{drawColor}{RGB}{0,0,0}
\definecolor{fillColor}{RGB}{0,0,0}

\path[draw=drawColor,line width= 0.4pt,line join=round,line cap=round,fill=fillColor] (681.72,222.55) circle (  1.49);
\definecolor{drawColor}{RGB}{255,0,0}
\definecolor{fillColor}{RGB}{255,0,0}

\path[draw=drawColor,line width= 0.4pt,line join=round,line cap=round,fill=fillColor] (682.24,205.47) circle (  1.49);
\definecolor{drawColor}{RGB}{0,0,0}
\definecolor{fillColor}{RGB}{0,0,0}

\path[draw=drawColor,line width= 0.4pt,line join=round,line cap=round,fill=fillColor] (682.28,221.24) circle (  1.49);
\definecolor{drawColor}{RGB}{255,0,0}
\definecolor{fillColor}{RGB}{255,0,0}

\path[draw=drawColor,line width= 0.4pt,line join=round,line cap=round,fill=fillColor] (682.75,204.81) circle (  1.49);
\definecolor{drawColor}{RGB}{0,0,0}
\definecolor{fillColor}{RGB}{0,0,0}

\path[draw=drawColor,line width= 0.4pt,line join=round,line cap=round,fill=fillColor] (682.77,222.88) circle (  1.49);
\definecolor{drawColor}{RGB}{255,0,0}
\definecolor{fillColor}{RGB}{255,0,0}

\path[draw=drawColor,line width= 0.4pt,line join=round,line cap=round,fill=fillColor] (683.24,205.14) circle (  1.49);
\definecolor{drawColor}{RGB}{0,0,0}
\definecolor{fillColor}{RGB}{0,0,0}

\path[draw=drawColor,line width= 0.4pt,line join=round,line cap=round,fill=fillColor] (683.26,221.24) circle (  1.49);
\definecolor{drawColor}{RGB}{255,0,0}
\definecolor{fillColor}{RGB}{255,0,0}

\path[draw=drawColor,line width= 0.4pt,line join=round,line cap=round,fill=fillColor] (683.73,206.13) circle (  1.49);
\definecolor{drawColor}{RGB}{0,0,0}
\definecolor{fillColor}{RGB}{0,0,0}

\path[draw=drawColor,line width= 0.4pt,line join=round,line cap=round,fill=fillColor] (683.75,223.21) circle (  1.49);
\definecolor{drawColor}{RGB}{255,0,0}
\definecolor{fillColor}{RGB}{255,0,0}

\path[draw=drawColor,line width= 0.4pt,line join=round,line cap=round,fill=fillColor] (684.18,205.80) circle (  1.49);
\definecolor{drawColor}{RGB}{0,0,0}
\definecolor{fillColor}{RGB}{0,0,0}

\path[draw=drawColor,line width= 0.4pt,line join=round,line cap=round,fill=fillColor] (684.19,223.53) circle (  1.49);
\definecolor{drawColor}{RGB}{255,0,0}
\definecolor{fillColor}{RGB}{255,0,0}

\path[draw=drawColor,line width= 0.4pt,line join=round,line cap=round,fill=fillColor] (684.67,205.14) circle (  1.49);
\definecolor{drawColor}{RGB}{0,0,0}
\definecolor{fillColor}{RGB}{0,0,0}

\path[draw=drawColor,line width= 0.4pt,line join=round,line cap=round,fill=fillColor] (684.72,217.29) circle (  1.49);
\definecolor{drawColor}{RGB}{255,0,0}
\definecolor{fillColor}{RGB}{255,0,0}

\path[draw=drawColor,line width= 0.4pt,line join=round,line cap=round,fill=fillColor] (685.17,205.80) circle (  1.49);
\definecolor{drawColor}{RGB}{0,0,0}
\definecolor{fillColor}{RGB}{0,0,0}

\path[draw=drawColor,line width= 0.4pt,line join=round,line cap=round,fill=fillColor] (685.19,221.56) circle (  1.49);
\definecolor{drawColor}{RGB}{255,0,0}
\definecolor{fillColor}{RGB}{255,0,0}

\path[draw=drawColor,line width= 0.4pt,line join=round,line cap=round,fill=fillColor] (685.62,205.80) circle (  1.49);
\definecolor{drawColor}{RGB}{0,0,0}
\definecolor{fillColor}{RGB}{0,0,0}

\path[draw=drawColor,line width= 0.4pt,line join=round,line cap=round,fill=fillColor] (685.63,224.52) circle (  1.49);
\definecolor{drawColor}{RGB}{255,0,0}
\definecolor{fillColor}{RGB}{255,0,0}

\path[draw=drawColor,line width= 0.4pt,line join=round,line cap=round,fill=fillColor] (686.11,207.11) circle (  1.49);
\definecolor{drawColor}{RGB}{0,0,0}
\definecolor{fillColor}{RGB}{0,0,0}

\path[draw=drawColor,line width= 0.4pt,line join=round,line cap=round,fill=fillColor] (686.12,224.52) circle (  1.49);
\definecolor{drawColor}{RGB}{255,0,0}
\definecolor{fillColor}{RGB}{255,0,0}

\path[draw=drawColor,line width= 0.4pt,line join=round,line cap=round,fill=fillColor] (686.57,206.45) circle (  1.49);
\definecolor{drawColor}{RGB}{0,0,0}
\definecolor{fillColor}{RGB}{0,0,0}

\path[draw=drawColor,line width= 0.4pt,line join=round,line cap=round,fill=fillColor] (686.58,223.21) circle (  1.49);
\definecolor{drawColor}{RGB}{255,0,0}
\definecolor{fillColor}{RGB}{255,0,0}

\path[draw=drawColor,line width= 0.4pt,line join=round,line cap=round,fill=fillColor] (687.06,206.13) circle (  1.49);
\definecolor{drawColor}{RGB}{0,0,0}
\definecolor{fillColor}{RGB}{0,0,0}

\path[draw=drawColor,line width= 0.4pt,line join=round,line cap=round,fill=fillColor] (687.07,224.52) circle (  1.49);
\definecolor{drawColor}{RGB}{255,0,0}
\definecolor{fillColor}{RGB}{255,0,0}

\path[draw=drawColor,line width= 0.4pt,line join=round,line cap=round,fill=fillColor] (687.58,205.47) circle (  1.49);
\definecolor{drawColor}{RGB}{0,0,0}
\definecolor{fillColor}{RGB}{0,0,0}

\path[draw=drawColor,line width= 0.4pt,line join=round,line cap=round,fill=fillColor] (687.60,224.52) circle (  1.49);
\definecolor{drawColor}{RGB}{255,0,0}
\definecolor{fillColor}{RGB}{255,0,0}

\path[draw=drawColor,line width= 0.4pt,line join=round,line cap=round,fill=fillColor] (688.06,205.47) circle (  1.49);
\definecolor{drawColor}{RGB}{0,0,0}
\definecolor{fillColor}{RGB}{0,0,0}

\path[draw=drawColor,line width= 0.4pt,line join=round,line cap=round,fill=fillColor] (688.07,222.88) circle (  1.49);
\definecolor{drawColor}{RGB}{255,0,0}
\definecolor{fillColor}{RGB}{255,0,0}

\path[draw=drawColor,line width= 0.4pt,line join=round,line cap=round,fill=fillColor] (688.55,205.47) circle (  1.49);
\definecolor{drawColor}{RGB}{0,0,0}
\definecolor{fillColor}{RGB}{0,0,0}

\path[draw=drawColor,line width= 0.4pt,line join=round,line cap=round,fill=fillColor] (688.56,222.88) circle (  1.49);
\definecolor{drawColor}{RGB}{255,0,0}
\definecolor{fillColor}{RGB}{255,0,0}

\path[draw=drawColor,line width= 0.4pt,line join=round,line cap=round,fill=fillColor] (689.01,205.14) circle (  1.49);
\definecolor{drawColor}{RGB}{0,0,0}
\definecolor{fillColor}{RGB}{0,0,0}

\path[draw=drawColor,line width= 0.4pt,line join=round,line cap=round,fill=fillColor] (689.02,223.53) circle (  1.49);
\definecolor{drawColor}{RGB}{255,0,0}
\definecolor{fillColor}{RGB}{255,0,0}

\path[draw=drawColor,line width= 0.4pt,line join=round,line cap=round,fill=fillColor] (689.51,205.80) circle (  1.49);
\definecolor{drawColor}{RGB}{0,0,0}
\definecolor{fillColor}{RGB}{0,0,0}

\path[draw=drawColor,line width= 0.4pt,line join=round,line cap=round,fill=fillColor] (689.53,223.53) circle (  1.49);
\definecolor{drawColor}{RGB}{255,0,0}
\definecolor{fillColor}{RGB}{255,0,0}

\path[draw=drawColor,line width= 0.4pt,line join=round,line cap=round,fill=fillColor] (689.97,205.47) circle (  1.49);
\definecolor{drawColor}{RGB}{0,0,0}
\definecolor{fillColor}{RGB}{0,0,0}

\path[draw=drawColor,line width= 0.4pt,line join=round,line cap=round,fill=fillColor] (689.99,223.21) circle (  1.49);
\definecolor{drawColor}{RGB}{255,0,0}
\definecolor{fillColor}{RGB}{255,0,0}

\path[draw=drawColor,line width= 0.4pt,line join=round,line cap=round,fill=fillColor] (690.43,205.14) circle (  1.49);
\definecolor{drawColor}{RGB}{0,0,0}
\definecolor{fillColor}{RGB}{0,0,0}

\path[draw=drawColor,line width= 0.4pt,line join=round,line cap=round,fill=fillColor] (690.45,224.19) circle (  1.49);
\definecolor{drawColor}{RGB}{255,0,0}
\definecolor{fillColor}{RGB}{255,0,0}

\path[draw=drawColor,line width= 0.4pt,line join=round,line cap=round,fill=fillColor] (690.95,205.47) circle (  1.49);
\definecolor{drawColor}{RGB}{0,0,0}
\definecolor{fillColor}{RGB}{0,0,0}

\path[draw=drawColor,line width= 0.4pt,line join=round,line cap=round,fill=fillColor] (690.97,223.53) circle (  1.49);
\definecolor{drawColor}{RGB}{255,0,0}
\definecolor{fillColor}{RGB}{255,0,0}

\path[draw=drawColor,line width= 0.4pt,line join=round,line cap=round,fill=fillColor] (691.41,204.16) circle (  1.49);
\definecolor{drawColor}{RGB}{0,0,0}
\definecolor{fillColor}{RGB}{0,0,0}

\path[draw=drawColor,line width= 0.4pt,line join=round,line cap=round,fill=fillColor] (691.43,222.22) circle (  1.49);
\definecolor{drawColor}{RGB}{255,0,0}
\definecolor{fillColor}{RGB}{255,0,0}

\path[draw=drawColor,line width= 0.4pt,line join=round,line cap=round,fill=fillColor] (691.85,205.80) circle (  1.49);
\definecolor{drawColor}{RGB}{0,0,0}
\definecolor{fillColor}{RGB}{0,0,0}

\path[draw=drawColor,line width= 0.4pt,line join=round,line cap=round,fill=fillColor] (691.87,224.19) circle (  1.49);
\definecolor{drawColor}{RGB}{255,0,0}
\definecolor{fillColor}{RGB}{255,0,0}

\path[draw=drawColor,line width= 0.4pt,line join=round,line cap=round,fill=fillColor] (692.30,206.78) circle (  1.49);
\definecolor{drawColor}{RGB}{0,0,0}
\definecolor{fillColor}{RGB}{0,0,0}

\path[draw=drawColor,line width= 0.4pt,line join=round,line cap=round,fill=fillColor] (692.31,222.55) circle (  1.49);
\definecolor{drawColor}{RGB}{255,0,0}
\definecolor{fillColor}{RGB}{255,0,0}

\path[draw=drawColor,line width= 0.4pt,line join=round,line cap=round,fill=fillColor] (692.72,208.43) circle (  1.49);
\definecolor{drawColor}{RGB}{0,0,0}
\definecolor{fillColor}{RGB}{0,0,0}

\path[draw=drawColor,line width= 0.4pt,line join=round,line cap=round,fill=fillColor] (692.74,222.55) circle (  1.49);
\definecolor{drawColor}{RGB}{255,0,0}
\definecolor{fillColor}{RGB}{255,0,0}

\path[draw=drawColor,line width= 0.4pt,line join=round,line cap=round,fill=fillColor] (693.15,207.44) circle (  1.49);
\definecolor{drawColor}{RGB}{0,0,0}
\definecolor{fillColor}{RGB}{0,0,0}

\path[draw=drawColor,line width= 0.4pt,line join=round,line cap=round,fill=fillColor] (693.18,220.58) circle (  1.49);
\definecolor{drawColor}{RGB}{255,0,0}
\definecolor{fillColor}{RGB}{255,0,0}

\path[draw=drawColor,line width= 0.4pt,line join=round,line cap=round,fill=fillColor] (693.65,205.47) circle (  1.49);
\definecolor{drawColor}{RGB}{0,0,0}
\definecolor{fillColor}{RGB}{0,0,0}

\path[draw=drawColor,line width= 0.4pt,line join=round,line cap=round,fill=fillColor] (693.69,223.21) circle (  1.49);
\definecolor{drawColor}{RGB}{255,0,0}
\definecolor{fillColor}{RGB}{255,0,0}

\path[draw=drawColor,line width= 0.4pt,line join=round,line cap=round,fill=fillColor] (694.18,206.13) circle (  1.49);
\definecolor{drawColor}{RGB}{0,0,0}
\definecolor{fillColor}{RGB}{0,0,0}

\path[draw=drawColor,line width= 0.4pt,line join=round,line cap=round,fill=fillColor] (694.19,224.52) circle (  1.49);
\definecolor{drawColor}{RGB}{255,0,0}
\definecolor{fillColor}{RGB}{255,0,0}

\path[draw=drawColor,line width= 0.4pt,line join=round,line cap=round,fill=fillColor] (694.65,205.80) circle (  1.49);
\definecolor{drawColor}{RGB}{0,0,0}
\definecolor{fillColor}{RGB}{0,0,0}

\path[draw=drawColor,line width= 0.4pt,line join=round,line cap=round,fill=fillColor] (694.67,222.55) circle (  1.49);
\definecolor{drawColor}{RGB}{255,0,0}
\definecolor{fillColor}{RGB}{255,0,0}

\path[draw=drawColor,line width= 0.4pt,line join=round,line cap=round,fill=fillColor] (695.11,205.80) circle (  1.49);
\definecolor{drawColor}{RGB}{0,0,0}
\definecolor{fillColor}{RGB}{0,0,0}

\path[draw=drawColor,line width= 0.4pt,line join=round,line cap=round,fill=fillColor] (695.13,221.56) circle (  1.49);
\definecolor{drawColor}{RGB}{255,0,0}
\definecolor{fillColor}{RGB}{255,0,0}

\path[draw=drawColor,line width= 0.4pt,line join=round,line cap=round,fill=fillColor] (695.63,206.13) circle (  1.49);
\definecolor{drawColor}{RGB}{0,0,0}
\definecolor{fillColor}{RGB}{0,0,0}

\path[draw=drawColor,line width= 0.4pt,line join=round,line cap=round,fill=fillColor] (695.65,221.56) circle (  1.49);
\definecolor{drawColor}{RGB}{255,0,0}
\definecolor{fillColor}{RGB}{255,0,0}

\path[draw=drawColor,line width= 0.4pt,line join=round,line cap=round,fill=fillColor] (696.08,207.44) circle (  1.49);
\definecolor{drawColor}{RGB}{0,0,0}
\definecolor{fillColor}{RGB}{0,0,0}

\path[draw=drawColor,line width= 0.4pt,line join=round,line cap=round,fill=fillColor] (696.09,223.53) circle (  1.49);
\definecolor{drawColor}{RGB}{255,0,0}
\definecolor{fillColor}{RGB}{255,0,0}

\path[draw=drawColor,line width= 0.4pt,line join=round,line cap=round,fill=fillColor] (696.50,205.47) circle (  1.49);
\definecolor{drawColor}{RGB}{0,0,0}
\definecolor{fillColor}{RGB}{0,0,0}

\path[draw=drawColor,line width= 0.4pt,line join=round,line cap=round,fill=fillColor] (696.52,221.56) circle (  1.49);
\definecolor{drawColor}{RGB}{255,0,0}
\definecolor{fillColor}{RGB}{255,0,0}

\path[draw=drawColor,line width= 0.4pt,line join=round,line cap=round,fill=fillColor] (697.03,203.50) circle (  1.49);
\definecolor{drawColor}{RGB}{0,0,0}
\definecolor{fillColor}{RGB}{0,0,0}

\path[draw=drawColor,line width= 0.4pt,line join=round,line cap=round,fill=fillColor] (697.06,220.91) circle (  1.49);
\definecolor{drawColor}{RGB}{255,0,0}
\definecolor{fillColor}{RGB}{255,0,0}

\path[draw=drawColor,line width= 0.4pt,line join=round,line cap=round,fill=fillColor] (697.50,204.16) circle (  1.49);
\definecolor{drawColor}{RGB}{0,0,0}
\definecolor{fillColor}{RGB}{0,0,0}

\path[draw=drawColor,line width= 0.4pt,line join=round,line cap=round,fill=fillColor] (697.55,222.55) circle (  1.49);
\definecolor{drawColor}{RGB}{255,0,0}
\definecolor{fillColor}{RGB}{255,0,0}

\path[draw=drawColor,line width= 0.4pt,line join=round,line cap=round,fill=fillColor] (698.06,204.16) circle (  1.49);
\definecolor{drawColor}{RGB}{0,0,0}
\definecolor{fillColor}{RGB}{0,0,0}

\path[draw=drawColor,line width= 0.4pt,line join=round,line cap=round,fill=fillColor] (698.07,222.55) circle (  1.49);
\definecolor{drawColor}{RGB}{255,0,0}
\definecolor{fillColor}{RGB}{255,0,0}

\path[draw=drawColor,line width= 0.4pt,line join=round,line cap=round,fill=fillColor] (698.50,202.84) circle (  1.49);
\definecolor{drawColor}{RGB}{0,0,0}
\definecolor{fillColor}{RGB}{0,0,0}

\path[draw=drawColor,line width= 0.4pt,line join=round,line cap=round,fill=fillColor] (698.53,217.95) circle (  1.49);
\definecolor{drawColor}{RGB}{255,0,0}
\definecolor{fillColor}{RGB}{255,0,0}

\path[draw=drawColor,line width= 0.4pt,line join=round,line cap=round,fill=fillColor] (698.96,203.83) circle (  1.49);
\definecolor{drawColor}{RGB}{0,0,0}
\definecolor{fillColor}{RGB}{0,0,0}

\path[draw=drawColor,line width= 0.4pt,line join=round,line cap=round,fill=fillColor] (698.99,222.55) circle (  1.49);
\definecolor{drawColor}{RGB}{255,0,0}
\definecolor{fillColor}{RGB}{255,0,0}

\path[draw=drawColor,line width= 0.4pt,line join=round,line cap=round,fill=fillColor] (699.40,203.50) circle (  1.49);
\definecolor{drawColor}{RGB}{0,0,0}
\definecolor{fillColor}{RGB}{0,0,0}

\path[draw=drawColor,line width= 0.4pt,line join=round,line cap=round,fill=fillColor] (699.42,223.21) circle (  1.49);
\definecolor{drawColor}{RGB}{255,0,0}
\definecolor{fillColor}{RGB}{255,0,0}

\path[draw=drawColor,line width= 0.4pt,line join=round,line cap=round,fill=fillColor] (700.02,204.48) circle (  1.49);
\definecolor{drawColor}{RGB}{0,0,0}
\definecolor{fillColor}{RGB}{0,0,0}

\path[draw=drawColor,line width= 0.4pt,line join=round,line cap=round,fill=fillColor] (700.04,222.88) circle (  1.49);
\definecolor{drawColor}{RGB}{255,0,0}
\definecolor{fillColor}{RGB}{255,0,0}

\path[draw=drawColor,line width= 0.4pt,line join=round,line cap=round,fill=fillColor] (700.59,204.48) circle (  1.49);
\definecolor{drawColor}{RGB}{0,0,0}
\definecolor{fillColor}{RGB}{0,0,0}

\path[draw=drawColor,line width= 0.4pt,line join=round,line cap=round,fill=fillColor] (700.61,221.56) circle (  1.49);
\definecolor{drawColor}{RGB}{255,0,0}
\definecolor{fillColor}{RGB}{255,0,0}

\path[draw=drawColor,line width= 0.4pt,line join=round,line cap=round,fill=fillColor] (701.09,204.48) circle (  1.49);
\definecolor{drawColor}{RGB}{0,0,0}
\definecolor{fillColor}{RGB}{0,0,0}

\path[draw=drawColor,line width= 0.4pt,line join=round,line cap=round,fill=fillColor] (701.10,223.86) circle (  1.49);
\definecolor{drawColor}{RGB}{255,0,0}
\definecolor{fillColor}{RGB}{255,0,0}

\path[draw=drawColor,line width= 0.4pt,line join=round,line cap=round,fill=fillColor] (701.63,205.14) circle (  1.49);
\definecolor{drawColor}{RGB}{0,0,0}
\definecolor{fillColor}{RGB}{0,0,0}

\path[draw=drawColor,line width= 0.4pt,line join=round,line cap=round,fill=fillColor] (701.64,223.86) circle (  1.49);
\definecolor{drawColor}{RGB}{255,0,0}
\definecolor{fillColor}{RGB}{255,0,0}

\path[draw=drawColor,line width= 0.4pt,line join=round,line cap=round,fill=fillColor] (702.10,204.48) circle (  1.49);
\definecolor{drawColor}{RGB}{0,0,0}
\definecolor{fillColor}{RGB}{0,0,0}

\path[draw=drawColor,line width= 0.4pt,line join=round,line cap=round,fill=fillColor] (702.12,223.21) circle (  1.49);
\definecolor{drawColor}{RGB}{255,0,0}
\definecolor{fillColor}{RGB}{255,0,0}

\path[draw=drawColor,line width= 0.4pt,line join=round,line cap=round,fill=fillColor] (702.54,203.50) circle (  1.49);
\definecolor{drawColor}{RGB}{0,0,0}
\definecolor{fillColor}{RGB}{0,0,0}

\path[draw=drawColor,line width= 0.4pt,line join=round,line cap=round,fill=fillColor] (702.56,221.56) circle (  1.49);
\definecolor{drawColor}{RGB}{255,0,0}
\definecolor{fillColor}{RGB}{255,0,0}

\path[draw=drawColor,line width= 0.4pt,line join=round,line cap=round,fill=fillColor] (703.00,204.48) circle (  1.49);
\definecolor{drawColor}{RGB}{0,0,0}
\definecolor{fillColor}{RGB}{0,0,0}

\path[draw=drawColor,line width= 0.4pt,line join=round,line cap=round,fill=fillColor] (703.03,216.64) circle (  1.49);
\definecolor{drawColor}{RGB}{255,0,0}
\definecolor{fillColor}{RGB}{255,0,0}

\path[draw=drawColor,line width= 0.4pt,line join=round,line cap=round,fill=fillColor] (703.46,204.16) circle (  1.49);
\definecolor{drawColor}{RGB}{0,0,0}
\definecolor{fillColor}{RGB}{0,0,0}

\path[draw=drawColor,line width= 0.4pt,line join=round,line cap=round,fill=fillColor] (703.48,223.86) circle (  1.49);
\definecolor{drawColor}{RGB}{255,0,0}
\definecolor{fillColor}{RGB}{255,0,0}

\path[draw=drawColor,line width= 0.4pt,line join=round,line cap=round,fill=fillColor] (703.93,203.50) circle (  1.49);
\definecolor{drawColor}{RGB}{0,0,0}
\definecolor{fillColor}{RGB}{0,0,0}

\path[draw=drawColor,line width= 0.4pt,line join=round,line cap=round,fill=fillColor] (703.95,223.53) circle (  1.49);
\definecolor{drawColor}{RGB}{255,0,0}
\definecolor{fillColor}{RGB}{255,0,0}

\path[draw=drawColor,line width= 0.4pt,line join=round,line cap=round,fill=fillColor] (704.34,204.16) circle (  1.49);
\definecolor{drawColor}{RGB}{0,0,0}
\definecolor{fillColor}{RGB}{0,0,0}

\path[draw=drawColor,line width= 0.4pt,line join=round,line cap=round,fill=fillColor] (704.36,214.67) circle (  1.49);
\definecolor{drawColor}{RGB}{255,0,0}
\definecolor{fillColor}{RGB}{255,0,0}

\path[draw=drawColor,line width= 0.4pt,line join=round,line cap=round,fill=fillColor] (704.95,203.17) circle (  1.49);
\definecolor{drawColor}{RGB}{0,0,0}
\definecolor{fillColor}{RGB}{0,0,0}

\path[draw=drawColor,line width= 0.4pt,line join=round,line cap=round,fill=fillColor] (704.97,222.55) circle (  1.49);
\definecolor{drawColor}{RGB}{255,0,0}
\definecolor{fillColor}{RGB}{255,0,0}

\path[draw=drawColor,line width= 0.4pt,line join=round,line cap=round,fill=fillColor] (705.55,204.48) circle (  1.49);
\definecolor{drawColor}{RGB}{0,0,0}
\definecolor{fillColor}{RGB}{0,0,0}

\path[draw=drawColor,line width= 0.4pt,line join=round,line cap=round,fill=fillColor] (705.57,223.86) circle (  1.49);
\definecolor{drawColor}{RGB}{255,0,0}
\definecolor{fillColor}{RGB}{255,0,0}

\path[draw=drawColor,line width= 0.4pt,line join=round,line cap=round,fill=fillColor] (705.98,204.48) circle (  1.49);
\definecolor{drawColor}{RGB}{0,0,0}
\definecolor{fillColor}{RGB}{0,0,0}

\path[draw=drawColor,line width= 0.4pt,line join=round,line cap=round,fill=fillColor] (706.00,223.21) circle (  1.49);
\definecolor{drawColor}{RGB}{255,0,0}
\definecolor{fillColor}{RGB}{255,0,0}

\path[draw=drawColor,line width= 0.4pt,line join=round,line cap=round,fill=fillColor] (706.41,204.16) circle (  1.49);
\definecolor{drawColor}{RGB}{0,0,0}
\definecolor{fillColor}{RGB}{0,0,0}

\path[draw=drawColor,line width= 0.4pt,line join=round,line cap=round,fill=fillColor] (706.42,222.88) circle (  1.49);
\definecolor{drawColor}{RGB}{255,0,0}
\definecolor{fillColor}{RGB}{255,0,0}

\path[draw=drawColor,line width= 0.4pt,line join=round,line cap=round,fill=fillColor] (706.83,203.83) circle (  1.49);
\definecolor{drawColor}{RGB}{0,0,0}
\definecolor{fillColor}{RGB}{0,0,0}

\path[draw=drawColor,line width= 0.4pt,line join=round,line cap=round,fill=fillColor] (706.85,222.22) circle (  1.49);
\definecolor{drawColor}{RGB}{255,0,0}
\definecolor{fillColor}{RGB}{255,0,0}

\path[draw=drawColor,line width= 0.4pt,line join=round,line cap=round,fill=fillColor] (707.26,203.50) circle (  1.49);
\definecolor{drawColor}{RGB}{0,0,0}
\definecolor{fillColor}{RGB}{0,0,0}

\path[draw=drawColor,line width= 0.4pt,line join=round,line cap=round,fill=fillColor] (707.27,222.55) circle (  1.49);
\definecolor{drawColor}{RGB}{255,0,0}
\definecolor{fillColor}{RGB}{255,0,0}

\path[draw=drawColor,line width= 0.4pt,line join=round,line cap=round,fill=fillColor] (707.68,203.50) circle (  1.49);
\definecolor{drawColor}{RGB}{0,0,0}
\definecolor{fillColor}{RGB}{0,0,0}

\path[draw=drawColor,line width= 0.4pt,line join=round,line cap=round,fill=fillColor] (707.72,222.88) circle (  1.49);
\end{scope}
\begin{scope}
\path[clip] (  0.00,  0.00) rectangle (722.70,289.08);
\definecolor{drawColor}{RGB}{0,0,0}

\path[draw=drawColor,line width= 0.4pt,line join=round,line cap=round] (530.91, 47.52) -- (707.70, 47.52);

\path[draw=drawColor,line width= 0.4pt,line join=round,line cap=round] (530.91, 47.52) -- (530.91, 43.56);

\path[draw=drawColor,line width= 0.4pt,line join=round,line cap=round] (560.37, 47.52) -- (560.37, 43.56);

\path[draw=drawColor,line width= 0.4pt,line join=round,line cap=round] (589.84, 47.52) -- (589.84, 43.56);

\path[draw=drawColor,line width= 0.4pt,line join=round,line cap=round] (619.30, 47.52) -- (619.30, 43.56);

\path[draw=drawColor,line width= 0.4pt,line join=round,line cap=round] (648.77, 47.52) -- (648.77, 43.56);

\path[draw=drawColor,line width= 0.4pt,line join=round,line cap=round] (678.23, 47.52) -- (678.23, 43.56);

\path[draw=drawColor,line width= 0.4pt,line join=round,line cap=round] (707.70, 47.52) -- (707.70, 43.56);

\node[text=drawColor,anchor=base,inner sep=0pt, outer sep=0pt, scale=  0.99] at (530.91, 33.26) {11:00};

\node[text=drawColor,anchor=base,inner sep=0pt, outer sep=0pt, scale=  0.99] at (589.84, 33.26) {12:00};

\node[text=drawColor,anchor=base,inner sep=0pt, outer sep=0pt, scale=  0.99] at (648.77, 33.26) {13:00};

\node[text=drawColor,anchor=base,inner sep=0pt, outer sep=0pt, scale=  0.99] at (707.70, 33.26) {14:00};

\path[draw=drawColor,line width= 0.4pt,line join=round,line cap=round] (524.04, 52.74) -- (524.04,216.97);

\path[draw=drawColor,line width= 0.4pt,line join=round,line cap=round] (524.04, 52.74) -- (520.08, 52.74);

\path[draw=drawColor,line width= 0.4pt,line join=round,line cap=round] (524.04, 85.58) -- (520.08, 85.58);

\path[draw=drawColor,line width= 0.4pt,line join=round,line cap=round] (524.04,118.43) -- (520.08,118.43);

\path[draw=drawColor,line width= 0.4pt,line join=round,line cap=round] (524.04,151.27) -- (520.08,151.27);

\path[draw=drawColor,line width= 0.4pt,line join=round,line cap=round] (524.04,184.12) -- (520.08,184.12);

\path[draw=drawColor,line width= 0.4pt,line join=round,line cap=round] (524.04,216.97) -- (520.08,216.97);

\node[text=drawColor,rotate= 90.00,anchor=base,inner sep=0pt, outer sep=0pt, scale=  0.99] at (514.54, 52.74) {0.6};

\node[text=drawColor,rotate= 90.00,anchor=base,inner sep=0pt, outer sep=0pt, scale=  0.99] at (514.54, 85.58) {0.7};

\node[text=drawColor,rotate= 90.00,anchor=base,inner sep=0pt, outer sep=0pt, scale=  0.99] at (514.54,118.43) {0.8};

\node[text=drawColor,rotate= 90.00,anchor=base,inner sep=0pt, outer sep=0pt, scale=  0.99] at (514.54,151.27) {0.9};

\node[text=drawColor,rotate= 90.00,anchor=base,inner sep=0pt, outer sep=0pt, scale=  0.99] at (514.54,184.12) {1.0};

\node[text=drawColor,rotate= 90.00,anchor=base,inner sep=0pt, outer sep=0pt, scale=  0.99] at (514.54,216.97) {1.1};

\path[draw=drawColor,line width= 0.4pt,line join=round,line cap=round] (524.04, 47.52) --
	(714.78, 47.52) --
	(714.78,241.56) --
	(524.04,241.56) --
	(524.04, 47.52);
\end{scope}
\begin{scope}
\path[clip] (484.44,  7.92) rectangle (722.70,281.16);
\definecolor{drawColor}{RGB}{0,0,0}

\node[text=drawColor,anchor=base,inner sep=0pt, outer sep=0pt, scale=  1.32] at (619.41,256.75) {\bfseries ZETA transmis};

\node[text=drawColor,anchor=base,inner sep=0pt, outer sep=0pt, scale=  0.99] at (619.41, 17.42) {Temps UTC};

\node[text=drawColor,rotate= 90.00,anchor=base,inner sep=0pt, outer sep=0pt, scale=  0.99] at (498.70,144.54) {ZETA};
\end{scope}
\begin{scope}
\path[clip] (524.04, 47.52) rectangle (714.78,241.56);
\definecolor{drawColor}{RGB}{0,255,0}
\definecolor{fillColor}{RGB}{0,255,0}

\path[draw=drawColor,line width= 0.4pt,line join=round,line cap=round,fill=fillColor] (531.25,172.95) circle (  1.49);

\path[draw=drawColor,line width= 0.4pt,line join=round,line cap=round,fill=fillColor] (531.71,172.29) circle (  1.49);

\path[draw=drawColor,line width= 0.4pt,line join=round,line cap=round,fill=fillColor] (532.15,172.95) circle (  1.49);

\path[draw=drawColor,line width= 0.4pt,line join=round,line cap=round,fill=fillColor] (532.61,174.27) circle (  1.49);

\path[draw=drawColor,line width= 0.4pt,line join=round,line cap=round,fill=fillColor] (533.05,176.24) circle (  1.49);

\path[draw=drawColor,line width= 0.4pt,line join=round,line cap=round,fill=fillColor] (533.51,176.24) circle (  1.49);

\path[draw=drawColor,line width= 0.4pt,line join=round,line cap=round,fill=fillColor] (533.95,176.24) circle (  1.49);

\path[draw=drawColor,line width= 0.4pt,line join=round,line cap=round,fill=fillColor] (534.43,176.56) circle (  1.49);

\path[draw=drawColor,line width= 0.4pt,line join=round,line cap=round,fill=fillColor] (534.89,175.25) circle (  1.49);

\path[draw=drawColor,line width= 0.4pt,line join=round,line cap=round,fill=fillColor] (535.34,173.94) circle (  1.49);

\path[draw=drawColor,line width= 0.4pt,line join=round,line cap=round,fill=fillColor] (535.79,173.28) circle (  1.49);

\path[draw=drawColor,line width= 0.4pt,line join=round,line cap=round,fill=fillColor] (536.26,173.61) circle (  1.49);

\path[draw=drawColor,line width= 0.4pt,line join=round,line cap=round,fill=fillColor] (536.70,174.92) circle (  1.49);

\path[draw=drawColor,line width= 0.4pt,line join=round,line cap=round,fill=fillColor] (537.16,175.58) circle (  1.49);

\path[draw=drawColor,line width= 0.4pt,line join=round,line cap=round,fill=fillColor] (537.62,174.92) circle (  1.49);

\path[draw=drawColor,line width= 0.4pt,line join=round,line cap=round,fill=fillColor] (538.08,174.59) circle (  1.49);

\path[draw=drawColor,line width= 0.4pt,line join=round,line cap=round,fill=fillColor] (538.59,172.95) circle (  1.49);

\path[draw=drawColor,line width= 0.4pt,line join=round,line cap=round,fill=fillColor] (539.04,170.32) circle (  1.49);

\path[draw=drawColor,line width= 0.4pt,line join=round,line cap=round,fill=fillColor] (539.50,170.65) circle (  1.49);

\path[draw=drawColor,line width= 0.4pt,line join=round,line cap=round,fill=fillColor] (539.94,173.61) circle (  1.49);

\path[draw=drawColor,line width= 0.4pt,line join=round,line cap=round,fill=fillColor] (540.40,174.92) circle (  1.49);

\path[draw=drawColor,line width= 0.4pt,line join=round,line cap=round,fill=fillColor] (540.86,173.94) circle (  1.49);

\path[draw=drawColor,line width= 0.4pt,line join=round,line cap=round,fill=fillColor] (541.32,172.29) circle (  1.49);

\path[draw=drawColor,line width= 0.4pt,line join=round,line cap=round,fill=fillColor] (541.79,172.62) circle (  1.49);

\path[draw=drawColor,line width= 0.4pt,line join=round,line cap=round,fill=fillColor] (542.25,172.62) circle (  1.49);

\path[draw=drawColor,line width= 0.4pt,line join=round,line cap=round,fill=fillColor] (542.69,172.95) circle (  1.49);

\path[draw=drawColor,line width= 0.4pt,line join=round,line cap=round,fill=fillColor] (543.17,172.29) circle (  1.49);

\path[draw=drawColor,line width= 0.4pt,line join=round,line cap=round,fill=fillColor] (543.63,173.61) circle (  1.49);

\path[draw=drawColor,line width= 0.4pt,line join=round,line cap=round,fill=fillColor] (544.09,174.27) circle (  1.49);

\path[draw=drawColor,line width= 0.4pt,line join=round,line cap=round,fill=fillColor] (544.56,174.59) circle (  1.49);

\path[draw=drawColor,line width= 0.4pt,line join=round,line cap=round,fill=fillColor] (545.02,174.27) circle (  1.49);

\path[draw=drawColor,line width= 0.4pt,line join=round,line cap=round,fill=fillColor] (545.48,174.59) circle (  1.49);

\path[draw=drawColor,line width= 0.4pt,line join=round,line cap=round,fill=fillColor] (545.94,174.27) circle (  1.49);

\path[draw=drawColor,line width= 0.4pt,line join=round,line cap=round,fill=fillColor] (546.39,173.28) circle (  1.49);

\path[draw=drawColor,line width= 0.4pt,line join=round,line cap=round,fill=fillColor] (546.85,172.62) circle (  1.49);

\path[draw=drawColor,line width= 0.4pt,line join=round,line cap=round,fill=fillColor] (547.31,170.65) circle (  1.49);

\path[draw=drawColor,line width= 0.4pt,line join=round,line cap=round,fill=fillColor] (547.77,171.97) circle (  1.49);

\path[draw=drawColor,line width= 0.4pt,line join=round,line cap=round,fill=fillColor] (548.23,169.67) circle (  1.49);

\path[draw=drawColor,line width= 0.4pt,line join=round,line cap=round,fill=fillColor] (548.69,171.64) circle (  1.49);

\path[draw=drawColor,line width= 0.4pt,line join=round,line cap=round,fill=fillColor] (549.14,171.31) circle (  1.49);

\path[draw=drawColor,line width= 0.4pt,line join=round,line cap=round,fill=fillColor] (549.59,169.67) circle (  1.49);

\path[draw=drawColor,line width= 0.4pt,line join=round,line cap=round,fill=fillColor] (550.04,164.74) circle (  1.49);

\path[draw=drawColor,line width= 0.4pt,line join=round,line cap=round,fill=fillColor] (550.49,160.47) circle (  1.49);

\path[draw=drawColor,line width= 0.4pt,line join=round,line cap=round,fill=fillColor] (550.93,165.07) circle (  1.49);

\path[draw=drawColor,line width= 0.4pt,line join=round,line cap=round,fill=fillColor] (551.37,166.38) circle (  1.49);

\path[draw=drawColor,line width= 0.4pt,line join=round,line cap=round,fill=fillColor] (551.83,169.01) circle (  1.49);

\path[draw=drawColor,line width= 0.4pt,line join=round,line cap=round,fill=fillColor] (552.29,168.35) circle (  1.49);

\path[draw=drawColor,line width= 0.4pt,line join=round,line cap=round,fill=fillColor] (552.74,165.40) circle (  1.49);

\path[draw=drawColor,line width= 0.4pt,line join=round,line cap=round,fill=fillColor] (553.19,137.81) circle (  1.49);

\path[draw=drawColor,line width= 0.4pt,line join=round,line cap=round,fill=fillColor] (553.65,135.18) circle (  1.49);

\path[draw=drawColor,line width= 0.4pt,line join=round,line cap=round,fill=fillColor] (554.10,134.85) circle (  1.49);

\path[draw=drawColor,line width= 0.4pt,line join=round,line cap=round,fill=fillColor] (554.61,143.06) circle (  1.49);

\path[draw=drawColor,line width= 0.4pt,line join=round,line cap=round,fill=fillColor] (555.07,147.33) circle (  1.49);

\path[draw=drawColor,line width= 0.4pt,line join=round,line cap=round,fill=fillColor] (555.51,149.63) circle (  1.49);

\path[draw=drawColor,line width= 0.4pt,line join=round,line cap=round,fill=fillColor] (555.99,167.70) circle (  1.49);

\path[draw=drawColor,line width= 0.4pt,line join=round,line cap=round,fill=fillColor] (556.46,166.71) circle (  1.49);

\path[draw=drawColor,line width= 0.4pt,line join=round,line cap=round,fill=fillColor] (556.94,170.32) circle (  1.49);

\path[draw=drawColor,line width= 0.4pt,line join=round,line cap=round,fill=fillColor] (557.43,168.68) circle (  1.49);

\path[draw=drawColor,line width= 0.4pt,line join=round,line cap=round,fill=fillColor] (557.93,166.71) circle (  1.49);

\path[draw=drawColor,line width= 0.4pt,line join=round,line cap=round,fill=fillColor] (558.41,159.16) circle (  1.49);

\path[draw=drawColor,line width= 0.4pt,line join=round,line cap=round,fill=fillColor] (558.88,137.15) circle (  1.49);

\path[draw=drawColor,line width= 0.4pt,line join=round,line cap=round,fill=fillColor] (559.37,110.22) circle (  1.49);

\path[draw=drawColor,line width= 0.4pt,line join=round,line cap=round,fill=fillColor] (559.83,119.08) circle (  1.49);

\path[draw=drawColor,line width= 0.4pt,line join=round,line cap=round,fill=fillColor] (560.31,117.44) circle (  1.49);

\path[draw=drawColor,line width= 0.4pt,line join=round,line cap=round,fill=fillColor] (560.77,116.13) circle (  1.49);

\path[draw=drawColor,line width= 0.4pt,line join=round,line cap=round,fill=fillColor] (561.26,111.53) circle (  1.49);

\path[draw=drawColor,line width= 0.4pt,line join=round,line cap=round,fill=fillColor] (561.75,100.36) circle (  1.49);

\path[draw=drawColor,line width= 0.4pt,line join=round,line cap=round,fill=fillColor] (562.24,156.53) circle (  1.49);

\path[draw=drawColor,line width= 0.4pt,line join=round,line cap=round,fill=fillColor] (562.73,125.98) circle (  1.49);

\path[draw=drawColor,line width= 0.4pt,line join=round,line cap=round,fill=fillColor] (563.22,136.49) circle (  1.49);

\path[draw=drawColor,line width= 0.4pt,line join=round,line cap=round,fill=fillColor] (563.78,104.96) circle (  1.49);

\path[draw=drawColor,line width= 0.4pt,line join=round,line cap=round,fill=fillColor] (564.27,112.84) circle (  1.49);

\path[draw=drawColor,line width= 0.4pt,line join=round,line cap=round,fill=fillColor] (564.78,114.49) circle (  1.49);

\path[draw=drawColor,line width= 0.4pt,line join=round,line cap=round,fill=fillColor] (565.28,129.92) circle (  1.49);

\path[draw=drawColor,line width= 0.4pt,line join=round,line cap=round,fill=fillColor] (565.76,148.65) circle (  1.49);

\path[draw=drawColor,line width= 0.4pt,line join=round,line cap=round,fill=fillColor] (566.25,114.16) circle (  1.49);

\path[draw=drawColor,line width= 0.4pt,line join=round,line cap=round,fill=fillColor] (566.72,132.88) circle (  1.49);

\path[draw=drawColor,line width= 0.4pt,line join=round,line cap=round,fill=fillColor] (567.22,130.58) circle (  1.49);

\path[draw=drawColor,line width= 0.4pt,line join=round,line cap=round,fill=fillColor] (567.67,116.13) circle (  1.49);

\path[draw=drawColor,line width= 0.4pt,line join=round,line cap=round,fill=fillColor] (568.15,112.19) circle (  1.49);

\path[draw=drawColor,line width= 0.4pt,line join=round,line cap=round,fill=fillColor] (568.62,120.40) circle (  1.49);

\path[draw=drawColor,line width= 0.4pt,line join=round,line cap=round,fill=fillColor] (569.10,111.86) circle (  1.49);

\path[draw=drawColor,line width= 0.4pt,line join=round,line cap=round,fill=fillColor] (569.59,162.11) circle (  1.49);

\path[draw=drawColor,line width= 0.4pt,line join=round,line cap=round,fill=fillColor] (570.06,121.71) circle (  1.49);

\path[draw=drawColor,line width= 0.4pt,line join=round,line cap=round,fill=fillColor] (570.57,125.33) circle (  1.49);

\path[draw=drawColor,line width= 0.4pt,line join=round,line cap=round,fill=fillColor] (571.05,119.74) circle (  1.49);

\path[draw=drawColor,line width= 0.4pt,line join=round,line cap=round,fill=fillColor] (571.52,127.30) circle (  1.49);

\path[draw=drawColor,line width= 0.4pt,line join=round,line cap=round,fill=fillColor] (572.01,122.37) circle (  1.49);

\path[draw=drawColor,line width= 0.4pt,line join=round,line cap=round,fill=fillColor] (572.49,126.97) circle (  1.49);

\path[draw=drawColor,line width= 0.4pt,line join=round,line cap=round,fill=fillColor] (572.96,119.41) circle (  1.49);

\path[draw=drawColor,line width= 0.4pt,line join=round,line cap=round,fill=fillColor] (573.45,111.86) circle (  1.49);

\path[draw=drawColor,line width= 0.4pt,line join=round,line cap=round,fill=fillColor] (573.93,142.08) circle (  1.49);

\path[draw=drawColor,line width= 0.4pt,line join=round,line cap=round,fill=fillColor] (574.40,118.76) circle (  1.49);

\path[draw=drawColor,line width= 0.4pt,line join=round,line cap=round,fill=fillColor] (574.89,101.35) circle (  1.49);

\path[draw=drawColor,line width= 0.4pt,line join=round,line cap=round,fill=fillColor] (575.35,123.35) circle (  1.49);

\path[draw=drawColor,line width= 0.4pt,line join=round,line cap=round,fill=fillColor] (575.84,117.11) circle (  1.49);

\path[draw=drawColor,line width= 0.4pt,line join=round,line cap=round,fill=fillColor] (576.33,105.95) circle (  1.49);

\path[draw=drawColor,line width= 0.4pt,line join=round,line cap=round,fill=fillColor] (576.81,107.92) circle (  1.49);

\path[draw=drawColor,line width= 0.4pt,line join=round,line cap=round,fill=fillColor] (577.32,161.78) circle (  1.49);

\path[draw=drawColor,line width= 0.4pt,line join=round,line cap=round,fill=fillColor] (577.86,169.34) circle (  1.49);

\path[draw=drawColor,line width= 0.4pt,line join=round,line cap=round,fill=fillColor] (578.38,104.96) circle (  1.49);

\path[draw=drawColor,line width= 0.4pt,line join=round,line cap=round,fill=fillColor] (578.89, 82.63) circle (  1.49);

\path[draw=drawColor,line width= 0.4pt,line join=round,line cap=round,fill=fillColor] (579.36, 77.37) circle (  1.49);

\path[draw=drawColor,line width= 0.4pt,line join=round,line cap=round,fill=fillColor] (579.84, 81.97) circle (  1.49);

\path[draw=drawColor,line width= 0.4pt,line join=round,line cap=round,fill=fillColor] (580.31, 81.31) circle (  1.49);

\path[draw=drawColor,line width= 0.4pt,line join=round,line cap=round,fill=fillColor] (580.80, 84.92) circle (  1.49);

\path[draw=drawColor,line width= 0.4pt,line join=round,line cap=round,fill=fillColor] (581.29, 86.24) circle (  1.49);

\path[draw=drawColor,line width= 0.4pt,line join=round,line cap=round,fill=fillColor] (581.77, 84.60) circle (  1.49);

\path[draw=drawColor,line width= 0.4pt,line join=round,line cap=round,fill=fillColor] (582.24, 89.85) circle (  1.49);

\path[draw=drawColor,line width= 0.4pt,line join=round,line cap=round,fill=fillColor] (582.85, 99.71) circle (  1.49);

\path[draw=drawColor,line width= 0.4pt,line join=round,line cap=round,fill=fillColor] (583.36, 48.79) circle (  1.49);

\path[draw=drawColor,line width= 0.4pt,line join=round,line cap=round,fill=fillColor] (583.90, 70.14) circle (  1.49);

\path[draw=drawColor,line width= 0.4pt,line join=round,line cap=round,fill=fillColor] (584.39, 43.21) circle (  1.49);

\path[draw=drawColor,line width= 0.4pt,line join=round,line cap=round,fill=fillColor] (585.07, 36.31) circle (  1.49);

\path[draw=drawColor,line width= 0.4pt,line join=round,line cap=round,fill=fillColor] (585.62, 37.30) circle (  1.49);

\path[draw=drawColor,line width= 0.4pt,line join=round,line cap=round,fill=fillColor] (586.12, 30.40) circle (  1.49);

\path[draw=drawColor,line width= 0.4pt,line join=round,line cap=round,fill=fillColor] (586.61, 39.60) circle (  1.49);

\path[draw=drawColor,line width= 0.4pt,line join=round,line cap=round,fill=fillColor] (587.12, 50.44) circle (  1.49);

\path[draw=drawColor,line width= 0.4pt,line join=round,line cap=round,fill=fillColor] (587.63, 41.57) circle (  1.49);

\path[draw=drawColor,line width= 0.4pt,line join=round,line cap=round,fill=fillColor] (588.14, 34.67) circle (  1.49);

\path[draw=drawColor,line width= 0.4pt,line join=round,line cap=round,fill=fillColor] (588.68, 28.76) circle (  1.49);

\path[draw=drawColor,line width= 0.4pt,line join=round,line cap=round,fill=fillColor] (589.18, 28.10) circle (  1.49);

\path[draw=drawColor,line width= 0.4pt,line join=round,line cap=round,fill=fillColor] (589.69, 32.70) circle (  1.49);

\path[draw=drawColor,line width= 0.4pt,line join=round,line cap=round,fill=fillColor] (590.20, 33.03) circle (  1.49);

\path[draw=drawColor,line width= 0.4pt,line join=round,line cap=round,fill=fillColor] (590.69, 35.33) circle (  1.49);

\path[draw=drawColor,line width= 0.4pt,line join=round,line cap=round,fill=fillColor] (591.21, 40.91) circle (  1.49);

\path[draw=drawColor,line width= 0.4pt,line join=round,line cap=round,fill=fillColor] (591.70, 47.81) circle (  1.49);

\path[draw=drawColor,line width= 0.4pt,line join=round,line cap=round,fill=fillColor] (592.21, 58.32) circle (  1.49);

\path[draw=drawColor,line width= 0.4pt,line join=round,line cap=round,fill=fillColor] (592.70, 58.98) circle (  1.49);

\path[draw=drawColor,line width= 0.4pt,line join=round,line cap=round,fill=fillColor] (593.21, 54.05) circle (  1.49);

\path[draw=drawColor,line width= 0.4pt,line join=round,line cap=round,fill=fillColor] (593.70, 50.77) circle (  1.49);

\path[draw=drawColor,line width= 0.4pt,line join=round,line cap=round,fill=fillColor] (594.26, 49.78) circle (  1.49);

\path[draw=drawColor,line width= 0.4pt,line join=round,line cap=round,fill=fillColor] (594.78, 38.94) circle (  1.49);

\path[draw=drawColor,line width= 0.4pt,line join=round,line cap=round,fill=fillColor] (595.27, 37.96) circle (  1.49);

\path[draw=drawColor,line width= 0.4pt,line join=round,line cap=round,fill=fillColor] (595.90, 33.03) circle (  1.49);

\path[draw=drawColor,line width= 0.4pt,line join=round,line cap=round,fill=fillColor] (596.39, 31.06) circle (  1.49);

\path[draw=drawColor,line width= 0.4pt,line join=round,line cap=round,fill=fillColor] (596.94, 19.56) circle (  1.49);

\path[draw=drawColor,line width= 0.4pt,line join=round,line cap=round,fill=fillColor] (597.63, 24.82) circle (  1.49);

\path[draw=drawColor,line width= 0.4pt,line join=round,line cap=round,fill=fillColor] (598.25, 27.12) circle (  1.49);

\path[draw=drawColor,line width= 0.4pt,line join=round,line cap=round,fill=fillColor] (598.78, 47.15) circle (  1.49);

\path[draw=drawColor,line width= 0.4pt,line join=round,line cap=round,fill=fillColor] (599.28, 30.07) circle (  1.49);

\path[draw=drawColor,line width= 0.4pt,line join=round,line cap=round,fill=fillColor] (600.02, 28.76) circle (  1.49);

\path[draw=drawColor,line width= 0.4pt,line join=round,line cap=round,fill=fillColor] (600.53, 28.10) circle (  1.49);

\path[draw=drawColor,line width= 0.4pt,line join=round,line cap=round,fill=fillColor] (601.02, 30.40) circle (  1.49);

\path[draw=drawColor,line width= 0.4pt,line join=round,line cap=round,fill=fillColor] (601.59, 35.00) circle (  1.49);

\path[draw=drawColor,line width= 0.4pt,line join=round,line cap=round,fill=fillColor] (602.08, 29.09) circle (  1.49);

\path[draw=drawColor,line width= 0.4pt,line join=round,line cap=round,fill=fillColor] (602.67, 33.36) circle (  1.49);

\path[draw=drawColor,line width= 0.4pt,line join=round,line cap=round,fill=fillColor] (603.25, 29.74) circle (  1.49);

\path[draw=drawColor,line width= 0.4pt,line join=round,line cap=round,fill=fillColor] (603.75, 23.17) circle (  1.49);

\path[draw=drawColor,line width= 0.4pt,line join=round,line cap=round,fill=fillColor] (604.26, 19.89) circle (  1.49);

\path[draw=drawColor,line width= 0.4pt,line join=round,line cap=round,fill=fillColor] (604.75, 30.73) circle (  1.49);

\path[draw=drawColor,line width= 0.4pt,line join=round,line cap=round,fill=fillColor] (605.26, 22.19) circle (  1.49);

\path[draw=drawColor,line width= 0.4pt,line join=round,line cap=round,fill=fillColor] (605.82, 18.25) circle (  1.49);

\path[draw=drawColor,line width= 0.4pt,line join=round,line cap=round,fill=fillColor] (606.39, 15.62) circle (  1.49);

\path[draw=drawColor,line width= 0.4pt,line join=round,line cap=round,fill=fillColor] (606.86, 14.96) circle (  1.49);

\path[draw=drawColor,line width= 0.4pt,line join=round,line cap=round,fill=fillColor] (607.35, 13.32) circle (  1.49);

\path[draw=drawColor,line width= 0.4pt,line join=round,line cap=round,fill=fillColor] (607.83, 13.65) circle (  1.49);

\path[draw=drawColor,line width= 0.4pt,line join=round,line cap=round,fill=fillColor] (608.30, 12.99) circle (  1.49);

\path[draw=drawColor,line width= 0.4pt,line join=round,line cap=round,fill=fillColor] (608.94, 10.36) circle (  1.49);

\path[draw=drawColor,line width= 0.4pt,line join=round,line cap=round,fill=fillColor] (609.43,  3.14) circle (  1.49);

\path[draw=drawColor,line width= 0.4pt,line join=round,line cap=round,fill=fillColor] (610.01,  1.82) circle (  1.49);

\path[draw=drawColor,line width= 0.4pt,line join=round,line cap=round,fill=fillColor] (610.53,  7.41) circle (  1.49);

\path[draw=drawColor,line width= 0.4pt,line join=round,line cap=round,fill=fillColor] (611.04,  4.12) circle (  1.49);

\path[draw=drawColor,line width= 0.4pt,line join=round,line cap=round,fill=fillColor] (611.54,  5.77) circle (  1.49);

\path[draw=drawColor,line width= 0.4pt,line join=round,line cap=round,fill=fillColor] (612.05,  2.81) circle (  1.49);

\path[draw=drawColor,line width= 0.4pt,line join=round,line cap=round,fill=fillColor] (612.61,  3.47) circle (  1.49);

\path[draw=drawColor,line width= 0.4pt,line join=round,line cap=round,fill=fillColor] (613.13,  5.11) circle (  1.49);

\path[draw=drawColor,line width= 0.4pt,line join=round,line cap=round,fill=fillColor] (613.62,  3.47) circle (  1.49);

\path[draw=drawColor,line width= 0.4pt,line join=round,line cap=round,fill=fillColor] (614.18,  0.51) circle (  1.49);

\path[draw=drawColor,line width= 0.4pt,line join=round,line cap=round,fill=fillColor] (614.69,  1.82) circle (  1.49);

\path[draw=drawColor,line width= 0.4pt,line join=round,line cap=round,fill=fillColor] (615.18,  5.44) circle (  1.49);

\path[draw=drawColor,line width= 0.4pt,line join=round,line cap=round,fill=fillColor] (615.67,  5.11) circle (  1.49);

\path[draw=drawColor,line width= 0.4pt,line join=round,line cap=round,fill=fillColor] (616.16,  3.47) circle (  1.49);

\path[draw=drawColor,line width= 0.4pt,line join=round,line cap=round,fill=fillColor] (616.65,  3.47) circle (  1.49);

\path[draw=drawColor,line width= 0.4pt,line join=round,line cap=round,fill=fillColor] (617.18,  5.44) circle (  1.49);

\path[draw=drawColor,line width= 0.4pt,line join=round,line cap=round,fill=fillColor] (617.67,  5.77) circle (  1.49);

\path[draw=drawColor,line width= 0.4pt,line join=round,line cap=round,fill=fillColor] (618.26,  2.15) circle (  1.49);

\path[draw=drawColor,line width= 0.4pt,line join=round,line cap=round,fill=fillColor] (618.80,  7.08) circle (  1.49);

\path[draw=drawColor,line width= 0.4pt,line join=round,line cap=round,fill=fillColor] (619.29, -1.13) circle (  1.49);

\path[draw=drawColor,line width= 0.4pt,line join=round,line cap=round,fill=fillColor] (619.76,  7.08) circle (  1.49);

\path[draw=drawColor,line width= 0.4pt,line join=round,line cap=round,fill=fillColor] (620.27,  6.09) circle (  1.49);

\path[draw=drawColor,line width= 0.4pt,line join=round,line cap=round,fill=fillColor] (620.79,  4.45) circle (  1.49);

\path[draw=drawColor,line width= 0.4pt,line join=round,line cap=round,fill=fillColor] (621.30,  6.42) circle (  1.49);

\path[draw=drawColor,line width= 0.4pt,line join=round,line cap=round,fill=fillColor] (621.79,  7.41) circle (  1.49);

\path[draw=drawColor,line width= 0.4pt,line join=round,line cap=round,fill=fillColor] (622.32,  9.38) circle (  1.49);

\path[draw=drawColor,line width= 0.4pt,line join=round,line cap=round,fill=fillColor] (622.79, 11.68) circle (  1.49);

\path[draw=drawColor,line width= 0.4pt,line join=round,line cap=round,fill=fillColor] (623.31, 24.16) circle (  1.49);

\path[draw=drawColor,line width= 0.4pt,line join=round,line cap=round,fill=fillColor] (623.82, 12.66) circle (  1.49);

\path[draw=drawColor,line width= 0.4pt,line join=round,line cap=round,fill=fillColor] (624.28, 11.35) circle (  1.49);

\path[draw=drawColor,line width= 0.4pt,line join=round,line cap=round,fill=fillColor] (624.80, 11.02) circle (  1.49);

\path[draw=drawColor,line width= 0.4pt,line join=round,line cap=round,fill=fillColor] (625.28,  8.72) circle (  1.49);

\path[draw=drawColor,line width= 0.4pt,line join=round,line cap=round,fill=fillColor] (625.80,111.20) circle (  1.49);

\path[draw=drawColor,line width= 0.4pt,line join=round,line cap=round,fill=fillColor] (626.36,  9.05) circle (  1.49);

\path[draw=drawColor,line width= 0.4pt,line join=round,line cap=round,fill=fillColor] (626.87,155.87) circle (  1.49);

\path[draw=drawColor,line width= 0.4pt,line join=round,line cap=round,fill=fillColor] (627.34, 55.04) circle (  1.49);

\path[draw=drawColor,line width= 0.4pt,line join=round,line cap=round,fill=fillColor] (627.86, 15.62) circle (  1.49);

\path[draw=drawColor,line width= 0.4pt,line join=round,line cap=round,fill=fillColor] (628.39, 73.10) circle (  1.49);

\path[draw=drawColor,line width= 0.4pt,line join=round,line cap=round,fill=fillColor] (628.90, 13.98) circle (  1.49);

\path[draw=drawColor,line width= 0.4pt,line join=round,line cap=round,fill=fillColor] (629.39, 14.31) circle (  1.49);

\path[draw=drawColor,line width= 0.4pt,line join=round,line cap=round,fill=fillColor] (629.89, 15.62) circle (  1.49);

\path[draw=drawColor,line width= 0.4pt,line join=round,line cap=round,fill=fillColor] (630.39, 13.32) circle (  1.49);

\path[draw=drawColor,line width= 0.4pt,line join=round,line cap=round,fill=fillColor] (630.93, 11.35) circle (  1.49);

\path[draw=drawColor,line width= 0.4pt,line join=round,line cap=round,fill=fillColor] (631.45, 10.36) circle (  1.49);

\path[draw=drawColor,line width= 0.4pt,line join=round,line cap=round,fill=fillColor] (631.92,  9.38) circle (  1.49);

\path[draw=drawColor,line width= 0.4pt,line join=round,line cap=round,fill=fillColor] (632.43,  9.71) circle (  1.49);

\path[draw=drawColor,line width= 0.4pt,line join=round,line cap=round,fill=fillColor] (632.94,  9.71) circle (  1.49);

\path[draw=drawColor,line width= 0.4pt,line join=round,line cap=round,fill=fillColor] (633.45,  7.74) circle (  1.49);

\path[draw=drawColor,line width= 0.4pt,line join=round,line cap=round,fill=fillColor] (633.95,  8.39) circle (  1.49);

\path[draw=drawColor,line width= 0.4pt,line join=round,line cap=round,fill=fillColor] (634.46, 12.99) circle (  1.49);

\path[draw=drawColor,line width= 0.4pt,line join=round,line cap=round,fill=fillColor] (634.95,  9.38) circle (  1.49);

\path[draw=drawColor,line width= 0.4pt,line join=round,line cap=round,fill=fillColor] (635.49,  9.05) circle (  1.49);

\path[draw=drawColor,line width= 0.4pt,line join=round,line cap=round,fill=fillColor] (636.05, 16.28) circle (  1.49);

\path[draw=drawColor,line width= 0.4pt,line join=round,line cap=round,fill=fillColor] (636.57,  3.14) circle (  1.49);

\path[draw=drawColor,line width= 0.4pt,line join=round,line cap=round,fill=fillColor] (637.05,  7.41) circle (  1.49);

\path[draw=drawColor,line width= 0.4pt,line join=round,line cap=round,fill=fillColor] (637.56,  3.47) circle (  1.49);

\path[draw=drawColor,line width= 0.4pt,line join=round,line cap=round,fill=fillColor] (638.08,  1.82) circle (  1.49);

\path[draw=drawColor,line width= 0.4pt,line join=round,line cap=round,fill=fillColor] (638.55,  4.45) circle (  1.49);

\path[draw=drawColor,line width= 0.4pt,line join=round,line cap=round,fill=fillColor] (639.27,  7.74) circle (  1.49);

\path[draw=drawColor,line width= 0.4pt,line join=round,line cap=round,fill=fillColor] (639.78,  4.45) circle (  1.49);

\path[draw=drawColor,line width= 0.4pt,line join=round,line cap=round,fill=fillColor] (640.27,  1.82) circle (  1.49);

\path[draw=drawColor,line width= 0.4pt,line join=round,line cap=round,fill=fillColor] (640.76,  3.80) circle (  1.49);

\path[draw=drawColor,line width= 0.4pt,line join=round,line cap=round,fill=fillColor] (641.29,  3.14) circle (  1.49);

\path[draw=drawColor,line width= 0.4pt,line join=round,line cap=round,fill=fillColor] (641.76,  7.08) circle (  1.49);

\path[draw=drawColor,line width= 0.4pt,line join=round,line cap=round,fill=fillColor] (642.32,  6.75) circle (  1.49);

\path[draw=drawColor,line width= 0.4pt,line join=round,line cap=round,fill=fillColor] (642.83,  6.42) circle (  1.49);

\path[draw=drawColor,line width= 0.4pt,line join=round,line cap=round,fill=fillColor] (643.35,  7.74) circle (  1.49);

\path[draw=drawColor,line width= 0.4pt,line join=round,line cap=round,fill=fillColor] (643.84,  7.41) circle (  1.49);

\path[draw=drawColor,line width= 0.4pt,line join=round,line cap=round,fill=fillColor] (644.35,  0.84) circle (  1.49);

\path[draw=drawColor,line width= 0.4pt,line join=round,line cap=round,fill=fillColor] (644.86,  1.17) circle (  1.49);

\path[draw=drawColor,line width= 0.4pt,line join=round,line cap=round,fill=fillColor] (645.99, 40.58) circle (  1.49);

\path[draw=drawColor,line width= 0.4pt,line join=round,line cap=round,fill=fillColor] (646.46,  2.48) circle (  1.49);

\path[draw=drawColor,line width= 0.4pt,line join=round,line cap=round,fill=fillColor] (647.03,  3.80) circle (  1.49);

\path[draw=drawColor,line width= 0.4pt,line join=round,line cap=round,fill=fillColor] (647.54,  2.15) circle (  1.49);

\path[draw=drawColor,line width= 0.4pt,line join=round,line cap=round,fill=fillColor] (648.08,  2.81) circle (  1.49);

\path[draw=drawColor,line width= 0.4pt,line join=round,line cap=round,fill=fillColor] (648.56,  2.81) circle (  1.49);

\path[draw=drawColor,line width= 0.4pt,line join=round,line cap=round,fill=fillColor] (649.08,  4.12) circle (  1.49);

\path[draw=drawColor,line width= 0.4pt,line join=round,line cap=round,fill=fillColor] (649.59,  5.11) circle (  1.49);

\path[draw=drawColor,line width= 0.4pt,line join=round,line cap=round,fill=fillColor] (650.16,201.20) circle (  1.49);

\path[draw=drawColor,line width= 0.4pt,line join=round,line cap=round,fill=fillColor] (650.73,  5.77) circle (  1.49);

\path[draw=drawColor,line width= 0.4pt,line join=round,line cap=round,fill=fillColor] (651.24,  6.09) circle (  1.49);

\path[draw=drawColor,line width= 0.4pt,line join=round,line cap=round,fill=fillColor] (651.73, 45.18) circle (  1.49);

\path[draw=drawColor,line width= 0.4pt,line join=round,line cap=round,fill=fillColor] (652.27,146.67) circle (  1.49);

\path[draw=drawColor,line width= 0.4pt,line join=round,line cap=round,fill=fillColor] (652.80, 32.70) circle (  1.49);

\path[draw=drawColor,line width= 0.4pt,line join=round,line cap=round,fill=fillColor] (653.27,124.34) circle (  1.49);

\path[draw=drawColor,line width= 0.4pt,line join=round,line cap=round,fill=fillColor] (653.75, 70.80) circle (  1.49);

\path[draw=drawColor,line width= 0.4pt,line join=round,line cap=round,fill=fillColor] (654.22,186.09) circle (  1.49);

\path[draw=drawColor,line width= 0.4pt,line join=round,line cap=round,fill=fillColor] (654.71,  5.77) circle (  1.49);

\path[draw=drawColor,line width= 0.4pt,line join=round,line cap=round,fill=fillColor] (655.19,  4.45) circle (  1.49);

\path[draw=drawColor,line width= 0.4pt,line join=round,line cap=round,fill=fillColor] (655.71, 40.25) circle (  1.49);

\path[draw=drawColor,line width= 0.4pt,line join=round,line cap=round,fill=fillColor] (656.32, 15.95) circle (  1.49);

\path[draw=drawColor,line width= 0.4pt,line join=round,line cap=round,fill=fillColor] (656.89,  5.44) circle (  1.49);

\path[draw=drawColor,line width= 0.4pt,line join=round,line cap=round,fill=fillColor] (657.41, 13.65) circle (  1.49);

\path[draw=drawColor,line width= 0.4pt,line join=round,line cap=round,fill=fillColor] (657.90,216.64) circle (  1.49);

\path[draw=drawColor,line width= 0.4pt,line join=round,line cap=round,fill=fillColor] (658.38,215.65) circle (  1.49);

\path[draw=drawColor,line width= 0.4pt,line join=round,line cap=round,fill=fillColor] (658.82,212.37) circle (  1.49);

\path[draw=drawColor,line width= 0.4pt,line join=round,line cap=round,fill=fillColor] (659.29,204.81) circle (  1.49);

\path[draw=drawColor,line width= 0.4pt,line join=round,line cap=round,fill=fillColor] (659.75,202.51) circle (  1.49);

\path[draw=drawColor,line width= 0.4pt,line join=round,line cap=round,fill=fillColor] (660.23,214.01) circle (  1.49);

\path[draw=drawColor,line width= 0.4pt,line join=round,line cap=round,fill=fillColor] (660.75,214.01) circle (  1.49);

\path[draw=drawColor,line width= 0.4pt,line join=round,line cap=round,fill=fillColor] (661.32,  6.09) circle (  1.49);

\path[draw=drawColor,line width= 0.4pt,line join=round,line cap=round,fill=fillColor] (661.91,  2.48) circle (  1.49);

\path[draw=drawColor,line width= 0.4pt,line join=round,line cap=round,fill=fillColor] (662.37,174.27) circle (  1.49);

\path[draw=drawColor,line width= 0.4pt,line join=round,line cap=round,fill=fillColor] (662.85,214.67) circle (  1.49);

\path[draw=drawColor,line width= 0.4pt,line join=round,line cap=round,fill=fillColor] (663.30, 20.22) circle (  1.49);

\path[draw=drawColor,line width= 0.4pt,line join=round,line cap=round,fill=fillColor] (663.80,215.98) circle (  1.49);

\path[draw=drawColor,line width= 0.4pt,line join=round,line cap=round,fill=fillColor] (664.35,  4.12) circle (  1.49);

\path[draw=drawColor,line width= 0.4pt,line join=round,line cap=round,fill=fillColor] (664.81,214.67) circle (  1.49);

\path[draw=drawColor,line width= 0.4pt,line join=round,line cap=round,fill=fillColor] (665.29,215.65) circle (  1.49);

\path[draw=drawColor,line width= 0.4pt,line join=round,line cap=round,fill=fillColor] (665.76,215.65) circle (  1.49);

\path[draw=drawColor,line width= 0.4pt,line join=round,line cap=round,fill=fillColor] (666.24,216.31) circle (  1.49);

\path[draw=drawColor,line width= 0.4pt,line join=round,line cap=round,fill=fillColor] (666.71,216.31) circle (  1.49);

\path[draw=drawColor,line width= 0.4pt,line join=round,line cap=round,fill=fillColor] (667.18,216.31) circle (  1.49);

\path[draw=drawColor,line width= 0.4pt,line join=round,line cap=round,fill=fillColor] (667.64,216.97) circle (  1.49);

\path[draw=drawColor,line width= 0.4pt,line join=round,line cap=round,fill=fillColor] (668.13,212.70) circle (  1.49);

\path[draw=drawColor,line width= 0.4pt,line join=round,line cap=round,fill=fillColor] (668.61,182.15) circle (  1.49);

\path[draw=drawColor,line width= 0.4pt,line join=round,line cap=round,fill=fillColor] (669.08,214.99) circle (  1.49);

\path[draw=drawColor,line width= 0.4pt,line join=round,line cap=round,fill=fillColor] (669.61,215.98) circle (  1.49);

\path[draw=drawColor,line width= 0.4pt,line join=round,line cap=round,fill=fillColor] (670.11,214.99) circle (  1.49);

\path[draw=drawColor,line width= 0.4pt,line join=round,line cap=round,fill=fillColor] (670.59,217.95) circle (  1.49);

\path[draw=drawColor,line width= 0.4pt,line join=round,line cap=round,fill=fillColor] (671.08,211.05) circle (  1.49);

\path[draw=drawColor,line width= 0.4pt,line join=round,line cap=round,fill=fillColor] (671.54,218.94) circle (  1.49);

\path[draw=drawColor,line width= 0.4pt,line join=round,line cap=round,fill=fillColor] (672.01,213.02) circle (  1.49);

\path[draw=drawColor,line width= 0.4pt,line join=round,line cap=round,fill=fillColor] (672.49,218.28) circle (  1.49);

\path[draw=drawColor,line width= 0.4pt,line join=round,line cap=round,fill=fillColor] (672.93,217.29) circle (  1.49);

\path[draw=drawColor,line width= 0.4pt,line join=round,line cap=round,fill=fillColor] (673.40,219.26) circle (  1.49);

\path[draw=drawColor,line width= 0.4pt,line join=round,line cap=round,fill=fillColor] (673.88,219.26) circle (  1.49);

\path[draw=drawColor,line width= 0.4pt,line join=round,line cap=round,fill=fillColor] (674.34,218.61) circle (  1.49);

\path[draw=drawColor,line width= 0.4pt,line join=round,line cap=round,fill=fillColor] (674.78,218.28) circle (  1.49);

\path[draw=drawColor,line width= 0.4pt,line join=round,line cap=round,fill=fillColor] (675.25,217.62) circle (  1.49);

\path[draw=drawColor,line width= 0.4pt,line join=round,line cap=round,fill=fillColor] (675.73,217.95) circle (  1.49);

\path[draw=drawColor,line width= 0.4pt,line join=round,line cap=round,fill=fillColor] (676.20,217.62) circle (  1.49);

\path[draw=drawColor,line width= 0.4pt,line join=round,line cap=round,fill=fillColor] (676.66,216.64) circle (  1.49);

\path[draw=drawColor,line width= 0.4pt,line join=round,line cap=round,fill=fillColor] (677.12,215.98) circle (  1.49);

\path[draw=drawColor,line width= 0.4pt,line join=round,line cap=round,fill=fillColor] (677.60,215.98) circle (  1.49);

\path[draw=drawColor,line width= 0.4pt,line join=round,line cap=round,fill=fillColor] (678.05,216.64) circle (  1.49);

\path[draw=drawColor,line width= 0.4pt,line join=round,line cap=round,fill=fillColor] (678.51,216.64) circle (  1.49);

\path[draw=drawColor,line width= 0.4pt,line join=round,line cap=round,fill=fillColor] (678.97,216.97) circle (  1.49);

\path[draw=drawColor,line width= 0.4pt,line join=round,line cap=round,fill=fillColor] (679.45,217.29) circle (  1.49);

\path[draw=drawColor,line width= 0.4pt,line join=round,line cap=round,fill=fillColor] (679.92,216.64) circle (  1.49);

\path[draw=drawColor,line width= 0.4pt,line join=round,line cap=round,fill=fillColor] (680.48,218.28) circle (  1.49);

\path[draw=drawColor,line width= 0.4pt,line join=round,line cap=round,fill=fillColor] (680.97,216.97) circle (  1.49);

\path[draw=drawColor,line width= 0.4pt,line join=round,line cap=round,fill=fillColor] (681.43,217.62) circle (  1.49);

\path[draw=drawColor,line width= 0.4pt,line join=round,line cap=round,fill=fillColor] (681.85,216.97) circle (  1.49);

\path[draw=drawColor,line width= 0.4pt,line join=round,line cap=round,fill=fillColor] (682.41,215.98) circle (  1.49);

\path[draw=drawColor,line width= 0.4pt,line join=round,line cap=round,fill=fillColor] (682.90,214.67) circle (  1.49);

\path[draw=drawColor,line width= 0.4pt,line join=round,line cap=round,fill=fillColor] (683.39,216.97) circle (  1.49);

\path[draw=drawColor,line width= 0.4pt,line join=round,line cap=round,fill=fillColor] (683.90,216.31) circle (  1.49);

\path[draw=drawColor,line width= 0.4pt,line join=round,line cap=round,fill=fillColor] (684.32,216.31) circle (  1.49);

\path[draw=drawColor,line width= 0.4pt,line join=round,line cap=round,fill=fillColor] (684.83,217.62) circle (  1.49);

\path[draw=drawColor,line width= 0.4pt,line join=round,line cap=round,fill=fillColor] (685.32,217.95) circle (  1.49);

\path[draw=drawColor,line width= 0.4pt,line join=round,line cap=round,fill=fillColor] (685.75,217.29) circle (  1.49);

\path[draw=drawColor,line width= 0.4pt,line join=round,line cap=round,fill=fillColor] (686.26,217.95) circle (  1.49);

\path[draw=drawColor,line width= 0.4pt,line join=round,line cap=round,fill=fillColor] (686.71,218.28) circle (  1.49);

\path[draw=drawColor,line width= 0.4pt,line join=round,line cap=round,fill=fillColor] (687.20,217.95) circle (  1.49);

\path[draw=drawColor,line width= 0.4pt,line join=round,line cap=round,fill=fillColor] (687.73,217.95) circle (  1.49);

\path[draw=drawColor,line width= 0.4pt,line join=round,line cap=round,fill=fillColor] (688.19,217.29) circle (  1.49);

\path[draw=drawColor,line width= 0.4pt,line join=round,line cap=round,fill=fillColor] (688.69,217.95) circle (  1.49);

\path[draw=drawColor,line width= 0.4pt,line join=round,line cap=round,fill=fillColor] (689.15,216.97) circle (  1.49);

\path[draw=drawColor,line width= 0.4pt,line join=round,line cap=round,fill=fillColor] (689.64,217.62) circle (  1.49);

\path[draw=drawColor,line width= 0.4pt,line join=round,line cap=round,fill=fillColor] (690.12,217.29) circle (  1.49);

\path[draw=drawColor,line width= 0.4pt,line join=round,line cap=round,fill=fillColor] (690.56,217.29) circle (  1.49);

\path[draw=drawColor,line width= 0.4pt,line join=round,line cap=round,fill=fillColor] (691.08,217.29) circle (  1.49);

\path[draw=drawColor,line width= 0.4pt,line join=round,line cap=round,fill=fillColor] (691.54,217.62) circle (  1.49);

\path[draw=drawColor,line width= 0.4pt,line join=round,line cap=round,fill=fillColor] (692.00,217.95) circle (  1.49);

\path[draw=drawColor,line width= 0.4pt,line join=round,line cap=round,fill=fillColor] (692.43,219.59) circle (  1.49);

\path[draw=drawColor,line width= 0.4pt,line join=round,line cap=round,fill=fillColor] (692.85,217.95) circle (  1.49);

\path[draw=drawColor,line width= 0.4pt,line join=round,line cap=round,fill=fillColor] (693.31,219.59) circle (  1.49);

\path[draw=drawColor,line width= 0.4pt,line join=round,line cap=round,fill=fillColor] (693.82,218.94) circle (  1.49);

\path[draw=drawColor,line width= 0.4pt,line join=round,line cap=round,fill=fillColor] (694.33,218.61) circle (  1.49);

\path[draw=drawColor,line width= 0.4pt,line join=round,line cap=round,fill=fillColor] (694.78,217.95) circle (  1.49);

\path[draw=drawColor,line width= 0.4pt,line join=round,line cap=round,fill=fillColor] (695.24,218.28) circle (  1.49);

\path[draw=drawColor,line width= 0.4pt,line join=round,line cap=round,fill=fillColor] (695.77,220.91) circle (  1.49);

\path[draw=drawColor,line width= 0.4pt,line join=round,line cap=round,fill=fillColor] (696.21,216.64) circle (  1.49);

\path[draw=drawColor,line width= 0.4pt,line join=round,line cap=round,fill=fillColor] (696.65,215.32) circle (  1.49);

\path[draw=drawColor,line width= 0.4pt,line join=round,line cap=round,fill=fillColor] (697.17,215.65) circle (  1.49);

\path[draw=drawColor,line width= 0.4pt,line join=round,line cap=round,fill=fillColor] (697.68,216.64) circle (  1.49);

\path[draw=drawColor,line width= 0.4pt,line join=round,line cap=round,fill=fillColor] (698.20,215.65) circle (  1.49);

\path[draw=drawColor,line width= 0.4pt,line join=round,line cap=round,fill=fillColor] (698.65,215.32) circle (  1.49);

\path[draw=drawColor,line width= 0.4pt,line join=round,line cap=round,fill=fillColor] (699.11,214.67) circle (  1.49);

\path[draw=drawColor,line width= 0.4pt,line join=round,line cap=round,fill=fillColor] (699.58,215.98) circle (  1.49);

\path[draw=drawColor,line width= 0.4pt,line join=round,line cap=round,fill=fillColor] (700.17,216.31) circle (  1.49);

\path[draw=drawColor,line width= 0.4pt,line join=round,line cap=round,fill=fillColor] (700.74,216.31) circle (  1.49);

\path[draw=drawColor,line width= 0.4pt,line join=round,line cap=round,fill=fillColor] (701.23,216.64) circle (  1.49);

\path[draw=drawColor,line width= 0.4pt,line join=round,line cap=round,fill=fillColor] (701.79,218.28) circle (  1.49);

\path[draw=drawColor,line width= 0.4pt,line join=round,line cap=round,fill=fillColor] (702.23,215.98) circle (  1.49);

\path[draw=drawColor,line width= 0.4pt,line join=round,line cap=round,fill=fillColor] (702.69,216.97) circle (  1.49);

\path[draw=drawColor,line width= 0.4pt,line join=round,line cap=round,fill=fillColor] (703.15,217.62) circle (  1.49);

\path[draw=drawColor,line width= 0.4pt,line join=round,line cap=round,fill=fillColor] (703.64,216.31) circle (  1.49);

\path[draw=drawColor,line width= 0.4pt,line join=round,line cap=round,fill=fillColor] (704.07,216.64) circle (  1.49);

\path[draw=drawColor,line width= 0.4pt,line join=round,line cap=round,fill=fillColor] (704.51,216.31) circle (  1.49);

\path[draw=drawColor,line width= 0.4pt,line join=round,line cap=round,fill=fillColor] (705.10,215.98) circle (  1.49);

\path[draw=drawColor,line width= 0.4pt,line join=round,line cap=round,fill=fillColor] (705.70,217.95) circle (  1.49);

\path[draw=drawColor,line width= 0.4pt,line join=round,line cap=round,fill=fillColor] (706.13,216.31) circle (  1.49);

\path[draw=drawColor,line width= 0.4pt,line join=round,line cap=round,fill=fillColor] (706.55,216.31) circle (  1.49);

\path[draw=drawColor,line width= 0.4pt,line join=round,line cap=round,fill=fillColor] (706.96,216.31) circle (  1.49);

\path[draw=drawColor,line width= 0.4pt,line join=round,line cap=round,fill=fillColor] (707.39,215.65) circle (  1.49);
\definecolor{drawColor}{RGB}{0,0,255}
\definecolor{fillColor}{RGB}{0,0,255}

\path[draw=drawColor,line width= 0.4pt,line join=round,line cap=round,fill=fillColor] (531.33,138.79) circle (  1.49);

\path[draw=drawColor,line width= 0.4pt,line join=round,line cap=round,fill=fillColor] (531.78,138.46) circle (  1.49);

\path[draw=drawColor,line width= 0.4pt,line join=round,line cap=round,fill=fillColor] (532.23,138.46) circle (  1.49);

\path[draw=drawColor,line width= 0.4pt,line join=round,line cap=round,fill=fillColor] (532.68,142.08) circle (  1.49);

\path[draw=drawColor,line width= 0.4pt,line join=round,line cap=round,fill=fillColor] (533.13,144.38) circle (  1.49);

\path[draw=drawColor,line width= 0.4pt,line join=round,line cap=round,fill=fillColor] (533.58,146.35) circle (  1.49);

\path[draw=drawColor,line width= 0.4pt,line join=round,line cap=round,fill=fillColor] (534.03,145.69) circle (  1.49);

\path[draw=drawColor,line width= 0.4pt,line join=round,line cap=round,fill=fillColor] (534.51,144.05) circle (  1.49);

\path[draw=drawColor,line width= 0.4pt,line join=round,line cap=round,fill=fillColor] (534.97,143.06) circle (  1.49);

\path[draw=drawColor,line width= 0.4pt,line join=round,line cap=round,fill=fillColor] (535.41,138.79) circle (  1.49);

\path[draw=drawColor,line width= 0.4pt,line join=round,line cap=round,fill=fillColor] (535.87,139.12) circle (  1.49);

\path[draw=drawColor,line width= 0.4pt,line join=round,line cap=round,fill=fillColor] (536.34,139.78) circle (  1.49);

\path[draw=drawColor,line width= 0.4pt,line join=round,line cap=round,fill=fillColor] (536.78,142.08) circle (  1.49);

\path[draw=drawColor,line width= 0.4pt,line join=round,line cap=round,fill=fillColor] (537.24,143.06) circle (  1.49);

\path[draw=drawColor,line width= 0.4pt,line join=round,line cap=round,fill=fillColor] (537.70,142.73) circle (  1.49);

\path[draw=drawColor,line width= 0.4pt,line join=round,line cap=round,fill=fillColor] (538.16,140.11) circle (  1.49);

\path[draw=drawColor,line width= 0.4pt,line join=round,line cap=round,fill=fillColor] (538.67,139.45) circle (  1.49);

\path[draw=drawColor,line width= 0.4pt,line join=round,line cap=round,fill=fillColor] (539.13,139.45) circle (  1.49);

\path[draw=drawColor,line width= 0.4pt,line join=round,line cap=round,fill=fillColor] (539.58,139.12) circle (  1.49);

\path[draw=drawColor,line width= 0.4pt,line join=round,line cap=round,fill=fillColor] (540.03,143.72) circle (  1.49);

\path[draw=drawColor,line width= 0.4pt,line join=round,line cap=round,fill=fillColor] (540.48,142.08) circle (  1.49);

\path[draw=drawColor,line width= 0.4pt,line join=round,line cap=round,fill=fillColor] (540.94,143.06) circle (  1.49);

\path[draw=drawColor,line width= 0.4pt,line join=round,line cap=round,fill=fillColor] (541.40,139.12) circle (  1.49);

\path[draw=drawColor,line width= 0.4pt,line join=round,line cap=round,fill=fillColor] (541.88,139.12) circle (  1.49);

\path[draw=drawColor,line width= 0.4pt,line join=round,line cap=round,fill=fillColor] (542.33,139.45) circle (  1.49);

\path[draw=drawColor,line width= 0.4pt,line join=round,line cap=round,fill=fillColor] (542.78,139.12) circle (  1.49);

\path[draw=drawColor,line width= 0.4pt,line join=round,line cap=round,fill=fillColor] (543.25,140.11) circle (  1.49);

\path[draw=drawColor,line width= 0.4pt,line join=round,line cap=round,fill=fillColor] (543.71,142.08) circle (  1.49);

\path[draw=drawColor,line width= 0.4pt,line join=round,line cap=round,fill=fillColor] (544.17,142.73) circle (  1.49);

\path[draw=drawColor,line width= 0.4pt,line join=round,line cap=round,fill=fillColor] (544.64,142.73) circle (  1.49);

\path[draw=drawColor,line width= 0.4pt,line join=round,line cap=round,fill=fillColor] (545.10,142.73) circle (  1.49);

\path[draw=drawColor,line width= 0.4pt,line join=round,line cap=round,fill=fillColor] (545.56,142.41) circle (  1.49);

\path[draw=drawColor,line width= 0.4pt,line join=round,line cap=round,fill=fillColor] (546.02,143.06) circle (  1.49);

\path[draw=drawColor,line width= 0.4pt,line join=round,line cap=round,fill=fillColor] (546.48,141.09) circle (  1.49);

\path[draw=drawColor,line width= 0.4pt,line join=round,line cap=round,fill=fillColor] (546.93,138.79) circle (  1.49);

\path[draw=drawColor,line width= 0.4pt,line join=round,line cap=round,fill=fillColor] (547.39,140.76) circle (  1.49);

\path[draw=drawColor,line width= 0.4pt,line join=round,line cap=round,fill=fillColor] (547.85,138.46) circle (  1.49);

\path[draw=drawColor,line width= 0.4pt,line join=round,line cap=round,fill=fillColor] (548.31,137.15) circle (  1.49);

\path[draw=drawColor,line width= 0.4pt,line join=round,line cap=round,fill=fillColor] (548.77,140.76) circle (  1.49);

\path[draw=drawColor,line width= 0.4pt,line join=round,line cap=round,fill=fillColor] (549.21,138.46) circle (  1.49);

\path[draw=drawColor,line width= 0.4pt,line join=round,line cap=round,fill=fillColor] (549.67,132.55) circle (  1.49);

\path[draw=drawColor,line width= 0.4pt,line join=round,line cap=round,fill=fillColor] (550.13,127.95) circle (  1.49);

\path[draw=drawColor,line width= 0.4pt,line join=round,line cap=round,fill=fillColor] (550.57,126.31) circle (  1.49);

\path[draw=drawColor,line width= 0.4pt,line join=round,line cap=round,fill=fillColor] (551.01,132.88) circle (  1.49);

\path[draw=drawColor,line width= 0.4pt,line join=round,line cap=round,fill=fillColor] (551.45,134.52) circle (  1.49);

\path[draw=drawColor,line width= 0.4pt,line join=round,line cap=round,fill=fillColor] (551.91,137.15) circle (  1.49);

\path[draw=drawColor,line width= 0.4pt,line join=round,line cap=round,fill=fillColor] (552.37,134.52) circle (  1.49);

\path[draw=drawColor,line width= 0.4pt,line join=round,line cap=round,fill=fillColor] (552.83,130.25) circle (  1.49);

\path[draw=drawColor,line width= 0.4pt,line join=round,line cap=round,fill=fillColor] (553.27,102.99) circle (  1.49);

\path[draw=drawColor,line width= 0.4pt,line join=round,line cap=round,fill=fillColor] (553.73, 97.41) circle (  1.49);

\path[draw=drawColor,line width= 0.4pt,line join=round,line cap=round,fill=fillColor] (554.19, 96.42) circle (  1.49);

\path[draw=drawColor,line width= 0.4pt,line join=round,line cap=round,fill=fillColor] (554.69, 94.45) circle (  1.49);

\path[draw=drawColor,line width= 0.4pt,line join=round,line cap=round,fill=fillColor] (555.13,105.29) circle (  1.49);

\path[draw=drawColor,line width= 0.4pt,line join=round,line cap=round,fill=fillColor] (555.59,116.46) circle (  1.49);

\path[draw=drawColor,line width= 0.4pt,line join=round,line cap=round,fill=fillColor] (556.07,128.28) circle (  1.49);

\path[draw=drawColor,line width= 0.4pt,line join=round,line cap=round,fill=fillColor] (556.54,135.18) circle (  1.49);

\path[draw=drawColor,line width= 0.4pt,line join=round,line cap=round,fill=fillColor] (557.02,138.79) circle (  1.49);

\path[draw=drawColor,line width= 0.4pt,line join=round,line cap=round,fill=fillColor] (557.51,134.19) circle (  1.49);

\path[draw=drawColor,line width= 0.4pt,line join=round,line cap=round,fill=fillColor] (558.02,132.55) circle (  1.49);

\path[draw=drawColor,line width= 0.4pt,line join=round,line cap=round,fill=fillColor] (558.49,124.34) circle (  1.49);

\path[draw=drawColor,line width= 0.4pt,line join=round,line cap=round,fill=fillColor] (558.97,104.96) circle (  1.49);

\path[draw=drawColor,line width= 0.4pt,line join=round,line cap=round,fill=fillColor] (559.44, 80.65) circle (  1.49);

\path[draw=drawColor,line width= 0.4pt,line join=round,line cap=round,fill=fillColor] (559.91, 84.27) circle (  1.49);

\path[draw=drawColor,line width= 0.4pt,line join=round,line cap=round,fill=fillColor] (560.39, 87.55) circle (  1.49);

\path[draw=drawColor,line width= 0.4pt,line join=round,line cap=round,fill=fillColor] (560.85, 80.65) circle (  1.49);

\path[draw=drawColor,line width= 0.4pt,line join=round,line cap=round,fill=fillColor] (561.32, 76.71) circle (  1.49);

\path[draw=drawColor,line width= 0.4pt,line join=round,line cap=round,fill=fillColor] (561.83, 73.10) circle (  1.49);

\path[draw=drawColor,line width= 0.4pt,line join=round,line cap=round,fill=fillColor] (562.32,128.94) circle (  1.49);

\path[draw=drawColor,line width= 0.4pt,line join=round,line cap=round,fill=fillColor] (562.81, 87.55) circle (  1.49);

\path[draw=drawColor,line width= 0.4pt,line join=round,line cap=round,fill=fillColor] (563.30,103.32) circle (  1.49);

\path[draw=drawColor,line width= 0.4pt,line join=round,line cap=round,fill=fillColor] (563.86, 62.26) circle (  1.49);

\path[draw=drawColor,line width= 0.4pt,line join=round,line cap=round,fill=fillColor] (564.35, 69.16) circle (  1.49);

\path[draw=drawColor,line width= 0.4pt,line join=round,line cap=round,fill=fillColor] (564.86, 61.60) circle (  1.49);

\path[draw=drawColor,line width= 0.4pt,line join=round,line cap=round,fill=fillColor] (565.37, 85.91) circle (  1.49);

\path[draw=drawColor,line width= 0.4pt,line join=round,line cap=round,fill=fillColor] (565.82, 99.38) circle (  1.49);

\path[draw=drawColor,line width= 0.4pt,line join=round,line cap=round,fill=fillColor] (566.33, 72.11) circle (  1.49);

\path[draw=drawColor,line width= 0.4pt,line join=round,line cap=round,fill=fillColor] (566.81, 79.01) circle (  1.49);

\path[draw=drawColor,line width= 0.4pt,line join=round,line cap=round,fill=fillColor] (567.30, 64.89) circle (  1.49);

\path[draw=drawColor,line width= 0.4pt,line join=round,line cap=round,fill=fillColor] (567.76, 58.98) circle (  1.49);

\path[draw=drawColor,line width= 0.4pt,line join=round,line cap=round,fill=fillColor] (568.23, 66.20) circle (  1.49);

\path[draw=drawColor,line width= 0.4pt,line join=round,line cap=round,fill=fillColor] (568.71, 69.49) circle (  1.49);

\path[draw=drawColor,line width= 0.4pt,line join=round,line cap=round,fill=fillColor] (569.18, 69.49) circle (  1.49);

\path[draw=drawColor,line width= 0.4pt,line join=round,line cap=round,fill=fillColor] (569.67,129.60) circle (  1.49);

\path[draw=drawColor,line width= 0.4pt,line join=round,line cap=round,fill=fillColor] (570.15, 77.70) circle (  1.49);

\path[draw=drawColor,line width= 0.4pt,line join=round,line cap=round,fill=fillColor] (570.64, 79.01) circle (  1.49);

\path[draw=drawColor,line width= 0.4pt,line join=round,line cap=round,fill=fillColor] (571.11, 75.73) circle (  1.49);

\path[draw=drawColor,line width= 0.4pt,line join=round,line cap=round,fill=fillColor] (571.59, 78.03) circle (  1.49);

\path[draw=drawColor,line width= 0.4pt,line join=round,line cap=round,fill=fillColor] (572.09, 71.79) circle (  1.49);

\path[draw=drawColor,line width= 0.4pt,line join=round,line cap=round,fill=fillColor] (572.57, 77.37) circle (  1.49);

\path[draw=drawColor,line width= 0.4pt,line join=round,line cap=round,fill=fillColor] (573.06, 65.87) circle (  1.49);

\path[draw=drawColor,line width= 0.4pt,line join=round,line cap=round,fill=fillColor] (573.53, 63.58) circle (  1.49);

\path[draw=drawColor,line width= 0.4pt,line join=round,line cap=round,fill=fillColor] (574.01,106.27) circle (  1.49);

\path[draw=drawColor,line width= 0.4pt,line join=round,line cap=round,fill=fillColor] (574.48, 70.47) circle (  1.49);

\path[draw=drawColor,line width= 0.4pt,line join=round,line cap=round,fill=fillColor] (574.96, 56.35) circle (  1.49);

\path[draw=drawColor,line width= 0.4pt,line join=round,line cap=round,fill=fillColor] (575.43, 80.98) circle (  1.49);

\path[draw=drawColor,line width= 0.4pt,line join=round,line cap=round,fill=fillColor] (575.92, 85.58) circle (  1.49);

\path[draw=drawColor,line width= 0.4pt,line join=round,line cap=round,fill=fillColor] (576.40, 59.31) circle (  1.49);

\path[draw=drawColor,line width= 0.4pt,line join=round,line cap=round,fill=fillColor] (576.89, 61.93) circle (  1.49);

\path[draw=drawColor,line width= 0.4pt,line join=round,line cap=round,fill=fillColor] (577.41,104.63) circle (  1.49);

\path[draw=drawColor,line width= 0.4pt,line join=round,line cap=round,fill=fillColor] (577.95,121.38) circle (  1.49);

\path[draw=drawColor,line width= 0.4pt,line join=round,line cap=round,fill=fillColor] (578.46, 61.60) circle (  1.49);

\path[draw=drawColor,line width= 0.4pt,line join=round,line cap=round,fill=fillColor] (578.95, 34.67) circle (  1.49);

\path[draw=drawColor,line width= 0.4pt,line join=round,line cap=round,fill=fillColor] (579.43, 41.90) circle (  1.49);

\path[draw=drawColor,line width= 0.4pt,line join=round,line cap=round,fill=fillColor] (579.92, 58.65) circle (  1.49);

\path[draw=drawColor,line width= 0.4pt,line join=round,line cap=round,fill=fillColor] (580.39, 36.97) circle (  1.49);

\path[draw=drawColor,line width= 0.4pt,line join=round,line cap=round,fill=fillColor] (580.88, 47.15) circle (  1.49);

\path[draw=drawColor,line width= 0.4pt,line join=round,line cap=round,fill=fillColor] (581.36, 37.96) circle (  1.49);

\path[draw=drawColor,line width= 0.4pt,line join=round,line cap=round,fill=fillColor] (581.85, 37.30) circle (  1.49);

\path[draw=drawColor,line width= 0.4pt,line join=round,line cap=round,fill=fillColor] (582.32, 60.29) circle (  1.49);

\path[draw=drawColor,line width= 0.4pt,line join=round,line cap=round,fill=fillColor] (582.93, 48.14) circle (  1.49);

\path[draw=drawColor,line width= 0.4pt,line join=round,line cap=round,fill=fillColor] (583.44,  6.42) circle (  1.49);

\path[draw=drawColor,line width= 0.4pt,line join=round,line cap=round,fill=fillColor] (583.98, 17.26) circle (  1.49);

\path[draw=drawColor,line width= 0.4pt,line join=round,line cap=round,fill=fillColor] (584.47,  7.08) circle (  1.49);

\path[draw=drawColor,line width= 0.4pt,line join=round,line cap=round,fill=fillColor] (585.16,  5.44) circle (  1.49);

\path[draw=drawColor,line width= 0.4pt,line join=round,line cap=round,fill=fillColor] (585.70,  2.81) circle (  1.49);

\path[draw=drawColor,line width= 0.4pt,line join=round,line cap=round,fill=fillColor] (586.20,  0.51) circle (  1.49);

\path[draw=drawColor,line width= 0.4pt,line join=round,line cap=round,fill=fillColor] (586.71,  3.47) circle (  1.49);

\path[draw=drawColor,line width= 0.4pt,line join=round,line cap=round,fill=fillColor] (587.20, 11.35) circle (  1.49);

\path[draw=drawColor,line width= 0.4pt,line join=round,line cap=round,fill=fillColor] (587.73,  5.11) circle (  1.49);

\path[draw=drawColor,line width= 0.4pt,line join=round,line cap=round,fill=fillColor] (592.29,  1.17) circle (  1.49);

\path[draw=drawColor,line width= 0.4pt,line join=round,line cap=round,fill=fillColor] (592.78,  4.45) circle (  1.49);

\path[draw=drawColor,line width= 0.4pt,line join=round,line cap=round,fill=fillColor] (593.29,  5.11) circle (  1.49);

\path[draw=drawColor,line width= 0.4pt,line join=round,line cap=round,fill=fillColor] (593.78,  4.45) circle (  1.49);

\path[draw=drawColor,line width= 0.4pt,line join=round,line cap=round,fill=fillColor] (594.34,  3.47) circle (  1.49);

\path[draw=drawColor,line width= 0.4pt,line join=round,line cap=round,fill=fillColor] (646.54, -0.15) circle (  1.49);

\path[draw=drawColor,line width= 0.4pt,line join=round,line cap=round,fill=fillColor] (648.18, 46.50) circle (  1.49);

\path[draw=drawColor,line width= 0.4pt,line join=round,line cap=round,fill=fillColor] (669.72,196.60) circle (  1.49);

\path[draw=drawColor,line width= 0.4pt,line join=round,line cap=round,fill=fillColor] (672.55,200.21) circle (  1.49);

\path[draw=drawColor,line width= 0.4pt,line join=round,line cap=round,fill=fillColor] (674.86,124.01) circle (  1.49);

\path[draw=drawColor,line width= 0.4pt,line join=round,line cap=round,fill=fillColor] (676.29, 15.62) circle (  1.49);

\path[draw=drawColor,line width= 0.4pt,line join=round,line cap=round,fill=fillColor] (676.73,184.45) circle (  1.49);

\path[draw=drawColor,line width= 0.4pt,line join=round,line cap=round,fill=fillColor] (677.19,195.29) circle (  1.49);

\path[draw=drawColor,line width= 0.4pt,line join=round,line cap=round,fill=fillColor] (677.68,166.05) circle (  1.49);

\path[draw=drawColor,line width= 0.4pt,line join=round,line cap=round,fill=fillColor] (678.14,196.60) circle (  1.49);

\path[draw=drawColor,line width= 0.4pt,line join=round,line cap=round,fill=fillColor] (678.58,197.59) circle (  1.49);

\path[draw=drawColor,line width= 0.4pt,line join=round,line cap=round,fill=fillColor] (679.07,186.42) circle (  1.49);

\path[draw=drawColor,line width= 0.4pt,line join=round,line cap=round,fill=fillColor] (679.53,158.50) circle (  1.49);

\path[draw=drawColor,line width= 0.4pt,line join=round,line cap=round,fill=fillColor] (680.05,198.24) circle (  1.49);

\path[draw=drawColor,line width= 0.4pt,line join=round,line cap=round,fill=fillColor] (680.54,197.59) circle (  1.49);

\path[draw=drawColor,line width= 0.4pt,line join=round,line cap=round,fill=fillColor] (681.49,197.26) circle (  1.49);

\path[draw=drawColor,line width= 0.4pt,line join=round,line cap=round,fill=fillColor] (683.96,197.59) circle (  1.49);

\path[draw=drawColor,line width= 0.4pt,line join=round,line cap=round,fill=fillColor] (684.44,197.26) circle (  1.49);

\path[draw=drawColor,line width= 0.4pt,line join=round,line cap=round,fill=fillColor] (684.91,199.23) circle (  1.49);

\path[draw=drawColor,line width= 0.4pt,line join=round,line cap=round,fill=fillColor] (685.39,198.24) circle (  1.49);

\path[draw=drawColor,line width= 0.4pt,line join=round,line cap=round,fill=fillColor] (685.81,197.59) circle (  1.49);

\path[draw=drawColor,line width= 0.4pt,line join=round,line cap=round,fill=fillColor] (686.34,198.24) circle (  1.49);

\path[draw=drawColor,line width= 0.4pt,line join=round,line cap=round,fill=fillColor] (686.78,198.24) circle (  1.49);

\path[draw=drawColor,line width= 0.4pt,line join=round,line cap=round,fill=fillColor] (687.27,197.91) circle (  1.49);

\path[draw=drawColor,line width= 0.4pt,line join=round,line cap=round,fill=fillColor] (687.79,196.93) circle (  1.49);

\path[draw=drawColor,line width= 0.4pt,line join=round,line cap=round,fill=fillColor] (688.27,197.26) circle (  1.49);

\path[draw=drawColor,line width= 0.4pt,line join=round,line cap=round,fill=fillColor] (688.76,198.24) circle (  1.49);

\path[draw=drawColor,line width= 0.4pt,line join=round,line cap=round,fill=fillColor] (689.22,196.60) circle (  1.49);

\path[draw=drawColor,line width= 0.4pt,line join=round,line cap=round,fill=fillColor] (689.73,196.93) circle (  1.49);

\path[draw=drawColor,line width= 0.4pt,line join=round,line cap=round,fill=fillColor] (690.18,197.26) circle (  1.49);

\path[draw=drawColor,line width= 0.4pt,line join=round,line cap=round,fill=fillColor] (690.63,197.26) circle (  1.49);

\path[draw=drawColor,line width= 0.4pt,line join=round,line cap=round,fill=fillColor] (691.17,197.59) circle (  1.49);

\path[draw=drawColor,line width= 0.4pt,line join=round,line cap=round,fill=fillColor] (691.61,198.24) circle (  1.49);

\path[draw=drawColor,line width= 0.4pt,line join=round,line cap=round,fill=fillColor] (692.07,197.26) circle (  1.49);

\path[draw=drawColor,line width= 0.4pt,line join=round,line cap=round,fill=fillColor] (692.51,203.50) circle (  1.49);

\path[draw=drawColor,line width= 0.4pt,line join=round,line cap=round,fill=fillColor] (692.92,198.24) circle (  1.49);

\path[draw=drawColor,line width= 0.4pt,line join=round,line cap=round,fill=fillColor] (693.42,200.54) circle (  1.49);

\path[draw=drawColor,line width= 0.4pt,line join=round,line cap=round,fill=fillColor] (693.92,198.24) circle (  1.49);

\path[draw=drawColor,line width= 0.4pt,line join=round,line cap=round,fill=fillColor] (694.41,196.93) circle (  1.49);

\path[draw=drawColor,line width= 0.4pt,line join=round,line cap=round,fill=fillColor] (694.87,167.04) circle (  1.49);

\path[draw=drawColor,line width= 0.4pt,line join=round,line cap=round,fill=fillColor] (695.85,202.18) circle (  1.49);

\path[draw=drawColor,line width= 0.4pt,line join=round,line cap=round,fill=fillColor] (696.29,196.60) circle (  1.49);

\path[draw=drawColor,line width= 0.4pt,line join=round,line cap=round,fill=fillColor] (696.73,195.94) circle (  1.49);

\path[draw=drawColor,line width= 0.4pt,line join=round,line cap=round,fill=fillColor] (697.26,194.30) circle (  1.49);

\path[draw=drawColor,line width= 0.4pt,line join=round,line cap=round,fill=fillColor] (697.75,195.94) circle (  1.49);

\path[draw=drawColor,line width= 0.4pt,line join=round,line cap=round,fill=fillColor] (698.29,194.96) circle (  1.49);

\path[draw=drawColor,line width= 0.4pt,line join=round,line cap=round,fill=fillColor] (698.73,195.94) circle (  1.49);

\path[draw=drawColor,line width= 0.4pt,line join=round,line cap=round,fill=fillColor] (699.19,130.58) circle (  1.49);

\path[draw=drawColor,line width= 0.4pt,line join=round,line cap=round,fill=fillColor] (699.66,185.10) circle (  1.49);

\path[draw=drawColor,line width= 0.4pt,line join=round,line cap=round,fill=fillColor] (700.23,196.27) circle (  1.49);

\path[draw=drawColor,line width= 0.4pt,line join=round,line cap=round,fill=fillColor] (700.81,196.27) circle (  1.49);

\path[draw=drawColor,line width= 0.4pt,line join=round,line cap=round,fill=fillColor] (701.30,196.27) circle (  1.49);

\path[draw=drawColor,line width= 0.4pt,line join=round,line cap=round,fill=fillColor] (701.86,197.26) circle (  1.49);

\path[draw=drawColor,line width= 0.4pt,line join=round,line cap=round,fill=fillColor] (702.31,195.29) circle (  1.49);

\path[draw=drawColor,line width= 0.4pt,line join=round,line cap=round,fill=fillColor] (702.77,195.29) circle (  1.49);

\path[draw=drawColor,line width= 0.4pt,line join=round,line cap=round,fill=fillColor] (703.21,196.27) circle (  1.49);

\path[draw=drawColor,line width= 0.4pt,line join=round,line cap=round,fill=fillColor] (703.71,195.62) circle (  1.49);

\path[draw=drawColor,line width= 0.4pt,line join=round,line cap=round,fill=fillColor] (704.13,196.27) circle (  1.49);

\path[draw=drawColor,line width= 0.4pt,line join=round,line cap=round,fill=fillColor] (704.57,196.60) circle (  1.49);

\path[draw=drawColor,line width= 0.4pt,line join=round,line cap=round,fill=fillColor] (705.16,195.29) circle (  1.49);

\path[draw=drawColor,line width= 0.4pt,line join=round,line cap=round,fill=fillColor] (705.78,195.94) circle (  1.49);

\path[draw=drawColor,line width= 0.4pt,line join=round,line cap=round,fill=fillColor] (706.21,195.94) circle (  1.49);

\path[draw=drawColor,line width= 0.4pt,line join=round,line cap=round,fill=fillColor] (706.62,195.62) circle (  1.49);

\path[draw=drawColor,line width= 0.4pt,line join=round,line cap=round,fill=fillColor] (707.04,195.62) circle (  1.49);

\path[draw=drawColor,line width= 0.4pt,line join=round,line cap=round,fill=fillColor] (707.47,194.63) circle (  1.49);
\definecolor{drawColor}{RGB}{0,0,0}
\definecolor{fillColor}{RGB}{255,255,255}

\path[draw=drawColor,line width= 0.4pt,line join=round,line cap=round,fill=fillColor] (636.21,241.56) rectangle (714.78,182.16);
\definecolor{fillColor}{RGB}{0,0,0}

\path[draw=drawColor,line width= 0.4pt,line join=round,line cap=round,fill=fillColor] (645.12,232.65) rectangle (652.24,226.71);
\definecolor{fillColor}{RGB}{255,0,0}

\path[draw=drawColor,line width= 0.4pt,line join=round,line cap=round,fill=fillColor] (645.12,220.77) rectangle (652.24,214.83);
\definecolor{fillColor}{RGB}{0,255,0}

\path[draw=drawColor,line width= 0.4pt,line join=round,line cap=round,fill=fillColor] (645.12,208.89) rectangle (652.24,202.95);
\definecolor{fillColor}{RGB}{0,0,255}

\path[draw=drawColor,line width= 0.4pt,line join=round,line cap=round,fill=fillColor] (645.12,197.01) rectangle (652.24,191.07);

\node[text=drawColor,anchor=base west,inner sep=0pt, outer sep=0pt, scale=  0.99] at (661.15,226.27) {Incident};

\node[text=drawColor,anchor=base west,inner sep=0pt, outer sep=0pt, scale=  0.99] at (661.15,214.39) {F. excelsior};

\node[text=drawColor,anchor=base west,inner sep=0pt, outer sep=0pt, scale=  0.99] at (661.15,202.51) {A. cordata};

\node[text=drawColor,anchor=base west,inner sep=0pt, outer sep=0pt, scale=  0.99] at (661.15,190.63) {M. alba};
\end{scope}
\end{tikzpicture}

  \end{tiny}
  \caption{Représentation visuelle de 3 flux de la lumière transmise\label{fig:flux}}
\end{figure}

\begin{figure}[h]
  \centering
  \begin{small}
    % Created by tikzDevice version 0.10.1 on 2016-11-20 12:15:34
% !TEX encoding = UTF-8 Unicode
\begin{tikzpicture}[x=1pt,y=1pt, scale=0.75]
\definecolor{fillColor}{RGB}{255,255,255}
\path[use as bounding box,fill=fillColor,fill opacity=0.00] (0,0) rectangle (650.43,216.81);
\begin{scope}
\path[clip] ( 47.52, 47.52) rectangle (214.17,169.29);
\definecolor{drawColor}{RGB}{0,0,0}

\path[draw=drawColor,line width= 1.2pt,line join=round] ( 56.26, 87.71) -- ( 76.84, 87.71);

\path[draw=drawColor,line width= 0.4pt,dash pattern=on 4pt off 4pt ,line join=round,line cap=round] ( 66.55, 75.05) -- ( 66.55, 82.82);

\path[draw=drawColor,line width= 0.4pt,dash pattern=on 4pt off 4pt ,line join=round,line cap=round] ( 66.55,130.12) -- ( 66.55,105.25);

\path[draw=drawColor,line width= 0.4pt,line join=round,line cap=round] ( 61.41, 75.05) -- ( 71.69, 75.05);

\path[draw=drawColor,line width= 0.4pt,line join=round,line cap=round] ( 61.41,130.12) -- ( 71.69,130.12);

\path[draw=drawColor,line width= 0.4pt,line join=round,line cap=round] ( 56.26, 82.82) --
	( 76.84, 82.82) --
	( 76.84,105.25) --
	( 56.26,105.25) --
	( 56.26, 82.82);

\path[draw=drawColor,line width= 1.2pt,line join=round] ( 81.98, 77.50) -- (102.56, 77.50);

\path[draw=drawColor,line width= 0.4pt,dash pattern=on 4pt off 4pt ,line join=round,line cap=round] ( 92.27, 68.94) -- ( 92.27, 72.98);

\path[draw=drawColor,line width= 0.4pt,dash pattern=on 4pt off 4pt ,line join=round,line cap=round] ( 92.27, 90.30) -- ( 92.27, 81.58);

\path[draw=drawColor,line width= 0.4pt,line join=round,line cap=round] ( 87.13, 68.94) -- ( 97.41, 68.94);

\path[draw=drawColor,line width= 0.4pt,line join=round,line cap=round] ( 87.13, 90.30) -- ( 97.41, 90.30);

\path[draw=drawColor,line width= 0.4pt,line join=round,line cap=round] ( 81.98, 72.98) --
	(102.56, 72.98) --
	(102.56, 81.58) --
	( 81.98, 81.58) --
	( 81.98, 72.98);

\path[draw=drawColor,line width= 1.2pt,line join=round] (107.70, 69.72) -- (128.27, 69.72);

\path[draw=drawColor,line width= 0.4pt,dash pattern=on 4pt off 4pt ,line join=round,line cap=round] (117.99, 64.55) -- (117.99, 66.31);

\path[draw=drawColor,line width= 0.4pt,dash pattern=on 4pt off 4pt ,line join=round,line cap=round] (117.99,115.06) -- (117.99, 74.88);

\path[draw=drawColor,line width= 0.4pt,line join=round,line cap=round] (112.84, 64.55) -- (123.13, 64.55);

\path[draw=drawColor,line width= 0.4pt,line join=round,line cap=round] (112.84,115.06) -- (123.13,115.06);

\path[draw=drawColor,line width= 0.4pt,line join=round,line cap=round] (107.70, 66.31) --
	(128.27, 66.31) --
	(128.27, 74.88) --
	(107.70, 74.88) --
	(107.70, 66.31);

\path[draw=drawColor,line width= 1.2pt,line join=round] (133.42, 61.59) -- (153.99, 61.59);

\path[draw=drawColor,line width= 0.4pt,dash pattern=on 4pt off 4pt ,line join=round,line cap=round] (143.70, 52.03) -- (143.70, 58.63);

\path[draw=drawColor,line width= 0.4pt,dash pattern=on 4pt off 4pt ,line join=round,line cap=round] (143.70,164.78) -- (143.70, 71.39);

\path[draw=drawColor,line width= 0.4pt,line join=round,line cap=round] (138.56, 52.03) -- (148.85, 52.03);

\path[draw=drawColor,line width= 0.4pt,line join=round,line cap=round] (138.56,164.78) -- (148.85,164.78);

\path[draw=drawColor,line width= 0.4pt,line join=round,line cap=round] (133.42, 58.63) --
	(153.99, 58.63) --
	(153.99, 71.39) --
	(133.42, 71.39) --
	(133.42, 58.63);

\path[draw=drawColor,line width= 1.2pt,line join=round] (159.13, 58.92) -- (179.71, 58.92);

\path[draw=drawColor,line width= 0.4pt,dash pattern=on 4pt off 4pt ,line join=round,line cap=round] (169.42, 56.20) -- (169.42, 57.24);

\path[draw=drawColor,line width= 0.4pt,dash pattern=on 4pt off 4pt ,line join=round,line cap=round] (169.42, 94.30) -- (169.42, 63.31);

\path[draw=drawColor,line width= 0.4pt,line join=round,line cap=round] (164.28, 56.20) -- (174.56, 56.20);

\path[draw=drawColor,line width= 0.4pt,line join=round,line cap=round] (164.28, 94.30) -- (174.56, 94.30);

\path[draw=drawColor,line width= 0.4pt,line join=round,line cap=round] (159.13, 57.24) --
	(179.71, 57.24) --
	(179.71, 63.31) --
	(159.13, 63.31) --
	(159.13, 57.24);

\path[draw=drawColor,line width= 1.2pt,line join=round] (184.85, 63.70) -- (205.43, 63.70);

\path[draw=drawColor,line width= 0.4pt,dash pattern=on 4pt off 4pt ,line join=round,line cap=round] (195.14, 59.55) -- (195.14, 60.65);

\path[draw=drawColor,line width= 0.4pt,dash pattern=on 4pt off 4pt ,line join=round,line cap=round] (195.14,106.33) -- (195.14, 68.98);

\path[draw=drawColor,line width= 0.4pt,line join=round,line cap=round] (190.00, 59.55) -- (200.28, 59.55);

\path[draw=drawColor,line width= 0.4pt,line join=round,line cap=round] (190.00,106.33) -- (200.28,106.33);

\path[draw=drawColor,line width= 0.4pt,line join=round,line cap=round] (184.85, 60.65) --
	(205.43, 60.65) --
	(205.43, 68.98) --
	(184.85, 68.98) --
	(184.85, 60.65);
\end{scope}
\begin{scope}
\path[clip] (  0.00,  0.00) rectangle (650.43,216.81);
\definecolor{drawColor}{RGB}{0,0,0}

\path[draw=drawColor,line width= 0.4pt,line join=round,line cap=round] ( 66.55, 47.52) -- (195.14, 47.52);

\path[draw=drawColor,line width= 0.4pt,line join=round,line cap=round] ( 66.55, 47.52) -- ( 66.55, 43.56);

\path[draw=drawColor,line width= 0.4pt,line join=round,line cap=round] ( 92.27, 47.52) -- ( 92.27, 43.56);

\path[draw=drawColor,line width= 0.4pt,line join=round,line cap=round] (117.99, 47.52) -- (117.99, 43.56);

\path[draw=drawColor,line width= 0.4pt,line join=round,line cap=round] (143.70, 47.52) -- (143.70, 43.56);

\path[draw=drawColor,line width= 0.4pt,line join=round,line cap=round] (169.42, 47.52) -- (169.42, 43.56);

\path[draw=drawColor,line width= 0.4pt,line join=round,line cap=round] (195.14, 47.52) -- (195.14, 43.56);

\node[text=drawColor,anchor=base,inner sep=0pt, outer sep=0pt, scale=  0.66] at ( 66.55, 33.26) {F. exc};

\node[text=drawColor,anchor=base,inner sep=0pt, outer sep=0pt, scale=  0.66] at ( 92.27, 33.26) {F. exc};

\node[text=drawColor,anchor=base,inner sep=0pt, outer sep=0pt, scale=  0.66] at (117.99, 33.26) {A. cor};

\node[text=drawColor,anchor=base,inner sep=0pt, outer sep=0pt, scale=  0.66] at (143.70, 33.26) {A. cor};

\node[text=drawColor,anchor=base,inner sep=0pt, outer sep=0pt, scale=  0.66] at (169.42, 33.26) {M. al};

\node[text=drawColor,anchor=base,inner sep=0pt, outer sep=0pt, scale=  0.66] at (195.14, 33.26) {M. al};

\path[draw=drawColor,line width= 0.4pt,line join=round,line cap=round] ( 47.52, 65.76) -- ( 47.52,157.55);

\path[draw=drawColor,line width= 0.4pt,line join=round,line cap=round] ( 47.52, 65.76) -- ( 43.56, 65.76);

\path[draw=drawColor,line width= 0.4pt,line join=round,line cap=round] ( 47.52, 84.11) -- ( 43.56, 84.11);

\path[draw=drawColor,line width= 0.4pt,line join=round,line cap=round] ( 47.52,102.47) -- ( 43.56,102.47);

\path[draw=drawColor,line width= 0.4pt,line join=round,line cap=round] ( 47.52,120.83) -- ( 43.56,120.83);

\path[draw=drawColor,line width= 0.4pt,line join=round,line cap=round] ( 47.52,139.19) -- ( 43.56,139.19);

\path[draw=drawColor,line width= 0.4pt,line join=round,line cap=round] ( 47.52,157.55) -- ( 43.56,157.55);

\node[text=drawColor,rotate= 90.00,anchor=base,inner sep=0pt, outer sep=0pt, scale=  0.66] at ( 38.02, 65.76) {100};

\node[text=drawColor,rotate= 90.00,anchor=base,inner sep=0pt, outer sep=0pt, scale=  0.66] at ( 38.02, 84.11) {150};

\node[text=drawColor,rotate= 90.00,anchor=base,inner sep=0pt, outer sep=0pt, scale=  0.66] at ( 38.02,102.47) {200};

\node[text=drawColor,rotate= 90.00,anchor=base,inner sep=0pt, outer sep=0pt, scale=  0.66] at ( 38.02,120.83) {250};

\node[text=drawColor,rotate= 90.00,anchor=base,inner sep=0pt, outer sep=0pt, scale=  0.66] at ( 38.02,139.19) {300};

\node[text=drawColor,rotate= 90.00,anchor=base,inner sep=0pt, outer sep=0pt, scale=  0.66] at ( 38.02,157.55) {350};
\end{scope}
\begin{scope}
\path[clip] (  7.92,  7.92) rectangle (222.09,208.89);
\definecolor{drawColor}{RGB}{0,0,0}

\node[text=drawColor,anchor=base,inner sep=0pt, outer sep=0pt, scale=  0.79] at (130.84,186.32) {\bfseries PAR transmis};

\node[text=drawColor,anchor=base,inner sep=0pt, outer sep=0pt, scale=  0.66] at (130.84, 17.42) {Espèces};

\node[text=drawColor,rotate= 90.00,anchor=base,inner sep=0pt, outer sep=0pt, scale=  0.66] at ( 22.18,108.41) {PAR (µmol.m$^{-2}$.s$^{-1}$)};
\end{scope}
\begin{scope}
\path[clip] (  0.00,  0.00) rectangle (650.43,216.81);
\definecolor{drawColor}{RGB}{0,0,0}

\path[draw=drawColor,line width= 0.4pt,line join=round,line cap=round] ( 47.52, 47.52) --
	(214.17, 47.52) --
	(214.17,169.29) --
	( 47.52,169.29) --
	( 47.52, 47.52);
\end{scope}
\begin{scope}
\path[clip] ( 47.52, 47.52) rectangle (214.17,169.29);
\definecolor{drawColor}{RGB}{0,0,0}

\node[text=drawColor,anchor=base,inner sep=0pt, outer sep=0pt, scale=  0.66] at ( 77.16, 63.36) {Valeur de l'incident};

\node[text=drawColor,anchor=base west,inner sep=0pt, outer sep=0pt, scale=  0.66] at ( 70.79, 53.17) {1396.27};
\definecolor{drawColor}{RGB}{255,0,0}

\path[draw=drawColor,line width= 0.4pt,line join=round,line cap=round] ( 66.55, 92.02) circle (  1.49);

\path[draw=drawColor,line width= 0.4pt,line join=round,line cap=round] ( 92.27, 78.30) circle (  1.49);

\path[draw=drawColor,line width= 0.4pt,line join=round,line cap=round] (117.99, 73.37) circle (  1.49);

\path[draw=drawColor,line width= 0.4pt,line join=round,line cap=round] (143.70, 68.44) circle (  1.49);

\path[draw=drawColor,line width= 0.4pt,line join=round,line cap=round] (169.42, 62.15) circle (  1.49);

\path[draw=drawColor,line width= 0.4pt,line join=round,line cap=round] (195.14, 67.36) circle (  1.49);
\end{scope}
\begin{scope}
\path[clip] (261.69, 47.52) rectangle (428.34,169.29);
\definecolor{drawColor}{RGB}{0,0,0}

\path[draw=drawColor,line width= 1.2pt,line join=round] (270.43,108.15) -- (291.01,108.15);

\path[draw=drawColor,line width= 0.4pt,dash pattern=on 4pt off 4pt ,line join=round,line cap=round] (280.72, 95.78) -- (280.72,104.17);

\path[draw=drawColor,line width= 0.4pt,dash pattern=on 4pt off 4pt ,line join=round,line cap=round] (280.72,146.16) -- (280.72,124.66);

\path[draw=drawColor,line width= 0.4pt,line join=round,line cap=round] (275.58, 95.78) -- (285.86, 95.78);

\path[draw=drawColor,line width= 0.4pt,line join=round,line cap=round] (275.58,146.16) -- (285.86,146.16);

\path[draw=drawColor,line width= 0.4pt,line join=round,line cap=round] (270.43,104.17) --
	(291.01,104.17) --
	(291.01,124.66) --
	(270.43,124.66) --
	(270.43,104.17);

\path[draw=drawColor,line width= 1.2pt,line join=round] (296.15,102.63) -- (316.73,102.63);

\path[draw=drawColor,line width= 0.4pt,dash pattern=on 4pt off 4pt ,line join=round,line cap=round] (306.44, 92.37) -- (306.44, 97.26);

\path[draw=drawColor,line width= 0.4pt,dash pattern=on 4pt off 4pt ,line join=round,line cap=round] (306.44,115.02) -- (306.44,106.69);

\path[draw=drawColor,line width= 0.4pt,line join=round,line cap=round] (301.30, 92.37) -- (311.58, 92.37);

\path[draw=drawColor,line width= 0.4pt,line join=round,line cap=round] (301.30,115.02) -- (311.58,115.02);

\path[draw=drawColor,line width= 0.4pt,line join=round,line cap=round] (296.15, 97.26) --
	(316.73, 97.26) --
	(316.73,106.69) --
	(296.15,106.69) --
	(296.15, 97.26);

\path[draw=drawColor,line width= 1.2pt,line join=round] (321.87, 95.89) -- (342.44, 95.89);

\path[draw=drawColor,line width= 0.4pt,dash pattern=on 4pt off 4pt ,line join=round,line cap=round] (332.16, 89.06) -- (332.16, 92.03);

\path[draw=drawColor,line width= 0.4pt,dash pattern=on 4pt off 4pt ,line join=round,line cap=round] (332.16,142.03) -- (332.16,100.88);

\path[draw=drawColor,line width= 0.4pt,line join=round,line cap=round] (327.01, 89.06) -- (337.30, 89.06);

\path[draw=drawColor,line width= 0.4pt,line join=round,line cap=round] (327.01,142.03) -- (337.30,142.03);

\path[draw=drawColor,line width= 0.4pt,line join=round,line cap=round] (321.87, 92.03) --
	(342.44, 92.03) --
	(342.44,100.88) --
	(321.87,100.88) --
	(321.87, 92.03);

\path[draw=drawColor,line width= 1.2pt,line join=round] (347.59, 70.51) -- (368.16, 70.51);

\path[draw=drawColor,line width= 0.4pt,dash pattern=on 4pt off 4pt ,line join=round,line cap=round] (357.87, 52.03) -- (357.87, 67.69);

\path[draw=drawColor,line width= 0.4pt,dash pattern=on 4pt off 4pt ,line join=round,line cap=round] (357.87,164.78) -- (357.87, 79.52);

\path[draw=drawColor,line width= 0.4pt,line join=round,line cap=round] (352.73, 52.03) -- (363.02, 52.03);

\path[draw=drawColor,line width= 0.4pt,line join=round,line cap=round] (352.73,164.78) -- (363.02,164.78);

\path[draw=drawColor,line width= 0.4pt,line join=round,line cap=round] (347.59, 67.69) --
	(368.16, 67.69) --
	(368.16, 79.52) --
	(347.59, 79.52) --
	(347.59, 67.69);

\path[draw=drawColor,line width= 1.2pt,line join=round] (373.30, 70.68) -- (393.88, 70.68);

\path[draw=drawColor,line width= 0.4pt,dash pattern=on 4pt off 4pt ,line join=round,line cap=round] (383.59, 67.00) -- (383.59, 68.52);

\path[draw=drawColor,line width= 0.4pt,dash pattern=on 4pt off 4pt ,line join=round,line cap=round] (383.59,104.07) -- (383.59, 74.99);

\path[draw=drawColor,line width= 0.4pt,line join=round,line cap=round] (378.45, 67.00) -- (388.73, 67.00);

\path[draw=drawColor,line width= 0.4pt,line join=round,line cap=round] (378.45,104.07) -- (388.73,104.07);

\path[draw=drawColor,line width= 0.4pt,line join=round,line cap=round] (373.30, 68.52) --
	(393.88, 68.52) --
	(393.88, 74.99) --
	(373.30, 74.99) --
	(373.30, 68.52);

\path[draw=drawColor,line width= 1.2pt,line join=round] (399.02, 79.51) -- (419.60, 79.51);

\path[draw=drawColor,line width= 0.4pt,dash pattern=on 4pt off 4pt ,line join=round,line cap=round] (409.31, 73.93) -- (409.31, 75.78);

\path[draw=drawColor,line width= 0.4pt,dash pattern=on 4pt off 4pt ,line join=round,line cap=round] (409.31,119.15) -- (409.31, 85.08);

\path[draw=drawColor,line width= 0.4pt,line join=round,line cap=round] (404.17, 73.93) -- (414.45, 73.93);

\path[draw=drawColor,line width= 0.4pt,line join=round,line cap=round] (404.17,119.15) -- (414.45,119.15);

\path[draw=drawColor,line width= 0.4pt,line join=round,line cap=round] (399.02, 75.78) --
	(419.60, 75.78) --
	(419.60, 85.08) --
	(399.02, 85.08) --
	(399.02, 75.78);
\end{scope}
\begin{scope}
\path[clip] (  0.00,  0.00) rectangle (650.43,216.81);
\definecolor{drawColor}{RGB}{0,0,0}

\path[draw=drawColor,line width= 0.4pt,line join=round,line cap=round] (280.72, 47.52) -- (409.31, 47.52);

\path[draw=drawColor,line width= 0.4pt,line join=round,line cap=round] (280.72, 47.52) -- (280.72, 43.56);

\path[draw=drawColor,line width= 0.4pt,line join=round,line cap=round] (306.44, 47.52) -- (306.44, 43.56);

\path[draw=drawColor,line width= 0.4pt,line join=round,line cap=round] (332.16, 47.52) -- (332.16, 43.56);

\path[draw=drawColor,line width= 0.4pt,line join=round,line cap=round] (357.87, 47.52) -- (357.87, 43.56);

\path[draw=drawColor,line width= 0.4pt,line join=round,line cap=round] (383.59, 47.52) -- (383.59, 43.56);

\path[draw=drawColor,line width= 0.4pt,line join=round,line cap=round] (409.31, 47.52) -- (409.31, 43.56);

\node[text=drawColor,anchor=base,inner sep=0pt, outer sep=0pt, scale=  0.66] at (280.72, 33.26) {F. exc};

\node[text=drawColor,anchor=base,inner sep=0pt, outer sep=0pt, scale=  0.66] at (306.44, 33.26) {F. exc};

\node[text=drawColor,anchor=base,inner sep=0pt, outer sep=0pt, scale=  0.66] at (332.16, 33.26) {A. cor};

\node[text=drawColor,anchor=base,inner sep=0pt, outer sep=0pt, scale=  0.66] at (357.87, 33.26) {A. cor};

\node[text=drawColor,anchor=base,inner sep=0pt, outer sep=0pt, scale=  0.66] at (383.59, 33.26) {M. al};

\node[text=drawColor,anchor=base,inner sep=0pt, outer sep=0pt, scale=  0.66] at (409.31, 33.26) {M. al};

\path[draw=drawColor,line width= 0.4pt,line join=round,line cap=round] (261.69, 71.29) -- (261.69,151.86);

\path[draw=drawColor,line width= 0.4pt,line join=round,line cap=round] (261.69, 71.29) -- (257.73, 71.29);

\path[draw=drawColor,line width= 0.4pt,line join=round,line cap=round] (261.69, 98.14) -- (257.73, 98.14);

\path[draw=drawColor,line width= 0.4pt,line join=round,line cap=round] (261.69,125.00) -- (257.73,125.00);

\path[draw=drawColor,line width= 0.4pt,line join=round,line cap=round] (261.69,151.86) -- (257.73,151.86);

\node[text=drawColor,rotate= 90.00,anchor=base,inner sep=0pt, outer sep=0pt, scale=  0.66] at (252.19, 71.29) {40};

\node[text=drawColor,rotate= 90.00,anchor=base,inner sep=0pt, outer sep=0pt, scale=  0.66] at (252.19, 98.14) {60};

\node[text=drawColor,rotate= 90.00,anchor=base,inner sep=0pt, outer sep=0pt, scale=  0.66] at (252.19,125.00) {80};

\node[text=drawColor,rotate= 90.00,anchor=base,inner sep=0pt, outer sep=0pt, scale=  0.66] at (252.19,151.86) {100};
\end{scope}
\begin{scope}
\path[clip] (222.09,  7.92) rectangle (436.26,208.89);
\definecolor{drawColor}{RGB}{0,0,0}

\node[text=drawColor,anchor=base,inner sep=0pt, outer sep=0pt, scale=  0.79] at (345.01,186.32) {\bfseries UVA-bleu transmis};

\node[text=drawColor,anchor=base,inner sep=0pt, outer sep=0pt, scale=  0.66] at (345.01, 17.42) {Espèces};

\node[text=drawColor,rotate= 90.00,anchor=base,inner sep=0pt, outer sep=0pt, scale=  0.66] at (236.35,108.41) {UVA-bleu (µmol.m$^{-2}$.s$^{-1}$)};
\end{scope}
\begin{scope}
\path[clip] (  0.00,  0.00) rectangle (650.43,216.81);
\definecolor{drawColor}{RGB}{0,0,0}

\path[draw=drawColor,line width= 0.4pt,line join=round,line cap=round] (261.69, 47.52) --
	(428.34, 47.52) --
	(428.34,169.29) --
	(261.69,169.29) --
	(261.69, 47.52);
\end{scope}
\begin{scope}
\path[clip] (261.69, 47.52) rectangle (428.34,169.29);
\definecolor{drawColor}{RGB}{0,0,0}

\node[text=drawColor,anchor=base,inner sep=0pt, outer sep=0pt, scale=  0.66] at (291.33, 63.36) {Valeur de l'incident};

\node[text=drawColor,anchor=base west,inner sep=0pt, outer sep=0pt, scale=  0.66] at (283.31, 53.17) {375.8148};
\definecolor{drawColor}{RGB}{255,0,0}

\path[draw=drawColor,line width= 0.4pt,line join=round,line cap=round] (280.72,112.23) circle (  1.49);

\path[draw=drawColor,line width= 0.4pt,line join=round,line cap=round] (306.44,102.77) circle (  1.49);

\path[draw=drawColor,line width= 0.4pt,line join=round,line cap=round] (332.16, 99.40) circle (  1.49);

\path[draw=drawColor,line width= 0.4pt,line join=round,line cap=round] (357.87, 76.27) circle (  1.49);

\path[draw=drawColor,line width= 0.4pt,line join=round,line cap=round] (383.59, 73.42) circle (  1.49);

\path[draw=drawColor,line width= 0.4pt,line join=round,line cap=round] (409.31, 82.62) circle (  1.49);
\end{scope}
\begin{scope}
\path[clip] (475.86, 47.52) rectangle (642.51,169.29);
\definecolor{drawColor}{RGB}{0,0,0}

\path[draw=drawColor,line width= 1.2pt,line join=round] (484.60,141.93) -- (505.18,141.93);

\path[draw=drawColor,line width= 0.4pt,dash pattern=on 4pt off 4pt ,line join=round,line cap=round] (494.89,112.82) -- (494.89,137.62);

\path[draw=drawColor,line width= 0.4pt,dash pattern=on 4pt off 4pt ,line join=round,line cap=round] (494.89,164.56) -- (494.89,154.86);

\path[draw=drawColor,line width= 0.4pt,line join=round,line cap=round] (489.75,112.82) -- (500.03,112.82);

\path[draw=drawColor,line width= 0.4pt,line join=round,line cap=round] (489.75,164.56) -- (500.03,164.56);

\path[draw=drawColor,line width= 0.4pt,line join=round,line cap=round] (484.60,137.62) --
	(505.18,137.62) --
	(505.18,154.86) --
	(484.60,154.86) --
	(484.60,137.62);

\path[draw=drawColor,line width= 1.2pt,line join=round] (510.32,148.07) -- (530.90,148.07);

\path[draw=drawColor,line width= 0.4pt,dash pattern=on 4pt off 4pt ,line join=round,line cap=round] (520.61,125.98) -- (520.61,138.91);

\path[draw=drawColor,line width= 0.4pt,dash pattern=on 4pt off 4pt ,line join=round,line cap=round] (520.61,162.84) -- (520.61,155.62);

\path[draw=drawColor,line width= 0.4pt,line join=round,line cap=round] (515.47,125.98) -- (525.75,125.98);

\path[draw=drawColor,line width= 0.4pt,line join=round,line cap=round] (515.47,162.84) -- (525.75,162.84);

\path[draw=drawColor,line width= 0.4pt,line join=round,line cap=round] (510.32,138.91) --
	(530.90,138.91) --
	(530.90,155.62) --
	(510.32,155.62) --
	(510.32,138.91);

\path[draw=drawColor,line width= 1.2pt,line join=round] (536.04, 92.78) -- (556.61, 92.78);

\path[draw=drawColor,line width= 0.4pt,dash pattern=on 4pt off 4pt ,line join=round,line cap=round] (546.33, 76.18) -- (546.33, 82.75);

\path[draw=drawColor,line width= 0.4pt,dash pattern=on 4pt off 4pt ,line join=round,line cap=round] (546.33,157.88) -- (546.33,104.74);

\path[draw=drawColor,line width= 0.4pt,line join=round,line cap=round] (541.18, 76.18) -- (551.47, 76.18);

\path[draw=drawColor,line width= 0.4pt,line join=round,line cap=round] (541.18,157.88) -- (551.47,157.88);

\path[draw=drawColor,line width= 0.4pt,line join=round,line cap=round] (536.04, 82.75) --
	(556.61, 82.75) --
	(556.61,104.74) --
	(536.04,104.74) --
	(536.04, 82.75);

\path[draw=drawColor,line width= 1.2pt,line join=round] (561.76, 66.04) -- (582.33, 66.04);

\path[draw=drawColor,line width= 0.4pt,dash pattern=on 4pt off 4pt ,line join=round,line cap=round] (572.04, 52.03) -- (572.04, 56.77);

\path[draw=drawColor,line width= 0.4pt,dash pattern=on 4pt off 4pt ,line join=round,line cap=round] (572.04,164.78) -- (572.04, 89.33);

\path[draw=drawColor,line width= 0.4pt,line join=round,line cap=round] (566.90, 52.03) -- (577.19, 52.03);

\path[draw=drawColor,line width= 0.4pt,line join=round,line cap=round] (566.90,164.78) -- (577.19,164.78);

\path[draw=drawColor,line width= 0.4pt,line join=round,line cap=round] (561.76, 56.77) --
	(582.33, 56.77) --
	(582.33, 89.33) --
	(561.76, 89.33) --
	(561.76, 56.77);

\path[draw=drawColor,line width= 1.2pt,line join=round] (587.47, 66.04) -- (608.05, 66.04);

\path[draw=drawColor,line width= 0.4pt,dash pattern=on 4pt off 4pt ,line join=round,line cap=round] (597.76, 56.13) -- (597.76, 58.71);

\path[draw=drawColor,line width= 0.4pt,dash pattern=on 4pt off 4pt ,line join=round,line cap=round] (597.76,156.59) -- (597.76, 77.68);

\path[draw=drawColor,line width= 0.4pt,line join=round,line cap=round] (592.62, 56.13) -- (602.90, 56.13);

\path[draw=drawColor,line width= 0.4pt,line join=round,line cap=round] (592.62,156.59) -- (602.90,156.59);

\path[draw=drawColor,line width= 0.4pt,line join=round,line cap=round] (587.47, 58.71) --
	(608.05, 58.71) --
	(608.05, 77.68) --
	(587.47, 77.68) --
	(587.47, 58.71);

\path[draw=drawColor,line width= 1.2pt,line join=round] (613.19,106.14) -- (633.77,106.14);

\path[draw=drawColor,line width= 0.4pt,dash pattern=on 4pt off 4pt ,line join=round,line cap=round] (623.48, 89.54) -- (623.48, 98.38);

\path[draw=drawColor,line width= 0.4pt,dash pattern=on 4pt off 4pt ,line join=round,line cap=round] (623.48,164.13) -- (623.48,122.96);

\path[draw=drawColor,line width= 0.4pt,line join=round,line cap=round] (618.34, 89.54) -- (628.62, 89.54);

\path[draw=drawColor,line width= 0.4pt,line join=round,line cap=round] (618.34,164.13) -- (628.62,164.13);

\path[draw=drawColor,line width= 0.4pt,line join=round,line cap=round] (613.19, 98.38) --
	(633.77, 98.38) --
	(633.77,122.96) --
	(613.19,122.96) --
	(613.19, 98.38);
\end{scope}
\begin{scope}
\path[clip] (  0.00,  0.00) rectangle (650.43,216.81);
\definecolor{drawColor}{RGB}{0,0,0}

\path[draw=drawColor,line width= 0.4pt,line join=round,line cap=round] (494.89, 47.52) -- (623.48, 47.52);

\path[draw=drawColor,line width= 0.4pt,line join=round,line cap=round] (494.89, 47.52) -- (494.89, 43.56);

\path[draw=drawColor,line width= 0.4pt,line join=round,line cap=round] (520.61, 47.52) -- (520.61, 43.56);

\path[draw=drawColor,line width= 0.4pt,line join=round,line cap=round] (546.33, 47.52) -- (546.33, 43.56);

\path[draw=drawColor,line width= 0.4pt,line join=round,line cap=round] (572.04, 47.52) -- (572.04, 43.56);

\path[draw=drawColor,line width= 0.4pt,line join=round,line cap=round] (597.76, 47.52) -- (597.76, 43.56);

\path[draw=drawColor,line width= 0.4pt,line join=round,line cap=round] (623.48, 47.52) -- (623.48, 43.56);

\node[text=drawColor,anchor=base,inner sep=0pt, outer sep=0pt, scale=  0.66] at (494.89, 33.26) {F. exc};

\node[text=drawColor,anchor=base,inner sep=0pt, outer sep=0pt, scale=  0.66] at (520.61, 33.26) {F. exc};

\node[text=drawColor,anchor=base,inner sep=0pt, outer sep=0pt, scale=  0.66] at (546.33, 33.26) {A. cor};

\node[text=drawColor,anchor=base,inner sep=0pt, outer sep=0pt, scale=  0.66] at (572.04, 33.26) {A. cor};

\node[text=drawColor,anchor=base,inner sep=0pt, outer sep=0pt, scale=  0.66] at (597.76, 33.26) {M. al};

\node[text=drawColor,anchor=base,inner sep=0pt, outer sep=0pt, scale=  0.66] at (623.48, 33.26) {M. al};

\path[draw=drawColor,line width= 0.4pt,line join=round,line cap=round] (475.86, 68.41) -- (475.86,154.65);

\path[draw=drawColor,line width= 0.4pt,line join=round,line cap=round] (475.86, 68.41) -- (471.90, 68.41);

\path[draw=drawColor,line width= 0.4pt,line join=round,line cap=round] (475.86, 89.97) -- (471.90, 89.97);

\path[draw=drawColor,line width= 0.4pt,line join=round,line cap=round] (475.86,111.53) -- (471.90,111.53);

\path[draw=drawColor,line width= 0.4pt,line join=round,line cap=round] (475.86,133.09) -- (471.90,133.09);

\path[draw=drawColor,line width= 0.4pt,line join=round,line cap=round] (475.86,154.65) -- (471.90,154.65);

\node[text=drawColor,rotate= 90.00,anchor=base,inner sep=0pt, outer sep=0pt, scale=  0.66] at (466.36, 68.41) {0.4};

\node[text=drawColor,rotate= 90.00,anchor=base,inner sep=0pt, outer sep=0pt, scale=  0.66] at (466.36, 89.97) {0.5};

\node[text=drawColor,rotate= 90.00,anchor=base,inner sep=0pt, outer sep=0pt, scale=  0.66] at (466.36,111.53) {0.6};

\node[text=drawColor,rotate= 90.00,anchor=base,inner sep=0pt, outer sep=0pt, scale=  0.66] at (466.36,133.09) {0.7};

\node[text=drawColor,rotate= 90.00,anchor=base,inner sep=0pt, outer sep=0pt, scale=  0.66] at (466.36,154.65) {0.8};
\end{scope}
\begin{scope}
\path[clip] (436.26,  7.92) rectangle (650.43,208.89);
\definecolor{drawColor}{RGB}{0,0,0}

\node[text=drawColor,anchor=base,inner sep=0pt, outer sep=0pt, scale=  0.79] at (559.18,186.32) {\bfseries ZETA transmis};

\node[text=drawColor,anchor=base,inner sep=0pt, outer sep=0pt, scale=  0.66] at (559.18, 17.42) {Espèces};

\node[text=drawColor,rotate= 90.00,anchor=base,inner sep=0pt, outer sep=0pt, scale=  0.66] at (450.52,108.41) {ZETA};
\end{scope}
\begin{scope}
\path[clip] (  0.00,  0.00) rectangle (650.43,216.81);
\definecolor{drawColor}{RGB}{0,0,0}

\path[draw=drawColor,line width= 0.4pt,line join=round,line cap=round] (475.86, 47.52) --
	(642.51, 47.52) --
	(642.51,169.29) --
	(475.86,169.29) --
	(475.86, 47.52);
\end{scope}
\begin{scope}
\path[clip] (475.86, 47.52) rectangle (642.51,169.29);
\definecolor{drawColor}{RGB}{0,0,0}

\node[text=drawColor,anchor=base,inner sep=0pt, outer sep=0pt, scale=  0.66] at (505.50, 63.36) {Valeur de l'incident};

\node[text=drawColor,anchor=base west,inner sep=0pt, outer sep=0pt, scale=  0.66] at (497.48, 53.17) {1.113237};
\definecolor{drawColor}{RGB}{255,0,0}

\path[draw=drawColor,line width= 0.4pt,line join=round,line cap=round] (494.89,143.78) circle (  1.49);

\path[draw=drawColor,line width= 0.4pt,line join=round,line cap=round] (520.61,147.19) circle (  1.49);

\path[draw=drawColor,line width= 0.4pt,line join=round,line cap=round] (546.33, 97.98) circle (  1.49);

\path[draw=drawColor,line width= 0.4pt,line join=round,line cap=round] (572.04, 77.44) circle (  1.49);

\path[draw=drawColor,line width= 0.4pt,line join=round,line cap=round] (597.76, 72.81) circle (  1.49);

\path[draw=drawColor,line width= 0.4pt,line join=round,line cap=round] (623.48,111.94) circle (  1.49);
\end{scope}
\end{tikzpicture}

  \end{small}
  \caption{Distribution des flux photoniques; la moyenne est représentée par un
    cercle rouge\label{fig:boxplots}}
\end{figure}


\subsection{Une très forte variation infraspécifique de la transmission selon la
  morphologie des houppiers }

Les mesures des caractères morphologiques et des lumières incidentes et
transmises nous ont fourni des données qui ont été combinées, j'ai ainsi obtenu
2 graphiques. Le premier graphique (voir figure~\ref{fig:zeta_min}) représente
l'évolution de l'intensité de l'ombre en fonction de la surface foliaire totale.
On remarque, pour les 3 espèces, que plus cette surface augmente, plus
l'intensité de l'ombrage augmente. De plus, les coefficients de pente des 3
espèces sont très proches puisqu'il y a une différence de 25\% entre le
\textit{M. alba} et le \textit{F. excelsior} et une différence de 3\% entre
\textit{F. excelsior} et \textit{A. cordata}. Il semblerait donc que l'intensité
de l'ombrage dépende d'un facteur initial et de la surface foliaire totale
quelque soit l'espèce.

Le second graphique (voir figure~\ref{fig:zeta_moyen}) représente l'évolution de $\zeta$
moyen en fonction de du LAD. $\zeta$ moyen est un indicateur de la densité du
feuillage, le LAD est une densité, donc au sein de chaque espèce, plus le LAD
est haut et plus $\zeta$ moyen est haut; autrement dit une augmentation de densité
induit une diminution des trouées, donc plus d'ombre. Le raisonnement se vérifie sur la
figure~\ref{fig:zeta_moyen} pour le frêne et l'aulne mais pas pour le mûrier. Deux
hypothèses peuvent expliquer cela, soit Tree Analyzer sous-estime les
surfaces foliaires puisque les feuilles au centre du houppier ne sont pas
visibles, soit la taille d'échantillonnage est trop faible.

\begin{figure}
  \centering
  \includegraphics[width=180mm]{zeta_min.png}
  \caption{Évolution de l'intensité de l'ombrage en fonction de la surface
    foliaire pour chacune des espèces\label{fig:zeta_min}}
\end{figure}

\begin{figure}
  \centering
  \includegraphics[width=180mm]{zeta_moyen.png}
  \caption{Évolution de la proportion de trouées en fonction du LAD pour chacune
    des espèces\label{fig:zeta_moyen}}
\end{figure}


\chapter{Discussion}

L'étude avait pour objectif mettre en évidence l'existence d'une relation entre
caractères morphologiques du houppier et transmission de la lumière. Elle s'est
déroulée dans un contexte agroforestier qui étudie l'interaction entre les
arbres et les herbacés.

\section{Relations allométriques}

L'analyse des échantillons de feuilles et de branches ont montré qu'il existe
des relations allométriques entre surface et longueur d'une feuille et entre
longueur d'un axe et nombre de feuilles sur l'axe. L'objectif de ces relations
était de calculer la surface foliaire d'un arbre pour la comparer avec la
surface foliaire calculée par Tree Analyzer. Il s'est avéré que les 2 méthodes de
mesure ne donnaient pas les mêmes résultats, il y a dans tous les cas au moins
3m$^{2}$ et parfois même jusqu'à 7m$^{2}$ de différence, autrement dit le double
du résultat de Tree Analyzer. La première hypothèse est que Tree Analyzer sous
estime trop la surface foliaire puisqu'il utilise des clichés photographiques, il
ne voit donc pas les feuilles au c$\oe$ur de l'arbre qui peuvent être cachées
par les feuilles périphériques du houppier. Il existe certes une incertitude de
mesure mais les tests menés par Phattaralerphong et Sinoquet (2005) ont montré
que cette incertitude est négligeable, les coefficient de détermination sont
égaux à 0.99 en moyenne lorsqu'ils ont fait leurs essais avec 8 photographies.

La deuxième hypothèse est que les relations allométriques sont fausses. La
définition d'une relation allométrique
a été respectée, les coefficients de détermination sont excellents pour ce qui
est des relations entre longueur et surface du limbe puisqu'on est aux alentours
de 0.94 et sont bons pour ce qui est des relations entre longueur de l'axe et
nombre de feuilles sur l'axe puisqu'on est aux alentours de 0.83. Des recherches
ont montré qu'il était possible de modéliser la croissance d'un arbre,
\citet{MAR_ref41} ont établi des modèles de distribution de la biomasse des
houppiers. Pour écrire leur article ils se sont appuyé sur un article qui
mettait en évidence l'existence de relations allométriques pour décrire des
organes des arbres. Ces éléments montrent que les relations allométriques
établis semblent justes, cette hypothèse est donc fausse, les relations
allométriques sont correctes.

La troisième hypothèse est que l'incertitude est issue d'un mauvais
échantillonnage. Les relations ont été établies en laboratoire à partir de
prélèvements. Il était impossible de prélever la branche portant le bourgeon
terminal car il ne fallait pas étêter les arbres. Il donc été impossible
d'échantillonner toutes les catégories de branches. Ces branches portaient des
feuilles souvent plus grande que les feuilles de la base, l'entre-n$\oe$ud était
plus important que l'entre-n$\oe$ud entre les feuilles à la base du houppier.
L'échantillonnage était donc biaisé et ne rendait pas compte de la réalité.
L'erreur est donc dû à un mauvais échantillonnage.


\section{La transmission de la lumière par les houppiers}

L'étude des flux photoniques globaux (PAR), UVA-bleu et de $\zeta$ ont mis en
évidence une forte influence d'un couvert arbustif sur la transmission de la
lumière. Cette filtration de la lumière avait été mise en évidence par
\citet{MAR_ref36} dans la cas du PAR et par \citet{MAR_ref30} dans le cas de
$\zeta$. Les résultats de ces expériences, qui ont des résultats du même ordre
de grandeur que ceux de ce rapport, montrent que le protocole expérimental
mis en place c'est avéré fiable.

Les mesures des flux photoniques n'ont pas été sans complication. Initialement,
il avait été prévu de faire des mesures pour un jour ensoleillé, un jour voilé
et un jour complètement couvert afin de mettre une évidence de transmission de
la lumière en fonction des conditions climatiques. L'idée initiale du protocole
était de procéder comme l'avaient fait \citet{MAR_ref36}. Ces 3 ciels (bleu,
voilé, bouché) n'ont pas été rencontré pendant la campagne de mesure, il a fallu
se contenter d'un ciel bleu au moins sur notre temps de mesure, c'est cela qui a
incité à choisir les enregistrements du 9 juillet. Certaines mesures ponctuelles
peuvent avoir été perturbées par des insectes. L'herbe était haute, environ
60~cm, et les capteurs étaient à 80~cm, il n'est pas à exclure que des insectes
ont pu passer par dessus les capteurs au moment de l'acquisition. Ces
perturbations peuvent expliquer l'existence de valeurs aberrantes mais elles ont
pu être éliminées facilement.


\chapter{Conclusion et perspectives}

Les recherches décrites dans ce rapport ont répondu favorablement à l'hypothèse
qui avait été formulée sous forme de problématique: Existe-t-il une différence de
transmission de la lumière par les houppiers, compte tenu des différences
morphologiques qui peuvent exister?

Les relations allométriques ont montré qu'il existe, au sein de chaque espèce,
des relations dans un organe ou entre différents organes de l'arbre. La
mise en relation des caractères morphologiques des houppiers et de $\zeta$
ont montré qu'il existe, chez chacune de ces espèces, une variation de la
transmission de la lumière selon les dimensions des caractères morphologiques.
Les résultats des deux principales expériences montrent donc qu'il existe une
différence de transmission de la lumière par les houppiers, compte-tenu des
différences morphologiques qui existent.\\

Bien que les expériences présentées dans ce rapport répondent favorablement à
la problématique, d'autres études peuvent venir renforcer la réponse ou peuvent
étudier d'autres aspects de la relation entre un arbre et les herbes en dessous.
Une étude statistique plus poussée permettrait de répondre à une question qui
n'a pas pu être résolue avec les résultats: Qu'est-ce qui provoque la différence
de transmission de la lumière entre les différentes espèces? Nous avons mis en
évidence une différence de transmission de la lumière compte tenu des
différences morphologiques au sein de chaque espèce, mais, les résultats des
expériences ne mettent pas en évidence les causes des différences de
transmission entre espèces. Une première étude statistique permettrait de
renforcer la réponse et mettrait en évidence si il existe une réelle différence
de transmission entre les 3 espèces. Il faudrait pour cela appliquer les mêmes
protocoles mais sur un plus grand nombre d'arbres.

Nous avons pris le parti de faire des mesures d'irradiance ``ponctuelles''
compte tenu des contraintes techniques imposées par les capteurs. Au lieu de
prendre des mesures sur une plage de temps restreinte, il aurait été possible de
faire des mesures de l'ombre tout au long de la journée. Cette proposition
de protocole a été avancée. Pour cela, il fallait placer les 6 capteurs sur un
demi arc de cercle de manière à suivre la course du soleil. Ce protocole n'était
pas envisageable avec les capteurs qui ont été utilisés du fait que les capteurs
cosinus pondéraient les rayons rasants. La valeur des flux photoniques aurait
été croissante jusqu'à midi, puis décroissante, indépendamment des conditions
naturelles. Des capteurs directionnels auraient été nécessaires, avec ces
capteurs il aurait fallu sélectionner les valeurs qui nous intéressaient en
fonction de la direction de mesure.

Des études sur le microclimat induit par les arbres peuvent apporter de nouveaux
éléments en accord avec cette transmission de la lumière. Des recherches sur
l'évapotranspiration des arbres peuvent apporter un complément d'informations,
il serait très intéressant de voir si l'association de l'évapotranspiration et
de l'ombrage induisent un microclimat favorable aux herbacés.



\bibliography{biblio}

\appendix

\chapter{Interface Tree Analyzer}\label{App:appxA}

\begin{figure}
  \centering
  \includegraphics[width=160mm]{entries.png}
  \caption{Fenêtre d'entrée des paramètres de l'image}
\end{figure}

\begin{figure}
  \centering
  \includegraphics[width=160mm]{entries_2.png}
  \caption{Fenêtre d'entrée des paramètres de calcul}
\end{figure}

\begin{figure}
  \centering
  \includegraphics[width=160mm]{TA_simple.png}
  \caption{Fenêtre principal d'un set de photographies}
\end{figure}



\chapter{Code source du script qui a produit les graphiques}\label{App:appxB}

\begin{minted}{r}
####### Script d'analyse des données de MARscope #######


## Part of the A7R project, this script was intended to produce figures
## for report.
## Copyright (C) 2016  Ludovic Hondet

## This program is free software: you can redistribute it and/or modify
## it under the terms of the GNU General Public License as published by
## the Free Software Foundation, either version 3 of the License, or
## (at your option) any later version.

## This program is distributed in the hope that it will be useful,
## but WITHOUT ANY WARRANTY; without even the implied warranty of
## MERCHANTABILITY or FITNESS FOR A PARTICULAR PURPOSE.  See the
## GNU General Public License for more details.

## You should have received a copy of the GNU General Public License
## along with this program.  If not, see <http://www.gnu.org/licenses/>.


# Initialisation du script
library(tikzDevice)
setwd("/media/win_shared/data_sync/Documents/INRA_PC/Data_MARscope/")


## Création des nuages de points
fluxLum = read.table("Carac_Lusignan_2016-07-09_002.txt", header = TRUE,
sep = ";", dec = ".")
attach(fluxLum)



# Sélection des données
capteur_01 = subset(fluxLum, Id == 'a01')    # Capteur n°1 + incident
capteur_02 = subset(fluxLum, Id == 'a02')    # Capteur n°2 + incident
capteur_03 = subset(fluxLum, Id == 'a03' & Ray == '2')
capteur_04 = subset(fluxLum, Id == 'a04' & Ray == '2')
capteur_05 = subset(fluxLum, Id == 'a05' & Ray == '2')
capteur_06 = subset(fluxLum, Id == 'a06' & Ray == '2')


# Tracer les graphiques

tikz("rapport_plots.tex", width = 10, height = 4, onefile = TRUE,
engine = "xetex")
par(mfrow = c(1,3), oma = c(1,1,1,0), mar = c(5,5,5,1))


HEADER = c("Flux_photons", "UVA_BLEU", "ZETA")
TITRE = c("PAR transmis", "UVA-bleu transmis", "ZETA transmis")
LABELS = c("PAR (µmol.m$^{-2}$.s$^{-1}$)", "Flux photonique UVA-bleu
(µmol.m$^{-2}$.s$^{-1}$)", "ZETA")


for(i in 1:3){
plot(strptime(capteur_01$Date_Heure, "%H:%M:%S"), capteur_01[,HEADER[i]],
     col = capteur_01$Ray, pch = 19,
     main = TITRE[i],
     xlab = "Temps UTC", ylab = LABELS[i],
     cex.main = 2, cex.lab = 1.5, cex.axis = 1.5)
points(strptime(capteur_03$Date_Heure, "%H:%M:%S"), capteur_03[,HEADER[i]],
       col = "green", pch = 19)
points(strptime(capteur_04$Date_Heure, "%H:%M:%S"), capteur_04[,HEADER[i]],
       col = "blue", pch = 19)

bozo = c("Incident", "F. excelsior", "A. cordata", "M. alba")
couleurs = c("black", "red", "green", "blue")
legend(x = "topright", legend = bozo, fill = couleurs, bty !='n', bg = "white",
       cex = 1.5)
}
dev.off()


## Création des boxplots

fluxLum = read.table("Carac_Lusignan_2016-07-09_004.txt", header = TRUE,
sep = ";", dec = ".")
attach(fluxLum)

TR = subset(fluxLum, Id=='a01'&Ray=='2' | Id=='a02'&Ray=='2' | Id=='a03'&Ray=='2'
| Id=='a05'&Ray=='2' | Id=='a04'&Ray=='2' | Id=='a06' & Ray=='2')
mean_in = c(1396.27,375.8148,1.113237)

tikz("rapport_barplots.tex", width = 9, height = 3, onefile = TRUE,
engine = "xetex")
par(mfrow = c(1,3), oma = c(1,1,1,0), mar = c(5,5,5,1))


HEADER = c("Flux_photons", "UVA_BLEU", "ZETA")
TITRE = c("PAR transmis", "UVA-bleu transmis", "ZETA transmis")
LABELS = c("PAR (µmol.m$^{-2}$.s$^{-1}$)", "UVA-bleu (µmol.m$^{-2}$.s$^{-1}$)",
"ZETA")


for(i in 1:3){
  boxplot(TR[,HEADER[i]]~Id, data = TR, range = 0, main = TITRE[i],
  xlab = "Espèces", ylab = LABELS[i],
  names = c("F","F","A","A","M","M"))
  legend(x = "bottomleft", legend = mean_in[i], title = "Incident",
  bty = "n")
  means = aggregate(TR[,HEADER[i]]~Id, data = TR, mean)
  means = as.list(means)
  points(1:6, means$`TR[, HEADER[i]]`, col = "red")
}
dev.off()
\end{minted}


\begin{abstract}
Dans le cadre d'une étude menée à l'INRA de Lusignan sur l'interaction entre les
arbres et les herbacés dans un système agroforestier, il s'agit de montrer que
la lumière tranmise par les arbres dépend des caractères morphologiques des
houppiers. Cette étude s'inscrit dans le cadre de la recherche sur des systèmes
de culture durable avec pour objectif de développer des systèmes durables et
rentables pour les éleveurs et agriculteurs. La question qui ce pose est de
savoir si il existe une différence de transmission de la lumière, compte tenu
des différences morphologiques qui peuvent exister dans les houppiers. Pour cela,
2 expériences ont été mises en place: mesure des caractères morphologiques des
houppiers et mesure des lumières incidentes et résultantes. La première
expérience a necessité un logiciel spécifique capable de déterminer les
caractères morphologiques à partir de 8 photographies. La seconde expérience a
necessité du matériel plus spécifique. Le système d'acquisition et de traitement
était capable de calculer des flux photoniques dans différentes longueurs d'onde
à partir des irradiances incidentes et transmises. Les résultats de ces 2
expériences ont ensuite été concaténé pour mettre en évidence un lien entre
morphologie des houppiers et transmission de la lumière. L'analyse des résultats
a mis en avant un lien entre la morphologie des houppiers et la transmission de
la lumière au sein de chacune des espèces.\\

\noindent In the context of a study at INRA of Lusignan on the interaction between trees
and meadows in an agroforestry system, the goal is to show that transmitted
light by trees depends on tree crowns morphological caracteristics. This study
is part of a research programm about sustainable agriculture with the aim of
developping sustainable cultural systems. The question is: Is there a diffrence
of transmitted light, knowing that there is a diffrence between tree crowns? To
answer this question, 2 experiments were conducted: measuring tree crowns
morphological caracteristics and mesuring of incident and transmitted lights.
The first one has needed a software which was able to compute tree crowns
morphological caracteristics with 8 pictures. The second one has needed some
more specific equipment. The acquisition system used was able to compute
photonic flows in different wavelenghts by measuring the incident and
transmitted irradiances. The results of those 2 experiments were concatenated in
order to show that there is a link between tree crowns morphological
caracteristics and transmitted light. The analysis of the results showed a link
between the transmission and tree crowns morphology in each species.

\end{abstract}


\end{document}

